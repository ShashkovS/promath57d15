% !TeX encoding = windows-1251
\documentclass[a4paper,12pt]{article}
\usepackage{newlistok}

%\УвеличитьВысоту{1.5cm}
\УвеличитьШирину{3mm}
\renewcommand{\spacer}{\smallskip}
\newcommand{\ndv}{\smash{\mskip3mu\not\lower1truept\hbox{\vdots}\mskip3mu}}

\Заголовок{Целые числа\т 1}
\НомерЛистка{8}
\ДатаЛистка{}

\begin{document}
\СоздатьЗаголовок

\задача
Докажите, что Ваше 28-летие будет в такой же день недели, в какой Вы родились. \кзадача

\опр
Пусть $a$ и~$b$\т целые числа, причём $b\ne 0$. Говорят, что \emph{$a$~делится на~$b$}, если существует такое целое число~$c$, что $a=bc$. В~этом случае говорят, что $a$~\выд кратно числу~$b$; число~$b$ называется \выд делителем числа~$a$, число $c$ называется \выд частным от деления~$a$ на~$b$. \\
Обозначение: $a \dv b$ ($a$~делится на~$b$) или $b \divs a$ ($b$~делит~$a$).
\копр

\задача
Докажите, что любые целые числа удовлетворяют следующим свойствам:\\
\пункт
если $a \dv c$ и~$b \dv c$, то $(a\pm b)\dv c$;
\пункт
если $a \dv c$ и~$b$\т произвольное целое число, то $ab \dv c$;\\
\пункт
если $a \dv b$ и~$b \dv c$, то $a \dv c$;
\пункт
если $a \dv b$, то либо $a=0$, либо $|a| \ge |b|$;\\
\пункт
если $a \dv b$ и~$b \dv a$, то $|a|=|b|$.
\кзадача

\задача
Верно ли, что любые целые числа удовлетворяют следующим свойствам:\\
\пункт
если $a \dv c$ и~$b \ndv c$, то $(a+b)\ndv c$;
\пункт
если $a \dv b$ и~$b \ndv c$, то $a\ndv c$;
\пункт
если $a\ndv b$ и~$b\dv c$, то $a\ndv c$;\\
\пункт
если $a\ndv c$ и~$b\ndv c$, то $ab\ndv c^2$;
\пункт
если $a\dv c$, $b\dv c$, то для любых целых $x$ и~$y$ выполнено $(ax+by)\dv c$?
\кзадача

\задача
Пусть $m,n$\т целые, и $5m + 3n\dv11$. Докажите, что
\пункт
$6m+8n\dv11$;
\пункт
$9m + n\dv11$.
\кзадача

\задача
Докажите, что если $(a^2+b^2)\dv 3$, то $a\dv 3$ и~$b\dv 3$.
\кзадача

\задача
Докажите, что
\пункт
$\overline{aaa}$ делится на $37$;
\пункт
$\overline{abc} - \overline{cba}$ делится на 99 (где $a$, $b$, $c$\т цифры). \кзадача

\задача
\пункт
Докажите, что целое число делится на~$4$ тогда и~только тогда, когда две его последние цифры образуют число, делящееся на~$4$.\\
\пункт
Сформулируйте и~докажите признаки делимости на~$2$, $5$, $8$, $10$.
\кзадача

\задача
\пункт
Из натурального числа $\overline{a_n\ldots a_1a_0}$ вычли сумму его цифр $a_n + \ldots + a_1 + a_0$. Докажите, что получилось число, делящееся на 9.
\пункт
Выведите из пункта а) признаки делимости на~3~и~на~9.
\кзадача

\задача
Докажите, что число, составленное из 81 единицы, делится на 81.
\кзадача

\задача
Докажите, что целое число $\overline{a_na_{n-1}\dots a_1a_0}$ делится на~$11$ если и~только если знакопеременная сумма его цифр $(a_0-a_1+a_2-a_3+\ldots+(-1)^n a_n)$ делится на~$11$.
\кзадача

\сзадача
Сформулируйте и~докажите признак делимости на~$7$.
\кзадача

\задача
Докажите, что $m(m + 1)(m + 2)$ делится на 6 при любом целом $m$.
\кзадача

\задача
Числа $a,b,c,d$\т натуральные. Обязательно ли число $\displaystyle\frac{(a + b + c + d)!}{a!\,b!\,c!\,d!}$ целое?
\кзадача

\задача
Докажите, что произведение~$n$ подряд идущих целых чисел делится на $n!$.
\кзадача

\задача
Целые числа $a$ и $b$ различны. Докажите, что $(a^n - b^n) \dv (a - b)$ при любом натуральном~$n$.
\кзадача

\задача
Найдите все целые $n$, при которых число $(n^3 + 3)/(n + 3)$ целое.
\кзадача

\задача
Решите в натуральных числах уравнения:
\пункт
$x^2 - y^2 = 31$;
\пункт
$x^2 - y^2 = 303$.
\кзадача

\задача
Может ли $n!$ оканчиваться ровно на 4 нуля? А ровно на 5 нулей?
\кзадача

\vfill
\ЛичныйКондуит{0mm}{6mm}
\ОбнулитьКондуит
\newpage

\опр Натуральное число $p > 1$ называется \выд{простым}, если оно имеет ровно два натуральных делителя: 1 и $p$. В противном случае оно называется \выд{составным}.
\копр

\задача
Докажите, что любое натуральное число, большее 1, либо само простое, либо раскладывается в произведение нескольких простых множителей.
\кзадача

\задача
\пункт
Даны натуральные числа $a_1$, \ldots, $a_n$, большие 1. Придумайте число, которое не делится ни на одно из чисел $a_1$, \ldots, $a_n$.
\пункт
Докажите, что простых чисел бесконечно много.\\
\пункт
Докажите, что простых чисел вида $3k + 2$ бесконечно много ($k$\т натуральное). \кзадача

\задача
\пункт
Могут ли 100 последовательных натуральных чисел все быть составными? \\
\пункт
Найдутся ли 100 последовательных натуральных чисел, среди которых ровно 5 простых?
\кзадача

\опр
Пусть $a$ и $b$\т целые числа, $b > 0$. \выд{Разделить} $a$ на $b$ \выд{с остатком} значит найти такие целые числа $k$ (неполное частное) и $r$ (остаток), что $a = kb + r$ и $0\le r < b$.
\копр

\задача
Числа $a$ и $b$\т целые, $b>0$. Отметим на числовой прямой все числа, кратные~$b$. Они разобьют прямую на отрезки длины $b$. Точка $a$ лежит на одном из них. Пусть $kb$\т левый конец этого отрезка. Докажите, что $k$\т частное, а $r = a - kb$\т остаток от деления $a$ на $b$.
\кзадача

\задача
Докажите, что частное и остаток определены однозначно.
\кзадача

\задача
Найдите частные и~остатки от деления $2012$ на~$23$, $-19$ на~$4$ и $n^2-n+1$~на~$n$.
\кзадача

\задача
Какой цифрой оканчивается число
\пункт
$14^{14}$;
\пункт
$14^{14^{14}}$;
\пункт
$7^{7^7}$?
\кзадача

\задача
Найдите остатки от деления
\пункт
$2^{2012}$ на~$3$;
\пункт
$57^{2012}$ на~$5$;
\пункт
$(12^{14}+14^{12})$ на~$13$;\\
\пункт
$(2222^{5555}+5555^{2222})$ на~$7$.
\кзадача

\задача
Найдите все такие натуральные $k$, что $2^k - 1$ делится на $7$.
\кзадача

\задача
Даны 20 целых чисел, ни одно из которых не делится на 5. Докажите, что сумма двадцатых степеней этих чисел делится на 5.
\кзадача

\задача
Докажите, что остаток от деления простого  числа на 30 есть простое число или 1.
\кзадача

\задача
Докажите, что из любых 52 целых чисел можно выбрать 2 таких числа, что\\
\пункт
их разность делится на~$51$;
\пункт
их сумма или их разность делится на~$100$.
\кзадача

\сзадача
Докажите, что из любых $n$ целых чисел всегда можно выбрать несколько, сумма которых делится на $n$ (или одно число, делящееся на $n$).
\кзадача

\сзадача
Из чисел $1$, $2$, $3$, \dots , $100$ выбрали произвольным образом 51 число. Докажите, что среди выбранных чисел найдутся два, одно из которых делится на другое.
\кзадача

\сзадача
Числа $2$, $3$, $7$ обладают следующим свойством: $(2\cdot3+1) \dv 7$, $(3\cdot7+1) \dv 2$, $(7\cdot2+1) \dv 3$. Существуют ли ещё тройки натуральных чисел, больших~$1$, с~таким свойством?
\кзадача

\vfill
\ЛичныйКондуит{0mm}{6mm}
%\GenXMLW
%\СделатьКондуит{6mm}{6mm}

\end{document}




