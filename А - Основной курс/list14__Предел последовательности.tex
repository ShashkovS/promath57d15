% !TeX encoding = windows-1251
\documentclass[a4paper,12pt]{article}
\usepackage{newlistok}
% \usepackage{tikz}
% \usetikzlibrary{calc}
\usepackage[framemethod=tikz]{mdframed}


\global\addtolength{\vsize}{-115mm}%
\global\advance\vsize by 10cm
\global\advance\vsize by 10cm

\ВключитьКолонитул

\УвеличитьВысоту{1.5cm}
\УвеличитьШирину{1.5cm}
%\renewcommand{\spacer}{\vfil}

\Заголовок{Предел последовательности}
\НомерЛистка{14}
\ДатаЛистка{09.2013}

\begin{document}
\СоздатьЗаголовок

\опр
\label{limit1}
Число $a$ называют \выд{пределом последовательности} $(x_n)$, если $(x_n)$ можно представить в виде $x_n = a + \al_n$, где последовательность $(\al_n)$ бесконечно малая.
Обозначение: $\limn x_n = a$.
Говорят также, что \выд{$(x_n)$ стремится к $a$ при $n$, стремящемся к бесконечности} (и пишут $x_n \to a$ при $n \to \infty$).
\копр

\опр
\выд{$\ep$-окрестность} точки $a$ (где $\ep > 0$)\т это интервал $(a-\ep, a+\ep)$. Обозначение: $\Uc_\ep(a)$.
\копр

\опр
\label{limit2}
Число $a$ называют \выд{пределом последовательности} $(x_n)$, если для всякого числа $\ep > 0$ найдётся такое число $N$, что при любом натуральном $k > N$ будет выполнено неравенство $|x_k - a| < \ep$.\\
Формально: $\forall\,\ep>0\ \ \exists\,k\in\N\ \ \forall\,n>k \ \ |x_n-a|<\ep$.
\копр

\опр
\label{limit3}
Число $a$ называют \выд{пределом последовательности} $(x_n)$, если в любом интервале, содержащем $a$, содержатся \выд{почти все} члены $(x_n)$ (то есть все, кроме конечного числа).
\копр

\утверждение
Последовательность не может иметь более одного предела.
\кутверждение

\утверждение
Пусть последовательность $(x_n)$ имеет предел. Тогда $(x_{n+1} - x_n)$\т бесконечно малая.
\кутверждение

\утверждение
Определения~\ref{limit1},~\ref{limit2}~и~\ref{limit3} эквивалентны.
\кутверждение

\bigskip
\hrule
\bigskip

\задача
В два сосуда разлили (не поровну) 1 л воды. Из 1-го сосуда перелили половину имеющейся~в~нём воды во 2-ой, затем из 2-го перелили половину оказавшейся в нём воды в 1-ый, снова из 1-го  перелили половину во 2-ой, \итд Сколько воды (с точностью до 1 мл) будет в 1-ом сосуде после 50 переливаний?
\кзадача

\задача
Найдите предел $(x_n)$ (если он существует):
%\невСтрочку
\тааа%
{\пункт $x_n=1+(-1)^n$;}%
{\пункт $x_n=1+(-0.1)^n$;}%
{\пункт $x_n=\dfrac{n}{n+1}$;}
\medskip
\тааа%
{\пункт $x_n=(2^n-1)/(2^n+1)$;}%
{\пункт $x_n=1+0.1+\ldots+(0.1)^n$;}%
{\пункт $x_n=\sqrt{n+1}-\sqrt{n}$.}%
\кзадача

\задача
Запишите, не используя отрицания:
\вСтрочку
\пункт
\лк число $a$ не предел $(x_n)$\пк;
\пункт
\лк $(x_n)$ не имеет~предела\пк.
\кзадача

\задача
Предел $(x_n)$ положителен. Верно ли, что все члены $(x_n)$, начиная с некоторого, положительны?
\кзадача

\begin{mdframed}[hidealllines=true, leftline=true, linecolor=gray!60,rightmargin=-10pt,outerlinewidth=4pt,innerbottommargin=.2\baselineskip,innertopmargin=.2\baselineskip,]
\взадача
Последовательность $(x_n)$ имеет предел $a$.
\вСтрочку
\пункт
Обязательно ли $(x_n)$ ограничена?\\
\пункт
Пусть $a>0$ и все члены $(x_n)$ положительны. Докажите, что последовательность $(1/x_n)$ ограничена.
\кзадача
\end{mdframed}

\begin{mdframed}[hidealllines=true, leftline=true, linecolor=gray!60,rightmargin=-10pt,outerlinewidth=4pt]
\взадача[Арифметика пределов]
Пусть
$\limn x_n = a$,
$\limn y_n = b$.
Докажите:
\невСтрочку
\пункт
$\limn (x_n \pm y_n)= a \pm b$;
\пункт
$\limn (x_n \cdot y_n) = ab$;
\пункт
если $b\ne0$ и все элементы последовательности $(y_n)$ отличны от нуля, то $\limn (x_n/y_n) = a/b$.
\кзадача
\end{mdframed}

\задача
Найдите предел $(x_n)$ (если он существует):
\тааа%
{\пункт $x_n=1+q+\ldots+q^{n}$;}%
{\пункт $x_n=(n^2+5n+7)/n^2$;}%
{\пункт $x_n=C^{50}_n/n^{50}$;}
\medskip
\тааа%
{\пункт $x_n=n^{50}/10^n$;}%
{\пункт $x_n=\sqrt[n]{n}$;}%
{\пункт $x_n=1/2+2/2^2+3/2^3+\ldots+n/2^n$.}%
\кзадача

\ЛичныйКондуит{0mm}{6mm}
\ОбнулитьКондуит

\newpage

\begin{mdframed}[hidealllines=true, leftline=true, linecolor=gray!60,rightmargin=-10pt,outerlinewidth=4pt]
\взадача
Пусть $A(x)=a_k x^k+\ldots+a_1x+a_0$ и $B(x)=b_m x^m+\ldots+b_1x+b_0$\т многочлены степеней $k$ и $m$ соответственно. Найдите пределы:
\вСтрочку
\пункт
$\limn A(n)/n^k$;
\пункт
$\limn A(n)/B(n)$.
\кзадача
\end{mdframed}


\задача
Последовательность $(x_n)$ с положительными членами такова, что последовательность $(x_{n+1}/x_n)$ имеет пределом некоторое число, меньшее 1. Докажите, что $(x_n)$ бесконечно малая.
\кзадача

\задача
Найдите:
\вСтрочку
\пункт $\limn \dfrac{4n^2}{n^2+n+1}$
\пункт $\limn \dfrac{n^2+2n-2}{n^3+n}$;
\пункт $\limn \dfrac{n^9-n^4+1}{2n^9+7n-5}$.
\кзадача

\задача
Найдите ошибку в рассуждении:
\лк Пусть $x_n=(n-1)/n$. Тогда $\limn x_n = \limn(1-1/n)=1$. С~другой стороны, $\limn x_n = \limn(1/n)\cdot\limn(n-1) = 0\cdot\limn (n-1)= 0$. Отсюда $0=1$.\пк
\кзадача

\задача
Пусть $\limn x_n =a$, $\limn y_n = b$, причём $x_n>y_n$ при $n\in\N$. Верно ли, что
\вСтрочку
\пункт $a>b$;
\пункт $a\geq b$?
\кзадача

\vspace*{-2.1truemm}

\задача
Пусть $\limn x_n = 1$. Найдите
\вСтрочку
\пункт  $\limn\dfrac{x_n^2}{7}$;
\спункт $\limn\dfrac{x_1+\ldots+x_n}n$.
\кзадача


\УстановитьГраницы{0cm}{3.2cm}
\задача
Возьмём любое положительное число $x_0$ и построим последовательность по такому закону: $x_{n+1}=0.5\cdot(x_n+a/x_n)$.
\невСтрочку
\пункт
Докажите, что $\limn x_n=\sqrt a$.
\спункт
Сколько понадобится последовательных приближений, чтобы найти $\sqrt{10}$ с точностью до $0.0001$, если в качестве начального приближения взять $x_0 = 3$?
\кзадача


\vspace*{-2truemm}

\задача
\пункт
\putpict{15.3cm}{1.4cm}{list14_1}{}
\putpict{15.3cm}{-1.5cm}{list14_2}{}
Рассмотрим фигуру, ограниченную графиком функции $y=x^2$, осью $Ox$ и прямой $x=1$. Разобьём отрезок $[0,1]$ на $n$ равных частей и построим на каждой части прямоугольник так, чтобы его правая верхняя вершина лежала на графике (см.~рис.). Сумму площадей прямоугольников обозначим $S_n$. Найдите $\limn S_n$.
\\\пункт
Построим прямоугольники так, чтобы их левые верхние вершины лежали на графике (см.~рис.). Сумму их площадей обозначим $s_n$. Докажите, что $(s_n)$ стремится к тому же числу, что и $(S_n)$.
\кзадача

\ВосстановитьГраницы

\vfill
\ЛичныйКондуит{0mm}{6mm}

%\GenXMLW

\end{document}




