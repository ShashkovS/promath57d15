% !TeX encoding = windows-1251
\documentclass[a4paper,12pt]{article}
\usepackage{newlistok}

\ВключитьКолонитул

\УвеличитьВысоту{2cm}
\УвеличитьШирину{1cm}
%\renewcommand{\spacer}{}

\Заголовок{Преобразования плоскости}
\НомерЛистка{25}
\ДатаЛистка{11.2014}

\begin{document}

\СоздатьЗаголовок

\раздел{Движения}

\опр
\выд{Движением} называется взаимно-однозначное преобразование плоскости, не меняющее расстояния между точками.\\
 Обозначим тождественное преобразование плоскости через $E$, параллельный перенос на вектор $\vec{a}$ через $T_{\vec{a}}$, осевую симметрию относительно прямой $l$ через $S_l$, центральную симметрию относительно точки $O$ через $Z_O$, поворот вокруг точки $O$ на угол $\alpha$ через~$R_O^\alpha$.
\копр

\задача
Найдите композиции:
\вСтрочку
\пункт $T_{\vec{x}}\circ T_{\vec{y}}$;
\пункт $Z_A\circ Z_B$;
\пункт $S_l\circ S_m$;
\пункт $T_{\vec{x}}\circ Z_A$; \\
\пункт $S_l\circ R_O^\al$, где $O\in l$;
\пункт $n$ центральных симметрий (с разными центрами).
\кзадача

\задача
При каких $n$ можно однозначно восстановить $n$-угольник по серединам сторон?
\кзадача

\задача
Верно ли, что для любых движений $F$, $G$ и $H$
\вСтрочку
\пункт $F\circ G = G\circ F$;
\пункт $F\circ (G\circ H) = (F\circ G)\circ H$?
\кзадача

\сзадача
Найдите композиции:
\вСтрочку
\пункт $T_{\vec{x}}\circ R_O^\al$;
\пункт $S_l\circ R_O^\al$, где $O\not\in l$;
\пункт $R_A^\al\circ R_B^\be$
\кзадача

\ввзадача
Пусть точки $A$, $B$ и $C$ не лежат на одной прямой.
Докажите, что движение однозначно определяется тем,
куда оно переводит эти точки.
\кзадача

\опр
\выд{Скользящей симметрией} называется движение, являющееся композицией осевой симметрии и параллельного переноса на вектор, параллельный оси. (Если вектор нулевой, получается обычная осевая симметрия; мы будем рассматривать её как частный случай скользящей симметрии.)
\копр

\задача
Докажите, что композиция осевой симметрии и параллельного переноса на произвольный вектор\т скользящая симметрия.
\кзадача

\опр
\выд{Неподвижной точкой} преобразования $F$ называется такая точка $x$, что $F(x)=x$.
\копр

\задача
\пункт
Найдите множество неподвижных точек для тождественного преобразования, поворота, параллельного переноса, симметрии и скользящей симметрии (с ненулевым вектором). \\
\пункт
Докажите, что множество неподвижных точек движения плоскости есть либо пустое множество, либо одна точка, либо прямая, либо вся плоскость.
\кзадача

\ввзадача[Теорема Шаля]
Докажите, что любое движение плоскости есть поворот, параллельный перенос или скользящая симметрия.
\кзадача

\взадача
Докажите, что любое движение плоскости можно представить как композицию не более чем трёх осевых симметрий.
\кзадача

\задача
Пусть композиция $n$ осевых симметрий равна композиции $m$ осевых симметрий. Докажите, что $(n-m)$ чётно.
\кзадача

\сзадача
Опишите все движения трёхмерного пространства, имеющие хотя бы одну неподвижную точку.
\кзадача

\ЛичныйКондуит{-0.3mm}{6mm}
\ОбнулитьКондуит
\newpage

\раздел{Преобразования подобия}

\опр
\выд{Преобразованием подобия} с коэффициентом $k>0$ называется преобразование плоскости, меняющее расстояния между точками ровно в $k$ раз. \выд{Гомотетия} $H_O^k$ с центром в точке $O$ и коэффициентом $k\neq 0$ переводит каждую точку $A$ в такую точку $A'$, что $\overrightarrow {OA'} = k\overrightarrow {OA}$.
\копр

\задача
Какое преобразование является композицией двух гомотетий с коэффициентами $k_1$ и $k_2$, если
\вСтрочку
\пункт $k_1 k_2 = 1$;
\пункт $k_1 k_2 \ne 1$?
\кзадача

\задача
\пункт
Даны два параллельных отрезка разной длины. Укажите все гомотетии, переводящие первый отрезок во второй.
\пункт[Замечательное свойство трапеции]
Докажите, что в любой трапеции точка пересечения диагоналей, точка пересечения продолжений боковых сторон и середины оснований лежат на одной прямой.
\кзадача

\задача
Какое преобразование является композицией гомотетии и параллельного переноса?
\кзадача

\задача
\пункт
Даны две окружности. Укажите все гомотетии, переводящие первую во вторую.
\пункт
Даны три окружности различных радиусов. Для каждой пары окружностей нашли точку пересечения их общих внешних касательных. Докажите, что эти три точки лежат на
одной прямой.
\кзадача

\задача
В окружности проведены два непараллельных радиуса. Постройте хорду, которая делится этими радиусами на три равные части.
\кзадача

\взадача
Докажите, что любое преобразование подобия есть композиция гомотетии и движения.
\кзадача

\задача
Можно ли перевести
\вСтрочку
\пункт
любую параболу в любую другую параболу преобразованием подобия;
\пункт
график функции $y=\sin x$ в график функции $y=\sin^2 x$ преобразованием подобия? А гомотетией?
\кзадача

\ввзадача
Докажите, что всякое преобразование подобия с коэффициентом, не равным $1$,\\
\вСтрочку
\пункт
имеет неподвижную точку;
\пункт
является композицией гомотетии и поворота с общим центром или композицией гомотетии и симметрии относительно оси, проходящей через центр гомотетии.
\кзадача

\задача
На стене висят двое часов, одни побольше, другие поменьше. Докажите, что прямые, соединяющие концы минутных стрелок в разные моменты времени, проходят через одну точку.
\кзадача

\ЛичныйКондуит{.3mm}{6mm}
% \GenXMLW

\end{document}

