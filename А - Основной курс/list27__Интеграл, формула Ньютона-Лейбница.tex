% !TeX encoding = windows-1251
\documentclass[a4paper,12pt]{article}
\usepackage[mag=950]{newlistok}

\УвеличитьШирину{1.2cm}
\УвеличитьВысоту{2.5cm}
\ВключитьКолонитул
\renewcommand{\spacer}{\vfil}

\begin{document}

\sloppy
\Заголовок{Интеграл --- 2}
\НомерЛистка{27}
\ДатаЛистка{\ммгг}
\СоздатьЗаголовок


\опр Пусть $f$ --- функция, определенная на некотором конечном или бесконечном
промежутке~$I$.
Её {\em первообразная} --- это такая дифференцируемая на этом
промежутке функция $F$, что $F'=f$.
Совокупность всех первообразных функции~$f$ на промежутке~$I$ называется {\it неопределенным интегралом} функции~$f$. Обозначение: $\int f(x)\,dx$.
\копр

\задача
Пусть $F_1$ и~$F_2$~---~две различные первообразные функции~$f$, определённой на промежутке $I$.
Докажите, что $(F_1-F_2)$~---~константа.
\кзадача


\задача Пусть функция $f$ непрерывна (на всём $\R$).
Зафиксируем произвольную точку $a$ и рассмотрим функцию $F(t) = \int\limits_a^tf(x)\, dx$ от
переменной $t$. \пункт Докажите, что функция $F$ непрерывна.
\\\пункт Докажите, что функция $F$ дифференцируема и найдите её производную.
\кзадача


\взадача[Формула Ньютона-Лейбница]
Пусть $f$ --- непрерывная функция на отрезке $[a,b]$ и $F$~--- её первообразная.
Докажите, что
\vspace*{-4mm}
$$\int\limits_b^af(x)\, dx=F(a)-F(b)$$
\vspace*{-5mm}
\кзадача


\задача
\впункт
Приведите пример функции, интегрируемой на отрезке $[a,b]$, но не имеющей первообразной на интервале $(a,b)$.
\спункт
Приведите пример функции, не имеющей первообразной ни на одном~интервале.
\спункт
Приведите пример функции, не интегрируемой на отрезке $[a,b]$, но имеющей первообразную на интервале $(a,b)$.
\спункт
Приведите пример разрывной функции, у~которой существует первообразная.
\кзадача

\взадача[Линейность неопределённого интеграла] Пусть на некотором промежутке существуют неопределенные
интегралы $\int f(x)\,dx$ и $\int g(x)\,dx$.
Тогда для любых постоянных $\alpha$ и $\beta$ на этом промежутке существует
неопределенный интеграл $\int (\alpha f(x)+\beta g(x))\,dx$ и $ \int
(\alpha f(x) + \beta g(x))\,dx = \alpha \int f(x)\,dx + \beta \int
g(x)\,dx. $
\кзадача


\взадача[Формула замены переменных]
Пусть $\omega (x)$ --- дифференцируемая функция с непрерывной производной.
Пусть $f$ --- непрерывная функция, и $F(x)$ --- её первообразная.
Докажите, что
\vspace*{-3mm}
$$\int f(\omega (x))\omega '(x)\, dx=F(\omega (x)) + С.$$
\vspace*{-5mm}
\кзадача

\задача
\пункт Пусть функция $\phi(x)$ монотонна и~дифференцируема на отрезке~$[\alpha,\beta]$, а~её производная $\phi'(x)$ непрерывна на этом отрезке. Пусть, кроме того, $\phi(\alpha)=a$ и~$\phi(\beta)=b$. Докажите, что для любой непрерывной на отрезке $[a,b]$ функции~$f$
\vskip -9mm
$$
\qquad\qquad\qquad\int\limits_a^bf(t)\,dt=\int\limits_{\alpha}^{\beta}f(\phi(x))\phi'(x)\,dx
$$
\vskip -2mm
\впункт Справедливо ли утверждение пункта а), если $\phi(x)$ не является монотонной?
\кзадача

\пример
С помощью формулы замены переменных можно эффективно вычислять многие определённые интегралы.
Приведём примеры того, как ей пользуются.
\vspace*{-2mm}
$$
\int\limits_{a=0}^{b=1} x \cdot (2 - x^2)^5 \, dx =
\left|
\begin{smallmatrix}
t = f(x)=2-x^2\\
dt = f'dx=-2x\,dx\\
x\,dx = -1/2 dt\\
A = f(a) = 2\\
B = f(b) = 1
\end{smallmatrix}
\right|
=\int\limits_{A=2}^{B=1} t^5 \hr{-\dfrac{1}{2}}\,dt
=-\dfrac{1}{2} \int\limits_{A=2}^{B=1} t^5\,dt
=\dfrac{1}{2} \int\limits_{1}^{2} t^5\,dt
=\dfrac{1}{2} \hr{\dfrac{2^6}{6} - \dfrac{1^6}{6}}
=\dfrac{21}{4}
$$
\vspace*{-2mm}
$$
\int\limits_{a=-1}^{b=1} \sqrt{1-x^2} \, dx =
\left|
\begin{smallmatrix}
t = f(x)=\arcsin x &           f^{-1}(t)= \sin t = x \\
dt = f'dx=dx/\sqrt{1-x^2}\ & (f^{-1})'dt = \cos t\,dt = dx\ \\
         A = f(a) = -\pi/2& B = f(b) = \pi/2
\end{smallmatrix}
\right|
=\int\limits_{A=-\pi/2}^{B=\pi/2} |\cos t\,| \cos t \,dt = \int\limits_{-\pi/2}^{\pi/2} \dfrac{\cos 2t +1}{2}\, dt = \dfrac{\pi}{2}
$$
\vspace*{-3mm}
\кпример


\задача
Вычислите:
\вСтрочку
\пункт $\int e^{e^x+x}\, dx$;
\пункт $\int xe^{x^2}\, dx$;
\пункт $\int \frac{\ln x}{x}\, dx$;
\пункт $\int \sin x \cos x \, dx$;
\пункт $\int \frac{\sin x }{\cos^3 x }\, dx$.
\кзадача

\взадача[Интегрирование по частям]
Пусть $u(x)$ и $v(x)$ --- дифференцируемые функции.
Пусть существует интеграл $\int u(x)v'(x)\, dx$.
Докажите, что существует интеграл $\int u'(x)v(x)\, dx$ и
\vspace*{-3mm}
$$
\int u'(x)v(x)\, dx=u(x)v(x)-\int u(x)v'(x)\, dx.
$$
\vspace*{-5mm}
\кзадача

\задача
Найдите следующие неопределённые интегралы ($k\in\N$): %интегралы:
\\
\вСтрочку
\пункт $\int \ln x \, dx$;
%\пункт $\int xe^x\, dx$;
\пункт $\int x^ke^x\, dx$; %, k\in \N$;
\пункт $\int e^x\sin x \, dx$;
%\пункт $\int e^x\cos x \, dx$;
\пункт $\int \ln^k x \, dx$; %, k\in \N$.
\пункт $\int\limits_0^\pi x\sin x\, dx$.
\кзадача
\vspace*{-3mm}

%\vfil
\ЛичныйКондуит{0mm}{6mm}
\ОбнулитьКондуит
\newpage

\ввзадача [Формула Тейлора] Пусть $f(x)$ --- функция с непрерывной $n+1$
производной. Докажите, что
\vspace*{-3mm}
$$
f(x)=f(x_0)+\sum_{k=1}^n\frac{f^{(k)}(x_0)}{k!}(x-x_0)^k+
\frac{1}{n!}\int\limits_{x_0}^x(x-t)^nf^{(n+1)}(t)\, dt.
$$
\vspace*{-3mm}
\кзадача



\раздел{Физические приложения интеграла}

\задача
\пункт Шпанская мушка летает по комнате так, что расстояние от неё до двух соседних стен и пола  в момент времени $t$ --- это $x(t)$, $y(t)$ и $z(t)$ соответственно. Найдите скорость мушки.
\\\пункт Найти длину произвольного куска параболы $y=x^2$.
\кзадача

\задача Пусть пара непрерывно дифференцируемых функций $(x(t),y(t))$, $0\leqslant t \leqslant T$ задаёт замкнутую
несамопересекающуюся кривую. Кривая ограничивает область площади $S$.
Доказать, что
\vspace*{-3mm}
$$
S = \left| \int\limits_0^T y(t) x'(t) dt \right|.
$$
\vspace*{-7mm}
\кзадача


\задача
Окружность радиуса $R$ катится по прямой с угловой скоростью $\omega$. На окружности зафиксировали точку. Кривая, по которой движется эта точка, называется \выд{циклоидой}. Задайте кривую параметрически (то есть в виде $(x(t),y(t))$) и найдите площадь одной арки циклоиды.
\кзадача

\задача
Пусть задана плотность проволоки $\varrho(x)$. Как с помощью интегрирования найти её
\\\пункт массу; \пункт центр масс?
\кзадача

\задача
Доказать, что объём тела, образованного вращением вокруг оси $Oy$ плоской фигуры $0 \leqslant a \leqslant x \leqslant b,\, 0\leqslant y \leqslant y(x)$, где $y(x)$
--- непрерывная функция, равен $V = 2 \pi \int\limits_a^b x y(x) \, dx$.
\кзадача

\задача
\пункт Найдите объём шара радиуса $R$.
\пункт Определите центр масс однородного полушария радиуса $R$.
\пункт Найдите площадь сферы радиуса $R$.
\сспункт Найдите объём четырёхмерного шара радиуса~$R$.
\кзадача

\раздел{Упражнения по вычислению интегралов}

\задача Найдите первообразные следующих функций:\\
\вСтрочку
\пункт $f=1$;
\пункт $f=x$;
\пункт $f=x^k, k\in \N$;
\пункт $f=1/x$;
\пункт $f=x^k, k\in \Z$;\\
\пункт $f=e^x$;
\пункт $f=\sin x $;
\пункт $f=\cos x $;
\пункт $f=\tg x $;
\пункт $f=\ctg x $.
\кзадача

\задача Найдите первообразные следующих функций:\\
\вСтрочку
\пункт $f=5x^2-1$;
\пункт $f=1-\cos 3x$;
\пункт $f=\dfrac{6}{(5x-7)^3}$;
\пункт $f=7\sin\dfrac{x}{3}+\dfrac{2}{\cos^2{4x}}$;
\кзадача


\задача
Найдите для функции $f$ первообразную $F$, проходящую через точку $M$\\
\пункт $f(x)=x^3$, $M=(2;1)$;
\пункт $f(x)=\sqrt{x}$, $M=\hr{-\frac12;3}$;
\пункт $f(x)=\sin x$, $M = (\pi, 7)$.
\кзадача


\задача
Вычислите следующие интегралы:\\
\пункт
$\displaystyle\int\limits_{ -1     }^{ 1       }  x^4         \,   dx;$
\пункт
$\displaystyle\int\limits_{   \pi  }^{  \pi/2  }  \cos x      \,   dx;$
\пункт
$\displaystyle\int\limits_{ -\pi/2 }^{ -\pi/3  }\frac{1}{\sin^2x}\,   dx;$
\пункт
$\displaystyle\int\limits_{  1     }^{  4     }\frac1{\sqrt x}\,   dx;$
\пункт
$\displaystyle\int\limits_{ -1      }^{2        } \frac{1}{(2x+1)^2}        \,   dx;$
\пункт
$\displaystyle\int\limits_{  1     }^{ 2       }  \frac{x+1}{(2x-1)^3}            \,   dx;$
\кзадача


\задача
Вычислите площадь фигуры, ограниченной линиями:\\
\пункт
$y=2x-x^2$, $y=0$;
\пункт
$y=(x+2)^2$, $y=0$, $x=0$;
\пункт
$y=\sin x$, $y=0$, $0\le x\le \pi$;
\пункт
$y = x^{2n} $, $ y = 1$;\\
\пункт
$у=-\dfrac{1}{\sqrt{x}}, y = 0, x=1, x=4$
\пункт
$y = \dfrac{4}{x^2} $, $ y = 7-3x$;
\пункт
$y = x^2+2x+2 $, $ y = 2+4x-x^2$;
\кзадача

\vfil
\ЛичныйКондуит{0.3mm}{6mm}

% \GenXMLW

\end{document}


