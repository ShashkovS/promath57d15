% !TeX encoding = windows-1251
\documentclass[a4paper,12pt]{article}
\usepackage{newlistok}
%\input{haidar_edition}

\УвеличитьШирину{1cm}
\УвеличитьВысоту{1.7cm}
\ВключитьКолонитул
\renewcommand{\spacer}{\vfill}

\newcommand{\Rim}[2]{{\cal R}([#1,#2])}
\newcommand{\RimAB}{\Rim {a}{b}}

\Заголовок{Определённый интеграл}
\НомерЛистка{26}
\ДатаЛистка{12.2014}
\sloppy

\begin{document}
\СоздатьЗаголовок


\опр
Всякий конечный набор точек $\sigma=\{x_0,x_1,\ldots,x_n\}$, удовлетворяющий условию $a=x_0 < x_1 < \ldots < x_{n-1} < x_n = b$, называется \emph{разбиением отрезка $[a,b]$}. Разность $x_i - x_{i-1}$ обозначается через $\Delta x_i$.
\копр

\опр
\label{Darbu}
Пусть $\sigma$\т некоторое разбиение отрезка $[a,b]$, а $f$\т функция, ограниченная на этом отрезке. 
Положим
$m_i=\inf\limits_{[x_{i-1}, x_i]}f(x)$,
$M_i=\sup\limits_{[x_{i-1}, x_i]}f(x)$.
Числа $s_{\sigma}=\sum\limits_{i=1}^n m_i\Delta x_i$ и $S_{\sigma}=\sum\limits_{i=1}^n M_i\Delta x_i$ называются соответственно \emph{нижней и верхней суммами Дарбу} функции~$f$ при разбиении~$\sigma$.
\копр

\задача
Объясните геометрический смысл верхней и нижней сумм Дарбу и \лк нарисуйте\пк их для~функций:
\пункт
$f(x)=x$ на отрезке $[0,1]$ при разбиении $\sigma=\{\frac i4 \mid i=0,\ldots,4\}$;\\ \пункт
$f(x)=(x-1)^2$ на отрезке~$[0,2]$ при разбиении $\sigma=\{\frac i4 \mid i=0,\ldots,8\}$.
\кзадача

\задача
Можно ли исключить из определения~\ref{Darbu} условие ограниченности функции~$f$ на отрезке~$[a,b]$?
\кзадача

\задача
Найдите нижние и верхние суммы Дарбу при разбиении $\sigma=\{\frac in \mid i=0,\ldots,n \}$ отрезка~$[0,1]$ для функций
\вСтрочку
\пункт $f(x)=x$;
\пункт $f(x)=x^2$;
\пункт $f(x)=\sin(\pi nx)$.
\кзадача

\задача
Что происходит с суммами Дарбу при добавлении к разбиению новых точек?
\кзадача

\задача
Докажите, что любая верхняя сумма Дарбу ограниченной на отрезке~$[a,b]$ функции~$f$ не~меньше любой нижней суммы Дарбу этой же функции.
\кзадача

\задача
Докажите, что для всякой ограниченной на отрезке~$[a,b]$ функции~$f$ существуют $I_*=\sup s_\sigma$ и $I^*=\inf S_\sigma$, где супремум и инфимум берутся по всем разбиениям $\sigma$ отрезка. Сравните $I_*$ и $I^*$. Эти числа называются
\emph{нижним и верхним интегралами Дарбу} функции~$f$ на отрезке~$[a,b]$.
\кзадача

\задача
Найдите $I_*$ и $I^*$ для\\
\пункт $f(x)=x$ на $[0,1]$;
\пункт $f(x)=x^2$ на $[0,1]$;
\пункт функции Дирихле на отрезке $[a,b]$.
\кзадача

\опр
Функция $f$ называется \emph{интегрируемой (по Риману)} на отрезке $[a,b]$, если она ограничена на этом отрезке, и $I_* =I^*$. Обозначение: $f\in\RimAB$. Число $I_* =I^*$ называется \emph{определённым интегралом (Римана)} функции~$f$ на отрезке~$[a,b]$. Обозначение: $\intl{a}{b}f(x)\,dx$.
\копр

\задача
Верно ли, что всякая ограниченная на отрезке функция интегрируема на нём?
\кзадача

\задача
\пункт Докажите, что интеграл от неотрицательной функции неотрицателен.\\
\пункт Верно ли, что если интеграл функции равен нулю, то и сама функция тождественно равна нулю?
\пункт А если эта функция неотрицательна?
\пункт А если эта функция неотрицательна и непрерывна?
\спункт Верно ли, что интеграл от строго положительной функции строго больше нуля?
\кзадача

\ввзадача
Докажите, что если $f\in\RimAB$ и для всех $x\in[a,b]$ выполнено $m\le f(x)\le M$, то
\vskip -4mm
$$
m(b-a) \le \intl{a}{b}f(x)\,dx \le M(b-a).
$$
\vskip -1mm
\кзадача

\взадача
Докажите, что если $f,g\in\RimAB$ и для всех $x\in[a,b]$ выполнено $f(x)\le g(x)$, то
\vskip -4mm
$$
\intl{a}{b} f(x)\,dx \le \intl{a}{b} g(x)\,dx.
$$
\vskip -1mm
\кзадача

\vfill
\ЛичныйКондуит{-0.3mm}{6mm}
\ОбнулитьКондуит
\newpage



\ввзадача [Линейность интеграла]
Докажите, что если $f,g\in\RimAB$, $\al,\be\in\R$, то функция $\al f+\be g$ также будет интегрируема на $[a,b]$, причём
\vskip -4mm
$$
\intl{a}{b}(\al f(x)+ \be g(x))\,dx = \al\intl{a}{b}f(x)\,dx + \be\intl{a}{b}g(x)\,dx.
$$
\vskip -1mm
\кзадача


%\newpage



\задача
Докажите, что
\невСтрочку
\пункт
$I_*=I^*$ тогда и только тогда, когда $\inf_{\sigma} (S_\sigma-s_\sigma) = 0$;
\пункт
если $f \in \RimAB$, то $|f|\in \RimAB$ и выполнено неравенство
$
\displaystyle
\left|\intl{a}{b}f(x)\,dx\right| \leq \intl{a}{b}|f(x)|\,dx;
$
\пункт
если $f \in \RimAB$, то $f^2 \in \RimAB$;
\smallskip
\ввпункт
если $f,g \in \RimAB$, то $fg \in \RimAB$.
\smallskip
\спункт
Пусть $f,g \colon [0,1] \to [0,1]$ и $f,g \in \Rim01$. Верно ли, что $f\circ g \in \Rim01$?
\кзадача

\smallskip
\задача
Докажите, что если $f\in\RimAB$ и $[c,d]\subset [a,b]$, то $f\in\Rim{c}{d}$.
\кзадача


\ввзадача
Докажите, что монотонная на отрезке функция интегрируема на этом отрезке.
\кзадача


\ввзадача [Аддитивность интеграла]
Пусть $a<b<c$. Докажите, что если функция $f(x)$ интегрируема на двух из трёх отрезков $[a,b]$, $[b,c]$ и $[a,c]$, то она интегрируема и на третьем, причём
\vskip -4mm
$$
\intl{a}{c} f(x)\,dx = \intl{a}{b} f(x)\,dx + \intl{b}{c} f(x)\,dx.
$$
\vskip -7mm
\кзадача


\задача
Найдите: 
\вСтрочку
\пункт
$\displaystyle\intl{-1}{2} x\,dx$;
\пункт
$\displaystyle\intl{-2}{1} |x|\,dx$;
\пункт
$\displaystyle\intl{0}{2} x^2\,dx$;
\пункт
$\displaystyle\intl{-1}{1} (x^2-3x+1)\,dx$;
\спункт
$\displaystyle\intl{0}{1} x^n\,dx$.
\кзадача

\опр
Функция $f$ называется \выд{равномерно непрерывной} на множестве $M$, если для всякого~$\ep > 0$ найдётся такое $\de > 0$, что для любых $x,y\in M$ таких, что $|x-y|<\de$, выполнено $|f(x)-f(y)|<\ep$.
\hbox to -1em{}
\копр

\взадача
Верно ли, что функция $f(x)=1/x$ равномерно непрерывна на множестве\\
\пункт
$M=(0,1)$;
\пункт
$M=(1,+\infty)$?
\кзадача

\взадача
Верно ли, что функция $f$ равномерно непрерывна на своей области определения, если
\пункт
$f(x)=x^2$;
\пункт
$f(x)=\sqrt{x}$;
\пункт
$f(x)=\sin x$.
\кзадача

\взадача
Докажите, что всякая равномерно непрерывная функция является непрерывной.
\кзадача

\ввзадача [Теорема Кантора]
Докажите, что непрерывная на отрезке функция равномерно непрерывна на нём.
\кзадача

\взадача
Верно ли утверждение, аналогичное теореме Кантора, для функции, заданной на \вСтрочку
\пункт интервале;
\пункт прямой?
\кзадача

\ввзадача
Докажите, что непрерывная на отрезке функция интегрируема на нём.
\кзадача

\ввзадача
Докажите интегрируемость на отрезке
\пункт ограниченной функции с конечным числом точек разрыва;
\пункт функции Римана;
\спункт ограниченной функции со счётным числом точек разрыва.
\кзадача

%\взадача [Теорема Ньютона-Лейбница]
%Пусть функция $f$ непрерывна на $\R$. Зафиксируем точку $a$ и рассмотрим функцию $F(t) = \intl{a}{t} f(x)\,dx$.
%\пункт
%Докажите, что функция $F$ непрерывна.
%\пункт
%Верно ли, что функция $F$ дифференцируема? Если верно, найдите производную.
%\кзадача
\vfill
\ЛичныйКондуит{0.3mm}{6.5mm}
%\СделатьКондуит{4.2mm}{8mm}

\end{document}
