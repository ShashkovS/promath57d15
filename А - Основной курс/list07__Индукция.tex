% !TeX encoding = windows-1251
\documentclass[a4paper,12pt]{article}
\usepackage{newlistok}

\УвеличитьВысоту{1.5cm}
\УвеличитьШирину{14mm}
\renewcommand{\spacer}{\vspace{2mm plus 1mm minus .5mm}}


\Заголовок{Математическая индукция}
\НомерЛистка{7}
\ДатаЛистка{11.2012}

\begin{document}
\СоздатьЗаголовок

\опр
\emph{Принцип математической индукции}\т это аксиома, которая заключается в следующем. Пусть имеется последовательность утверждений $A_1,A_2,\ldots,A_n,\ldots$, про которую известно, что выполнены условия:
\begin{nums}{-1}
\item\emph{(база индукции)} утверждение~$A_1$ истинно;
\item\emph{(индуктивный переход)} из истинности утверждения~$A_k$ следует истинность утверждения~$A_{k+1}$.
\end{nums}
\vspace{-2mm}
Тогда все утверждения~$A_n$ истинны.
\копр

\замечание
Зачастую оказывается удобным считать базой индукции не $A_1$, а утверждение с каким-нибудь б\'ольшим номером $A_l$. В этом случае рассуждение будет выглядеть так:
\begin{nums}{-1}
\item утверждение~$A_l$ истинно;
\item из того, что $k \ge l$ и утверждение~$A_k$ истинно, следует истинность утверждения~$A_{k+1}$.
\end{nums}
\vspace{-2mm}
Тогда для любого $n \ge l$ утверждение~$A_n$ истинно.
\кзамечание

\задача
Докажите, что части, на которые $n$ прямых делят плоскость, можно раскрасить в два цвета, так чтобы соседние части (имеющие общий отрезок или луч) были окрашены в разные цвета.
\кзадача

\задача
В компании из $k$ человек $(k\geq4)$ у каждого появилась новость, известная лишь ему одному. За один телефонный разговор двое сообщают друг другу все известные им новости. Докажите, что за $2k-4$ разговора все они могут узнать все новости.
\кзадача

\задача
Известно, что $a_1=1$ и  $a_{n+1}=2a_n+1$ при $n\geq1$. Найдите $a_n$.
\кзадача

\задача
Докажите: модуль суммы любого числа слагаемых не больше суммы модулей слагаемых.
\кзадача

\задача
Верна ли теорема: \лк Если треугольник разбит отрезками на треугольники, то хотя бы один из треугольников разбиения не остроугольный\пк? Вот её доказательство (нет ли в нём ошибки?):\\
{\лк
1. Если треугольник разбит отрезком на два треугольника, то один из них не остроугольный (ясно).\\
2. Пусть имеется треугольник, как-то разбитый на $n$ треугольников. Проведём ещё один отрезок, разбив один из маленьких треугольников на два. Получим разбиение на $n+1$ треугольник, причём один из двух новых треугольников будет не остроугольный.\\
По индукции теорема доказана.\пк}
\кзадача

\задача
Докажите неравенство Бернулли: $(1+a)^n\geq1+na$ при $a\geq-1$.
\кзадача

\задача
Докажите, что
\пункт $2^n>n$;
\пункт $2^n>n^2$ при $n\geqslant4$;
\пункт $n!>2^n$ при $n>3$;
\кзадача

\задача
На какое максимальное число частей могут разбить плоскость $n$ прямых?
\кзадача

\сзадача
На какое максимальное число частей могут разбить пространство $n$ плоскостей?
\кзадача

\задача
Докажите, что $2^{5n-2}+5^{n-1}\cdot3^{n+1}$ делится на 17 при любом натуральном $n$.
\кзадача

\задача
Докажите для любого натурального $n$ неравенство: $\displaystyle{1+\frac1{2^2}+\frac1{3^2}+\dots+\frac1{n^2}\leq 2-\frac1{n}}$.
\кзадача

\задача
На кольцевой автотрассе стоят несколько машин. Известно, что общего количества бензина во всех стоящих автомобилях достаточно для того, чтобы заправленная этим бензином машина смогла объехать всю трассу и вернуться на прежнее место. Докажите, что хотя бы один из автомобилей, стоящих на дороге, может объехать все кольцо, забирая по пути бензин у остальных машин.
\кзадача

\vfill
\ЛичныйКондуит{0mm}{6mm}
\ОбнулитьКондуит
\newpage


\опр
\emph{Принцип наименьшего элемента} гласит: \лк всякое непустое подмножество множества натуральных чисел содержит наименьшее число\пк.
\копр

\опр
\emph{Обобщённый принцип математической индукции} заключается в~следующем. Пусть имеется последовательность утверждений $A_1,A_2,\ldots,A_n,\ldots$ Предположим, дополнительно известно, что выполнены условия:
\begin{nums}{-1}
\item утверждение~$A_1$ истинно;
\item из истинности утверждения~$A_k$ для всех~$k$ таких, что $k \le m$, следует истинность утверждения~$A_{m+1}$.
\end{nums}
\vspace{-2mm}
Тогда все утверждения~$A_n$ истинны.
\копр

\сзадача
Докажите, что принцип математической индукции и~принцип наименьшего элемента эквивалентны.
\кзадача

\сзадача
Докажите, что принцип математической индукции и~обобщённый принцип математической индукции эквивалентны друг другу.
\кзадача

\задача
Докажите, что уравнение $n^2=2m^2$ не имеет решений в натуральных числах.
\кзадача

\задача
Докажите, что любое натуральное число можно представить как сумму нескольких разных степеней двойки (возможно, включая и нулевую).
\кзадача

\задача
Число $x+\dfrac1x$\т целое. Докажите, что $x^n+\dfrac1{x^n}$\т тоже целое при любом натуральном $n$.
\кзадача

\задача
Докажите, что для любого натурального~$n>3$ число~$n!$ можно разложить на два множителя, отношение которых будет не меньше~$2/3$ и не больше~$3/2$.
\кзадача

\задача[Ханойские башни]
Есть детская пирамида с $n$ кольцами и два пустых стержня той~же~высоты.
Разрешается перекладывать верхнее кольцо с одного стержня на другой, но нельзя класть~большее кольцо на меньшее. Докажите, что
\вСтрочку
\пункт можно переложить все кольца на один из пустых стержней;
\пункт можно сделать это за $2^n-1$ перекладываний;
\пункт меньшим числом перекладываний не обойтись.
\кзадача

\задача
$k$ воров хотят поделить добычу. Каждый уверен, что он поделил бы добычу
на равные части, но остальные ему не верят.  Как действовать ворам, чтобы после раздела каждый был уверен, что у него не менее $\frac1k$ части добычи? Разберите случаи, когда:
\вСтрочку
\пункт
$k=2$;
\спункт
$k=3$;
\спункт
$k$\т любое.
\кзадача

\сзадача
При каких $n$ гири весом 1, 2, \dots, $n$ кг можно разложить на три равные по весу кучи?
\кзадача

\сзадача
При каких $n$ можно соединить каждые два из данных $n$ сел односторонним маршрутом так, чтобы из любого села в любое другое можно было доехать не более чем с одной пересадкой?
\кзадача

\сзадача
Двое играют в игру, исход которой не зависит от случая. Игроки ходят по очереди, причём по правилам игра продолжается не более $n$ ходов. Ничьих не бывает. Докажите, что у одного из игроков есть выигрышная стратегия.
\кзадача

\vfill
\ЛичныйКондуит{0mm}{6mm}
%\СделатьКондуит{6mm}{6mm}
%\GenXML

\end{document}




