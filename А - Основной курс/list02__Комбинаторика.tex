% !TeX encoding = windows-1251
\documentclass[a4paper,12pt]{article}
\usepackage[mag=990]{newlistok}

\usepackage{tikz}
\usetikzlibrary{calc}

\УвеличитьШирину{1.5cm}
\УвеличитьВысоту{2.5cm}
\renewcommand{\spacer}{\vfil}


\DeclareTextSymbolDefault{\textquotesingle}{TS1}

\begin{document}

%\УвеличитьШирину{5cm}
%\УвеличитьВысоту{8cm}
\Заголовок{Комбинаторика}
\НомерЛистка{2}
\ДатаЛистка{09.2012}



\СоздатьЗаголовок


\задача
В школьной столовой 5 кранов для умывания. Каждый может быть закрыт или
открыт. Сколькими способами может течь вода в столовой?
\кзадача

\задача Некое современное здание имеет форму куба, стоящего на
четырёх колоннах. Имеется 6 красок. Сколькими способами
можно покрасить грани здания этими красками в 6 цветов?
(Каждая грань красится целиком в один цвет,
разные грани красятся в разные цвета.)
\кзадача

\задача \пункт
В заборе 20 досок, каждую надо покрасить в синий, зелёный или жёлтый
цвет, причём соседние доски красятся в разные цвета.
Сколькими способами это можно сделать?
\\\пункт А если требуется ещё, чтобы хоть одна из досок
обязательно была синей?
\кзадача

\задача
\вСтрочку
\пункт
Сколько можно составить различных
(не обязательно осмысленных) слов из $k$ букв,
используя русский алфавит?
\\\пункт
А если потребовать, чтобы буквы в словах не повторялись?
\\\пункт
Сколькими способами можно переставить буквы в слове из $k$ различных букв?
\кзадача


\задача
\пункт
Сколько существует 10-значных чисел, не содержащих цифру 1?
\\\пункт
Сколько из них содержит  цифру 9 (хотя бы одну)?
\кзадача

\задача
\пункт
Десять девушек водят хоровод. Сколькими способами они могут встать
в круг?
\\\пункт
Сколько ожерелий можно составить из 10 различных бусин?
\\\пункт А если в ожерелье всего 3 белых и 7 синих бусин?
\кзадача

\задача
\пункт
Сколько строк можно составить из 0 и 1, чтобы в каждой строке было 10 цифр?\\
\пункт
На дереве растут 10 яблок. Сколькими способами можно сорвать
несколько из них?
\кзадача

\задача
Меню в школьном буфете постоянно и состоит из $n$ разных блюд.
Петя хочет~каж\-дый день выбирать себе завтрак по-новому
(за раз он может съесть от 0 до $n$ разных блюд).\\
\вСтрочку
\пункт  Сколько дней ему удастся это делать?
\пункт  Сколько блюд он съест за это время?\\
\пункт Вася решил последовать примеру Пети,
но съедать каждый день нечетное число блюд.
Сколько дней ему удастся это делать?
\пункт  Сколько блюд он съест за это время?
\кзадача


\задача В классе учатся 20 человек. Сколькими способами из них можно
выбрать двоих школьников: старосту и ответственного за проездные билеты?
А просто двоих школьников?
\кзадача

\задача
Сколько разных слов (не только осмысленных) можно получить,
переставляя буквы в словах
\вСтрочку
\пункт
{\tt РОК};
\пункт
{\tt КУРОК};
\пункт
{\tt КОЛОБОК};
\пункт
$\underbrace{\text{{\tt А}{\tt А}\dots{\tt А}}}_a
\underbrace{\text{{\tt Б}{\tt Б}\dots{\tt Б}}}_b$?
\спункт
$\underbrace{\text{\tt б}_1\dots\text{\tt б}_1}_{k_1}
\underbrace{\text{\tt б}_2\dots\text{\tt б}_2}_{k_2}
\,\dots\,\dots\,
\underbrace{\text{\tt б}_m\dots\text{\tt б}_m}_{k_m}$.
\кзадача

\задача
\пункт Сколькими способами можно выбрать трёх дежурных в классе
из 20 человек?\\
\пункт А сколькими способами можно выбрать старосту, его помощника
и трёх дежурных?
\кзадача



\задача
Сколькими способами можно расставить на шахматной доске
\\\пункт 8 различных ладей;
\пункт 8~неразличимых ладей так, чтобы они не били друг друга?
\кзадача


\задача Фабрика игрушек выпускает разноцветные кубики. У всякого кубика
каждая грань окрашена целиком одной из шести красок, имеющихся на фабрике,
причём разные грани одного кубика окрашены разными красками.
Сколько видов кубиков выпускает фабрика?
%(Два кубика считаются
%одинаково раскрашенными, если можно так расположить их в пространстве,
%чтобы одинаково расположенные грани имели одинаковый цвет.)
\кзадача

\сзадача Фабрика из предыдущей задачи начала выпуск параллелепипедов
$1\times1\times2$, склеивая по два из выпускаемых ею кубиков.
Сколько получится различных видов новой игрушки?
\кзадача

\сзадача Решите две предыдущие задачи, заменив куб на тетраэдр
(и 6 цветов на 4).
\кзадача

\задача
\пункт Какое наибольшее число неразличимых
слонов можно расставить на шахматной доске
так, чтобы они не били друг друга?
\\\пункт Докажите, что число способов такой расстановки ---
квадрат некоторого числа.
\\\спункт Найдите это число. (Сначала решите такую же задачу для досок
$2\times2$, $3\times3$, $4\times4$, \ldots)
\кзадача

\задача
Сколько существует строк из 20 цифр, в которых встречаются только
нули и единицы, причём никакие два нуля не стоят рядом?
\кзадача

\сзадача
В таблицу размера $k\times l$ записывают числа
$+1$ и $-1$ так, чтобы произведение чисел в каждой строке и в каждом
столбце равнялось $+1$. Сколькими способами это можно сделать?
\кзадача

%\GenXML



\ЛичныйКондуит{0mm}{6mm}

\vspace*{-4mm}

%\СделатьКондуит{5.7mm}{8.2mm}



\end{document}
