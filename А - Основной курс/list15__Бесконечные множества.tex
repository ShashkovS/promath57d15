% !TeX encoding = windows-1251
\documentclass[a4paper,12pt]{article}
\usepackage{newlistok}
\usepackage{framed}
% \usepackage{tikz}
% \usetikzlibrary{calc}

\ВключитьКолонитул

\УвеличитьВысоту{.5cm}
% \УвеличитьШирину{1.5cm}
%\renewcommand{\spacer}{\vfil}

\Заголовок{Бесконечные множества}
\НомерЛистка{15}
\ДатаЛистка{09.2013}


\begin{document}
\СоздатьЗаголовок


%\zz{AAAAAA}
\опр
Множество называется \выд конечным, если оно пусто или равномощно множеству $\{1,2,\ldots,n\}$ для некоторого натурального~$n$. Говорят, что множество \выд бесконечно, если оно не~является конечным.
\копр

\опр
Множества $X$ и $Y$ называются \выд равномощными, если существует взаимно однозначное отображение $f \from X \to Y$. Обозначение: $|X|=|Y|$.
\копр

\опр
Множество называется \выд конечным, если оно пусто или равномощно множеству $\{1,2,\ldots,n\}$ для некоторого натурального~$n$. Говорят, что множество \выд бесконечно, если оно не~является конечным.
\копр

\опр
Множество называется \выд счётным, если оно равномощно множеству натуральных чисел $\N$. Говорят, что множество \выд{не более чем счётно}, если оно конечно или счётно. Множество называется \выд несчётным, если оно бесконечно и~не является счётным.
\копр

\ввзадача
Докажите, что всякое подмножество счётного множества не более чем счётно.
\кзадача

\задача%
Докажите, что следующие множества счётны:
\невСтрочку
\пункт $\{x\in\N \mid x \text{ делится на } 9\}$;
\пункт $\Z$;
%\сНовойСтроки
\впункт конечное объединение счётных множеств;
\ввпункт счётное объединение счётных множеств.
\кзадача

\сзадача
Найдите алгебраическое выражение от двух переменных $x$ и $y$, задающее взаимно однозначное соответствие между множеством неотрицательных целых чисел и множеством точек плоскости, координаты которых\т неотрицательные целые числа.
\кзадача

\задача%
Докажите, что счётно
\невСтрочку
\пункт множество точек плоскости, координаты которых\т целые числа;
\пункт множество $\Q$;
\впункт декартово произведение счётных множеств;
\пункт множество предложений в русском языке;
\пункт множество алгебраических\footnote{Число $a$ алгебраично, если найдётся многочлен $P(x)$ с рациональными коэффициентами, такой что $P(a)=0$} чисел.
\пункт множество конечных подмножеств множества $\N$.
\кзадача

\задача
Счётно ли
\пункт множество точек плоскости, обе координаты которых рациональны;
\пункт множество всех треугольников на плоскости, координаты вершин которых рациональны;
\спункт множество всех многоугольников на плоскости, координаты вершин которых рациональны?
\кзадача



\задача
Счётно ли любое бесконечное множество непересекающихся
\невСтрочку
\пункт интервалов длины более $1$ на прямой;
\ввпункт интервалов на прямой;
\пункт кругов на плоскости;
\пункт восьмёрок на плоскости (восьмёрка\т это две касающиеся внешним образом окружности; восьмёрки могут быть разных размеров);
\спункт букв \лк Т\пк\ (любых размеров) на плоскости?
\кзадача

\ЛичныйКондуит{-0.6mm}{6mm}
\ОбнулитьКондуит

\newpage

\взадача
\невСтрочку
\пункт
Докажите, что в любом бесконечном множестве найдется счётное подмножество.
\пункт
Пусть $A$~не более чем счётно, а~$B$~бесконечно. Докажите, что $|A \cup B|=|B|$.
\кзадача




\задача
Равномощны ли следующие множества точек:
\невСтрочку
\пункт интервал и отрезок;
\пункт полуокружность (без~концов) и прямая;
\пункт интервал и прямая;
\пункт квадрат\footnote{Квадрат в этом листке\т это квадрат с внутренностью, например множество точек $(x,y)$, где $0\leq x, y\leq1$.} и круг;
\пункт квадрат и плоскость;
\пункт отрезок и счётное объединение множеств, равномощных отрезку?
\кзадача


% Несчетные множества

\задача
% \label{continuum}
Докажите, что множество $S$ бесконечных последовательностей из 0 и 1 и множество всех подмножеств множества $\N$ равномощны.
\кзадача


%\раздел{Дополнительные задачи}

\ввзадача
\пункт
Дана бесконечная вправо и вниз таблица из 0 и 1. Как по этой таблице составить бесконечную строку из 0 и 1, которая не совпадёт ни с одной из строк таблицы?\\
\help{надо, чтобы новая строка отличалась от каждой строки таблицы хотя бы в одном месте.}\\
\пункт
Докажите, что множество $S$ из задачи~9 несчётно.
\кзадача

\опр
Говорят, что множество \выд{имеет мощность континуум} (\emph{континуально}), если оно равномощно множеству $S$ из задачи~9.
\копр

\ввзадача [Теорема Кантора-Бернштейна]
Если множество $A$ равномощно подмножеству множества $B$ и множество $B$ равномощно подмножеству множества $A$, то $A$ и $B$ равномощны.
\кзадача

\задача
Докажите, что любой круг и любое круговое кольцо на плоскости равномощны.
\кзадача

% \сзадача
% \пункт
% Квадрат представлен в виде объединения двух множеств. Докажите, что одно из них равномощно отрезку.
% \пункт
% Та же задача, но вместо квадрата\т отрезок.
% \кзадача


\задача
Докажите, что следующие множества континуальны:
\невСтрочку
\пункт
множество взаимно однозначных отображений из $\N$ в $\N$;
\пункт
множество бесконечных последовательностей натуральных чисел.
\кзадача


\задача[Теорема Кантора]
Может ли множество быть равномощно множеству всех своих подмножеств?
\кзадача

\сзадача[Парадокс Деда Мороза]
Ровно за минуту до Нового Года Дед Мороз выдаёт Васе 10 конфет, после чего одну конфету у него забирает. За полминуты до НГ он ещё раз повторяет эту операцию. За четверть минуты\т ещё раз. И так далее до бесконечности. Сколько конфет будет у Васи в Новом Году?
\кзадача

\сзадача
Дано множество $M$ положительных чисел. Известно, что для любого его конечного подмножества $N \subset M$, сумма всех чисел из $N$ не превосходит 1. Докажите, что множество $M$ не более чем счётно.
\кзадача
\сзадача
Пусть множество $S$ имеет мощность континуум. Докажите, что $|S \times S| = S$.
\кзадача

\vfill
\ЛичныйКондуит{0mm}{6mm}


%\GenXMLW

\end{document}




