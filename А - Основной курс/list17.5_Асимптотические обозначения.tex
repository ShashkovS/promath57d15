% !TeX encoding = windows-1251
\documentclass[a4paper,12pt]{article}
\usepackage{newlistok}

\УвеличитьВысоту{2.0cm}
\УвеличитьШирину{1.0cm}

\ВключитьКолонитул
\Заголовок{Асимптотические обозначения}
\НомерЛистка{17$\dfrac12$}
\ДатаЛистка{\ммгг}

\begin{document}
\СоздатьЗаголовок

\опр
Пусть $(x_n)$ и $(y_n)$ --- две последовательности.
Говорят, что $x_n = O(y_n)$ (читается как \лк икс-эн есть о большое от игрек-эн\пк), 
если существуют константа $C$ и число $N$ такие, что $|x_n|\le C\cdot|y_n|$ при $n>N$.
Говорят, что $x_n = o(y_n)$ (читается как \лк икс-эн есть о малое от игрек-эн\пк),
если для любого числа $\ep>0$ найдётся такое число $N$, что $|x_n|\le \ep\cdot|y_n|$ при $n>N$.
\копр

\задача
Докажите, что 
\невСтрочку
\пункт
$x_n = O(1)$ тогда и только тогда, когда последовательность $(x_n)$ ограничена;
\пункт
$x_n = o(1)$ тогда и только тогда, когда последовательность $(x_n)$ бесконечно малая.
\кзадача

\задача
Какие из следующих утверждений верны:
\пункт
$\sin n = O(1)$;
\пункт
$\sin n = o(1)$;
\smallskip%
\\
\пункт
$n^2 = O(n^3)$;
\пункт
$n^2 = o(n^3)$;
\пункт
$n^2 = O(n)$;
\пункт
$n^2 = o(n)$;
\пункт
$\dfrac{1}{n^2} = O\hr{\dfrac{1}{n^3}}$;
\пункт
$\dfrac{1}{n^2} = o\hr{\dfrac{1}{n}}$.
\кзадача

\ввзадача[основные асимптотики]
Докажите, что выполнены следующие асимптотические равенства:
\пункт
$n^k = o(n^l)$ при $k < l$;
\пункт
$n^k = o(a^n)$ при $a > 1$;
\пункт
$a^n = o(n!)$;
\пункт
$n! = o(n^n)$;
\кзадача

\задача
Можно ли утверждать, что $x_n = o(z_n)$, если
\\
\пункт
$x_n = o(y_n)$ и $y_n = o(z_n)$;
\пункт
$x_n = O(y_n)$ и $y_n = O(z_n)$;
\\\пункт
$x_n = o(y_n)$ и $y_n = O(z_n)$;
\пункт
$x_n = O(y_n)$ и $y_n = o(z_n)$.
\кзадача

\задача
Известно, что $x_n=O(n^4)$ и $y_n = o(n^3)$.
Что можно сказать про $x_n + y_n$ и $x_n \cdot y_n$?
\кзадача

\задача
Что можно сказать про последовательность $(a_n\cdot b_n)$, если известно, что
\\
\пункт
$a_n = o(x_n)$ и $b_n = o(y_n)$;
\пункт
$a_n = O(x_n)$ и $b_n = O(y_n)$;
\\\пункт
$a_n = o(x_n)$ и $b_n = O(y_n)$;
\пункт
$a_n = O(x_n)$ и $b_n = o(y_n)$.
\кзадача

\опр
Используя асимптотические обозначения очень удобно выделять самую \лк весомую\пк часть последовательности.
Например, пишут $(n+1)^2 = n^2 + o(n^2)$ или $(n+1)^2 = n^2 + O(n)$, имея в виду, что заменив каждое асимптотическое выражение на подходящую последовательность, удовлетворяющую этой асимптотике, можно получить тождество.
В нашем примере в качестве такой последовательности выступает $(2n+1)$, ведь $2n+1 = o(n^2)$ и $2n+1 = O(n)$.
\копр

\задача
Докажите, что
\пункт
$(n+1)^3 = n^3 + o(n^3)$;
\пункт
$(n+1)^4 = n^4 + 4 n^3 + O(n^2)$;
\пункт
$o(1) + o(1) = o(1)$;
\\\пункт
$1 + 2 + \ldots + n = n^2/2 + O(n)$;
\пункт
$1^2 + 2^2 + \ldots + n^2 = n^3/6 + O(n^2)$;
\кзадача

\задача
\пункт
Предполагая, что формула $\sqrt{1+\dfrac1n} = 1 + \dfrac{1}{2n} + \dfrac{a}{n^2} + O\hr{\dfrac{1}{n^3}}$ верна для некоторой константы $a$, найдите значение $a$.
\спункт Доказать, что при этом $a$ формула действительно верна.
\кзадача

\сзадача
Укажите такое число $a$, что $\sqrt[3]{1+\dfrac1n} = 1 + \dfrac{a}{n} + O\hr{\dfrac{1}{n^2}}$.
\кзадача

\задача
\пункт
При анализе алгоритма выяснилось, что время его работы $T(n)$ на входе длины $n$ удовлетворяет соотношению $T(n) = T(\hs{n/2}) + T(\hs{n/3}) + O(n)$. Докажите, что $T(n) = O(n)$.
\\\спункт
Что можно сказать о $T(n)$, если $T(n) = 2T(\hs{n/2}) + O(n)$?
\кзадача

\сзадача
Докажите, что при неких $a$ и $b$ верна формула 
$1 + 1/2^2 + 1/3^2 + \ldots + 1/n^2 = a + b/n + O(1/n^2)$ и найдите $b$.
(Найти $a$ гораздо сложнее, $a=\pi^2/6$.)
\кзадача


\задача[асимптотика факториала]
\невСтрочку
\пункт 
Докажите, что для любого натурального числа~$n$ выполнены неравенства $\left(\dfrac n4\right)^n\leqslant n!\leqslant\left(\dfrac{n+1}2\right)^n$;
\сспункт[формула Стирлинга]
Докажите, что $n! = \sqrt{2\pi n}\,\hr{\dfrac{n}{e}}^n \cdot \bbbr{1 + O\hr{\dfrac{1}{n}}}$.
\кзадача



\ЛичныйКондуит{0mm}{6mm}
\end{document}




