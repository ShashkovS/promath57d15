% !TeX encoding = windows-1251
\documentclass[a4paper,12pt]{article}
\usepackage{newlistok}
% \usepackage{tikz}
% \usetikzlibrary{calc}
\usepackage[framemethod=tikz]{mdframed}


\global\addtolength{\vsize}{-115mm}%
\global\advance\vsize by 10cm
\global\advance\vsize by 10cm

\ВключитьКолонитул

% \УвеличитьВысоту{1.5cm}
% \УвеличитьШирину{1.5cm}
%\renewcommand{\spacer}{\vfil}

\Заголовок{Поле действительных чисел}
\НомерЛистка{17}
\ДатаЛистка{11.2013}

\begin{document}
\СоздатьЗаголовок

\опр
Пусть $F$\т упорядоченное поле. Для любого натурального числа~$n\in\N$ рассмотрим элемент $\ol{n} = \overbrace{1+1+\ldots+1}^{n\text{ раз}} \in F$.
Для любого целого числа~$m\in\Z$ рассмотрим элемент~$\ol{m}$, если~$m\in\N$, элемент~$0$, если~$m=0$, и элемент~$-(\ol{-m})$, если~$-m\in\N$ (то есть противоположный элемент для натурального $\ol{-m}$.
Для любого рационального числа $q = \frac{m}{n} \in \Q$ рассмотрим элемент~$\ol{q} = \dfrac{\ol{m}}{\ol{n}} \in F$.
Таким образом множества $\N$, $\Z$, $\Q$ естественным образом вложены~в~$F$. В дальнейшем, говоря о натуральных/целых/рациональных числах как об элементах поля~$F$, мы не будем писать черту над ними.
\копр

\аксиома[Аксиома Архимеда]
Для любого $a\in\F$ найдется такое натуральное~$n\in F$, что $n>a$.
\каксиома

\аксиома[Принцип вложенных отрезков]
Пусть дана последовательность вложенных отрезков $[a_1,b_1] \supset[a_2,b_2] \supset\ldots$. Тогда пересечение $\capnui [a_n, b_n]$ не пусто.
\каксиома

\опр
Упорядоченное поле~$F$, называется \emph{полным}, если в нём выполнена аксиома Архимеда и принцип вложенных отрезков.
\копр

\опр
Полное линейно упорядоченное поле называется \emph{полем действительных чисел}. Обозначение:~$\R$. Множество $\R \setminus \Q$ называется множеством \emph{иррациональных чисел}.
\копр

\задача
Докажите, что $\limn \dfrac1n = 0$.
\кзадача

\ввзадача
Докажите, что между любыми двумя различными числами из~$\R$ найдётся бесконечно много рациональных чисел.
\кзадача

\задача
Докажите, что пересечение последовательности вложенных отрезков $\br{[a_n, b_n]}$ состоит из одной точки тогда и только тогда, когда $\limn (b_n - a_n) = 0$.
\кзадача

\задача
Приведите пример упорядоченного поля, в котором не выполняется\\
\пункт
аксиома Архимеда;
\пункт
принцип вложенных отрезков.
\кзадача

\сзадача
Выполняется ли принцип вложенных отрезков в поле $\R(x)$ рациональных функций с вещественными коэффициентами?
\кзадача

\задача
\пункт
Докажите, что не существует такого~$q\in\Q$, что~$q^2=2$.
\\\пункт
Рассмотрим последовательность вложенных отрезков $\br{\hs{2/{b_n}, b_n}}$, где $b_1 = 2$ и $b_{n} = \dfrac12\left(b_{n-1}+\dfrac2{b_{n-1}}\right)$ при $n>1$. Докажите, что они в $\R$ имеют единственную общую точку. Какую?
\\\впункт
Докажите, что множество иррациональных чисел непусто.
\кзадача

\задача
Докажите, что между любыми двумя различными числами из~$\R$ найдётся бесконечно много иррациональных чисел.
\кзадача

\ввзадача[Аксиома о разделяющем числе]
Пусть~$A$ и~$B$\т два непустых подмножества поля~$\R$, такие что для всех $a\in A$ и~$b\in B$ справедливо неравенство $a\le b$. Докажите, что существует такое число $c\in\R$, что при всех $a\in A$ и~$b\in B$ выполнено $a\le c\le b$.
\кзадача

\ЛичныйКондуит{-0.6mm}{6mm}
\ОбнулитьКондуит
\newpage

\опр
Говорят, что подмножество~$M$ упорядоченного поля~$F$ \emph{ограничено сверху}, если существует такой элемент~$C$, что для всех $x\in M$ выполняется неравенство $x\le C$. Число~$C$ в~этом случае называется \emph{верхней гранью} множества~$M$.\\
Формально: $\exi C\in F \quad \fa x\in M:\quad x\le C$.\\
Аналогично определяется \emph{ограниченность снизу} и \emph{нижняя грань} множества.
\копр

\задача
Верно ли, что множество положительных чисел~$P$ ограничено сверху? А снизу?
\кзадача

\опр
Говорят, что подмножество~$M$ упорядоченного поля~$F$ \emph{ограничено}, если оно ограничено сверху и~снизу одновременно.
\копр

\опр
\emph{Модулем} (\emph{абсолютной величиной}) элемента~$a$ упорядоченного поля~$F$ называется элемент
$
|a|=
\left\{
\begin{array}{rcl}
a,& \;\mbox{если}& a\ge0\\
-a,& \;\mbox{если}& a<0.\\
\end{array}
\right.
$
\копр

\опр
\label{sup-1}
Элемент~$С$ упорядоченного поля~$F$ называется \emph{точной верхней гранью} множества~$M$, если выполняются следующие два условия:
\begin{items}{-3}
\item[1)] $\fa x\in M:\quad x\le C$;
\item[2)] $\fa C_1<C\quad \exi x\in M:\quad x>C_1$.
\end{items}
\vskip -2mm
Условие 2) иногда записывают в следующей форме:\quad
$2'$) $\fa \ep>0 \quad \exi x\in M:\quad x>C-\ep$.\\
Обозначение: $C=\sup M$ (читается: \emph{супр\'емум}).
\копр

\опр
\label{sup-2}
Элемент~$С$ упорядоченного поля~$F$ называется \emph{точной верхней гранью} множества~$M$, если $C$~есть наименьшая из всех верхних граней множества~$M$.
\копр

\задача
Докажите эквивалентность определений~\ref{sup-1} и~\ref{sup-2}.
\кзадача

\опр
Аналогично определяется \emph{точная нижняя грань} множества.\\
Обозначение: $\inf M$ (читается: \emph{инф\'имум}).
\копр

\задача
Может ли у~множества быть более одной точной верхней (нижней) грани?
\кзадача

\задача
Найдите точные нижнюю и~верхнюю грани множества~$M$, если:\\
\пункт
$M=\left\{\frac{1}{a} \mid a>2 \right\}$;
\пункт
$M=\{a+b \mid -5 < a \le 3,\, |b| < 1\}$;
\пункт
$M=\{ab \mid -5 < a \le 3,\, |b| < 1\}$.
\кзадача

\ввзадача[аксиома о точной верхней грани]
Докажите, что всякое непустое ограниченное сверху подмножество поля $\R$ имеет в~$\R$ точную верхнюю грань.
\кзадача

\взадача
Пусть множества $A,B\subset\R$ ограничены и~непусты. Докажите, что:\\
\пункт
$\sup\{a+b \mid a\in A,\, b\in B\}=\sup A +\sup B$;
\пункт
$\sup(A\cup B)=\max(\sup A,\,\sup B)$.
\кзадача

\ввзадача
Докажите, что в~упорядоченном поле~$F$ полнота эквивалентна
\невСтрочку
\пункт
аксиоме о~разделяющем числе;
\пункт
аксиоме о~точной верхней грани.
\кзадача

\сзадача
Докажите, что поле действительных чисел континуально.
\кзадача

\сзадача
Докажите, что поле действительных чисел не более, чем единственно.
\кзадача

\сзадача
\пункт
Докажите, что $\{a\in\R\mid a\ge0\} = \{b^2\mid b\in\R\}$.
\\\пункт
Докажите, что поле $\R$ нельзя упорядочить двумя разными способами.
\кзадача


%\vfill
\ЛичныйКондуит{0mm}{6mm}
\GenXMLW
\immediate\write18{WinEdt.exe}

\end{document}




