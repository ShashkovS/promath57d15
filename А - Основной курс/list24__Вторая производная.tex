% !TeX encoding = windows-1251
\documentclass[a4paper,12pt]{article}
\usepackage{newlistok}

\УвеличитьШирину{1.5cm}
\УвеличитьВысоту{2cm}
\def\dfrac{\displaystyle\frac}
\renewcommand{\spacer}{\vfil}
\sloppy


\begin{document}

\Заголовок{Вторая производная, выпуклость и асимптоты}
\НомерЛистка{24}
\ДатаЛистка{10.2014}

\СоздатьЗаголовок



\опр Функция $f:(a,b)\rightarrow\R$
называется \выд{выпуклой вниз} на $(a;b)$, если для каждого отрезка
$[x_1;x_2]\subseteq(a;b)$ выполнено условие:
график функции $f$ лежит не выше графика прямой $L$,
соединяющей точки $(x_1;f(x_1))$ и $(x_2;f(x_2))$,
то есть $f(x)\le L(x)$ при любом $x\in[x_1;x_2]$.
\копр



\задача Пусть функция $f$ определена и два раза дифференцируема на
интервале $(a,b)$.  Выясните, какие из следующих условий
эквивалентны тому, что $f$ выпукла вниз на $(a,b)$:
\сНовойСтроки
\пункт
$\al\cdot f(x)+(1-\al)\cdot f(y)\geqslant
f(\al\cdot x+(1-\al)\cdot y)$
для любых $x,y\in(a,b)$ и любого $\al\in[0,1]$;
\пункт
надграфик $f$ на $(a;b)$,
то есть $\{(x,y)\in\R^2\ |\ x\in(a,b),y\geqslant f(x)\}$
--- выпуклое множество;
\пункт
$f'$ монотонно неубывает на интервале $(a,b)$;
\пункт
$f''(x)\geqslant0$ для любого $x\in(a,b)$;
\пункт
любая касательная $l$ к графику $f$ расположена не выше его:
$f(x)\geqslant l(x)$ при всех $x\in(a,b)$;
\пункт[неравенство Йенсена]
для любых чисел~$x_1,\dots,x_n\in(a,b)$ и любых положительных чисел~$\al_1,\dots,\al_n$ выполнено неравенство:
\vspace*{-3mm}
$$
\dfrac{\al_1 f(x_1)+\dots+\al_n f(x_n)}{\al_1+\dots+\al_n}
\ge
f\hr{\dfrac{\al_1 x_1+\dots+\al_n x_n}{\al_1+\dots+\al_n}}
$$
\vspace*{-7mm}
\кзадача

\задача Дайте эквивалентные определения функции, выпуклой
вверх на~$(a,b)$ (можно устно).
\кзадача

\задача \label{funcs}
Найдите промежутки выпуклости вверх и выпуклости вниз
следующих функций:\\
\вСтрочку
\пункт $\sin x$;
\пункт $x^3$;
%\пункт $x^4$;
%\пункт $\ln x$;
\пункт $\sqrt{|x|}$;
%\пункт  $5x^4+7x^3$;
%\пункт $\sin x+\cos x$;
\пункт $(x(x-1))^{-1}$;
\пункт $x^2+\frac1x$.
\кзадача

\задача
Что больше: $\root 3 \of{60}$ или $2+\root 3 \of 7$?
\кзадача

\задача Докажите %следующие
неравенства:
\сНовойСтроки
\пункт
$\hr{\dfrac{x_1+\dots+x_n}{n}}^2\le\dfrac{x_1^2+\dots+x_n^2}{n}$ для любых чисел $x_1,\dots,x_n$;
\пункт [неравенство Коши-Буняковского]
$(x_1y_1+\dots+x_ny_n)^2\le(x_1^2+\dots+x_n^2)(y_1^2+\dots+y_n^2)$;
\спункт
$\sin \al \, \sin \be\, \sin \ga \le 3\sqrt{3}/8$, если $\al,\be,\ga$ --- углы
некоторого треугольника.
\кзадача

\опр Точка $x_0$ называется {\it точкой перегиба} функции $f$, если
существует $\varepsilon>0$ такое, что $f$ выпукла вниз
на $(x_0-\varepsilon,x_0)$ и выпукла вверх на $(x_0,x_0+\varepsilon)$
(или наоборот).
\копр

\задача Пусть функция $f$ дважды дифференцируема в некоторой окрестности
точки $x_0$.
\сНовойСтроки
\пункт
Пусть $x_0$ --- точка перегиба функции $f$.
Верно ли, что $f''(x_0)=0$? Верно ли обратное? %утверждение?
%\пункт
%Докажите, что если $x_0$ --- точка перегиба $f$, то
%$x_0$ --- точка локального экстремума~$f'$. %Верно ли обратное?
\пункт
Докажите, что $x_0$ --- точка перегиба $f$ если и только если
$f''$ меняет знак в точке $x_0$.
\кзадача

\задача
Нарисуйте графики функций из задачи \ref{funcs}
и найдите точки перегиба этих функций.
\кзадача

\задача
Пусть $f$ дважды дифференцируема в некоторой окрестности точки $x_0$,
причём $f'(x_0)=0$ и
\вСтрочку
\пункт
$f''(x_0)>0$;
\пункт
$f''(x_0)<0$.
Имеет ли $f$ в $x_0$ локальный экстремум, и если да, то какого типа?
\кзадача

%\задача
%Сколько перегибов у графика $y=(x+1)/(x^2+1)$?
%Лежат ли они на одной прямой?
%\кзадача

%\раздел{Асимптоты}

\опр Прямая $y=kx+b$ называется %(\выд{наклонной\/})
\выд{асимптотой\/}
%\footnote{асимптоты вида $y=b$ называются также {\it горизонтальными\/}}
графика функции $y=f(x)$ при $x\to+\infty$,
если %$f(x)$ определена при всех $x\gg0$
%существует $\lim\limits_{x\to+\infty}(f(x)-(kx+b))=0$.
$f(x)-(kx+b)\to0$ при $x\to+\infty$.
Прямая $x=x_0$ называется {\it вертикальной асимптотой\/}
графика функции $y=f(x)$ при $x$, стремящемся к $x_{0}$ слева,
%(соотв. справа),
если $\lim\limits_{x\rightarrow x_{0}-0}f(x)=\infty$.
%(соотв. $\lim\limits_{x\searrow x_{0}}f(x)=\infty$).
\копр

\задача
Дайте определение асимптот графика функции $y=f(x)$
при $x\to-\infty$ и при $x\rightarrow x_{0}+0$.
\кзадача

\задача
Верно ли, что график функции $y=f(x)$ имеет асимптоту
$y=kx+b$ при $x\to+\infty$ тогда и только тогда, когда
существуют пределы
$\lim\limits_{x\to+\infty}\displaystyle\frac{f(x)}{x}=k$
и $\lim\limits_{x\to+\infty}(f(x)-kx)=b$?
%Как найти $k$ и $b$?
\кзадача

%\задача
%Нарисуйте (с точным указанием всех асимптот) графики функций:\\
%\вСтрочку
%\пункт $x+\frac1x$;
%\пункт $x^2+\frac1x$;
%\пункт $\frac{x+3}{2-x}$;
%\пункт $\frac{x^{2}-4\,x+3}{x+1}$;
%\пункт $\frac{(x+1)(x-2)(x+3)}{x^{2}+1}$;
%\пункт $\sqrt{x\,(1+x)}$.
%\кзадача

\задача
%Постройте (с полным исследованием) %\footnote{т.~е. как можно более
%точным указанием области определения, множества значений,
%промежутков возрастания - убывания, локальных экстремумов,
%направлений выпуклости, перегибов и асимптот})
Найдите асимптоты следующих функций:\\
\вСтрочку
\пункт $x+\frac1x$;
\пункт $\dfrac{x+3}{2-x}$;
%\пункт $\frac{x^{2}-4\,x+3}{x+1}$;
%\пункт $\frac{(x+1)(x-2)(x+3)}{x^{2}+1}$;
\пункт $\sqrt{x\,(1+x)}$;
%\пункт $x\arctg x$;
%\пункт $\frac{x}{(x+1)^{2}}$;
%\пункт  $\root 3 \of{9-x^3}$;
%\пункт $\frac{x^3}{1-x^2}$;
%\пункт $\frac{1-2\,x+4\,{x}^{2}}{1-2\,x+2\,{x}^{2}}$;
%\пункт $\frac{\arcsin x}{\sqrt{1-x^2}}$;
%\пункт $\sqrt[3]{\frac{x^2}{x+1}}$;
%\спункт $\frac{\cos x}{\cos2x}$.
%\пункт $\frac{x^3+x^2-x+2}{x^2+x}$.
\кзадача



\ЛичныйКондуит{0mm}{6mm}

\end{document}

