% !TeX encoding = windows-1251
\documentclass[a4paper,12pt]{article}
\usepackage{newlistok}
% \usepackage{tikz}
% \usetikzlibrary{calc}
\usepackage[framemethod=tikz]{mdframed}


\global\addtolength{\vsize}{-115mm}%
\global\advance\vsize by 10cm
\global\advance\vsize by 10cm

\ВключитьКолонитул

% \УвеличитьВысоту{1.5cm}
% \УвеличитьШирину{1.5cm}
%\renewcommand{\spacer}{\vfil}

\Заголовок{Аксиомы поля}
\НомерЛистка{16}
\ДатаЛистка{10.2013}

\begin{document}
\СоздатьЗаголовок

\раздел{Поле}

\опр
Множество~$F$ называется \emph{полем}, если на нём заданы операции \emph{сложения} и~\emph{умножения} (отображения $ + \from F\times F \to F$ и~$\cdot \from F\times F \to F$ соответственно), удовлетворяющие следующим условиям (\emph{аксиомам поля}):
\begin{items}{-3}
\item[(A1)]
$\fa a,b\in F:\quad a+b=b+a$ (\emph{коммутативность сложения}).
\item[(A2)]
$\fa a,b,c\in F:\quad(a+b)+c=a+(b+c)$ (\emph{ассоциативность сложения}).
\item[(A3)]
В~$F$ существует такой элемент~$0$, что $\fa a\in F:\quad a+0=a$ (\emph{существование нуля}).
\item[(A4)]
$\fa a\in F \quad \exi b\in F:\quad a+b=0$
(\emph{существование противоположного элемента}).\\
Элемент $b$ называется \emph{противоположным} к $a$ и обозначается $-a$.
\item[(M1)]
$\fa a,b\in F:\quad a\cdot b=b\cdot a$ (\emph{коммутативность умножения}).
\item[(M2)]
$\fa a,b,c\in F:\quad (a\cdot b)\cdot c=a\cdot (b\cdot c)$ (\emph{ассоциативность умножения}).
\item[(M3)]
В~$F\setminus\{0\}$ существует такой элемент $1$, что $\fa a\in F:\quad a\cdot 1=a$ (\emph{существование единицы}).
\item[(M4)]
$\fa a\in F,\,a\ne0,\quad\exi b\in F:\quad a\cdot b=1$ (\emph{существование обратного элемента}).\\
Элемент $b$ называется \emph{обратным} к $a$ и обозначается $1/a$ или~$a^{-1}$).
\item[(AM)]
$\fa a,b,c\in F:\quad a\cdot(b+c)=a\cdot b + a\cdot c$\\ (\emph{дистрибутивность умножения относительно сложения}).
\end{items}
\vskip -3mm
\копр


\опр
Для любых элементов поля $a, b \in F$ уравнение $a+x=b$ имеет единственное решение $x = b + (-a)$, которое обозначается $b-a$ и называется \выд разностью элементов $b$ и $a$. Таким образом, в~поле определена операция \emph{вычитания}.
\копр

\опр
Для любых элементов поля $a, b \in F,\, a\ne 0$, уравнение $a \cdot x=b$ имеет единственное решение $x = b \cdot a^{-1}$, которое обозначается $\frac{b}{a}$ и называется \выд частным элементов $b$ и $a$. Таким образом, в~поле определена операция \emph{деления} на ненулевые элементы.
\копр

\задача
Пусть $F$\т поле, $a,b\in F$. Докажите, что\\
\пункт
$-(a+b)=(-a)+(-b)$;
\пункт
если $a,b\ne0$, то $(a\cdot b)^{-1}=a^{-1}\cdot b^{-1}$.
\кзадача

\задача
Пусть $F$\т поле. Докажите, что
\невСтрочку
\пункт
для любого $a$ из $F$ выполнено равенство $a\cdot 0=0$;
\пункт
если для элементов $a$ и~$b$ из $F$ справедливо равенство $a\cdot b=0$, то либо $a=0$, либо $b=0$.
\спункт
Останется ли верным утверждение пункта~б), если отказаться от аксиомы~M4?
\кзадача

\задача
Пусть $F$\т поле, $a\in F$. Докажите, что\\
\пункт
$a\cdot(-1)=-a$;
\пункт
$(-a)\cdot(-a)=a\cdot a$;
\пункт
если $a\ne0$, то $(-a)^{-1}=-a^{-1}$.
\кзадача

\задача
Пусть $F$\т поле, $a,b,c,d\in F$, причём $b,d\ne0$. Докажите, что\\
\smallskip
\пункт
$\dfrac a b\cdot\dfrac c d=\dfrac{a\cdot c}{b\cdot d}$;
\пункт
$\dfrac a b+\dfrac c d=\dfrac{a\cdot d+b\cdot c}{b\cdot d}$.
\кзадача

\задача
Существует ли поле, состоящее из\\
\пункт
одного элемента;
\пункт
двух элементов;
\пункт
трёх элементов;
\спункт
четырёх элементов?
\кзадача

\задача
Пусть $p$\т произвольное простое число. Постройте поле, состоящее из
\ввпункт
$p$~элементов;
\спункт
$p^2$~элементов.
\кзадача

\сзадача
Существует ли поле, состоящее из шести элементов?
\кзадача

\ЛичныйКондуит{-0.6mm}{6mm}
\ОбнулитьКондуит
\newpage

\раздел{Линейно упорядоченное поле}

\опр
Множество $E$ называется \emph{линейно упорядоченным}, если на нём задано отношение \лк меньше или равно\пк (то есть известно, для каких $a,b\in E$ выполнено неравенство $a\le b$), причём выполнены следующие \emph{аксиомы порядка}:
\begin{items}{-3}
\item[(O1)]
$\fa a\in E:\quad a\le a$ (\emph{рефлексивность}).
\item[(O2)]
Если $a,b\in E$, причём $a\le b$ и $b\le a,\,$ то $a=b$ (\emph{антисимметричность}).
\item[(O3)]
Если  $a,b,c\in E$, причём $a\le b$ и $b\le c,\,$ то $a\le c$ (\emph{транзитивность}).
\item[(O4)]
$\fa a,b\in E:\quad$ либо $a\le b$, либо $b\le a$
(\emph{линейная упорядоченность}).
\end{items}
\vskip -3mm

\noindent Вместо $a\le b$ пишут также $b\ge a$, а~записи $a<b$ и~$b>a$ означают, что $a\le b$ и~$a\ne b$.
\копр

\опр
Поле~$F$ называется \emph{упорядоченным полем}, если множество~$F$ линейно упорядочено, причём выполнены следующие аксиомы:
\begin{items}{-3}
\item[(AO)]
Если $a,b,c\in F$ и $a\le b$, то $a+c\le b+c$.
\item[(MO)]
Если $a,b,c\in F$, $0\le c$ и $a\le b$, то $a\cdot c\le b\cdot c$.
\end{items}
\vskip -3mm
\копр

\задача
Пусть $F$\т упорядоченное поле. Докажите, что
\невСтрочку
\пункт
если $a\le b$ и $c\le d$, то $a+c\le b+d$;
\пункт
если $0\le a\le b$ и $0\le c\le d$, то $a\cdot c\le b\cdot d$;
\пункт
если $a\le b$ и $c\le 0$, то $a\cdot c\ge b\cdot c$;
\пункт
если $0<a\le b$, то $\dfrac{1}{a} \ge \dfrac{1}{b}$;
\пункт
$1>0$.
\кзадача

\задача
Пусть $F$\т упорядоченное поле, $P$\т множество всех его \emph{положительных} элементов, то есть $P=\{a\in F\mid a>0\}$. Докажите, что тогда выполнены следующие свойства:
\begin{items}{-3}
\item[(P1)]
$\fa a\in F$: \quad либо $a\in P$, либо $a=0$, либо $-a\in P$.
\item[(P2)]
Если  $a,b\in P$, то $a+b\in P$ и $a\cdot b\in P$.
\end{items}
\кзадача

\ввзадача
Пусть $F$\т поле, $P\subset F$\т подмножество, удовлетворяющее условиям~(P1) и~(P2) из предыдущей задачи. Докажите, что поле~$F$ можно сделать упорядоченным таким образом, что $P$~будет множеством положительных элементов, причём отношение порядка $\le$ однозначно определяется множеством~$P$.
\кзадача

\задача
Докажите, что в~любом упорядоченном поле бесконечно много элементов.
\кзадача

\взадача
Пусть $F$\т упорядоченное поле. Рассмотрим множество рациональных функций от переменной~$x$
\[
F(x) = \left\{\dfrac{P(x)}{Q(x)} \mid P(x),\,Q(x) \text{\т многочлены с~коэффициентами в~поле}~F,\quad Q(x)\ne0\right\}
\]
с~естественными операциями сложения и~умножения.
\невСтрочку
\ввпункт
Проверьте, что $F(x)$\т это поле.
\пункт
Докажите, что поле $F(x)$ можно сделать упорядоченным.
\спункт
Докажите, что поле $F(x)$ можно сделать упорядоченным как минимум двумя различными способами.
\кзадача

%\vfill
\ЛичныйКондуит{0mm}{6mm}
%\GenXMLW

\end{document}




