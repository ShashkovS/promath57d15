% !TeX encoding = windows-1251
\documentclass[a4paper,12pt]{article}
\usepackage{newlistok}
% \usepackage{tikz}
% \usetikzlibrary{calc}
\usepackage[framemethod=tikz]{mdframed}


\global\addtolength{\vsize}{-115mm}%
\global\advance\vsize by 10cm
\global\advance\vsize by 10cm

\ВключитьКолонитул

% \УвеличитьВысоту{1.5cm}
% \УвеличитьШирину{0.5cm}
%\renewcommand{\spacer}{\vfil}

\Заголовок{Производная}
\НомерЛистка{20}
\ДатаЛистка{03.2014}

\begin{document}
\СоздатьЗаголовок

\опр
Пусть функция~$f$ определена в~некоторой окрестности точки~$x_0$. Функция~$f$ называется {\it дифференцируемой в~точке~$x_0$}, если существует предел
\vspace*{-3mm}$$\limx{x_0} \dfrac{f(x)-f(x_0)}{x-x_0}\vspace*{-6mm}$$
(называемый \emph{производной функции~$f$ в~точке}~$x_0$). Обозначения: $f'(x_0)$, $\dfrac{df}{dx}(x_0)$.
\копр

\задача
Докажите, что если функция~$f$ дифференцируема в~точке~$x_0$, то
\[
f'(x_0)=\lim\limits_{t\to 0}\dfrac{f(x_0+t)-f(x_0)}{t}.
\]
\кзадача

\опр
Функция~$f$ называется \emph{дифференцируемой на множестве}~$M$, если она дифференцируема в~каждой точке этого множества. В~этом случае функция $g\from M \to \R$, $g(x) = f'(x)$ называется \emph{производной функции~$f$ на множестве}~$M$. Обозначения: $f'$, $\dfrac{df}{dx}$.
\копр

\взадача
Найдите производные следующих функций:\medskip\\
\пункт
$c$\qquad
\пункт
$x$\qquad
\пункт
$x^2$\qquad
\пункт
$1/x$\qquad
\пункт
$\sqrt{x}$\qquad
\пункт
$\sin x$\qquad
\пункт
$\cos x$\qquad
\пункт
$e^x$\qquad
\пункт
$\ln x$.
\кзадача

\задача
Приведите пример функции, определённой на множестве $\R$, которая дифференцируема ровно в~одной точке.
\кзадача

\ввзадача
Пусть функции $f$ и~$g$ дифференцируемы на множестве~$M$. Докажите,
что:
\medskip\\
\пункт
$(cf)'=cf' \quad (c\in\R)$;
\пункт
$(f\pm g)'=f'\pm g'$;
\пункт[Правило Лейбница]
$(fg)' = f'g+fg'$;
\medskip\\
\пункт
если $g(x_0) \ne 0$, то $\left(\dfrac{f}{g}\right)'(x_0) = \dfrac{f'(x_0)g(x_0) - f(x_0)g'(x_0)}{g^2(x_0)}$.
\кзадача

\задача
Найдите производные следующих функций:\\
\пункт
$x^m \quad (m\in\Z)$;
\пункт
$|x-3|$;
\пункт
$x^2+\dfrac{1}{x^3}$;
\пункт
$\tg x$;
\пункт
$\ctg x$.
\кзадача

\ввзадача
Докажите, что $f'(x_0)=A$ тогда и~только тогда, когда для некоторой функции~$o(t)$ \выд приращение $f(x_0+t)-f(x_0)$ представимо в~виде $At+o(t)$, причём $\lim\limits_{t\to 0}o(t)/t = 0$.
\кзадача

\ввзадача[Производная сложной функции]
Пусть функция~$f$ дифференцируема в~точке~$x_0$, а~функция~$g$ дифференцируема в~точке $y_0 = f(x_0)$. Докажите, что композиция $h=g\circ f$ функций~$f$ и~$g$ дифференцируема в~точке~$x_0$, причём $h'(x_0) = g'(y_0)f'(x_0)$.
\кзадача


\ЛичныйКондуит{-0.3mm}{6mm}
\ОбнулитьКондуит
\newpage


\задача
Найдите производные следующих функций:\medskip\\
\пункт
$2x^3+5x^2-3x+4$;
\пункт
$\dfrac{x^2+8x-15}{x+2}$;
\пункт
$(2x^2-5x+3)^{100}$;
\пункт
$\sqrt{1-x^2}$;
\пункт
$\dfrac{(3x+1)^2}{(2x-5)^3}$; \medskip\\
\пункт
$\sin(\cos e^x)$;
\спункт
$f(x)=\bcase{
x^2 \sin \frac{1}{x}, & \text{ если } x\ne 0\\
0, & \text{ если } x=0\\
}$
\спункт
$f(x)=\bcase{
x \sin \frac{1}{x}, & \text{ если } x\ne 0\\
0, & \text{ если } x=0\\
}$
\кзадача

\ввзадача
Пусть функция~$f$ определена и строго монотонна в некоторой окрестности точки~$x_0$, причём существует $f'(x_0) \ne 0$. Докажите, что обратная к~$f$ функция~$g$ дифференцируема в~точке~$y_0=f(x_0)$ и~$g'(y_0) = \dfrac{1}{f'(x_0)}$.
\кзадача

\сзадача
Найдите и исправьте ошибку в предыдущей задаче.
\кзадача

\задача
Найдите производные следующих функций:\medskip \\
\пункт
$x^a \quad (a\in\R)$;
\пункт
$a^x \quad (a>0)$;
\пункт
$\log_{a}{x} \quad (a>0,\,a\ne 1)$; \medskip\\
\пункт
$\arcsin x$;
\пункт
$\arccos x$;
\пункт
$\arctg x$;
\пункт
$\arcctg x$.
\кзадача


\опр
\label{tangent}
Пусть функция~$f$ определена в~окрестности точки~$x_0$. Прямая~$L$
называется \emph{касательной к~графику функции в~точке~$x_0$}, если
$\limx{x_0} \dfrac{\rho(x)}{x-x_0}=0$, где $\rho(x)$\т расстояние от точки $(x,f(x))$ до прямой~$L$.
\копр

\задача
Пусть функция~$f$ дифференцируема в~точке~$x_0$. Докажите, что касательная к~графику функции~$f$ в~точке~$x_0$ существует и~её уравнение имеет вид: $y(x) = f'(x_0)(x-x_0)+f(x_0)$.
\кзадача

\задача
Докажите, что в~случае окружности определение~\ref{tangent} эквивалентно определению касательной, известному из геометрии.
\кзадача

\взадача
Пусть функция~$f$ дифференцируема в~точке~$x_0$ и~$f'(x_0)>0$. Докажите, что существует такая окрестность точки~$x_0$, что для всех~$x$ из левой полуокрестности $f(x) < f(x_0)$, а~для всех~$x$ из правой полуокрестности $f(x) > f(x_0)$.
\кзадача

\ввзадача[Теорема Ферма]
Пусть функция~$f$ определена на интервале~$(a,b)$ и~в~точке~$x_0$ принимает наибольшее или наименьшее значение на $(a,b)$. Докажите, что если производная $f'(x_0)$ существует, то она равна нулю. Верно ли обратное?
\кзадача

% \задача
% Докажите, что при всех $x\in\R$ выполняется неравенство:\\
% \пункт
% $x^4+x^3 \ge -\frac{3^3}{4^4}$;
% \пункт
% $x^6-6x+5 \ge 0$.
% \кзадача
%
% \задача
% Найдите наибольшее и~наименьшее значение функции~$f$ на отрезке~$[0;2]$, если \\ $f(x) = x^4-4x^3+10x^2-12x+5$.
% \кзадача

% \взадача[Теорема Ролля]
% Пусть функция~$f$ непрерывна на отрезке~$[a,b]$, дифференцируема на интервале~$(a,b)$ и~$f(a)=f(b)$.
% Докажите, что найдётся такое $c\in(a,b)$, что $f'(c)=0$.
% \кзадача
%
% \задача
% ({\it Теорема Лагранжа}) Пусть функция~$f$ непрерывна на
% отрезке~$[a,b]$ и~дифференцируема на интервале~$(a,b)$. Докажите,
% что в~этом случае существует такое $c\in(a,b)$, что
% $f'(c)=\dfrac{f(b)-f(a)}{b-a}$. Каков геометрический смысл этого
% утверждения?
% \кзадача

\задача
\пункт
Дайте определения левой и~правой производных функции~$f$ в~точке~$x_0$.\\
Обозначение: $f'_-(x_0)$ и~$f'_+(x_0)$\\
\пункт
При каких~$a$ и~$b$ функция $f(x)=
\bcase{
x^2, & \text{ если } x \le x_0\\
ax+b, & \text{ если } x > x_0\\
}$ дифференцируема на~$\R$?
\кзадача

% \сзадача
% Пусть непрерывная функция $f:[0,1]\to\R$ дифференцируема на
% интервале~$(0,1)$, причём $f(0)=f(1)=0$ и~$\sup\limits_{x\in[0,1]}
% f(x)=1$. Докажите, что найдётся такая точка $x_0\in(0,1)$, что
% $|f'(x_0)|>2$.
% \кзадача

% \сзадача[Теорема Дарбу]
% Пусть функция~$f$ дифференцируема на интервале $(a,b)$ и~существуют правая производная~$f'_+(a)$ и~левая производная~$f'_-(b)$. Докажите, что функция~$f'$ принимает на отрезке $[a,b]$ все промежуточные значения между $f'_+(a)$ и~$f'_-(b)$.
% \кзадача
%
% \сзадача
% Существует ли непрерывная на $\R$ функция, которая нигде не
% дифференцируема?
% \кзадача

\vfill
\ЛичныйКондуит{0mm}{6mm}
% \GenXMLW

\end{document}



