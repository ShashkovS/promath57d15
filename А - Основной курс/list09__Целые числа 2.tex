% !TeX encoding = windows-1251
\documentclass[a4paper,12pt]{article}
\usepackage{newlistok}

%\УвеличитьВысоту{1.5cm}
%\УвеличитьШирину{3mm}
\renewcommand{\spacer}{\medskip}
\newcommand{\ndv}{\smash{\mskip3mu\not\lower1truept\hbox{\vdots}\mskip3mu}}

\Заголовок{Целые числа\т 2. Алгоритм Евклида.}
\НомерЛистка{9}
\ДатаЛистка{01.2013}

\begin{document}
\СоздатьЗаголовок

\опр
Если число~$d$ делит числа~$a$ и~$b$, то $d$~называется \выд{общим делителем} чисел $a$ и~$b$. Наибольший среди общих делителей чисел $a$ и~$b$ называется \выд{наибольшим общим делителем} $a$ и~$b$ (обозначение: $(a,b)$). В~том случае, когда $(a,b)=1$, говорят, что числа $a$ и~$b$ \выд{взаимно простые}.
\копр

\задача
Для каких целых $a$ и~$b$ число $(a,b)$ существует?
\кзадача

\задача
Докажите, что
\пункт
для каждого целого~$x$ справедливо $(a,b)=(a,\,b+ax)$;
\пункт
если $a$~кратно~$b$, то $(a,b)=|b|$;
\пункт
если~$r$\т остаток от деления~$a$ на~$b$, то $(a,b)=(b,r)$.
\кзадача

\задача
Найдите
\пункт
$(n,1)$;
\пункт
$(n,\,n+1)$;
\пункт
$(2n+3,\,7n+6)$;
\пункт
$(n^2,\,n+1)$.
\кзадача

\задача
Пусть $a$ и~$b$\т два фиксированных целых числа. Обозначим через $I$ множество всех чисел, представимых в~виде $ax+by$ ($x$ и~$y$\т целые числа). Пусть $d$\т
наименьшее положительное число в~$I$. Докажите, что
\невСтрочку
\пункт
каждое число из $I$ делится на любой общий делитель чисел $a$ и~$b$ (а~значит, и~на $(a,b)$);
\пункт
каждое число из $I$ делится на~$d$;
\пункт
$d=(a,b)$.
\кзадача

\задача
Пусть $a$ и~$b$\т два фиксированных целых числа. Обозначим через $d$ наименьшее натуральное число, делящееся на любой общий делитель $a$ и $b$. Докажите, что $d=(a,b)$.
\кзадача


\задача[Алгоритм Евклида]
Пусть $a$ и $b$\т два фиксированных натуральных числа. Будем последовательно заменять большее из этих чисел на их разность. Докажите, что:\\
\пункт
в некоторый момент мы получим пару $(d, 0)$, $d\ne 0$;
\пункт
$(a,b)=d$;\\
\пункт
все промежуточные числа представимы в виде $ax + by$ для некоторых целых $x$ и $y$;\\
\пункт
найдутся целые числа $x$ и $y$, что $ax + by = (a,b)$;
\кзадача

\задача
Найдите
\пункт
$(7\,777\,777,\,7\,777)$;
\пункт
$(3289,969)$;
\пункт
$(7581,1767)$;
\пункт
$(10946,17711)$;\\
\спункт
$(2^m-1,\,2^n-1)$;
\спункт
$(2^{2^m}+1,\,2^{2^n}+1)$.
\кзадача

\задача
Как для данных чисел~$a$ и~$b$ при помощи алгоритма Евклида найти такие целые числа~$x$ и~$y$, что $ax+by=(a,b)$?
\кзадача

\задача
Найдите целые числа~$x$ и~$y$ такие, что $ax+by=(a,b)$, в~следующих случаях:\\ \пункт
$a=525,\,b=231$;
\пункт
$a=645,\,b=381$.
\кзадача

\задача
\пункт
Докажите, что для любого натурального~$k$ выполнено $(ka,kb)=k\cdot(a,b)$.\\
\пункт
Докажите, что если $m$\т общий натуральный делитель чисел~$a$ и~$b$, то $(a/m,b/m)=(a,b)/m$.
\кзадача

\задача
Докажите, что числа~$a$ и~$b$ взаимно просты тогда и~только тогда, когда существуют такие целые~$x$ и~$y$, что $ax+by=1$.
\кзадача

\задача
Числа $a$, $b$ и $c$ целые, $(a,b) = 1$. Докажите, что \\
\пункт
если $ac \dv b$,  то $c \dv b$;
\пункт
если $c \dv a$ и $c \dv b$,  то $c \dv ab$.
\кзадача

\сзадача
Даны $m$~целых чисел. За один ход разрешается прибавить по единице к~любым~$n$ из них. При каких $m$~и~$n$ всегда можно за несколько таких ходов сделать все числа одинаковыми?
\кзадача

\vfill
\ЛичныйКондуит{0mm}{6mm}
\ОбнулитьКондуит
\newpage

\опр
Уравнение, которое требуется решить в~целых числах, называется \выд диофантовым. \выд{Линейным диофантовым уравнением} называется уравнение вида $ax+by=c$. Для данного линейного уравнения уравнение $ax+by=0$ называется \выд однородным.
\копр

\задача
\невСтрочку
\пункт
Пусть $(x_{1}, y_{1})$ и~$(x_{2}, y_{2})$\т решения однородного уравнения. Докажите, что пары $(x_{1}+x_{2},y_{1}+y_{2})$ и~$(x_{1}-x_{2}, y_{1}-y_{2})$ также являются решениями однородного уравнения.
\пункт
Пусть $(x_{1}, y_{1})$ и~$(x_{1}, y_{2})$\т два решения линейного диофантового уравнения. Докажите, что пара $(x_{1}-x_{2}, y_{1}-y_{2})$ является решением соответствующего однородного уравнения.
\пункт
Докажите, что линейное уравнение $ax+by=c$ имеет решение в~целых числах тогда и~только тогда, когда $c$~делится на $(a,b)$.
\пункт
Придумайте способ, как находить хотя бы одно решение уравнения $ax+by=c$.
\пункт
Пусть $(x_{0}, y_{0})$\т одно из решений линейного диофантового уравнения $ax+by=c$. Докажите, что в~таком случае множество всех решений $\{(x,y)\}$ описывается следующими формулами:
\[
x=x_{0}+\dfrac{bt}{(a,b)},\quad y=y_{0}-\dfrac{at}{(a,b)},\qquad t\in\mathbb Z.
\]
\кзадача

\задача
Решите следующие диофантовы уравнения:
\пункт
$21x+9y=7$;
\пункт
$17x+23y=36$;
\пункт
$31x-133y=2$;
\пункт
$7581x-1767y=171$;
\пункт
$nx+(2n-1)y=3$.
\кзадача

\задача
При каких~$a$ и~$b$ можно заплатить в~кассу один рубль, имея на руках неограниченное количество $a$-рублёвых купюр, если в~кассе есть неограниченное количество $b$-рублёвых купюр?
\кзадача

\задача
По окружности длины $a$~см катится колесо, длина обода которого равна $b$~см ($a$~и~$b$ натуральные, $(a,b)=d$). В~колесо вбит гвоздь, он оставляет отметки на окружности. Сколько отметок оставит гвоздь?
\кзадача

\сзадача
Слонопотам ходит по бесконечной клетчатой доске на $m$~клеток в~одном направлении и~на $n$ в~направлении, перпендикулярном первому (конь является слонопотамом с параметрами $m=2$, $n=1$). При каких $m$ и~$n$ он сможет попасть
\пункт
в соседнюю по диагонали клетку;
\пункт
в~соседнюю справа клетку?
\кзадача

\сзадача
Решите следующие диофантовы уравнения:
\пункт
$x+y=xy$;
\пункт
$x+y+z=xyz$;
\пункт
$x^{2}-3xy+2y^{2}=3$;
\пункт
$15x^{2}-9y^{2}= 9$;
\пункт
$1!+2!+\ldots+x!=y^{2}$;
\пункт
$1!+2!+\ldots+x!=y^{z}$.
\кзадача

\сзадача
Пусть $a$ и $b$\т взаимно простые натуральные числа. Рассмотрим множество всех чисел вида $ax+by$, где целые числа~$x$ и~$y$ неотрицательны. Докажите, что это множество содержит ровно одно число из каждой пары $(z,\, ab-a-b-z)$ где $z\in\Z$.
\кзадача

\vfill
\ЛичныйКондуит{0mm}{6mm}
%\GenXML

\end{document}




