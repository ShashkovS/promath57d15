% !TeX encoding = windows-1251
\documentclass[a4paper,12pt]{article}

\usepackage{newlistok}

\УвеличитьШирину{1cm}
\УвеличитьВысоту{2cm}
\renewcommand{\spacer}{\vfil}


\begin{document}



\Заголовок{Прогрессии}
\НомерЛистка{3}
\ДатаЛистка{09.2012}




\СоздатьЗаголовок

\vspace*{.5truemm}

%Последовательность --- это набор занумерованных чисел
%\опр Арифметической прогрессией называется после
%\копр

\опр {\it Арифметическая прогрессия\/}  --- это (конечная или бесконечная)
последовательность чисел $\ldots\,,\,a_1,\,a_2,\,a_3,\,\ldots\,$,
в которой разность $d=a_k-a_{k-1}$ между соседними числами
одинакова для всех $k$; она называется {\it разностью\/} или {\it
приращением\/} прогрессии.
\копр

\задача
Выразите $n$-ый член арифметической прогрессии через
первый член и разность.
\кзадача

\задача
Найдите 50-ое натуральное число, большее 90, c остатком 3 от деления на~4.
\кзадача


\задача Каждый член некоторой последовательности (кроме крайних, если
такие есть) равен среднему арифметическому двух соседних членов:
$a_k=(a_{k-1}+a_{k+1})/2$. Верно ли, что эта последовательность~--- арифметическая прогрессия?
Верно ли обратное утверждение?
\кзадача

%\задача Найдите 100-ый член последовательности $a_n$, заданной условиями
%$a_1=1$, $a_2=4$, $a_{n}=2a_{n-1}-a_{n-2}$ при $n\geq3$.
%\кзадача


%\задача
%Будет ли арифметической прогрессией последовательность с $k$-тым
%членом, равным:
%\вСтрочку
%\пункт
%$\underbrace{1\,1\,\ldots\,1}_{k}$\,,
%\пункт
%$k$-тому натуральному числу, оканчивающемуся на 13\,?
%\кзадача


%\задача
%Какие из перечисленных ниже свойств набора чисел $\OFAM a,n$
%необходимы, а какие --- достаточны для того, чтобы этот набор был
%арифметической прогрессией:
%\сНовойСтроки
%\пункт
%каждый элемент (кроме крайних) равен среднему арифметическому
%двух соседних:
%$a_k=(a_{k-1}+a_{k+1})/2$;
%\пункт
%$2\,a_i=a_{i-2}+a_{i+2}$ при всех $2\leq i\leq n-2$;\hfill
%\пункт
%сумма $a_i+a_{n-i}$ одна и та же для всех $0\leq i\leq n$?
%\кзадача

\задача
Выразите сумму всех членов конечной арифметической
прогрессии $a_1,\ \!a_2,\ \!\ldots,\ \!a_n$
\сНовойСтроки
\пункт через два крайних члена и количество слагаемых;
\пункт через начальный член, количество слагаемых и приращение.
\кзадача

\задача
Найдите сумму всех тр\"ехзначных чисел, оканчивающихся на 7.
\кзадача

\задача
\пункт
Дан квадратный тр\"ехчлен $f(x)=ax^2+bx+c$.
При каких условиях на $a$, $b$ и $c$ найд\"ется такая арифметическая прогрессия
$(a_n)$, что $a_1+\ldots+a_n=f(n)$ при всех натуральных~$n$?
\пункт Найдите арифметическую прогрессию,
сумма первых $n$ членов которой равна $2n^2-3n$.
\кзадача

%\задача Найдите все арифметические прогрессии, у которых каждый член,
%начиная с третьего, равен сумме двух предыдущих.
%\кзадача

\сзадача
Можно ли покрыть натуральный ряд $k$ арифметическими
прогрессиями с различными целыми разностями, не равными 1,
если
\вСтрочку
\пункт $k=2$;
\пункт $k=3$;
\пункт $k=4$;
\пункт $k=5$.
\кзадача

\опр {\it Геометрическая прогрессия\/}  --- это (конечная или бесконечная)
последовательность ненулевых чисел $\ldots\,,\,a_1,\,a_2,\,a_3,\,\ldots\,$,
в которой отношение $q=a_k/a_{k-1}$ соседних чисел
одинаково для всех $k$; оно называется {\it знаменателем\/} прогрессии.
\копр


\задача
Будет ли геометрической прогрессией последовательность, $k$-ый
член~\hbox{которой}~ра\-вен
\сНовойСтроки%
\пункт
$0,\underbrace{0\ldots0}_{k}3$;
\вСтрочку
%\пункт
%  $3^{-k}$;
\пункт
 $\underbrace{1\ldots1}_{k}$;
\пункт
 $g_k\cdot h_k$, где $(g_k)$, $(h_k)$ --- геометрические~\hbox{прогрессии?}
\кзадача

\задача
Выразите $n$-ый член геометрической прогрессии через
первый член~и~\hbox{знаменатель.}
\кзадача


%\задача Найдите сумму %:\вСтрочку
%$1+x+x^2+\ldots+x^n$.
%\кзадача

\задача Выразите сумму
всех элементов конечной геометрической прогрессии через начальный член,
количество слагаемых и знаменатель.
\кзадача

%\задача Можно ли покрыть натуральный ряд конечным
%числом геометрических прогрессий?
%\кзадача

\задача Торговец прин\"ес на рынок мешок орехов.
Первый покупатель купил 1 орех, второй --- 2 ореха,
третий --- 4, и так далее: каждый следующий покупатель
покупал вдвое больше орехов, чем предыдущий.
Орехи, купленные последним, весили 50 кг, после чего у продавца
остался один орех. Сколько килограммов орехов было у продавца вначале?
(Все орехи одинаковые.)
\кзадача

\задача Найдите все геометрические прогрессии, у которых каждый член,
начиная с третьего, равен сумме двух предыдущих.
\кзадача

\опр {\it Числами Фибоначчи} называют элементы последовательности
$F_0,\,F_1,\,F_2,\,\ldots,$
в которой первые два члена $F_0$, $F_1$ равны 1, а
%остальные вычисляются по формуле $u_n=u_{n-1}+u_{n-2}$ (при $n\geq3$).
каждый следующий член равен сумме двух предыдущих:
%:$u_1=u_2=1$,
$F_{n}=F_{n-1}+F_{n-2}$ при всех натуральных $n\geq2$.
\копр

\задача
\пункт Вычислите первые 10 чисел Фибоначчи.
\пункт Представьте последовательность Фибоначчи в виде
суммы двух геометричес\-ких прогрессий,
т.~е.~найдите такие прогрессии $(g_n)$ и $(h_n)$, что $F_n=g_n+h_n$ при
всех~\hbox{$n\in\N$.}  %Найдите формулу для чисел Фибоначчи.
\кзадача

%\раздел{* * *}


%\GenXMLW

\ЛичныйКондуит{0mm}{6mm}

\vspace*{-4mm}

%\СделатьКондуит{5.7mm}{8.2mm}



%%\СделатьКондуит{6mm}{6.7mm}

\end{document}

