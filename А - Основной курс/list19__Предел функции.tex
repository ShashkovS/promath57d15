% !TeX encoding = windows-1251
\documentclass[a4paper,12pt]{article}
\usepackage[mag=1010]{newlistok}
% \usepackage{tikz}
% \usetikzlibrary{calc}
\usepackage[framemethod=tikz]{mdframed}


\global\addtolength{\vsize}{-115mm}%
\global\advance\vsize by 10cm
\global\advance\vsize by 10cm

\ВключитьКолонитул
% \УвеличитьВысоту{1.5cm}
% \УвеличитьШирину{1.5cm}
%\renewcommand{\spacer}{\vfil}

\newcommand{\sm}{\setminus}

\Заголовок{Предел функции}
\НомерЛистка{19}
\ДатаЛистка{02.2014}

\begin{document}
\СоздатьЗаголовок
\опр
Пусть $\ep > 0$, $a \in \R$. Множество $\dot U_\ep(a) = U_\ep(a) \sm \{a\} = \{x \in \R \mid 0<|x-a|<\ep\}$ называется \emph{проколотой $\ep$-окрестностью точки}~$a$. Множества $\dot U_\ep^+(a) = \{x\in \R \mid a<x<a+\ep\}$ и~$\dot U_\ep^-(a)=\{x\in \R \mid a-\ep<x<a\}$ называются \emph{правой} и~\emph{левой проколотыми полуокрестностями точки}~$a$ соответственно.
\копр

% \опр
% Точка~$a$ называется {\it предельной точкой\/} множества~$M$, если любая проколотая
% окрестность точки~$a$ содержит хотя бы одну точку множества~$M$. Точка~$a$ из множества~$M$, не являющаяся предельной точкой множества~$M$, называется {\it изолированной} точкой этого множества.
% \копр
%
% \задача
% Докажите, что точка~$a$ является предельной точкой множества~$M$ в~том и~только в~том случае, когда
% \невСтрочку
% \пункт
% существует последовательность из элементов множества $M\sm\{a\}$,
% сходящаяся к~точке~$a$;
% \пункт
% в~любой окрестности точки~$a$ лежит бесконечно много элементов множества~$M$.
% \кзадача

\опр
\label{Heine}
(\emph{Предел функции в~смысле Гейне})\; Пусть функция~$f$ определена на множестве~$M$
% и~точка~$a$ является предельной точкой этого множества.
и~некоторая проколотая окрестность точки~$a$ вложена в~$M$.
Число~$b$ называется \emph{пределом функции $f$ в~точке~$a$}, если для любой последовательности~$(x_n)$ элементов множества $M\sm\{a\}$, сходящейся к~$a$, последовательность $(f(x_n))$ сходится к~$b$.\\
Обозначение: $\limx{a}f(x) = b$ или $f(x)\to b$ при $x\to a$.
\копр

\опр
\label{Cauchy}
(\emph{Предел функции в~смысле Коши})\; Пусть функция~$f$ определена на множестве~$M$
% и~точка~$a$ является предельной точкой этого множества.
и~некоторая проколотая окрестность точки~$a$ вложена в~$M$.
Число~$b$ называется \emph{пределом функции~$f$ в~точке~$a$}, если для каждого $\ep>0$ существует такое число $\de>0$, что для всех~$x$ из множества $\dot U_\de(a)\cap M$ выполняется условие $f(x)\in U_\ep(b)$.\\
Формально:  $\fa \ep>0 \,\exi \de>0 \,\fa x\in\dot U_\de(a)\cap M:\,f(x)\in U_\ep(b)$.
\копр

\взадача
Докажите эквивалентность определений~\ref{Heine} и~\ref{Cauchy}.
\кзадача

\задача
Может ли функция иметь более одного предела в~данной точке?
\кзадача

\задача
Пусть $a,b\in\R$, $k\in\N$. Найдите следующие пределы (если они существуют):\\
\пункт
$\limx{a}b$;
\пункт
$\limx{a}x$;
\пункт
$\limx{a}\{x\}$;
\пункт
$\limx{a}[x]$;
\пункт
$\limx{a}x^k$;
\пункт
$\limx{a}\sqrt[k]x$;
\пункт
$\limx{0}x\sin\frac1x$.
\кзадача

% \опр
% Функция~$f$ называется {\it бесконечно малой в~точке~$a$}, если
% $\limx{a}f(x)=0$.
% \копр
%
% \задача
% Докажите, что $\limx{a} f(x)=b$ тогда и~только тогда, когда функция
% $f$~представима в~виде $f(x)=g(x)+b$, где функция~$g$~---~бесконечно малая
% в~точке~$a$.
% \кзадача

\ввзадача[Арифметика пределов]
Пусть области определения функций~$f$ и~$g$ совпадают, \\
$\limx{x_0} f(x)=a$, $\limx{x_0} g(x)=b$. Докажите, что в~этом случае выполняются следующие равенства:\\
\таа{\пункт
$\fa c\in\R:\,\limx{x_0}cf(x)=ca$;}
{\пункт
$\limx{x_0}(f(x)\pm g(x))=a\pm b$;}
\таа{
\пункт
$\limx{x_0}(f(x)g(x))=ab$;}
{\пункт
если $b\ne0$, то $\limx{x_0}\,\dfrac{f(x)}{g(x)}=\dfrac{a}{b}$.}
\кзадача

\задача
Найдите следующие пределы:\\
\пункт
$\limx{1}\,\dfrac{x^2-5x+2}{4x+5};$
\пункт
$\limx{2}\,\dfrac{x^2-6x+8}{x^2-4};$
\пункт
$\limx{0}\,\dfrac{x^2+3x^4}{3x^2+x^4}$;
\пункт
$\limx{0}\,\dfrac{\sqrt{1+x}-1}{x}$.
\кзадача

\ввзадача[Принцип двух милиционеров для функций]
Пусть функции $f$, $g$ и~$h$ определены на множестве~$M$ и~для любого $x\in M$ имеют место неравенства $f(x) \le g(x) \le h(x)$. Тогда если $\limx{a}f(x) = \limx{a}h(x) = b$, то существует предел функции~$g$ в~точке~$a$, причём $\limx{a}g(x) = b$.
\кзадача

\задача
Докажите, что:
\пункт
Если $0 < x < \frac{\pi}2$, то $\sin(x) < x < \tg(x)$;
\пункт
$\limn n(e^{\frac1n}-1)=1$.
\кзадача

\ввзадача[\лк Замечательные\пк пределы]
Докажите следующие равенства:\\
\пункт
$\limx{0} \frac{\sin x}{x}=1$;
\пункт
$\limx{0} \frac{e^x-1}{x}=1$.
\кзадача

\ЛичныйКондуит{-0.6mm}{6mm}
\ОбнулитьКондуит
\newpage

\опр
\label{left limit}
Пусть функция~$f$ определена на множестве~$M$
% и~точка~$a$ является предельной точкой  множества $M\cap\{x\mid x<a\}$.
и~некоторая левая проколотая полуокрестность окрестность точки~$a$ вложена в~$M$.
Число~$b$ называется \emph{пределом слева функции~$f$ в~точке~$a$}, если для каждого числа $\ep > 0$ существует такое число $\de > 0$, что для всех~$x$ из множества $\dot U_\de^-(a)\cap M$ выполняется условие $f(x)\in U_\ep(b)$.\\
Обозначение: $\limx{a-0} f(x)=b$.
\копр

\задача
\пункт
Запишите определение~\ref{left limit} формально (при помощи кванторов).\\
\пункт
Сформулируйте (в т.ч. на языке кванторов) определение предела справа.
\кзадача

\взадача
Пусть функция~$f$ имеет пределы справа и~слева в~точке~$a$. Докажите, что предел $\limx{a} f(x)$ существует тогда и~только тогда, когда $\limx{a-0} f(x) = \limx{a+0} f(x)$.
\кзадача

\задача
Приведите пример функции на~$\R$, которая в~точке~$a$:
\невСтрочку
\пункт
не имеет предела ни слева, ни справа;
\пункт
имеет предел слева, но не имеет предела справа;
\пункт
имеет разные пределы слева и~справа.
\кзадача

\ввзадача
Докажите, что функция, монотонная на интервале $(a,b)$, имеет предел как слева, так и~справа в~каждой точке этого интервала.
\кзадача

\опр
\label{limit on infty}
Пусть функция~$f$ определена на
% неограниченном
множестве~$M$
и найдётся число $S \in \R$, такое что множество $\{x \in \R \colon |x| > S\}$ вложено в $M$.
Число~$b$ называется \emph{пределом функции~$f$ при~$x$, стремящемся к~$\infty$}, если для каждого числа $\ep > 0$ существует такое число $C>0$, что для всех $x\in M$ из неравенства $|x| > C$ следует $f(x) \in U_\ep(b)$.\\
Обозначение: $\limx{\infty} f(x)=b$ или $f(x)\to b$ при $x\to \infty$.
\копр

\задача
\пункт
Запишите определение~\ref{limit on infty} формально (при помощи кванторов).\\
\пункт
Сформулируйте определение предела функции~$f$ при~$x \to +\infty$ и при~$x \to -\infty$.\\
\пункт
Сформулируйте следующее определение: предел функции~$f$ при~$x$, стремящемся к~$a$, равен~$\infty$ (соответственно $+\infty$, $-\infty$).
\qquad Обозначение: $\limx{a}f(x) = \infty$.
\кзадача

\задача
Пусть функция~$f$ не обращается в~ноль в~некоторой окрестности точки~$a$.  Докажите, что $\limx{a}f(x) = \infty$ тогда и~только тогда, когда $\limx{a}\,\dfrac{1}{f(x)}=0$.
\кзадача

\сзадача
Верно ли, что если $\limx{a} f(x) = b$ и~$\limx{b} g(x) = c$, то $\limx{a} g(f(x)) = c$?
\кзадача

\сзадача
Приведите пример функции, определённой на~$\R$, не равной тождественно нулю ни на каком интервале, но имеющей в~каждой точке нулевой предел.
\кзадача

\сзадача
Приведите пример функции, определённой на~$\Q$ и~имеющей в~каждой точке бесконечный предел.
\кзадача







\vfill
\ЛичныйКондуит{0mm}{6mm}
%\GenXMLW

\end{document}



