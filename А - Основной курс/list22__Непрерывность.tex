% !TeX encoding = windows-1251
\documentclass[a4paper,12pt]{article}
\usepackage[mag=970]{newlistok}



\ВключитьКолонитул

\УвеличитьВысоту{1.8cm}
\УвеличитьШирину{1.2cm}
%\renewcommand{\spacer}{\vfil}

\newcommand{\sm}{\setminus}

\Заголовок{Непрерывность}
\НомерЛистка{22}
\ДатаЛистка{09.2013}

\begin{document}
\СоздатьЗаголовок

\раздел{Напоминание}
\renewcommand{\spacer}{\vspace{1mm}}

\опр
\label{limit1}
Число $a$ называют \выд{пределом последовательности} $(x_n)$, если $(x_n)$ можно представить в виде $x_n = a + \al_n$, где последовательность $(\al_n)$ бесконечно малая.
Обозначение: $\limn x_n = a$.
Говорят также, что \выд{$(x_n)$ стремится к $a$ при $n$, стремящемся к бесконечности} (и пишут $x_n \to a$ при $n \to \infty$).
\копр

\опр
\label{limit2}
Число $a$ называют \выд{пределом последовательности} $(x_n)$, если для всякого числа $\ep > 0$ найдётся такое число $N$, что при любом натуральном $k > N$ будет выполнено неравенство $|x_k - a| < \ep$.\\
Формально: $\forall\,\ep>0\ \ \exists\,k\in\N\ \ \forall\,n>k \ \ |x_n-a|<\ep$.
\копр

\опр
\label{limit3}
Число $a$ называют \выд{пределом последовательности} $(x_n)$, если в любом интервале, содержащем $a$, содержатся \выд{почти все} члены $(x_n)$ (то есть все, кроме конечного числа).
\копр

\утверждение
Определения~\ref{limit1},~\ref{limit2}~и~\ref{limit3} эквивалентны.
\кутверждение

\утверждение[Теорема Вейерштрасса]
Любая ограниченная монотонная последовательность сходится.\hspace{-2mm}
\кутверждение

\утверждение[Теорема Больцано-Вейерштрасса]
Из всякой ограниченной последовательности можно выделить сходящуюся подпоследовательность.
\кутверждение

\опр
Последовательность~$(x_n)$ называется \emph{фундаментальной}, если для всякого числа~$\ep>0$ существует такое $k\in\N$, что для любых натуральных~$m$ и~$n$, больших~$k$, выполняется неравенство $|x_m - x_n|<\ep$.\\
Формально: $\fa\ep>0\,\exi k\in\N\,\fa m,n>k \colon |x_m-x_n|<\ep$.
\копр

\утверждение[Критерий Коши]
Последовательность сходится тогда и только тогда, когда она является фундаментальной.
\кутверждение

\опр
Пусть $\ep > 0$, $a \in \R$. Множество $\dot U_\ep(a) = U_\ep(a) \sm \{a\} = \{x \in \R \mid 0<|x-a|<\ep\}$ называется \emph{проколотой $\ep$-окрестностью точки}~$a$. Множества $\dot U_\ep^+(a) = \{x\in \R \mid a<x<a+\ep\}$ и~$\dot U_\ep^-(a)=\{x\in \R \mid a-\ep<x<a\}$ называются \emph{правой} и~\emph{левой проколотыми полуокрестностями точки}~$a$ соответственно.
\копр

\опр
\label{Heine}
(\emph{Предел функции в~смысле Гейне})\; Пусть функция~$f$ определена на множестве~$M$
% и~точка~$a$ является предельной точкой этого множества.
и~некоторая проколотая окрестность точки~$a$ вложена в~$M$.
Число~$b$ называется \emph{пределом функции $f$ в~точке~$a$}, если для любой последовательности~$(x_n)$ элементов множества $M\sm\{a\}$, сходящейся к~$a$, последовательность $(f(x_n))$ сходится к~$b$.\\
Обозначение: $\limx{a}f(x) = b$ или $f(x)\to b$ при $x\to a$.
\копр

\опр
\label{Cauchy}
(\emph{Предел функции в~смысле Коши})\; Пусть функция~$f$ определена на множестве~$M$
% и~точка~$a$ является предельной точкой этого множества.
и~некоторая проколотая окрестность точки~$a$ вложена в~$M$.
Число~$b$ называется \emph{пределом функции~$f$ в~точке~$a$}, если для каждого $\ep>0$ существует такое число $\de>0$, что для всех~$x$ из множества $\dot U_\de(a)\cap M$ выполняется условие $f(x)\in U_\ep(b)$.\\
Формально:  $\fa \ep>0 \,\exi \de>0 \,\fa x\in\dot U_\de(a)\cap M:\,f(x)\in U_\ep(b)$.
\копр

\утверждение
Определения~\ref{Heine} и~\ref{Cauchy} эквиваленты.
\кутверждение


%\newpage
%\vspace*{-17mm}
\раздел{Непрерывность}
\renewcommand{\spacer}{\vspace{2mm plus 2mm minus 1.5mm}}

\опр[непрерывность в смысле Коши]
Пусть $M \subseteq\R$.
Функция $f:M\to\R$ называется \выд{непрерывной в точке $a\in M$},
если для любого $\ep > 0$ найдётся число $\de>0$ такое,
что для всех $x\in M \cap (a-\de, a+\de)$ выполнено неравенство $|f(x) - f(a)| < \ep$.
\копр
\vspace*{-1mm}


\задача
Укажите множество точек непрерывности функций:
\пункт
$x$;
\пункт
$\sgn x$;
\пункт
$x^2$;
\пункт
$\{x\}$;
\пункт
$\dfrac{1}{x}$.
\кзадача

\задача
Сформулируйте определение непрерывности, аналогичное определению предела по Гейне.
\кзадача


\задача Запишите без отрицаний: \лк  $f:M \to \R$
\выд{разрывна} в точке $a\in M$\пк\ (для определения по Коши).
\кзадача

\задача
Будет ли функция, непрерывная и положительная в точке $a$\\
\пункт ограниченной; \пункт положительной в некоторой окрестности точки $a$?
\кзадача


\задача Функции $f$, $g$ непрерывны в точке $a\in \R$. Докажите: %, что:
\вСтрочку\\
%\сНовойСтроки
\пункт $|f|\in C(a)$;
\пункт $f\pm g\in C(a)$;% непрерывна в точке $a$;
%\пункт $\max(f, g)\in C(a), \min(f, g)\in C(a)$;% непрерывна в точке $a$;
\пункт $f\cdot g\in C(a)$; %непрерывна в точке $a$;
\пункт если %, кроме того,
$g(a) \neq 0$,
то ${f}/{g}\in C(a)$.
\кзадача

\задача
Сформулируйте и докажите теорему о непрерывности композиции
двух непрерывных функций.
%(Композицией функций $f(x)$ и $g(x)$ называется функция $f(g(x))$.)
\кзадача

\задача Докажите непрерывность функции (во всех точках её области
определения):
\вСтрочку
\пункт $x^n$, где $n\in\N$;
\пункт многочлен из $\R[x]$;%\footnote{$\R[x]$ --- все многочлены с вещественными коэффициентами.};
\пункт $P(x)/Q(x)$, где $P,Q\in\R[x]$, $Q\ne0$;
\пункт $\root n \of x$,~где~$n\in\N$;
\пункт $x^\al$, где $\al\in\R$;
\пункт $\sin x$;
\пункт $e^x$;
\пункт $a^x$, где $a>0$;
\пункт $\ln x$;
\пункт $\arctg x$.
\кзадача

\задача Придумайте функцию  $f:\R\to\R$, %определённая на $\R$ и\\
\вСтрочку
\пункт всюду разрывную;
\пункт непрерывную лишь в одной точке;
\пункт разрывную в точках вида $1/n$, где $n\in\N$, и только в них;
\спункт разрывную в точках из $\Q$ и только в них;
\сспункт \hspace*{2mm} на каждом отрезке принимающую все действительные значения.
\кзадача

\задача
Пусть функция $f$ определена и непрерывна в каждой точке отрезка $[a,b]$ ($f\in C([a,b])$),
причём $f(a) > 0$, $f(b) <0$. Всегда ли найдётся  такое $c\in (a,b)$, что $f(c)=0$?
\кзадача

\задача Докажите, что любой многочлен нечётной степени
с действительными коэффициентами имеет хотя бы один действительный корень.
\кзадача

\ввзадача[Теорема Коши о промежуточном значении]
Пусть $f\in C([a;b]),$ $f(a) < f(b)$.
Верно ли, что для любого числа $c \in [f(a), f(b)]$ существует такая
точка $x\in [a,b]$, что $f(x) = c$?
\кзадача



\задача[Теорема Л.~Бр\'ауэра о неподвижной точке для отрезка]
Пусть $f\in C([0;1])$ и все значения функции $f$ содержатся в
отрезке~$[0;1]$. Докажите, что уравнение $f(x)=x$ имеет корень.
\кзадача



% \опр
% Функция $f:M\rightarrow\R$ %, где $M\subseteq\R$,
% называется \выд{равномерно непрерывной на} %множестве}
% $M$, если
% для каждого $\varepsilon>0$ найд\"ется такое $\delta>0$, что
% для любой пары точек $x,y\in M$,
% расстояние между которыми меньше $\delta$, %будет
% таких что $|x-y|<\delta$,
% выполнено %неравенство
% $|f(x)-f(y)|<\varepsilon$.
% \копр
%
% \задача
% Обязательно ли
% Функция $f$ равномерно непрерывна на множестве $M\subseteq\R$.
% Обязательно ли тогда $f\in C(M)$?
% будет непрерывной на этом множестве?
% \кзадача
%
%
% \задача
% \вСтрочку
% Какие из %следующих
% функций $x^2$, $\sqrt x$, $1/x$ равномерно непрерывны\\
% \пункт
% на луче $[1;\infty)$;
% \пункт
% на интервале~$(0;1)$?
% \кзадача



\ввзадача Пусть функция непрерывна на отрезке. Докажите, что она
на этом отрезке
\\\пункт[1-я теорема Вейерштрасса] ограничена;
\\\пункт[2-я теорема Вейерштрасса] достигает своего наибольшего и наименьшего значения;
\\\пункт Верны ли утверждения пунктов а), б) для функции, непрерывной
на интервале или на прямой?
\кзадача

% \задача
% \пункт
% Докажите, что из любого покрытия отрезка интервалами можно выбрать конечное подпокрытие;
% \пункт
% Докажите, что функция, непрерывная на отрезке, равномерно непрерывна на нём.
% \кзадача

\задача
Функции $f$ и $g$ непрерывны на $\R$ , причём $f(x)=g(x)$ при $x\in\Q$.
Докажите, что $f=g$.
\кзадача

\опр \выд{Промежутком} называют любой
отрезок, полуинтервал, интервал, открытый или замкнутый луч на прямой,
а также всю прямую.
\копр

\ввзадача[Теорема о монотонной функции]
Пусть функция непрерывна на некотором промежутке $I\subseteq \R$.
Докажите, что $f$ обратима на этом промежутке тогда и только тогда,
когда $f$ строго монотонна на нём, причём обратная функция также будет строго
монотонной и непрерывной.
\кзадача


\задача
Непостоянная функция $f$ определена и непрерывна на множестве
$I\subseteq\R$.  Каким может быть множество значений этой функции
на $I$, если $I$ --- это
\вСтрочку
\пункт
отрезок;
\пункт
интервал;
\пункт
прямая?
\кзадача

\задача Выпуклый многоугольник $M$, прямая $l$ и точка $A$ лежат в одной
плоскости. Докажите, что~найдётся прямая $l'$, которая
делит $M$ на две равновеликие части и
\вСтрочку
\пункт параллельна~$l$;
\пункт проходит через~$A$.
\кзадача


\задача Докажите, что функция, \пункт непрерывная на некотором интервале; \пункт монотонная на
некотором интервале, имеет предел как слева, так и справа в каждой точке этого интервала.
\кзадача

\задача Докажите, что монотонная функция, определённая на промежутке, непрерывна во всех точках
этого промежутка, за исключением не более чем счётного числа точек.
\кзадача

\сзадача Пусть функция $f$ определена на промежутке и в каждой точке этого
промежутка имеет конечный предел, не обязательно совпадающий со значением в точке. Насколько $f$
может отличаться от непрерывной? Более точно, каким может быть у $f$ множество точек разрыва?
\кзадача



\vfill
\ЛичныйКондуит{0mm}{6mm}
%\GenXMLW

\end{document}



