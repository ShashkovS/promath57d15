% !TeX encoding = windows-1251
\documentclass[a4paper,12pt]{article}
\usepackage[mag=960]{newlistok}



\ВключитьКолонитул

\УвеличитьВысоту{2.5cm}
\УвеличитьШирину{1.5cm}
%\renewcommand{\spacer}{\vfil}

\newcommand{\sm}{\setminus}

\Заголовок{Производная --- 2}
\НомерЛистка{23}
\ДатаЛистка{09.2013}

\begin{document}
\СоздатьЗаголовок


\задача
  Однажды утром (в 9:00 ) турист вышел из лагеря к вершине горы и добрался туда в 20:00. В 9:00
следующего дня он начал спуск с вершины (по той же тропе, что и поднимался) и в 20:00 вернулся в лагерь.
Найдётся ли на тропе точка, которую турист проходил в одно и то же время в день подъёма и в день спуска?
\кзадача




\задача
\пункт Пусть функция $f$
непрерывна на отрезке $[a,b]$  и дифференцируема на интервале $(a,b)$,
и $f'(x_0)>0$ в некоторой точке \hbox{$x_0\in (a,b)$.}
Докажите, что найдется такая окрестность $U$ точки $x_0$, что
для всех \hbox{$x\in U$} если $x>x_0$, то
$f(x)>f(x_0)$, а если $x<x_0$, то $f(x)<f(x_0)$?
\спункт Верно ли, что $f$ из пункта~а) монотонно
возрастает в некоторой окрестности точки $x_0$?
\кзадача


\опр Точка $x_0$ называется точкой \выд{локального максимума} функции $f$,
если $f(x_0)\ge f(x)$ для всех $x$ из некоторой окрестности $x_0$.
Аналогично определяется точка \выд{локального минимума}.
\копр

\задача Докажите, что у любой функции $f$,
непрерывной на отрезке $[a,b]$  и дифференцируемой на интервале $(a,b)$,
существует точка локального максимума и точка локального минимума.
\кзадача

\задача \выд{(Теорема Ферма)} Пусть $f$
непрерывна на $[a,b]$  и дифференцируема на $(a,b)$.
Докажите, что если $x\in (a,b)$ --- точка локального
максимума (минимума) $f$, то $f'(x)=0$. Верно ли обратное?
%\пункт Верно ли, что если $f'(x)=0$ для $x\in (a,b)$, то $x$ --- точка
%локального максимума или минимума?
\кзадача

\задача Докажите для всех $x$:
\label{ex}
\вСтрочку
\пункт $x^4+x^3\ge -\frac{3^3}{4^4}$;
\пункт $x^6-6x+5\ge 0$;
\пункт $x^4-4x^3+10x^2-12x+5\ge 0$.
\кзадача


%\задача Найдите все точки локальных экстремумов:
%функций на интервале $(-\infty ,\infty )$:
%\label{ex}
%\вСтрочку
%\пункт $x+1$;
%\пункт $x^2-1$;
%\пункт $x^3+x$
%\пункт $\sin x$.
%\кзадача

\задача Найдите наибольшее и наименьшее значение функций из задачи~\ref{ex} при $x\in [0,1]$.
\кзадача



\задача Найдите наименьшее значение функций при $x>0$:
\вСтрочку
\пункт $x+\frac{1}{x}$;
\пункт $x+\frac{1}{x^2}$;
\пункт $x^2+2x+\frac{4}{x}$.
\кзадача

\задача
Какую наибольшую площадь может иметь трапеция,
три стороны которой равны~1?
\кзадача

\задача
Даны две точки $A$ и $B$ по разные стороны от прямой $l$, разделяющей
две среды. Требуется найти такую точку $D$ на прямой $l$,
чтобы время преодоления
светом пути $ADB$ было минимальным при условии, что скорость распространения
света в верхней среде $v_1$, а в нижней --- $v_2$.
Докажите, что такая точка $D$ существует и определяется условием
$\sin\alpha_1/\sin\alpha_2=v_1/v_2$, где $\alpha_1$ и $\alpha_2$ ---
углы, образованные прямыми $AD$ и $BD$ с прямой, проходящей через точку
$D$ перпендикулярно $l$.
\кзадача

\задача
Найдите точку параболы $y=x^2$, ближайшую к точке $(-1;2)$.
\кзадача


\задача \выд{(Теорема Ролля)} %Пусть функция $f$ удовлетворяет условию $(*)$
Пусть $f$ непрерывна на $[a,b]$  и дифференцируема на $(a,b)$,
и, кроме того, $f(a)=f(b)$. Докажите, что найдётся такая точка $x\in
(a,b)$, что $f'(x)=0$.
\кзадача

\задача  \выд{(Теорема Лагранжа)}
Пусть $f$ непрерывна на $[a,b]$  и дифференцируема на $(a,b)$.
Докажите, что найдётся такое $x\in (a,b)$, что
$f'(x)=\frac{f(b)-f(a)}{b-a}$ и
объясните геометрический смысл этой теоремы. % Лагранжа.
\кзадача

\задача %Пусть функция $f$ удовлетворяет условию $(*)$
Пусть $f$ непрерывна на $[a,b]$  и дифференцируема на $(a,b)$.
Докажите, что если для всех $x\in (a,b)$ выполнено:
\вСтрочку
\пункт $f'(x)=0$, то $f$ постоянна
на $[a,b]$.
\пункт $f'(x)>0$,
то $f$ возрастает на $[a,b]$.
\кзадача

\задача
Функции $f$ и $g$ непрерывны на $[a,b]$ и дифференцируемы на $(a,b)$, причём $f(a) = g(a)$ и $\fa x \in (a,b) \colon f'(x) \ge g'(x)$. Докажите, что $\fa x \in [a,b] \colon f(x) \ge g(x)$.
\кзадача

\задача Докажите для всех $x>0$: %выполнены неравенства:\\
\вСтрочку
%\пункт $e^x>1+x$;
%\пункт $e^x>1+x+\frac{x^2}{2}$;
\пункт $\sin x>x-\frac{x^3}{6}$;
\пункт $1-\frac{x^2}{2}<\cos x<1-\frac{x^2}{2!}+\frac{x^4}{4!}$;\\
\спункт $e^x>1+x+\frac{x^2}{2}+\ldots +\frac{x^n}{n!}$, где $n\in \N$.
\кзадача

\задача
\пункт
Докажите, что двигаясь по прямой со скоростью, не большей $v$, нельзя за время $t$ уехать дальше, чем на $vt$.
\спункт
Докажите, что двигаясь по плоскости со скоростью, не большей $v$, нельзя за время $t$ уехать дальше, чем на $vt$.
\кзадача


\сзадача Пусть %функция
$f$ определена и дифференцируема на $(a,b)$.
\вСтрочку
\пункт Верно ли, что $f'$ непрерывна на $(a,b)$?\\
\пункт Пусть у $f'$ существуют пределы слева и справа в точке $x_0\in(a,b)$.
Верно ли, что они совпадают?\\
\пункт [Теорема Дарбу]
Пусть $[c,d]\subset(a,b)$.
Докажите, что $f'$  принимает на $[c,d]$ все значения между
$f'(c)$ и~$f'(d)$.
\кзадача


\сзадача
Среди ровной степи стоит гора. На вершину ведут две тропы
(считаем их графиками непрерывных функций),
не опускающиеся ниже уровня степи.
Два альпиниста одновременно начали подъём (по разным тропам),
соблюдая условие:
всё время быть на одинаковой высоте.
Смогут ли они достичь вершины, двигаясь непрерывно, если
\пункт
тропы состоят из конечного числа подъёмов и спусков;
\\\пункт в общем случае?
\кзадача


\сзадача %{\small\sc (Н.~Н.~Константинов)}
Из $A$ в $B$ ведут две дороги,
не пересекающие друг друга и сами себя.
Две машины,~\hbox{связанные}~верёвкой длины $15$ м,
проехали из $A$ в $B$ по разным дорогам,
не порвав верёвки.
Два круглых воза радиуса $8$~м
выезжают одновременно по разным дорогам, один из $A$ в $B$,
другой из $B$ в $A$.
Могут ли они разминуться?

\кзадача


\vfill
\ЛичныйКондуит{0mm}{6mm}
%\GenXMLW

\end{document}



