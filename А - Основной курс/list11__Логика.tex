% !TeX encoding = windows-1251
\documentclass[12pt]{article}

\usepackage[mag=1030]{newlistok}

%\УвеличитьШирину{1.5cm}
%\УвеличитьВысоту{1.5cm}

\global\addtolength{\vsize}{-115mm}%
\global\advance\vsize by 10cm
\global\advance\vsize by 10cm

%\renewcommand{\spacer}{\vfil}
\ВключитьКолонитул

\begin{document}

\Заголовок{Логика}
\НомерЛистка{11}
\ДатаЛистка{03.2013}

\СоздатьЗаголовок

\задача
В парламенте некой страны 100 депутатов. Каждый депутат либо честный, либо продажный. Известно, что среди любых двух депутатов хотя бы один\т продажный. Сколько честных?
\кзадача

\задача
    Три человека\т $A$, $B$ и $C$\т обладают абсолютными логическими способностями. Кроме того, каждый из них знает, что двое других мыслят абсолютно логично. Этой троице показали 7~марок: 2~красных, 2~зелёных и 3~синих. Затем всем троим завязали глаза и наклеили на лоб по марке, а остальные марки спрятали. После того, как повязки с глаз были сняты, у $A$ спросили: \лк Можете ли вы назвать хотя бы один цвет, которого на вас наверняка нет?\пк, и $A$ ответил: \лк Нет\пк. Когда тот же вопрос задали $B$, он тоже ответил: \лк Нет\пк. Про кого из $A$, $B$ и $C$ можно сказать, какая на нём марка?
\кзадача

\задача
Двум гениальным математикам сообщили по натуральному числу, сказав, что эти числа различаются на 1. После этого они по очереди задают друг другу один и тот же вопрос: \лк Знаешь ли ты моё число?\пк\ (отвечают только \лк да\пк\ или \лк нет\пк). Сможет ли каждый из математиков узнать оба числа?
\кзадача


\задача\сНовойСтроки
Каждый туземец с острова Амба\т либо честный, либо лжец. Честные изрекают только истинные высказывания, лжецы\т только ложные.
\пункт
Вам навстречу идут двое туземцев. На вопрос \лк Вы\т честный?\пк\ первый из них буркает что-то неразборчивое. Второй туземец приходит Вам на помощь: \лк Мой друг ответил \лк да\пк. Но верить ему не стоит\т он лжец\пк. Что вы можете сказать про этих туземцев?
\пункт
Один из следующей пары туземцев говорит: \лк Я лжец или мой друг лжец\пк. Ваши выводы?
\пункт
Что вы подумаете, услышав высказывание: \лк Я лжец и мой друг лжец\пк?
\пункт
А услышав: \лк Если я честный, то мой друг лжец\пк?
\кзадача

\опр
Назовём {\it высказыванием} любое повествовательное предложение, которое либо истинно, либо ложно. Если $A$ и $B$\т некоторые высказывания, то можно определить следующие высказывания:
%Высказывание
\лк не $A$\пк\ (обозначение $\overline A$)\т {\it отрицание\/} высказывания $A$, истинно если и только если $A$ ложно;\\
%Высказывание
\лк $A$ и $B$\пк\ ($A\wedge B$)\т {\it конъюнкция\/} $A$ и $B$, истинно если и только если и $A$, и $B$ истинны; \\
%Высказывание
\лк $A$ или $B$\пк\ ($A\vee B$)\т {\it дизъюнкция\/} $A$ и $B$, истинно если и только если хотя бы одно из $A$ и $B$ истинно; \\
%Высказывание
\лк если $A$, то $B$\пк\ ($A\to B$) \т {\it импликация\/}, истинно если и только если $A$ ложно или и $A$, и $B$ истинны.
\копр

\задача
Выразите\\
\вСтрочку
\пункт
$A\to B$;
\пункт
$A\wedge B$\\
через $A$ и $B$, используя только дизъюнкцию и отрицание.
\кзадача

\задача
Выразите\\
\вСтрочку
\пункт
$\overline{A\to B}$;
\пункт
$A\vee B$\\
через $A$ и $B$, используя только конъюнкцию и отрицание.
\кзадача
\vfill
\ЛичныйКондуит{0mm}{6mm}
\ОбнулитьКондуит
\newpage

\соглашение
Для упрощения записи логических утверждений удобно использовать \выд кванторы.
Квантор существования: {$\exists$} (перевёрнутое \лк E\пк от \выд{Exists\/}\rm); Квантор всеобщности: {$\forall$} (перевёрнутое \лк А\пк от \выд{All\/}\rm);
Например, высказывание \лк Найдётся такой $x$, что $f(x)=0$\пк можно записать так: \лк $\exists\, x \colon f(x) = 0$\пк.
\ксоглашение

\задача
Локсодрома считается хорошей, если, во-первых, брахистохрона длиннее морской мили, и, во-вторых, строфоида не самопересекается. Определите без отрицания плохую (не являющуюся хорошей) локсодрому.
\кзадача

\задача
Запишите высказывания с помощью кванторов и постройте отрицания, не используя \лк не\пк:\\
\пункт
Для всякого $n\in \N$ верно: $f(n) < n^2$; \\
\пункт
Найдётся $\alpha\in\R$, что для любого $x\in\N$ выполнено: $\pi(x) \le \alpha \dfrac{x}{\ln x}$.
\кзадача

\сзадача
Докажите, что высказывание, истинность которого зависит
только от истинности высказываний $A_1,\dots,A_n$,
выражается через них с помощью только дизъюнкции, конъюнкции и отрицания.
\кзадача

\задача Рассмотрим два определения лёгкой контрольной:\\
\hbox{\phantom{bbbbbbbbbb}}
I. \выд{В каждом варианте каждую задачу решил хотя бы один ученик.}\\
\hbox{\phantom{bbbbbbbbb}}
II. \выд{В каждом варианте хотя бы один ученик решил все задачи.}\\
Может ли контрольная быть лёгкой в смысле определения~I и трудной в смысле
определения~II?
\кзадача

\задача
Солдату-цирюльнику пришел приказ: брить тех солдат его взвода,
которые не бреются сами (а остальных не брить). Сможет ли он его выполнить?
\кзадача

\задача
Являются ли следующие утверждения истинными или ложными:\\
\phantom{\rule{0mm}{7mm}}\hfil
\hbox{\framebox{Утверждение в рамке ложно}}
\hfil\hfil
\hbox{\framebox{\framebox{Утверждение в двойной рамке истинно}}}
\hfil
\кзадача

\ссзадача [Истинное происшествие]
Н.Н.Константинов сказал участникам своего %математического
семинара: \лк В январе занятия %семинара
проходят 13, 17, 20, 24, 27 и 31 числа. В один из этих дней
вам будет предложена %неожиданная
контрольная работа на логическую тему, но в какой
именно день, вы накануне знать ещё не будете.\пк
\сНовойСтроки
\пункт
Докажите, что эта контрольная не могла быть предложена 31 января.
\пункт
Докажите, что эта контрольная не могла быть предложена 27 января.
\пункт
Докажите, что эта контрольная не могла быть предложена 20 января.
\пункт
Однако 20 января эта контрольная состоялась (кстати говоря,
единственную задачу этой контрольной вы сейчас читаете). Ясное
дело, накануне ни один участник семинара об этом не знал. Как это
совместить с решением предыдущих пунктов задачи?
\кзадача

\vfill
\ЛичныйКондуит{0mm}{6mm}

%\GenXMLW

\end{document}




\сзадача [Истинное происшествие]
Н.Н.Константинов сказал участникам своего %математического
семинара: \лк В январе занятия %семинара
проходят 13, 17, 20, 24, 27 и 31 числа. В один из этих дней
вам будет предложена %неожиданная
контрольная работа на логическую тему, но в какой
именно день, вы накануне знать ещё не будете.\пк
\сНовойСтроки
\пункт
Докажите, что эта контрольная не могла быть предложена 31 января.
%\пункт
%Докажите, что эта контрольная не могла быть предложена 27 января.
\пункт
Докажите, что эта контрольная не могла быть предложена 20 января.
\пункт
Однако 20 января эта контрольная состоялась (кстати говоря,
единственную задачу этой контрольной вы сейчас читаете). Ясное
дело, накануне ни один участник семинара об этом не знал. Как это
совместить с решением предыдущих пунктов задачи?
\кзадача

