% !TeX encoding = windows-1251
\documentclass[a4paper,12pt]{article}
\usepackage{newlistok}
% \usepackage{tikz}
% \usetikzlibrary{calc}

\global\addtolength{\vsize}{-115mm}%
\global\advance\vsize by 10cm
\global\advance\vsize by 10cm

\ВключитьКолонитул

\УвеличитьВысоту{1.5cm}
\УвеличитьШирину{1.5cm}
%\renewcommand{\spacer}{\vfil}

\Заголовок{Последовательности}
\НомерЛистка{13}
\ДатаЛистка{09.2013}

\begin{document}
\СоздатьЗаголовок

\опр
Последовательность $(x_n)$ называется \emph{ограниченной сверху}, если найдётся такое число~$C$, что при всех натуральных~$n$ будет выполнено неравенство $x_n<C$.\\
Формально: $\exists \, C\in\R \ \ \forall\, n\in\N \ \ x_n<C$.\\
Аналогично определяется последовательность, \выд{ограниченная снизу}.\\
Если последовательность ограничена и сверху и снизу, говорят, что она \выд ограничена.
\копр

\опр
Последовательность $(x_n)$ называется \emph{возрастающей}, если при всех натуральных~$n$ выполнено неравенство $x_n<x_{n+1}$.\\
Формально: $\forall\, n\in\N \Rightarrow x_n<x_{n+1}$.\\
Аналогично определяются \выд убывающая, \выд невозрастающая, \выд неубывающая последовательности.
\копр

\опр
Последовательность называется \emph{монотонной}, если она является либо возрастающей, либо убывающей, либо невозрастающей, либо неубывающей.
\копр

\опр
Последовательность~$(y_k)$ называется \emph{подпоследовательностью} последовательности~$(x_n)$, если существует возрастающая последовательность натуральных чисел~$(n_k)$ такая, что $y_k=x_{n_k}$.
\копр

\опр
\emph{Суммой} последовательностей $(x_n)$ и~$(y_n)$ называется последовательность $(z_n)$, задаваемая соотношением $z_n=x_n+y_n$ при каждом натуральном~$n$. Аналогично определяются \emph{разность}, \emph{произведение} и~\emph{отношение} двух последовательностей.
\копр

\опр
Последовательность $(x_n)$ называется \emph{бесконечно малой}, если для любого положительного числа $\ep$ при $n\gg0$ выполняется неравенство $|x_n|<\ep$.\\
Формально: $\forall\,\ep>0\ \ \exists\,k\in\N\ \ \forall\,n>k \ \ |x_n|<\ep$.
\копр

\утверждение
Бесконечно малая последовательность является ограниченной.
\кутверждение

\утверждение
Сумма, разность и~произведение бесконечно малых последовательностей являются бесконечно малыми последовательностями.
\кутверждение

\bigskip
\hrule
\bigskip

\задача
\пункт Придумайте две различные последовательности, являющиеся подпоследовательностями друг друга.
\спункт Придумайте такую последовательность натуральных чисел, чтобы каждая последовательность натуральных чисел являлась её подпоследовательностью.
\кзадача

\сзадача
Докажите, что у любой последовательности найдётся монотонная подпоследовательность.
\кзадача

\задача
Есть ли последовательность, члены которой найдутся в любом интервале числовой оси?
\кзадача

\задача
Докажите, что следующие последовательности являются бесконечно малыми (то есть для каждой последовательности~$(x_n)$ по заданному положительному числу~$\ep$ найдите какой-нибудь номер~$k$, начиная с~которого выполнено неравенство $|x_n|<\ep$):\\
\пункт $x_n=\dfrac1n$;
\пункт $x_n=\dfrac{14}{n^3}$;
\пункт $x_n=\dfrac1{2n^2+3n-1}$;
\пункт $x_n=\dfrac{\sin n^\circ}{n^2}$.
\кзадача

\взадача
Пусть $(x_n)$\т бесконечно малая, а~$(y_n)$\т ограниченная последовательность. Докажите, что $(x_n+y_n)$\т ограниченная, а~$(x_ny_n)$\т бесконечно малая последовательность.
\кзадача

\сзадача
Любую ли последовательность можно представить как отношение\\
\пункт двух ограниченных;
\пункт двух бесконечно малых последовательностей?
\кзадача

\задача
Дана последовательность $(x_n)$ с положительными членами. Верно ли, что $(x_n)$ бесконечно малая тогда и только тогда, когда последовательность $(\sqrt x_n)$ бесконечно малая?
\кзадача

\задача
В бесконечно малой последовательности $(x_n)$ переставили члены (то есть взяли взаимно однозначное соответствие $f\from\N\to\N$ и получили новую последовательность $(y_n)$, где $y_n=x_{f(n)}$ для всех $n\in\N$). Обязательно ли полученная последовательность будет бесконечно малой?
\кзадача

\задача
Последовательность состоит из положительных членов, причём сумма любого количества любых её членов не превосходит 1. Докажите, что эта последовательность бесконечно малая.
\кзадача

\vfill
\ЛичныйКондуит{0mm}{6mm}

%\GenXMLW

\end{document}




