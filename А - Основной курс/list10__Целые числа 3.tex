% !TeX encoding = windows-1251
\documentclass[a4paper,12pt]{article}
\usepackage{newlistok}

\УвеличитьВысоту{1.5cm}
%\УвеличитьШирину{3mm}
\renewcommand{\spacer}{\medskip}

\Заголовок{Целые числа\т 3. Основная теорема арифметики.}
\НомерЛистка{10}
\ДатаЛистка{02.2013}

\begin{document}



\СоздатьЗаголовок

\задача
Пусть $n\in\N$. Докажите, что наименьший натуральный делитель числа~$n$, отличный от единицы, является простым числом.
\кзадача

\взадача
Докажите, что если натуральное число~$n$, большее единицы, не делится ни на одно из простых чисел, не превосходящих $\sqrt{n}$, то число~$n$ простое.
\кзадача

\задача
Найдите все простые числа, лежащие на числовой оси между~$2010$ и~$2050$.
\кзадача

\задача
Найдите все~$p\in\N$, для которых числа~$p$, $(p+10)$ и~$(p+14)$ простые.
\кзадача

\задача
Докажите, что существует бесконечно много простых чисел следующего вида:\\
\пункт
$p=4n-1$,
\пункт
$p=6n+5$,
\спункт
$p=4n+1$.
\кзадача

\взадача[Основная теорема арифметики]
\невСтрочку
\пункт
Докажите, что любое натуральное число можно разложить в~произведение простых чисел.
\пункт
Докажите, что это разложение единственно с~точностью до перестановки сомножителей.
\кзадача

\замечание
Более точно, представление натурального числа в~виде $n=p_1^{\al_1}\cdot p_2^{\al_2}\cdot\ldots\cdot p_k^{\al_k}$, где $p_1<p_2<\ldots<p_k$\т простые числа, а~$\al_1,\al_2,\ldots,\al_k$\т натуральные числа, единственно. Такое разложение на простые сомножители называется \emph{каноническим}.
\кзамечание

\задача
Числа $a$, $b$, $c$, $n$ натуральные, $(a,b) = 1$, $ab = c^n$. Найдётся ли такое целое $x$, что~$a = x^n$?
\кзадача

\задача
Решите в натуральных числах уравнение $x^{42} = y^{55}$.
\кзадача

\взадача[Теорема Лежандра]
Докажите, что простое число~$p$ входит в~каноническое разложение числа~$n!$ в~степени $[n/p]+[n/p^2]+[n/p^3]+\ldots$ (здесь $[x]$\т это целая часть числа~$x$).
\кзадача

\задача
Найдите каноническое разложение числа
\пункт
$2013$,
\пункт
$1002001$,
\пункт
$17!$,
\пункт
$C_{20}^{10}$.
\кзадача

\задача
\пункт
На какое число нулей оканчивается число~$2013!$?
\пункт
Может ли~$n!$ делиться на~$2^n$ при каком-либо натуральном~$n$?
\кзадача

% \задача[Малая теорема Ферма]
% Пусть $p$\т простое число, $n$\т целое число. Докажите, что
% \вСтрочку
% \пункт
% $n^p - n$ делится на $p$;
% \пункт
% если $(n,p) = 1$, то $n^{p - 1} - 1$ делится на $p$.
% \кзадача

\опр
\emph{Общим кратным} ненулевых целых чисел~$a$ и~$b$ называется целое число, которое делится как на~$a$, так и~на~$b$. Наименьшее среди положительных общих кратных называется \emph{наименьшим общим кратным} чисел~$a$ и~$b$. Обозначение:~$[a,b]$.
\копр

\взадача
Пусть $a=p_1^{\al_1}\cdot p_2^{\al_2}\cdot\ldots\cdot p_n^{\al_n},\,\,b=p_1^{\beta_1}\cdot p_2^{\beta_2}\cdot\ldots\cdot p_n^{\beta_n}$, причём $\al_i,\,\beta_i\geqslant0$.\\
\пункт
Найдите~$(a,b)$ и~$[a,b]$.
\пункт
Докажите, что $ab=(a,b)\cdot[a,b]$.
\кзадача

\взадача
Докажите, что любое общее кратное чисел~$a$ и~$b$ делится на~$[a,b]$.
\кзадача

\задача
Верно ли, что
\невСтрочку
\пункт
$[ca,cb]=c\cdot[a,b]$ при $c\in\N$;
\пункт
числа $[a,b]/a$ и~$[a,b]/b$ взаимно просты?
\кзадача

\задача
Про натуральные числа $a$ и $b$ известно, что $(a,b) = 15$, $[a,b] = 840$. Найдите $a$ и $b$.
\кзадача


\задача
Найдите все натуральные числа c нечётным числом натуральных делителей.
\кзадача


\vfill
\ЛичныйКондуит{0mm}{6mm}
\ОбнулитьКондуит
\newpage

\задача
Пусть $n=p_1^{\al_1}\cdot p_2^{\al_2}\cdot\ldots\cdot p_k^{\al_k}$.
\невСтрочку
\пункт
Найдите количество натуральных делителей числа~$n$.
\пункт
Докажите, что сумма натуральных делителей числа~$n$ равна
\[
\frac{p_1^{\al_1+1}-1}{p_1-1}\cdot\frac{p_2^{\al_2+1}-1}{p_2-1}\cdot\ldots\cdot \frac{p_k^{\al_k+1}-1}{p_k-1}.
\]
\спункт
Выведите формулу для суммы квадратов делителей числа~$n$.
\кзадача



\сзадача
Число, равное сумме всех своих натуральных делителей за исключением самого себя, называется \emph{совершенным}. Докажите, что если числа~$p$ и~$(2^p-1)$\т простые, то число $2^{p-1}\cdot(2^p-1)$ совершенно.
\кзадача

\опр
Простые числа вида $(2^n-1)$ называются \emph{числами Мерсенна}.
\копр

{\small
Числа Мерсенна получили известность в связи с эффективным критерием простоты, благодаря которому простые числа Мерсенна давно удерживают лидерство как самые большие известные простые числа. Часть этого критерия простоты дана в следующей задаче. На февраль 2013 года самым большим известным простым числом является число Мерсенна $M_{57\,885\,161}=2^{57885161}-1$, найденное в январе 2013 года в рамках проекта распределённых вычислений GIMPS. Десятичная запись числа $M_{57\,885\,161}$ содержит $17\,425\,170$ цифр.
\par}

\сзадача
Докажите, что
\невСтрочку
\пункт
если $m \dv n$, то $(a^m-1) \dv (a^n-1)$;
\пункт
множество чисел $\{n \mid (a^n-1)\dv r\}$ является арифметической прогрессией для любого натурального~$r$;
\пункт
если $(a^n-1)$\т простое, то $a=2$ и~$n$\т простое.
\кзадача


\опр
Простые числа вида $(2^{2^k}+1)$ называются \emph{числами Ферма}.
\копр

{\small
Изучение чисел такого вида начал Ферма, который выдвинул гипотезу, что все они простые. Однако, эта гипотеза была опровергнута Эйлером в 1732 году, нашедшим разложение числа $F_5=4\,294\,967\,297$  на простые делители (в худшем случае понадобится проверить больше 6000 простых делителей, однако при должном трудолюбии вы сможете найти простой делитель этого числа на калькуляторе).
Особый интерес числа Ферма представляют в связи с теоремой Гаусса — Ванцеля:
Правильный $n$-угольник можно построить с помощью циркуля и линейки тогда и только тогда, когда $n=2^r\cdot p_1\cdot\ldots\cdot p_k$,  где $p_i$ --- различные простые числа Ферма.
На январь 2013 года известно лишь 5 простых чисел Ферма: 3, 5, 17, 257, 65537. Существование других простых чисел Ферма является открытой проблемой.
\par}

\сзадача
\пункт
Докажите, что если число $(2^n+1)$\т простое, то $n=2^k$.
\пункт
Докажите, что числа вида $(2^{2^k}+1)$ являются взаимно простыми при различных~$k$.
\пункт
Докажите, что все делители чисел Ферма имеют вид $k\cdot 2^{n+1}+1$.
\кзадача


\сзадача
Может ли быть целым число\\
\пункт
$1+\dfrac12+\dfrac13+\ldots+\dfrac1n$;
\пункт
$1+\dfrac13+\dfrac15+\ldots+\dfrac1{2n+1}$?
\кзадача

\сзадача
На доске написано $n$~натуральных чисел. За одну операцию вместо двух чисел, ни одно из которых не делится на другое, можно написать их наибольший общий делитель и~наименьшее общее кратное. Докажите, что
\невСтрочку
\пункт
можно провести лишь конечное число таких операций;
\пункт
финальный результат не зависит от порядка выполнения действий.\\
Например, $(4,6,9)\to(2,12,9)\to(2,3,36)\to(1,6,36)$.\\
Или так: $(4,6,9)\to(4,3,18)\to(1,12,18)\to(1,6,36)$.
\кзадача

\vfill
\ЛичныйКондуит{0mm}{6mm}
%\GenXMLW

\end{document}
