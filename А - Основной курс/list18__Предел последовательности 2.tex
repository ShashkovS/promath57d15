% !TeX encoding = windows-1251
\documentclass[a4paper,12pt]{article}
\usepackage{newlistok}
% \usepackage{tikz}
% \usetikzlibrary{calc}
\usepackage[framemethod=tikz]{mdframed}




\ВключитьКолонитул

% \УвеличитьВысоту{1.5cm}
% \УвеличитьШирину{1.5cm}
%\renewcommand{\spacer}{\vfil}

\Заголовок{Предел последовательности\т 2}
\НомерЛистка{18}
\ДатаЛистка{12.2013}

\begin{document}
\СоздатьЗаголовок

\ввзадача[Теорема Вейерштрасса]
Докажите, что любая ограниченная монотонная последовательность сходится.
\кзадача


\задача
Докажите, что последовательность ($x_{n}$) сходится, и~найдите её предел, если
\невСтрочку
\пункт
$x_n = \dfrac{a^n}{n!}$, $a>0$;
\пункт
$x_1=\sqrt{2}$, $x_2=\sqrt{2\sqrt{2}}$, $x_3=\sqrt{2\sqrt{2\sqrt{2}}}$,  $\ldots$;
\пункт
$x_1=\sqrt{2}$, $x_2=\sqrt{2+\sqrt{2}}$, $x_3=\sqrt{2+\sqrt{2+\sqrt{2}}}$, $\ldots$;
\пункт
$x_1=\dfrac{1}{2}$, $x_{n+1}=x_{n}-x^2_{n}$.
\кзадача

\задача
Докажите, что существует предел $\limn \left(\dfrac{1}{n+1} + \dfrac{1}{n+2} + \ldots + \dfrac{1}{2n}\right)$.
\кзадача

\взадача[Число~$e$]
Докажите, что:
\невСтрочку
\пункт
последовательность $x_n=\left(1+\dfrac1n\right)^n$ сходится (её предел обозначают буквой~$e$);\\
\пункт
$\limn\left(1+\dfrac kn\right)^n=e^k \quad (k\in\N)$;
\пункт
$\limn\left(1-\dfrac1n\right)^n = \dfrac1e$;
\спункт
$\limn\left(1+\dfrac1{1!}+\dfrac1{2!}+\ldots+\dfrac1{n!}\right)=e$.
\кзадача

\опр
\label{Acc1}
Число~$a$ называется \emph{предельной точкой} последовательности~$(x_n)$, если для всякого числа $\ep>0$ и~для любого $k\in\N$ существует такое натуральное $n>k$, что выполняется неравенство $|x_n-a|<\ep$.\\
Формально: $\fa \ep>0\, \fa k\in\N\, \exi n>k \colon |x_n-a|<\ep$.
\копр

\опр
\label{Acc2}
Точка~$a$ называется {\it предельной точкой} последовательности~$(x_n)$, если любая окрестность точки~$a$ содержит бесконечно много точек последовательности~$(x_n)$.
\копр

\задача
Докажите эквивалентность определений~\ref{Acc1} и~\ref{Acc2}.
\кзадача

\задача
\пункт
Докажите, что если последовательность имеет предел, то этот предел является предельной точкой и~других предельных точек нет.\\
\пункт
Верно ли, что если последовательность имеет единственную предельную точку, то она (последовательность) является сходящейся?
\кзадача

\задача
Для следующих последовательностей укажите все их предельные точки:\\
\пункт
$x_{n}=\dfrac{n+1}{n}$;
\пункт
$x_{n}=(-1)^{n}$;
\пункт
$x_{n}=n$;
\пункт
$x_{n}=n^{(-1)^n}$.
\кзадача


\vfill
\ЛичныйКондуит{-0.6mm}{6mm}
\ОбнулитьКондуит
\newpage

\задача
Существует ли последовательность, множество предельных точек которой есть\\
\пункт
$\{1,2,\ldots,n\} \quad (n\in\N)$;
\пункт
$\N$;
\пункт
$[0,1]$;
\впункт
$(0,1)$;
\пункт
$\Q$;
\пункт
$\R$?
\кзадача

\задача
Докажите, что
\невСтрочку
\пункт
если $a$~является предельной точкой последовательности, то из этой последовательности можно выделить подпоследовательность, сходящуюся к~$a$; \пункт
всякая ограниченная последовательность имеет хотя бы одну предельную точку;
\ввпункт[Теорема Больцано-Вейерштрасса]
из всякой ограниченной последовательности можно выделить сходящуюся подпоследовательность.
\кзадача

\опр
Последовательность~$(x_n)$ называется \emph{фундаментальной}, если для всякого числа~$\ep>0$ существует такое $k\in\N$, что для любых натуральных~$m$ и~$n$, больших~$k$, выполняется неравенство $|x_m - x_n|<\ep$.\\
Формально: $\fa\ep>0\,\exi k\in\N\,\fa m,n>k \colon |x_m-x_n|<\ep$.
\копр

\ввзадача[Критерий Коши]
\вСтрочку
\пункт Докажите, что сходящаяся последовательность является фундаментальной;
\пункт Докажите, что фундаментальная последовательность имеет предел.
\кзадача

\задача[Признак Лейбница сходимости рядов]
Дана бесконечно малая монотонная последовательность $(x_n)$. Докажите, что существует предел $\limn \sumkun(-1)^k x_k$.
\кзадача

\задача
Последовательность $(x_n)$ строится по следующему закону: первый член выбирается произвольно, а каждый следующий вычисляется по формуле $x_{n+1}=ax_n+1$. При каких $a$ последовательность $(x_n)$ имеет предел?
\кзадача

\задача
Про последовательность $(x_n)$ известно, что для любого $n$ верно $|x_n - x_{n+1}|\le\dfrac{1}{2^n}$. Докажите, что она сходится.
\кзадача

\задача
Последовательность $(x_n)$ такова, что существует предел $\limn\sumkun |x_n|$ (ряд \emph{сходится абсолютно}). Докажите, что тогда существует предел $\limn\sumkun x_n$.
\кзадача

\взадача
Докажите, что в~упорядоченном поле~$F$ полнота эквивалентна
\невСтрочку
\пункт теореме Больцано-Вейерштрасса;
\пункт теореме Вейерштрасса.
\кзадача

\vfill
\ЛичныйКондуит{0mm}{6mm}
\GenXMLW

\end{document}


\задача
Найдите предел последовательности $(x_n)$, если $x_1=\dfrac12$ и $x_{n+1}=\dfrac{1}{2-x_{n}}$ при $n\in\N$.
\кзадача

\задача
Последовательность $(x_n)$ такова, что существует предел $\limn(|x_1-x_2|+\ldots+|x_{n-1}-x_n|)$. Докажите, что она сходится.
\кзадача

\задача
На отрезке $AB$ строится последовательность точек $(M_n)$ по следующему
правилу: $M_1=A$, $M_2=B$, $M_{n+2}$ --- середина отрезка $M_nM_{n+1}$ при
$n\in\N$. К какой точке отрезка $AB$ стремится последовательность $(M_n)$?
\кзадача

\задача
Найдите предел
$\limn \underbrace{\frac{1}{1+\frac{1}{1+\frac{1}{\cdots}}}}_n $.
\кзадача

