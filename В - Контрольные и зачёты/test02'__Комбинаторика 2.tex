% !TeX encoding = windows-1251
\documentclass[a4paper,12pt]{article}
\usepackage{newlistok}
\usepackage{tikz}

\УвеличитьВысоту{1.5cm}
\УвеличитьШирину{1.5cm}
\renewcommand{\spacer}{\vspace{1mm}}

\Заголовок{Самостоятельная работа}
\def\словоЛисток{СР \No\/}
\НомерЛистка{$2'$}
\ДатаЛистка{14 декабря 2012г.}

\begin{document}


\ncopy{3}{
%\vspace*{-12mm}
\СоздатьЗаголовок

\задача
Докажите, что для любых натуральных $n$ и $k\le n$ выполнено неравенство: $C_n^k<C_{n+1}^k$.
\кзадача

\задача
Найдите сумму $\sum\limits_{i=0}^n \br{2^i \cdot C_n^i} = 2^0\,C_n^0+2^1\,C_n^1+2^2\,C_n^2+\ldots+2^n\,C_n^n$.
\кзадача

\задача
Ведущий игры \лк Русское лото\пк Михаил Борисов достаёт 86 бочонков из 99, но он --- знатный жулик.
В тех случаях, когда число 49 --- год его рождения --- не выпало, он подменяет бочонок с минимальным номером на 49.
Во сколько раз меньше комбинаций становится от таких вот махинаций? (порядок бочонков не имеет значения)
\кзадача  


\задача
Докажите, что для любых натуральных $n>1$ и $k<n$ выполнено неравенство: $C_{2n}^k<C_{2n}^n$.
\кзадача

\задача
В коробке имеется 100 различных бусин. Мы собираем из них бусы по 40 бусин (очевидно, что бусины по-прежнему различны).
Какое количество разных бус можно собрать?
\кзадача


\ВосстановитьГраницы


\hrl
\note{Для получения оценки $n$ необходимо правильно решить $n-1$ задачу. Решившие все 5 задач получают две пятёрки.\\
Можно пользовать любыми бумажными носителями информации. Задачи необходимо \выд качественно записать. }
}


%\ЛичныйКондуит{0mm}{6mm}
%\СделатьКондуит{6mm}{6mm}

\end{document}
