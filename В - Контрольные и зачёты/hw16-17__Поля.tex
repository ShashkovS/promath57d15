% !TeX encoding = windows-1251
\documentclass[a4paper,12pt]{article}
\usepackage{newlistok}

\УвеличитьВысоту{0.05cm}
\УвеличитьШирину{1.5cm}
\renewcommand{\spacer}{\smallskip}

\Заголовок{Домашняя работа}
\def\словоЛисток{ДЗ \No\/}
\НомерЛистка{16-17}
\ДатаЛистка{декабрь 2013г.}

\begin{document}

\ncopy{3}{
\vspace*{-12mm}
\СоздатьЗаголовок

\опр
Пусть $F$\т поле. Обозначим $X(F) = \{n\in\N \mid \overbrace{1+\ldots+1}^{n} = 0\}$. Если множество~$X(F)$ не~пусто, \выд характеристикой поля $F$ называют число $\chi(F) = \min X(F)$. В противном случае говорят, что $\chi(F) = 0$.
\копр

\задача
Пусть $\chi(F) > 0$. Верно ли, что $\chi(F)$\т простое число?
\кзадача

\задача
Пусть поле $F$ бесконечно. Обязательно ли $\chi(F) = 0$?
\кзадача

\задача
Можно ли поле $\{a + b\sqrt{2} \mid a, b \in \Q\}$ упорядочить другим (не обычным) способом?
\кзадача

\утверждение[Принцип вложенных интервалов]
Пусть дана последовательность вложенных интервалов $(a_1,b_1) \supset(a_2,b_2) \supset\ldots$. Тогда пересечение $\capnui (a_n, b_n)$ не пусто.
\кутверждение

\задача
Верен ли принцип вложенных интервалов в полном поле?
\кзадача

\hrl
}

\end{document}
