% !TeX encoding = windows-1251
\documentclass[a4paper,12pt]{article}
\usepackage{newlistok}

\УвеличитьВысоту{1.5cm}
%\УвеличитьШирину{1.5cm}
\renewcommand{\spacer}{\smallskip}

\Заголовок{Домашняя работа}
\def\словоЛисток{ДЗ \No\/}
\НомерЛистка{22}
\ДатаЛистка{сентябрь 2014г.}

\begin{document}

\ncopy{3}{
\vspace*{-12mm}
\СоздатьЗаголовок

\опр[условие Липшица]
Функция $f$, определённая на множестве $M$, называется \выд липшицевой  (названо в честь Рудольфа Липшица),
если найдётся такая константа $C$, что для любых $x,y\in M$ выполнено неравенство $|f(x)-f(y)|\le C|x-y|$.
\копр

\задача
Пусть даны две функции $f(x)$ и $g(x)$, удовлетворяющие условию Липшица, и некоторая константа $c\in\R$.
Докажите, что следующие функции также являются липшицевыми:\\
\пункт $c f(x)$;
\пункт $f(x) \pm g(x)$;
\пункт $f(g(x))$;
\пункт $f(x)g(x)$, если область $M$ ограничена;
\кзадача

\задача
Докажите, что следующие функции являются липшицевыми:\\
\пункт $x$;
\пункт $\cos x$;
\пункт $\arcctg x$;
\пункт $x^n$ на любом ограниченном множестве;
\кзадача

\задача
Докажите, что липшицева функция на множестве $M$ непрерывна в каждой точке области~$M$.
\кзадача

\сзадача
На сковородке лежат две котлеты (можно считать, что котлеты --- выпуклые многоугольники).
Докажите, что их можно разрезать каждую на две равновеликих части одним прямолинейным разрезом.
\кзадача

% \ссзадача
% Есть бутерброд из колбасы и хлеба.
% Доказать, что можно его так разрезать на две части, чтобы хлеба и колбасы в каждой из них оказалось поровну.
% \кзадача

\hrl
}

\end{document}
