% !TeX encoding = windows-1251
\documentclass[a4paper,12pt]{article}
\usepackage{newlistok}

\УвеличитьВысоту{1.9cm}
\УвеличитьШирину{1.5cm}
%\renewcommand{\spacer}{\vfil}

\Заголовок{Домашняя работа}
\def\словоЛисток{ДЗ \No\/}
\НомерЛистка{1}
\ДатаЛистка{13 февраля 2013г.}

\begin{document}

\ncopy{1}{
\vspace*{-12mm}
\СоздатьЗаголовок
\задача
Арифметическая прогрессия состоит из 57 членов и её сумма равна 2013. Найдите сумму первого и последнего её членов.
\кзадача


\задача
Дана арифметическая прогрессия $p_n = 3+2(n-1)$. Найдите сумму первых ста её членов.
\кзадача


\задача
Дана геометрическая прогрессия $p_n = 3 \cdot 2^{n-1}$. Найдите сумму первых ста её членов.
\кзадача


\задача
Даны две арифметические прогрессии $p_n = 3 - 2(n+7)$ и $q_n = 1 + 5(n-3)$.
Найдите сумму $p_1q_1 + p_2q_2 + \ldots + p_nq_n$.
\кзадача


\hrl
{\small Для получения оценки $n$ необходимо правильно решить $n-1$ задачу.}
}

\end{document}
