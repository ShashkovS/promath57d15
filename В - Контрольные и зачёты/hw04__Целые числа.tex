% !TeX encoding = windows-1251
\documentclass[a4paper,12pt]{article}
\usepackage{newlistok}

\УвеличитьВысоту{2.5cm}
\УвеличитьШирину{1.5cm}
\renewcommand{\spacer}{\smallskip}

\Заголовок{Домашняя работа}
\def\словоЛисток{ДЗ \No\/}
\НомерЛистка{4}
\ДатаЛистка{13 апреля 2013г.}

\begin{document}

\ncopy{5}{
\vspace*{-12mm}
\СоздатьЗаголовок
\задача
Найдите наименьшее общее кратное чисел 2537 и 2773.
\кзадача


\задача
Какое наименьшее значение может принимать выражение $x^2 + y^2$ при условии, что $x$ и $y$\т целые числа, и $23x + 37y = 12$?
\кзадача


\задача
Известно, что у числа $n$ ровно 17 натуральных делителей. Может ли $n$ делиться на 57? 
\кзадача


\задача
Докажите, что если $x^{57} \dv y^{57}$, то $x \dv y$.
\кзадача


\hrl
\note{Для получения оценки $n$ необходимо правильно решить $n-1$ задачу.}
}

\end{document}
