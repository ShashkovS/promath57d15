% !TeX encoding = windows-1251
\documentclass[a4paper,12pt]{article}
\usepackage{newlistok}

\УвеличитьВысоту{1.5cm}
\УвеличитьШирину{1.5cm}
\renewcommand{\spacer}{\smallskip}

\Заголовок{Самостоятельная работа}
\def\словоЛисток{СР \No\/}
\НомерЛистка{22}
\ДатаЛистка{\ммгг}

\begin{document}

\ncopy{5}{
\vspace*{-12mm}
\СоздатьЗаголовок

\задача
Докажите, что функция $f$, дифференцируемая в точке $a$, непрерывна в точке $a$.
\кзадача

\задача
Школьник на контрольной дал такое определение непрерывности функции $f(x)$
в точке $a$:\\
\лк существует такое $\varepsilon>0$,
что для любого $\delta>0$ и для любого $x\in(a-\delta,a+\delta)$
выполнено %неравенство
$|f(x)-f(a)|\leq\varepsilon$\пк. \\
Какое свойство функции описано этим определением?
\кзадача

\задача
Функция $f$ непрерывна на $\R$ и принимает все действительные значения.
Докажите, что $f$  либо строго монотонна на $\R$, либо принимает какое-то
значение трижды.
\кзадача


\задача
  Функция $f^2(x)$ непрерывна на отрезке $[0,1]$.
  Обязательно ли функция $f(x)$ непрерывна?
\кзадача  


% \задача
% Пусть $f:\R\rightarrow\R$ --- непрерывная функция, $(a;b)$ --- любая точка
% на координатной плоскости. Докажите, что среди всех точек графика функции $f$
% найдётся такая, расстояние от которой до точки $(a,b)$ минимально
% (то есть не больше, чем расстояние от любой другой точки графика $f$ до
% $(a;b)$).
% \кзадача

\hrl
}

\end{document}
