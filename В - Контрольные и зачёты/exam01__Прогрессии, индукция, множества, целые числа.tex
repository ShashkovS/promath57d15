% !TeX encoding = windows-1251
\documentclass[a4paper,12pt]{article}
\usepackage{newlistok}

\УвеличитьВысоту{2.5cm}
\УвеличитьШирину{1.5cm}
%\renewcommand{\spacer}{\vfil}

\Заголовок{Программа зачёта: прогрессии, множества, комбинаторика, целые числа}
\НомерЛистка{}
\ДатаЛистка{}

\begin{document}
\СоздатьЗаголовок

\раздел{Прогрессии}
\begin{enumerate}
\item Определение арифметической прогрессии. Две формулы суммы арифметической прогрессии.
\item Определение геометрической прогрессии. Формула суммы геометрической прогрессии.
\item Числа Фибоначчи. Явная формула.
\\


\раздел{Теория множеств}
\item Множества и~их подмножества. Операции над множествами (объединение, пересечение, разность, декартово произведение) и~их свойства.
\item Формула включений-исключений для $n=3$.
\item Отображения множеств. Образы, прообразы и~их свойства. Композиция отображений.
\item Взаимно однозначное отображение. Обратимое и~обратное отображения. Критерий обратимости отображения.
\\


\раздел{Комбинаторика}
\item Число сочетаний и число перестановок. Явные формулы.
\item Основные комбинаторные свойства сочетаний (и их комбинаторные доказательства).
\item Треугольник Паскаля и его свойства.
\item Бином Ньютона.
\\


\раздел{Математическая индукция}
\item Формулировка принципа математической индукции и~обобщённого принципа математической индукции.
\item Неравенство Бернулли и иррациональность $\sqrt2$, $\sqrt3$, и т.д.
\item Формулы сокращённого умножения. Формулы для сумм $1^k+2^k+3^k+\ldots+n^k$ при $k=1;2;3$.
\\


\раздел{Целые числа}
\item Понятие делимости целых чисел и~его основные свойства. Делимое, делитель и~частное.
\item Признаки делимости на 2, 3, 4, 5, 8, 9 и~11.
\item Деление с~остатком (существование и~единственность неполного частного и~остатка).
\item Наибольший общий делитель двух целых чисел и~его простейшие свойства. Его представимость в~виде линейной комбинации.
\item Алгоритм Евклида. Его использование при поиске выражения для наибольшего общего делителя двух чисел через их целочисленную линейную комбинацию.
\item Линейные диофантовы уравнения и~общий метод их решения.
\item Основная теорема арифметики.
\item Теорема Лежандра о~каноническом разложении числа~$n!$.
\item Наименьшее общее кратное и~его простейшие свойства. Связь наибольшего общего делителя и~наименьшего общего кратного двух чисел с~этими числами.
\\


\end{enumerate}

\newpage

\noindent
\begin{enumerate}
\setlength\itemsep{12truept}
\item Определение арифметической прогрессии. Две формулы суммы арифметической прогрессии.
\item Определение геометрической прогрессии. Формула суммы геометрической прогрессии.
\item Числа Фибоначчи. Явная формула.
\item Множества и~их подмножества. Операции над множествами (объединение, пересечение, разность, декартово произведение) и~их свойства.
\item Формула включений-исключений для $n=3$.
\item Отображения множеств. Образы, прообразы и~их свойства. Композиция отображений.
\item Взаимно однозначное отображение. Обратимое и~обратное отображения. Критерий обратимости отображения.
\item Число сочетаний и число перестановок. Явные формулы.
\item Основные комбинаторные свойства сочетаний (и их комбинаторные доказательства).
\item Треугольник Паскаля и его свойства.
\item Бином Ньютона.
\item Формулировка принципа математической индукции и~обобщённого принципа математической индукции.
\item Неравенство Бернулли и иррациональность $\sqrt2$, $\sqrt3$, и т.д.
\item Формулы сокращённого умножения. Формулы для сумм $1^k+2^k+3^k+\ldots+n^k$ при $k=1;2;3$.
\item Понятие делимости целых чисел и~его основные свойства. Делимое, делитель и~частное.
\item Признаки делимости на 2, 3, 4, 5, 7, 8, 9 и~11.
\item Деление с~остатком (существование и~единственность неполного частного и~остатка).
\item Наибольший общий делитель двух целых чисел и~его простейшие свойства. Его представимость в~виде линейной комбинации.
\item Алгоритм Евклида. Его использование при поиске выражения для наибольшего общего делителя двух чисел через их целочисленную линейную комбинацию.
\item Линейные диофантовы уравнения и~общий метод их решения.
\item Основная теорема арифметики.
\item Теорема Лежандра о~каноническом разложении числа~$n!$.
\item Наименьшее общее кратное и~его простейшие свойства. Связь наибольшего общего делителя и~наименьшего общего кратного двух чисел с~этими числами.
\end{enumerate}
%\ЛичныйКондуит{0mm}{6mm}
%\СделатьКондуит{6mm}{6mm}

\newpage
\renewcommand{\spacer}{\vspace{4.1mm minus .1mm}}


\раздел{Теория множеств}

\задача
Докажите, что для любых множеств $A$, $B$, $C$\\
\пункт $A\cap(B\cup C)=(A\cap B)\cup(A\cap C)$;
\пункт $A\cup(B\cap C)=(A\cup B)\cap(A\cup C)$;
\кзадача

\задача
Докажите, что для любых множеств $A$, $B$, $C$\\
\пункт $A\setminus(B\cup C)=(A\setminus B)\cap(A\setminus C)$;
\пункт $A\setminus(B\cap C)=(A\setminus B)\cup(A\setminus C)$.
\кзадача

\задача
Верно ли, что для любых множеств $A,B,C$\\
\пункт $A\setminus(A\setminus B)=A\cap B$;
\пункт $A\setminus(B\setminus C)=(A\setminus B)\cup(A\cap C)$;
\кзадача

\задача
Верно ли, что для любых множеств $A,B,C$\\
\пункт $(A\setminus B)\cup(B\setminus A)=A\cup B$;
\пункт $(A\setminus B)\cup B=A$;
\кзадача

\задача
Верно ли, что для любых множеств $A,B,C$\\
\пункт $(B\setminus A)\cap C=(B\cap C)\setminus A$;
\пункт $(B\setminus A)\cup C=(B\cup C)\setminus(A\setminus C)$;
\кзадача

\задача
Верно ли, что для любых множеств $A,B,C$\\
\пункт $(A\times B)\cup(C\times D)=(A\cup C)\times(B\cup D)$;
\пункт $(A\times B)\cap(C\times D)=(A\cap C)\times(B\cap D)$?
\кзадача

\задача
Существуют ли такие множества~$A$, $B$ и~$C$, что условия $A\cap B\ne\varnothing$, $A\cap C=\varnothing$ и~$(A\cap B)\setminus C=\varnothing$ выполнены одновременно?
\кзадача

\задача
Пусть $f:X\to Y,\,\,$ $A_1,A_2\subset X$.\quad Верно ли, что
\пункт $f(X)=Y$;
\пункт $f^{-1}(Y)=X$;
\кзадача

\задача
Пусть $f:X\to Y,\,\,$ $A_1,A_2\subset X$.\quad Верно ли, что
\пункт $f^{-1}(f(X))=X$;
\пункт $f(f^{-1}(Y))=Y$;
\кзадача

\задача
Пусть $f:X\to Y,\,\,$ $A_1,A_2\subset X$.\quad Верно ли, что\\
\пункт $f(A_{1}\cup A_{2})=f(A_{1})\cup f(A_{2})$;
\пункт $f(A_{1}\cap  A_{2})=f(A_{1})\cap  f(A_{2})$;
\кзадача

\задача
Пусть $f:X\to Y,\,\,$ $A_1,A_2\subset X$.\quad Верно ли, что\\
\пункт $f(A_{1}\setminus A_{2})=f(A_{1})\setminus f(A_{2})$;
\пункт если $A_{1}\subset A_{2}$, то $f(A_{1})\subset  f(A_{2})$;
\пункт если $f(A_{1})\subset f(A_{2})$, то $A_{1}\subset A_{2}$?
\кзадача

\задача
Пусть $f:X\to Y,\,\,$ $B_1,B_2\subset Y$.\quad Верно ли, что\\
\пункт $f^{-1}(B_{1}\cup B_{2})=f^{-1}(B_{1})\cup f^{-1}(B_{2})$;
\пункт $f^{-1}(B_{1}\cap  B_{2})=f^{-1}(B_{1})\cap f^{-1}(B_{2})$;
%\пункт $f^{-1}(B_{1}\setminus B_{2})=f^{-1}(B_{1})\setminus  f^{-1}(B_{2})$;
\кзадача

\задача
Пусть $f:X\to Y,\,\,$ $B_1,B_2\subset Y$.\quad Верно ли, что\\
\пункт если $B_{1}\subset B_{2}$, то $f^{-1}(B_{1})\subset f^{-1}(B_{2})$;
\пункт если $f^{-1}(B_{1})\subset f^{-1}(B_{2})$, то $B_{1}\subset B_{2}$?
\кзадача

\задача
Пусть множество~$X$ состоит из $m$~элементов, а~множество~$Y$ из $n$~элементов.\\
\пункт Сколько существует различных отображений из множества~$X$ в~множество~$Y$?\\
\пункт Как много среди этих отображений взаимно однозначных?
\кзадача

\задача
Верно ли, что если отображение $f:X\to Y$ удовлетворяет условиям $f(X)=Y$ и~$f^{-1}(Y)=X$, то $f$~---~взаимно однозначно?
\кзадача

\задача
Пусть для отображений $f:X\to Y$ и~$g:Y\to X$ отображение $f\circ g$ тождественно. Верно ли, что $g=f^{-1}$?
\кзадача

\задача
Верно ли, что если $|A|=|B|$ и $|C|=|D|$, то\par\noindent
\пункт $|A\times C|=|B \times D|$;
\пункт $|A\cup C|=|B \cup D|$;
\пункт $|A\cap C|=|B \cap D|$.
\кзадача
%
%\задача
% Верно ли, что следующие множества счётны:\par\noindent
%
%\пункт множество всевозможных конечных последовательностей нулей и~единиц;\par\noindent
%
%\пункт множество всевозможных русских \лк слов\пк;\par\noindent
%
%\пункт множество конечных подмножеств множества~$\mathbb N$?
%\кзадача

\задача
Являются ли равномощными:\\
\пункт любые два отрезка на плоскости;
\пункт любые две окружности на плоскости;
\кзадача

\задача
Являются ли равномощными:\\
\пункт интервал и~полуокружность без концов;
\пункт интервал и~прямая;
\кзадача
%
%\задача
%очевидная
% Счётно ли множество бесконечных последовательностей из нулей и~единиц, в~которых число нулей конечно?
% \кзадача
%
%\задача
%очевидная
% Верно ли, что любое подмножество прямой равномощно некоторому подмножеству отрезка?
% \кзадача
%
%\задача
%простая
% Рассмотрим на клетчатой плоскости всевозможные замкнутые несамопересекающиеся ломаные, звенья которых идут по сторонам клеток. Счётно ли множество всех таких ломаных?
% \кзадача
%
%\задача
%простая
% Докажите, что множество точек любого треугольника (с внутренностью) равномощно множеству точек любого прямоугольника (с внутренностью).
% \кзадача
%
%\задача
%простая
% Равномощно ли множество всех лучей множеству всех окружностей (на плоскости)?
% \кзадача
%
%\задача
%простая
% Докажите, что множество точек плоскости, абсцисса которых неотрицательна, равномощно множеству всех точек плоскости.
% \кзадача
%
%\задача
% средняя
%Равномощно ли отрезку множество точек $[0;1]\cup[2;3]\cup[4;5]\cup\dots$?
%\кзадача
%
%\задача
% средняя
%Множество~$A$ счётно, а~множество~$B$ равномощно отрезку. Докажите, что множество $A\times B$ равномощно отрезку.
%\кзадача
%
%\задача
% средняя
%Счётно ли множество всевозможных возрастающих бесконечных последовательностей из целых чисел?
%\кзадача

%\newpage
\раздел{Целые числа}

\задача
% очевидная
Делится ли число $C_{100}^{50}$ на $83$?
\кзадача

\задача
% очевидная
Число~$1270$ при делении на некоторое число даёт неполное частное~$74$. Найдите делитель и~остаток.
\кзадача

\задача
% очевидная
Число делится на~$2$, но не делится на~$4$. Докажите, что количество чётных делителей этого числа равно количеству его нечётных делителей.
\кзадача

\задача
% очевидная
Пусть $p$~---~простое число, $a$~и~$b$~---~целые числа. Докажите, что $(a+b)^p-a^p-b^p$ делится на~$p$.
\кзадача

\задача
% очевидная
Верно ли, что из любых $5$~целых чисел можно выбрать два числа, разность квадратов которых делится на~$7$?
\кзадача

\задача
% очевидная
При каких натуральных~$k$ число $(k-1)!$ не делится на~$k$?
\кзадача

\задача
% очевидная
При каких целых значениях числа~$m$ дробь $\dfrac{14m+17}{21m+25}$ сократима?
\кзадача

\задача
Найдите остаток от деления
\пункт $2005^{2005^{2005}}$ на~$17$;
\пункт $7^{7^7}$ на $10$;
\кзадача

\задача
Найдите остаток от деления
\пункт $(2222^{5555}+5555^{2222})$ на~$7$;
\пункт $996^{996^{996}}$ на~$19$.
\кзадача

\задача
При каких~$a$ и~$b$ можно заплатить в~кассу один рубль, имея на руках неограниченное количество $a$-рублёвых купюр, если в~кассе есть неограниченное количество $b$-рублёвых купюр?
\кзадача

\задача
По окружности длины $a$~см катится колесо, длина обода которого равна $b$~см ($a$~и~$b$ натуральные, $(a,b)=d$). В~колесо вбит гвоздь, он оставляет отметки на окружности. Сколько отметок оставит гвоздь?
\кзадача

\задача
% безыдейный счёт
Решите в целых числах уравнение\\
\пункт $21x+48y=6;$
\пункт $105x+42y=56;$
\кзадача

\задача
% безыдейный счёт
Решите в целых числах уравнение\\
\пункт $1990x-173y=11;$
\пункт $nx+(2n-1)y=3$.
\кзадача

\задача
Докажите, что $C_{1000}^{500}$  не делится на 1024.
\кзадача

\задача
% очевидная
Найдите каноническое разложение на простые множители числа~$C_{50}^{25}$.
\кзадача

\задача
Может ли~$n!$ делиться на~$2^n$ при каком-либо натуральном~$n$?
\кзадача

\задача
% простая
При каких~$n$ число $3(n^2+n)+7$ делится на~$5$?
\кзадача

\задача
% простая, красивая
В~некотором числе переставили цифры и~получили число, в~$3$ раза меньшее исходного. Докажите, что исходное число делится на~$27$.
\кзадача

\задача
Докажите, что число, составленное из 81~единицы, делится на~$81$.
\кзадача

\задача
% очевидная
В~последовательности Фибоначчи (1, 1, 2, 3, 5, 8, 13, \ldots) каждый член равен сумме двух предыдущих. Заменим каждое число в~этой последовательности остатком от деления его на~$11$. Найдите 1000-ый член получившейся последовательности.
\кзадача

\задача
% средняя
В~последовательности Фибоначчи (1, 1, 2, 3, 5, 8, 13, \ldots) каждый член равен сумме двух предыдущих. Заменим каждое число в~этой последовательности остатком от деления его на $57^{2010}$. Правда ли, что получившаяся последовательность будет периодической?
\кзадача

\задача
Докажите, что из любых 52 целых чисел можно выбрать 2 таких числа, что\\
\пункт их разность делится на~$51$;
\пункт их сумма или их разность делится на~$100$.
\кзадача

\задача
% мне лениво...
Пусть $p_1^{\alpha_1}\ldots p_k^{\alpha_k}$~---~каноническое разложение числа~$n$. %\\
%а)~Найдите количество натуральных делителей числа~$n$.\par\noindent
%б)~Докажите, что сумма натуральных делителей числа~$n$ равна $$\dfrac{p_1^{\alpha_1+1}-1}{p_1-1}\cdot\dfrac{p_2^{\alpha_2+1}-1}{p_2-1}\cdot\ldots\cdot \dfrac{p_k^{\alpha_k+1}-1}{p_k-1}.$$
%в)~
Найдите произведение всех натуральных делителей~$n$.
\кзадача

\задача
% мне лениво...
Найдите число, произведение делителей которого равно
\пункт $108$;
\пункт $3^{30}\cdot5^{40}$.
\кзадача

\задача
% мне лениво...
Число имеет вид~$2^l\cdot3^m$, где $l$~и~$m$~---~неотрицательные целые числа.
Найдите данное число, если сумма его делителей равна~$403$.
\кзадача

\раздел{Математическая индукция}

\задача
% очевидная
Верно ли следующее рассуждение?\par
\лк Для любого натурального~$n$ произвольные $n$~точек плоскости лежат на одной прямой. При $n=1$ утверждение, очевидно, верно. Предположив, что утверждение верно для $n=k$,
докажем его для $n=k+1$. Возьмём произвольные~$k+1$ точек. Точки $1,\ldots, k$ лежат
на одной прямой, точки $2,\ldots, k+1$ лежат на одной прямой (по предположению индукции), следовательно, все они лежат на единственной прямой, проходящей через точки $2,\ldots, k$\пк
\кзадача

\задача
Дано $n$~прямых на плоскости. Докажите, что части, на которые эти прямые делят плоскость, можно раскрасить в~два цвета так, чтобы соседние части (граничащие по отрезку, по лучу или по прямой) были окрашены в~разные цвета.
\кзадача

\задача
% очевидная
Про последовательность чисел $a_1$, $a_2$, $a_3$, \dots известно, что $a_1=1$, $a_2=2$ и~$a_{n+1}=a_n-a_{n-1}$ при всех натуральных $n>2$. Докажите, что $a_{n+6}=a_n$ для всех~$n\in\mathbb N$.
\кзадача

\задача
Докажите, что\quad а)~$2^n>n$;\qquad б)~$2^n>n^2$ при $n>4$;\qquad в)~$n!>2^n$ при $n>3$.
\кзадача

\задача
% очевидная
Найдите все натуральные~$n$, при которых $2^n>n^3$.
\кзадача

\задача
% очевидная
Докажите, что при любом натуральном~$n$ выполняется равенство
$$
\left(1-\frac14\right)\left(1-\frac19\right)\cdot\ldots\cdot\left(1-\frac1{n^2}\right)=\frac{n+1}{2n}.
$$
\кзадача

\задача
Докажите, что при любом натуральном~$n$ выполняются равенства:\\
\пункт $\dfrac{1}{1\cdot2}+\dfrac{1}{2\cdot3}+\ldots+\dfrac{1}{n(n+1)}=\dfrac{n}{n+1};$\qquad
\кзадача

\задача
Докажите, что при любом натуральном~$n$ выполняются равенства:\\
\пункт $1\cdot2+2\cdot3+\ldots+n(n+1)=\dfrac{n(n+1)(n+2)}{3}.$
\кзадача

\задача
% очевидная
Докажите, что при любом натуральном~$n$ выполняется неравенство
$$
\frac{1\cdot3\cdot5\cdot\ldots\cdot(2n-1)}{2\cdot4\cdot6\cdot\ldots\cdot(2n)}
\leqslant\frac1{\sqrt{2n+1}}.
$$
\кзадача

\задача
% очевидная
Докажите, что число $11^{2n}-2^{6n}$ делится на~$57$ при любом натуральном~$n$.
\кзадача

\задача
Докажите, что $(2^{5n-2}+5^{n-1}\cdot3^{n+1})$ делится на~$17$ при любом натуральном~$n$.
\кзадача

\задача
% очевидная
Докажите, что
$
\,\,\underbrace{44\ldots4}_{n\ {\rm раз}}\underbrace{88\dots8}_{n-1\ {\rm раз}}9=
(\underbrace{66\ldots6}_{n-1\ {\rm раз}}7)^2.
$
\кзадача

\задача
% почти очевидная
Докажите, что сумма кубов любых трёх последовательных натуральных чисел делится на~$9$.
\кзадача

\задача
% простая
Дана последовательность чисел Фибоначчи:
$F_1=F_2=1$ и~$F_{k+1}=F_k+F_{k-1}$ при всех натуральных $k>1$.
Докажите, что
$F_{m+n}=F_{n-1}F_m+F_nF_{m-1}$ для любых натуральных~$m,n\geqslant 2$.
\кзадача

\задача
Докажите, что любое натуральное число можно представить как сумму нескольких различных степеней двойки (возможно, включая нулевую степень).
\кзадача

\задача
Докажите, что если число $a+\dfrac1a$~---~целое, то и~$a^n+\dfrac{1}{a^n}$~---~тоже целое.
\кзадача

\задача
Докажите, что всякие $n$~квадратов можно разрезать конечным числом прямолинейных разрезов на части так, что из полученных частей можно сложить квадрат.
\кзадача

\задача
% простая
На окружности расставлены~$2^n$ чисел, каждое из которых равно~$1$ или~$-1$. Каждую секунду все числа одновременно умножаются на своего правого соседа. Докажите, что настанет момент, когда все числа будут равны~$1$.
\кзадача

\задача
% простая
В~колоде часть карт лежит рубашкой вниз. Время от времени Петя вынимает из колоды пачку из одной или нескольких подряд идущих карт, в~которой верхняя и~нижняя карты лежат  рубашкой вниз, переворачивает всю пачку как одно целое и~вставляет в~то же место колоды. Докажите, что в~конце концов все карты лягут рубашкой вверх вне зависимости от того, как Петя выбирает пачки. (Примечание: если \лк пачка\пк состоит лишь из одной карты, то требуется только, чтобы она лежала рубашкой вниз.)
\кзадача

\задача
% простая
Докажите, что квадрат~$2^n\times2^n$, из которого вырезана произвольная клетка, можно разрезать на \лк уголки\пк из трёх клеток (\лк уголок\пк~---~это квадрат $2\times2$ без одной клетки).
\кзадача

\задача
Докажите, что длина любой стороны произвольного $n$-угольника меньше суммы длин остальных его сторон.
\кзадача

\задача
% средняя
Докажите, что любое натуральное число, не превосходящее~$n!$, можно представить в~виде суммы не более, чем~$n$ попарно различных слагаемых, каждое из которых является делителем числа~$n!$.
\кзадача

\задача
% сложная
Последовательность~$a_n$ удовлетворяет соотношению $a_{n+1}=3a_n-2a_{n-1}$. Известно, что числа~$a_1$ и~$a_0$ натуральны, причём $a_1>a_0$. Докажите, что $a_n\geqslant2^n$.
\кзадача

\раздел{Комбинаторика}

\задача
% очевидная
Сколько различных диагоналей можно провести в~выпуклом $n$-угольнике?
\кзадача

\задача
% очевидная
Сколькими способами можно посадить за круглый стол пять мужчин и~пять женщин так, чтобы никакие два лица одного пола не сидели рядом?
\кзадача

\задача
% очевидная
Сколькими способами можно распределить $3n$~различных предметов между тремя людьми так, чтобы каждый получил $n$~предметов?
\кзадача

\задача
% очевидная
Сколько слагаемых получится, если в~выражении $(1+x+y)^{20}$ раскрыть скобки, но не привести подобные члены?
\кзадача

\задача
В~школьной столовой имеются булочки $m$~разновидностей. Вася хочет купить $n$~булочек. Сколькими способами он может это сделать?
\кзадача

\задача
% очевидная
В~каком количестве шестизначных чисел хотя бы $2$~цифры совпадают?
\кзадача

\задача
% очевидная, простой счёт
Сколькими способами можно расставить на шахматной доске двух королей так, чтобы они не били друг друга (короли бьют друг друга, если они находятся в~клетках, имеющих общую вершину).
\кзадача

\задача
Имеется $2m$~одинаковых белых шаров и~$3n$~одинаковых чёрных шаров. Сколькими способами из всего этого набора можно взять $(m+n)$~шаров?
\кзадача

\задача
Сколькими способами можно выложить в~ряд $m$~белых и~$n$~чёрных шаров так, чтобы никакие два чёрных шара не лежали рядом?
\кзадача

\задача
Из $245$~кубиков $88$~имеют красную грань, $93$~---~синюю и~$103$~---~зелёную. При этом красную и~зелёную грани имеют $32$~кубика, красную и~синюю~---~$28$~кубиков, зелёную и~синюю~---~$37$~кубиков, а~грани всех трёх цветов имеет $13$~кубиков. Сколько кубиков не имеют грани ни одного из указанных цветов?
\кзадача

\задача
В~ряд записали $105$~единиц, поставив перед каждой знак~$\пк+\пк$. Сначала изменили знак на противоположный перед каждой третьей единицей, затем~---~перед каждой пятой, а~затем~---~перед каждой седьмой. Найдите значение полученного выражения.
\кзадача

\задача
Меню в~школьном буфете постоянно и~состоит из $n$~разных блюд. Петя хочет каждый день выбирать себе завтрак по-новому (за раз он может съесть от~$0$ до~$n$ разных блюд).
\пункт Сколько дней ему удастся это делать?
\пункт Сколько блюд он съест за это время?
\кзадача

\задача
Сколько различных слов (не обязательно осмысленных) можно получить, переставляя буквы в~словах
\пункт $\underbrace{AA\ldots A}_{\mbox{$a$ букв}}\underbrace{BB\ldots B}_{\mbox{$b$ букв}}$,
\пункт $\underbrace{A_1A_1\ldots A_1}_{\mbox{$a_1$ букв}}\underbrace{A_2A_2\ldots A_2}_{\mbox{$a_2$ букв}}\ldots\underbrace{A_nA_n\ldots A_n}_{\mbox{$a_n$ букв}}$?
\кзадача

\задача
Пусть множество~$C$ содержит $n$~элементов. Сколькими способами можно выбрать такие два его подмножества~$A$ и~$B$, что
\пункт $A\cap B=\varnothing$;
\пункт $A\subset B$?
\кзадача

\задача
Сколькими способами семь школьников могут вместе покататься
\пункт на аттракционе \лк поезд\пк, состоящем из $7$~одноместных вагончиков;
\пункт на аттракционе \лк поезд\пк, состоящем из $11$~одноместных вагончиков;
\кзадача

\задача
Сколькими способами семь школьников могут вместе покататься
\пункт на карусели, у~которой $7$~мест;
\пункт на карусели, у~которой $11$~мест;
\пункт на аттракционе \лк поезд\пк, состоящем из $7$~двухместных каруселей?
\кзадача

\задача
% почти очевидная
Сколькими нулями оканчивается число
\пункт $11^{100}-1$;
\пункт $9^{11}+1$?
\кзадача

\задача
% простая
Найдите сумму всех трёхзначных чисел, каждое из которых можно записать с~помощью цифр $1$, $2$, $3$ и~$4$ (цифры могут повторяться).
\кзадача

\задача
% нетривиальных идей нет, но надо много считать
Сколько телефонных номеров содержат комбинацию $12$?
\кзадача


\end{document}




