% !TeX encoding = windows-1251
\documentclass[a4paper,12pt,fleqn]{article}
\usepackage[mag=1000]{newlistok}

\УвеличитьВысоту{.6cm}
\УвеличитьШирину{1.0cm}
\renewcommand{\spacer}{\vspace{1mm minus 1mm}}
\pagestyle{plain}
%\sloppy 
\begin{document}
\vspace*{-1.7cm}
\АвторЛекции{Поршнев Е.}
\НазваниеЛекции{Понятие степени}
\СоздатьЗаголовокЛекции

Основная цель этой лекции\т придать смысл выражению $a^b$ (\emph{$a$ в степени $b$}). С самого начала сформулируем те свойства степени, к которым мы все привыкли:
\begin{nums}{-3}
\item
\label{additivity}
$a^b \cdot a^c = a^{b+c}$.
\item
\label{composition}
$(a^b)^c = a^{bc}$.
\item
\label{multiplicativity}
$a^c \cdot b^c = (ab)^c$.
\item
\label{base_monotoneness}
Пусть $a > b > 0$. Если $c > 0$, то $a^c > b^c$; если $c < 0$, то $a^c < b^c$.
\item
\label{exponent_monotoneness}
Пусть $b > c$. Если $a > 1$, то $a^b > a^c$; если $1 > a > 0$, то $a^b < a^c$.
\end{nums}

Напомним определение степени с натуральным показателем.
\опр
\label{natural}
Пусть $a\in\R$, $b\in\N$. Тогда по определению $a^b = \underbrace{a\cdot\ldots\cdot a}_{b\text{ раз}}$.
\копр

\лемма
Свойства 1--5 выполняются для степени с натуральным показателем.
\клемма
\доказательство
Доказательство оставляется читателю в качестве упражнения.
\кдоказательство

Несложно расширить определение степени на случай $b \in \Z$ (правда при этом придётся ограничить себя случаем $a \ne 0$).

\опр
\label{integer}
Пусть $a\in\R\setminus\{0\}$, $b\in\Z$. Тогда по определению
\[
a^b = \bcase{
a^b,&\quad b\in\N;\\
1,&\quad b=0;\\
1/a^{-b},&\quad -b\in\N.\\
}
\]
\копр

\лемма
Свойства 1--5 выполняются для степени с целым показателем.
\клемма
\доказательство
Свойства степени с целым показателем обычно выводят из свойств степени с натуральным показателем. Детальное доказательство оставляется читателю в качестве упражнения.
\кдоказательство

Перед тем, как определять степень с рациональным показателем, введём понятие корня.

\опр
Пусть $n\in\N$. \emph{Арифметическим корнем $n$-ой степени из неотрицательного числа~$a$} называется такое неотрицательное число~$x$, что $x^n=a$.\\ Обозначение:~$x=\sqrt[n]a$.
\копр

\теорема
Для любого неотрицательного вещественного числа $a$ и для любого натурального числа~$n$ существует корень $x=\sqrt[n]a$.
\ктеорема

\доказательство
Случай $a=0$ тривиален, поэтому будем считать, что $a > 0$.

Рассмотрим множество $M = \{t \mid t^n \le a,\, t\ge 0\} \subset \R$. Это множество очевидно не пусто (ведь $0 \in M$) и ограничено сверху (числом $\max(a,1)$). Поэтому из аксиомы о точной верхней грани следует, что существует $x = \sup M$. Покажем, что $x^n = a$.

Предположим, что $x^n = a + \ep$, где $\ep > 0$. Рассмотрим маленькое число $\de \in (0, x)$ и $y = x - \de$. Оценим~$y^n$.
\[
|x^n - y^n| = |x-y|\cdot|x^{n-1} + x^{n-2}y + \ldots + xy^{n-2} + y^{n-1}| < \de \cdot (nx^{n-1})
\]
Последнее неравенство обусловлено тем, что в силу $y < x$ каждое из слагаемых меньше, чем $x^{n-1}$, а всего слагаемых в точности $n$.
В частности, если выбрать $\de < \dfrac{\ep}{nx^{n-1}}$, то $|x^n - y^n| < \ep$ и тем самым $y^n > a$. Значит, $y$\т верхняя грань множества $M$ и при этом $y < x$. Это противоречит выбору $x$.

Предположим, что $x^n = a - \ep$, где $\ep > 0$. Рассмотрим маленькое число $\de \in (0, x)$ и $y = x + \de$. Оценим~$y^n$.
\[
|x^n - y^n| = |x-y|\cdot|x^{n-1} + x^{n-2}y + \ldots + xy^{n-2} + y^{n-1}| < \de \cdot (n(2x)^{n-1})
\]
Последнее неравенство обусловлено тем, что $y < 2x$.
Если выбрать $\de < \dfrac{\ep}{n(2x)^{n-1}}$, то $|x^n - y^n| < \ep$ и тем самым $y^n < a$. Значит, $y\in M$, что противоречит выбору $x$.

Теорема доказана.
\кдоказательство

\опр
\label{rational}
Пусть $a\in\R$, $a > 0$, $b = \frac{m}{n} \in \Q$. Тогда по определению $a^b = \sqrt[n]{a^m}$.
\копр

\утверждение
Определение~\ref{rational} корректно, то есть не зависит от представления числа $b$ в виде дроби.
\кутверждение
\доказательство
Пусть $b = \frac{m_1}{n_1} = \frac{m_2}{n_2}$. Обозначим $r_1 = \sqrt[n_1]{a^{m_1}}$, $r_2 = \sqrt[n_2]{a^{m_2}}$. Нам нужно проверить, что $r_1 = r_2$.

Из определения корня следует, что $r_1^{n_1} = a^{m_1}$ и $r_2^{n_2} = a^{m_2}$. Из свойства~\ref{composition} степени с целым показателем следует, что
\[
r_1^{n_1 m_2} = (r_1^{n_1})^{m_2} = (a^{m_1})^{m_2} = a^{m_1 m_2} = (a^{m_2})^{m_1} = (r_2^{n_2})^{m_1} = r_2^{n_2 m_1}.
\]
Из равенства $\frac{m_1}{n_1} = \frac{m_2}{n_2}$ следует, что $n_1 m_2 = n_2 m_1$. А значит, в силу свойства~\ref{base_monotoneness} числа $r_1$ и $r_2$ совпадают.
\кдоказательство

\утверждение
В случае $b\in\Z$ определение~\ref{rational} согласуется с определением~\ref{integer}.
\кутверждение
\доказательство
Действительно, если $b = \frac{m}{1}$, то $\sqrt[1]{a^m} = a^m$.
\кдоказательство

\лемма
\label{rational_properties}
Свойства 1--5 выполняются для степени с рациональным показателем.
\клемма
\доказательство
Свойства степени с рациональным показателем обычно выводят из свойств степени с целым показателем. Детальное доказательство оставляется читателю в качестве упражнения.
\кдоказательство

\лемма
\label{zero_limit}
Пусть $a > 0$\т вещественное число и $(x_n)$\т последовательность рациональных чисел, стремящаяся к нулю. Тогда $\lim a^{x_n} = 1$.
\клемма
\доказательство
Предположим, что $a \ge 1$. Зададимся произвольным числом $\ep > 0$. По аксиоме Архимеда существует натуральное число $k > \frac{a-1}{\ep}$. Тогда $(1+\ep)^k \ge 1 + k\ep > a$. Значит, $a^{1/k} < 1+\ep$. Кроме того $a^{1/k} < 1+\ep < \frac{1}{1-\ep}$, поэтому $a^{-1/k} > 1-\ep$.

Поскольку $\limn x_n = 0$, найдётся такое $N$, что для всех $n > N$ выполнено $|x_n| < \frac{1}{k}$. \\
Тогда $a^{x_n} \in (a^{-1/k}, a^{1/k}) \subset (1-\ep, 1+\ep)$.

Случай $a < 1$ сводится к уже разобранного с помощью равенства $a^{x_n} = (1/a)^{-x_n}$.
\кдоказательство

\bigskip
Ну и наконец перейдём к случаю $b\in\R$.

\утверждение
\label{exists}
Пусть $a$ и $b$\т вещественные числа, причём $a > 0$. Тогда найдётся  последовательность рациональных чисел $(x_n)$, такая что $\limn x_n = b$ и существует предел $\limn a^{x_n}$.
\кутверждение
\доказательство
Рассмотрим произвольную последовательность рациональных чисел, \выдж монотонно стремящуюся к $b$, все члены которой лежат на интервале $(b/2, b)$. Тогда её можно взять в качестве $(x_n)$. Действительно, в силу свойства~\ref{exponent_monotoneness} степени с рациональным показателем последовательность $(a^{x_n})$ тоже будет монотонной (возрастающей или убывающей в зависимости от того, больше единицы $a$ или меньше) и все её члены заключены между $a^{b/2}$ и $a^b$. Значит, по теореме Вейерштрасса у неё есть предел.
\кдоказательство

\утверждение
\label{only}
Пусть $a$ и $b$\т вещественные числа, причём $a > 0$. Для любой последовательности рациональных чисел $(y_n)$, такой что $\limn y_n = b$, существует предел $\limn a^{y_n}$, и этот предел не зависит от выбора последовательности~$(y_n)$.
\кутверждение
\доказательство
Пусть $(x_n)$\т последовательность из предыдущего утверждения. Мы уже знаем, что существует предел $s = \limn a^{x_n}$. Докажем, что $\limn a^{y_n} = s$. 

Действительно, $a^{y_n} = a^{x_n} + (a^{y_n} - a^{x_n}) = a^{x_n} + a^{x_n}(a^{y_n - x_n} - 1)$.
Из леммы~\ref{zero_limit} следует, что $\limn a^{y_n - x_n} = 1$. Поэтому $\limn (a^{y_n - x_n} - 1) = 0$ и $\limn a^{x_n}(a^{y_n - x_n} - 1) = 0$. Значит, последовательность $(a^{y_n})$ имеет предел, и этот предел равен $s$.
\кдоказательство

\опр
\label{real}
Пусть $a$ и $b$\т вещественные числа, причём $a > 0$. Рассмотрим произвольную последовательность рациональных чисел $(y_n)$, стремящуюся к $b$. По определению полагаем $a^b = \limn a^{y_n}$. Утверждение~\ref{only} гарантирует нам, что этот предел существует и не зависит от выбора последовательности~$(y_n)$.
\копр

\утверждение
В случае $b \in \Q$, определение~\ref{real} согласуется с определением~\ref{rational}.
\кутверждение
\доказательство
Поскольку $b \in \Q$, можно взять $y_n = b$. Очевидно, что $\limn a^{y_n} = a^b$.
\кдоказательство

\теорема
\label{real_properties}
Свойства 1--5 выполняются для степени с вещественным показателем.
\ктеорема

Доказательство теоремы мы разобьём на цепочку утверждений.


\утверждение
\label{real_additivity}
Выполнено равенство $a^b \cdot a^c = a^{b+c}$.
\кутверждение
\доказательство
Рассмотрим произвольные последовательности рациональных чисел $x_n \to b$ и $y_n \to c$. Тогда последовательность $(x_n+y_n)$ стремится к $b+c$ и
\[
a^b \cdot a^c = \limn a^{x_n} \cdot \limn a^{y_n} = \limn a^{x_n}\cdot a^{y_n} = \limn a^{x_n+y_n} = a^{b+c}.
\]
\кдоказательство

\следствие
\label{inversion_1}
Выполнено равенство $a^{-b} = 1/a^b$.
\кследствие
\доказательство
$1 = a^0 = a^{b + (-b)} = a^b \cdot a^{-b}$.
\кдоказательство

\утверждение
\label{real_multiplicativity}
Выполнено равенство $a^c \cdot b^c = (ab)^c$.
\кутверждение
\доказательство
Рассмотрим произвольную последовательность рациональных чисел $x_n \to c$.\\ Тогда $a^c \cdot b^c = \limn a^{x_n} \cdot \limn b^{x_n} = \limn a^{x_n}\cdot b^{x_n} = \limn (ab)^{x_n} = (ab)^c$.
\кдоказательство

\следствие
\label{inversion_2}
Выполнено равенство $\frac{1}{a^b} = \left(\frac1a\right)^b$.
\кследствие
\доказательство
Доказательство оставляется читателю в качестве упражнения.
\кдоказательство

\утверждение
\label{real_exponent_monotoneness}
Пусть $b > c$. Если $a > 1$, то $a^b > a^c$; если $1 > a > 0$, то $a^b < a^c$.
\кутверждение
\доказательство
Пусть $a > 1$. Возьмём какое-нибудь рациональное число $q \in (0, b-c)$ и последовательность рациональных чисел $x_n \to b-c$, все члены которой больше $q$. Тогда все члены последовательности $a^{x_n}$ не меньше, чем $a^q$. Поэтому $a^{b-c} = \limn a^{x_n} \ge a^q > 1$. Теперь применим утверждение~\ref{additivity}: $a^b = a^c \cdot a^{b-c} > a^c$.

Если же $a < 1$, то наоборот $a^{b-c} \le a^q < 1$ и $a^b < a^c$.
\кдоказательство

\следствие
\label{real_zero_limit}
Пусть $a > 0$\т вещественное число и $(x_n)$\т последовательность вещественных чисел, стремящаяся к нулю. Тогда $\limn a^{x_n} = 1$.
\кследствие
\доказательство
Для определённости будем считать, что $a \ge 1$. Для каждого натурального $n$ выберем пару рациональных чисел $y_n$ и $z_n$ так, что $y_n \le x_n \le z_n$ и при этом $|y_n| \le 2|x_n|$ и $|z_n| \le 2|x_n|$. Тогда обе последовательности $(y_n)$ и $(z_n)$ стремятся к нулю. Поэтому из леммы~\ref{zero_limit} следует, что $\limn a^{y_n} = 1$ и $\limn a^{z_n} = 1$. Из утверждения~\ref{real_exponent_monotoneness} следует, что $a^{y_n} \le a^{x_n} \le a^{z_n}$. Применяя теорему о двух милиционерах, получаем $\limn a^{x_n} = 1$.

Если же $a < 1$, всё аналогично за исключением того, что $a^{y_n} \ge a^{x_n} \ge a^{z_n}$.
\кдоказательство

\следствие
\label{real_unit_limit}
Пусть $b > 0$\т вещественное число и $(a_n)$\т последовательность вещественных чисел, стремящаяся к единице. Тогда $\limn a_n^b = 1$.
\кследствие
\доказательство
Выберем натуральное число $N \ge |b|$. Аксиома Архимеда гарантирует нам, что это можно сделать. Очевидно, что $\limn a_n^N = 1$ и $\limn a_n^{-N} = 1$. Из утверждения~\ref{real_exponent_monotoneness} следует, что для каждого $n$ выполнено либо $a_n^{-N} \le a_n^b \le a_n^N$, либо $a_n^{N} \le a_n^b \le a_n^{-N}$ (в зависимости от того, больше единицы $a_n$ или меньше). Применяя теорему о двух милиционерах, получаем $\limn a_n^b = 1$.
\кдоказательство

\утверждение
\label{real_base_monotoneness}
Пусть $a > b > 0$. Если $c > 0$, то $a^c > b^c$; если $c < 0$, то $a^c < b^c$.
\кутверждение
\доказательство
Пусть $c > 0$. Возьмём какое-нибудь рациональное число $q \in (0, c)$ и последовательность рациональных чисел $x_n \to c$, все члены которой больше $q$. Тогда все члены последовательности $(a/b)^{x_n}$ не меньше, чем $(a/b)^q$. Поэтому $(a/b)^с = \limn (a/b)^{x_n} \ge (a/b)^q > 1$. Значит, $a^c > b^c$.
Теперь применим утверждение~\ref{multiplicativity}: $a^c = b^c \cdot (a/b)^b > b^c$.

Если же $c < 0$, то $a^{-c} > b^{-c}$ и из следствия~\ref{inversion_1} получаем $a^c < b^c$.
\кдоказательство


\утверждение
\label{real_composition}
Выполнено равенство $(a^b)^c = a^{bc}$.
\кутверждение
\доказательство
Рассмотрим произвольные последовательности рациональных чисел $x_n \to b$ и $y_n \to c$. Распишем 
\[(a^b)^c - a^{bc} = ((a^b)^c - (a^{x_n})^c) + ((a^{x_n})^c - (a^{x_n})^{y_n}) + ((a^{x_n})^{y_n} - a^{x_n y_n}) + (a^{x_n y_n} - a^{bc}).\] 
Теперь будем рассматривать получившиеся слагаемые по одному.
\[
\begin{aligned}
(a^b)^c - (a^{x_n})^c
&= (a^b)^c \left(1 - (a^{x_n})^c \frac{1}{(a^b)^c}\right) \\
&= (a^b)^c \left(1 - (a^{x_n})^c \left(\frac{1}{a^b}\right)^c\right) \\
&= (a^b)^c \left(1 - (a^{x_n})^c (a^{-b})^c\right) \\
&= (a^b)^c \left(1 - (a^{x_n} \cdot a^{-b})^c\right) \\
&= (a^b)^c \left(1 - (a^{x_n - b})^c\right). \\
\end{aligned}
\]
Здесь мы последовательно воспользовались следствием~\ref{inversion_2}, следствием~\ref{inversion_1}, утверждением~\ref{real_multiplicativity} и утверждением~\ref{real_additivity}.

Из следствия~\ref{real_zero_limit} мы знаем, что $\limn a^{x_n - b} = 1$. Далее, из следствия~\ref{real_unit_limit} заключаем, что $\limn (a^{x_n - b})^c = 1$. Значит, $\limn (a^b)^c - (a^{x_n})^c = 0$.

С первым слагаемым разобрались. Теперь второе:
\[
\begin{aligned}
(a^{x_n})^c - (a^{x_n})^{y_n}
&= (a^{x_n})^{y_n} \left((a^{x_n})^c \frac{1}{(a^{x_n})^{y_n}} - 1\right) \\
&= (a^{x_n})^{y_n} \left((a^{x_n})^{c} (a^{x_n})^{-y_n} - 1\right) \\
&= (a^{x_n})^{y_n} \left((a^{x_n})^{c-y_n} - 1\right) \\
\end{aligned}
\]
Обозначим $\al = \frac12 a^b,\, \be = 2a^b$. Последовательность $(a^{x_n})$ стремится к $a^b$, поэтому начиная с некоторого номера все её члены лежат в интервале $(\al, \be)$. Из следствия~\ref{real_zero_limit} мы знаем, что $\limn \al^{c-y_n} = 1$ и $\limn \be^{c-y_n} = 1$. Из утверждения~\ref{real_base_monotoneness} следует, что для каждого достаточно большого $n$ выполнено либо $\al^{c-y_n} \le (a^{x_n})^{c-y_n} \le \be^{c-y_n}$, либо $\be^{c-y_n} \le (a^{x_n})^{c-y_n} \le \al^{c-y_n}$ (в зависимости от того, больше нуля $c-y_n$ или меньше). Применяя теорему о двух милиционерах, получаем $\limn (a^{x_n})^{c-y_n} = 1$. Кроме того $\limn (a^{x_n})^{y_n} = a^{bc}$, поэтому $\limn (a^{x_n})^c - (a^{x_n})^{y_n} = 0$.

Пошли дальше. Третье слагаемое просто равно нулю, а четвёртое стремится к нулю по определению степени. Значит, $\limn \left((a^b)^c - a^{bc}\right) = 0$. Но под знаком предела стоит константа, не зависящая от $n$. Это возможно только в том случае, если $(a^b)^c - a^{bc} = 0$. Утверждение доказано.
\кдоказательство

\medskip

Ура! На этом доказательство теоремы~\ref{real_properties} завершено.

\bigskip
\hrule
\bigskip

\задача
Докажите лемму~\ref{rational_properties}.
\кзадача

\задача
Докажите следствие~\ref{inversion_2}.
\кзадача

\ввзадача
Пусть $a,b \in \R$, $a > 0$, $a \ne 1$, $b > 0$.
\невСтрочку
\пункт
Докажите, что уравнение $a^x = b$ имеет решение.\\
(Указание: здесь поможет аксиома о точной верхней грани)
\пункт
Докажите, что это решение единственно.
\кзадача

\сзадача
Возможно ли такое, что $a, b \not\in \Q$ и $a^b \in \Q$.
\кзадача


\end{document}
