% !TeX encoding = windows-1251
\documentclass[a4paper,12pt]{article}
\usepackage[mag=970]{newlistok}

\УвеличитьВысоту{2.3cm}
\УвеличитьШирину{1.5cm}
\renewcommand{\spacer}{\vspace{1mm minus 1mm}}
\sloppy

\begin{document}
\АвторЛекции{Поршнев Е.}
\НазваниеЛекции{Вещественные числа. Дедекиндовы сечения.}
\vspace*{-6mm}
\СоздатьЗаголовокЛекции
\vspace*{-6mm}

\опр
Пусть $\Q = A \cup B$, причём:
\begin{nums}{-3}
\item
Для любых элементом $a \in A$ и $b \in B$ выполнено неравенство $a < b$;
%$\fa a\in A \, \fa b \in B \colon a < b$;
\item
Множество $A$ не пусто;
%$A \ne \nothing$;
\item
Множество $A$ не совпадает со всем $\Q$;
%$A \ne \Q$;
\item
\label{nomax}
В множестве $A$ нет наибольшего элемента.
\end{nums}
В этом случае пару множеств $(A, B)$ называют \emph{дедекиндовым сечением}. \\ Отметим, что множество $B$ однозначно восстанавливается по множеству $A$, поэтому для задания сечения достаточно зафиксировать множество $A$.
\копр

\лемма
\label{density1}
Пусть $(A,B)$\т сечение; $\ep > 0$\т рациональное число. Тогда найдутся такие $a\in A$ и $b\in B$, что $b-a \le \ep$.
Иначе говоря, для найдутся сколь угодно близкие числа из $A$ и $B$.
\клемма
\доказательство
% Выберем произвольную пару чисел $(a\in A, b\in B)$. Если $b-a < \ep$, нам повезло и больше ничего делать не нужно. В противном случае рассмотрим число $c = (a+b)/2$. Если $c \in A$, заменим нашу пару на $(c,b)$; если $c \in B$, заменим её на $(a,c)$. Будем действовать таким образом до тех пор, пока разность чисел в паре не станет меньше $\ep$. Поскольку разность на каждом шаге уменьшается вдвое, процесс завершится.
Рассмотрим множество $A' = \{a+\ep\mid a\in A\}$. 
Очевидно, что $A \subset A'$.
Предположим, что $A = A'$. 
Выберем произвольный $a \in A$. 
Так как $A = A'$, то $a + n\ep \in A$ для любого натурального~$n$. 
Но мы знаем, что множество $A$ ограничено сверху! 
Противоречие.
Значит, существует $b \in A' \setminus A$. 
Но тогда $b\in B$, и при этом $b = a+\ep$, где $a \in A$. 
Эту пару $(a,b)$ и выберем.
\кдоказательство

\лемма
\label{density2}
Пусть $(A,B)$\т сечение, причём $0 \in A$; $t > 1$\т любое рациональное число. Тогда найдутся такие $a\in A$ и $b\in B$, что $b/a \le t$. Иначе говоря, найдётся пара чисел из $A$ и $B$ такая, что их отношение сколь угодно близко к 1.
\клемма
\доказательство
Рассмотрим множество $A' = \{at\mid a\in A\}$. 
Покажем, что $A \subset A'$.
Выберем произвольный элемент $a \in A$. 
Если $a \le 0$, то $a/t < 0$. 
Отсюда, $a/t \in A$ и $a = (a/t)t \in A'$.\\
Если же $a > 0$, то $a/t < a$. 
Значит, $a/t \in A$ и $a = (a/t)t \in A'$.

Предположим, что $A = A'$. Выберем любое положительное число $a \in A$. В силу $A = A'$ получаем $at^n \in A$ для любого натурального $n$. Опять получаем противоречие с ограниченностью множества $A$ сверху.

Значит, существует $b \in A' \setminus A \subset B$. При этом $b = at$, где $a \in A$. Эту пару $(a,b)$ и выберем.
\кдоказательство

\опр
Пусть $(A,B)$\т дедекиндово сечение. Тогда множество $A$ называется \emph{вещественным числом}.
Множество всех дедекиндовых сечений обозначается через $\R$ и называется множеством действительных чисел.
\копр

\пример
Зафиксируем $q \in \Q$. Тогда множество $Q = \{a \mid a < q\} \subset \Q$ является вещественным числом. Оно соответствует сечению $(Q, \Q \setminus Q)$.\\
Необходимо проверить свойство~\ref{nomax}. Оно следует из того, что на любом интервале $(a,q)$ есть рациональные числа.
\кпример

\bigskip
Выше мы определили само множество вещественных чисел. Теперь необходимо определить операции сложения и умножения и ввести отношение порядка.
\bigskip

\опр
Пусть $X,Y$\т вещественные числа. Тогда суммой $X + Y$ будем называть множество $\{x+y \mid x\in X, y\in Y\}$.
\копр

\bigskip
Нам необходимо проверить, что такое определения сложения корректно, то есть что множество $A = X+Y$ также будет являться вещественным числом. В качестве множества $B$ возьмём $\Q \setminus A$ и проверим, что пара $(A,B)$ образует дедекиндово сечение.
\begin{nums}{-3}
\item
$\fa a\in A \, \fa b \in B \colon a < b$.\\
Рассмотрим любой элемент $a\in A$. Из определения множества $X+Y$ следует, что найдутся числа $x\in X, y\in Y$ такие, что $a = x+y$. Рассмотрим также любое число $a' < a$ и покажем, что $a' \in A$. Действительно, $a' = x + (y+a'-a)$. Число $y+a'-a$ меньше $y$ и поэтому лежит в $Y$.
\item
$A \ne \nothing$.\\
Очевидно.
\item
$A \ne \Q$.\\
Выберем $x_+ \in \Q\setminus X$ и $y_+ \in \Q\setminus Y$. Тогда для любых $x\in X$ и $y\in Y$ выполнено $x+y\le x_+ + y_+$. Поэтому $x_+ + y_+ \not\in A$.
\item
В множестве $A$ нет наибольшего элемента.\\
Предположим, что он есть: $a_0 = x_0 + y_0$. В множестве $X$ наибольшего элемента нет, поэтому существует $x_1 \in X$, $x_1 > x_0$. Значит, число $a_1 = x_1+y_0$ будет больше, чем $a_0$. Однако оно также принадлежит $A$. Противоречие.
\end{nums}

Теперь проверим, что сложение получилось \лк хорошее\пк. То есть, что выполняются все необходимые аксиомы:
\begin{items}{-3}
\item[(A1)]
$\fa X,Y\in \R:\quad X+Y=Y+X$.\\
Поскольку сложение рациональных чисел коммутативно, получаем\\ $X + Y = \{x+y \mid x\in X, y\in Y\} = \{y+x \mid x\in X, y\in Y\} = Y + X$.
\item[(A2)]
$\fa X,Y,Z\in \R:\quad(X+Y)+Z=X+(Y+Z)$.\\
Так же как и (A1) напрямую следует из ассоциативности сложения рациональных чисел.
\item[(A3)]
В~$\R$ существует такой элемент~$O$, что $\fa X\in \R:\quad X+O=X$.\\
Нулём в множестве вещественных чисел является множество $O = \{a \mid a < 0\}$. Распишем по определению: $X + O = \{x + y\mid x\in X, y < 0\}$. Ясно, что $X+O \subset X$. Проверим, что $X \subset X+O$. Выберем какой-нибудь элемент $x' \in X$. Поскольку в $X$ нет наибольшего элемента, то найдётся $x \in X$ такой, что $x > x'$. Тогда число $y = x' - x$ лежит в $O$, и при этом $x' = x+y$. Доказали.
\item[(A4)]
$\fa X\in \R \quad \exi Y\in \R:\quad X+Y=O$.\\
Положим $Y = \{a - x_+ \mid a < 0, x_+\in\Q\setminus X\}$. Вначале проверим, что $Y$\т это вещественное число:
\begin{nums}{-3}
\item
$\fa y\in Y \, \fa z \in \Q\setminus Y \colon y < z$.\\
Выберем произвольный элемент $y \in Y$ и покажем, что если $y' < y$, то $y' \in Y$. Пусть $y = a - x_+$. Тогда $y' = (a+y'-y) - x_+ \in Y$.
\item
$Y \ne \nothing$.\\
Очевидно.
\item
$Y \ne \Q$.\\
Зафиксируем любой элемент $x\in X$. Тогда для всех $x_+ \in \Q\setminus X$ выполнено $x_+ > x$. Значит, $a - x_+ < -x$. Таким образом множество $Y$ ограничено сверху и не может совпадать с $\Q$.
\item
В множестве $Y$ нет наибольшего элемента.\\
Рассмотрим произвольный элемент $y = a-x_+ \in Y$. Элемент $a/2 - x_+$ тоже лежит в $Y$ и при этом он больше, чем $y$.
\end{nums}
Покажем, что $X + Y = O$. По определению сложения: $X+Y = \{x + a - x_+ \mid x\in X, a<0, x_+ \in \Q\setminus X\}$. Заметим, что для любых $x \in X$ и $x_+ \in \Q\setminus X$ число $x - x_+$ отрицательно. Поэтому $X+Y \subset O$. Осталось показать, что $O \subset X+Y$. Выберем любое число $a' \in O$. В силу леммы~\ref{density1} найдётся пара $(x \in X, x_+ \in \Q\setminus X)$, такая что $x_+ - x < -a'$. Значит, $a' = x + (a'+x_+ - x) - x_+ \in X+Y$.
\end{items}


\опр
Пусть $X,Y$\т вещественные числа. Будем говорить, что $X \le Y$, если $X \subseteq Y$.
\копр

Проверим аксиомы:
\begin{items}{-3}
\item[(O1)]
$\fa X\in \R:\quad X\le X$.\\
Очевидно в силу $X\subseteq X$.
\item[(O2)]
Если $X,Y\in \R$, причём $X\le Y$ и $Y\le X$, то $X=Y$.\\
Из $X\subseteq Y$, $Y\subseteq X$ следует $X=Y$.
\item[(O3)]
Если  $X,Y,Z\in \R$, причём $X\le Y$ и $Y\le Z$, то $X\le Z$.\\
Из $X\subseteq Y$, $Y\subseteq Z$ следует $X\subseteq Z$.
\item[(O4)]
$\fa X,Y\in \R:\quad$ либо $X\le Y$, либо $Y\le X$.\\
Нам необходимо проверить, что если $(X, X_+)$ и $(Y, Y_+)$\т сечения, то $X\subseteq Y$ или $Y\subseteq X$. Предположим, что $X\not\subseteq Y$. Тогда найдётся $y_+ \in X \cap Y_+$. Но для всех $y\in Y$ выполнено $y < y_+$. Значит, $Y \subseteq X$.
\item[(AO)]
Если $X,Y,Z\in \R$ и $X\le Y$, то $X+Z\le Y+Z$.\\
По определению $X+Z = \{x+z \mid x\in X, z\in Z\}$. В силу $X\subseteq Y$ это подмножество множества $\{y+z \mid y\in Y, z\in Z\} = Y+Z$. Значит, $X+Z\le Y+Z$.
\end{items}

\опр
Пусть $X,Y$\т вещественные числа, причём $X \ge O$, $Y \ge O$. Тогда произведением $X \cdot Y$ будем называть множество $\{x \cdot y \mid x\in X, x \ge 0, y\in Y, y\ge 0\} \cup \{a \mid a<0\}$.\\
В прочих случаях определим умножение с помощью равенств $X\cdot Y = -((-X)\cdot Y) = -(X\cdot(-Y)) = (-X)\cdot(-Y)$.
\копр

\bigskip
Опять же вначале необходимо проверить, что такое определения умножения корректно, то есть что множество $A = X\cdot Y$ будет являться вещественным числом. В качестве множества $B$ возьмём $\Q \setminus A$ и проверим, что пара $(A,B)$ образует дедекиндово сечение. Ясно, что это достаточно сделать в случае $X \ge O$, $Y \ge O$.
\begin{nums}{-3}
\item
$\fa a\in A \, \fa b \in B \colon a < b$.\\
Рассмотрим любой элемент $a\in A$. Из определения множества $X\cdot Y$ следует, что либо $a<0$, либо найдутся неотрицательные числа $x\in X, y\in Y$ такие, что $a = xy$. Рассмотрим также любое число $a' < a$ и покажем, что $a' \in A$.\\
Пусть $a \le 0$. Тогда $a' < 0$ и $a' \in A$.\\
Пусть $a > 0$ и $a = xy$. Тогда $a' = (xa'/a)y$. При этом $0 \le xa'/a \le x$. Поэтому $xa'/a \in X$ и $a' \in X\cdot Y$.
\item
$A \ne \nothing$.\\
Очевидно.
\item
$A \ne \Q$.\\
Выберем $x_+ \in \Q\setminus X, x_+ > 0$ и $y_+ \in \Q\setminus Y, y_+ > 0$. Тогда для всех неотрицательных $x\in X,\,y\in Y$ выполнено $xy \le x_+ y_+$. Поэтому $x_+ y_+ \not\in A$.
\item
В множестве $A$ нет наибольшего элемента.\\
Если множество $\{x \cdot y \mid x\in X, x \ge 0, y\in Y, y\ge 0\}$ пусто, то $A = O$ и наибольшего элемента нет.
Если множество $\{x \cdot y \mid x\in X, x \ge 0, y\in Y, y\ge 0\}$ непусто, то любой его элемент больше любого элемента из $\{a \mid a<0\}$. Значит, если наибольший элемент есть, то он должен иметь вид $a_0 = x_0 y_0$, $x_0 \ge 0, y_0 \ge 0$. В множестве $X$ наибольшего элемента нет, поэтому существует $x_1 \in X$, $x_1 > x_0 \ge 0$. Аналогично существует $y_1 \in Y$, $y_1 > y_0 \ge 0$. Значит, число $a_1 = x_1y_1$ будет больше, чем $a_0$. Однако оно также принадлежит $A$. Противоречие.
\end{nums}

\замечание
\label{positivity}
Если $X\ge O$, $Y\ge O$, то $X\cdot Y \ge O$. Это сразу следует из определения.
\кзамечание

Теперь проверим, что выполнены аксиомы умножения. Мы докажем их в случае, когда все числа неотрицательны.
\begin{items}{-3}
\item[(M1)]
$\fa X,Y\in \R,\,X\ge O, Y\ge O:\quad X\cdot Y = Y\cdot X$.\\
$
\begin{aligned}
X \cdot Y &= \{x \cdot y \mid x\in X, x \ge 0, y\in Y, y\ge 0\} \cup \{a \mid a<0\}\\
&= \{y \cdot x \mid x\in X, x \ge 0, y\in Y, y\ge 0\} \cup \{a \mid a<0\}\\
&= Y \cdot X.\\
\end{aligned}
$
\item[(M2)]
$\fa X,Y,Z\in \R,\,X\ge O, Y\ge O, Z\ge O:\quad (X\cdot Y)\cdot Z=X\cdot (Y\cdot Z)$.\\
В силу замечания~\ref{positivity} числа $X\cdot Y$ и $Y\cdot Z$ также неотрицательны. Поэтому\\
$
\begin{aligned}
(X\cdot Y)\cdot Z &= \{(xy)z \mid x\in X, x \ge 0, y\in Y, y\ge 0\, z\in Z, z\ge 0\} \cup \{a \mid a<0\}\\
&= \{x(yz) \mid x\in X, x \ge 0, y\in Y, y\ge 0\, z\in Z, z\ge 0\} \cup \{a \mid a<0\}\\
&= X\cdot (Y\cdot Z).\\
\end{aligned}
$
\item[(M3)]
В~$\R\setminus\{O\}$ существует такой элемент $I$, что $\fa X\in \R,\,X\ge O:\quad X\cdot I=X$.\\
Единицей в множестве вещественных чисел является множество $I = \{a \mid a < 1\}$. Распишем по определению: $X \cdot I = \{xy\mid x\in X, x \ge 0, 0\le y < 1\}\cup \{a \mid a<0\}$. Ясно, что $X\cdot I \subset X$. Проверим, что $X \subset X\cdot I$. Выберем какой-нибудь элемент $x' \in X$.\\
Если $x' < 0$, то $x' \in \{a \mid a<0\} \subset X \cdot I$.\\
Если $x' = 0$, то возьмём $y = 1/2$. Значит, $x' = x'y \in X \cdot I$.\\
Пусть $x' > 0$. Поскольку в $X$ нет наибольшего элемента, найдётся $x \in X$, $x > x'$. Тогда число $y = x'/x < 1$ лежит в $I$ и при этом $x' = xy$. Доказали.
\item[(M4)]
$\fa X \in \R,\,X > O,\quad\exi Y\in \R:\quad X\cdot Y=I$.\\
Положим $Y = \{a / x_+ \mid a \in I, x_+\in\Q\setminus X\}$. Это определение корректно, так как из $X > O$ следует $0 \in X$ и $0 \not\in \Q\setminus X$.
Вначале проверим, что $Y$\т это вещественное число:
\begin{nums}{-3}
\item
$\fa y\in Y \, \fa z \in \Q\setminus Y \colon y < z$.\\
Выберем произвольный элемент $y \in Y$ и покажем, что если $y' < y$, то $y' \in Y$. Если $y' \le 0$, это очевидно. Пусть $y' > 0$ и $y = a/x_+$. Тогда $y' = (ay'/y)/x_+ \in Y$.
\item
$Y \ne \nothing$.\\
Очевидно.
\item
$Y \ne \Q$.\\
Зафиксируем любой элемент $x\in X$. Тогда для всех $x_+ \in \Q\setminus X$ выполнено $x_+ > x$. Значит, $a/x_+ < 1/x$. Таким образом множество $Y$ ограничено сверху и не может совпадать с $\Q$.
\item
В множестве $Y$ нет наибольшего элемента.\\
Рассмотрим произвольный элемент $y = a/x_+ \in Y$. Элемент $((1+a)/2)/x_+$ тоже лежит в $Y$ и при этом он больше, чем $y$.
\end{nums}
Теперь покажем, что $X \cdot Y = I$. По определению умножения:\\
$X \cdot Y = \{xa/x_+ \mid x\in X, x \ge 0, a\in I, x_+\in\Q\setminus X, a/x_+\ge 0\} \cup \{a \mid a<0\}$.\\
Заметим, что $x_+ > 0$, поэтому условие $a/x_+\ge 0$ равносильно $a \ge 0$.
Поскольку $x < x_+$, число $xa/x_+$ меньше $1$. Поэтому $X\cdot Y \subset I$. Осталось показать, что $I \subset X\cdot Y$. Выберем любое число $a' \in I$.\\
Если $a' < 0$, то $a' \in \{a \mid a<0\} \subset X\cdot Y$.\\
Пусть $a' = 0$. Заметим, что $0 \in X$ и $0 \in Y$. Поэтому $0 \in X\cdot Y$.\\
Пусть $1 > a' > 0$.
В силу леммы~\ref{density2} найдётся пара положительных чисел $(x \in X, x_+ \in \Q\setminus X)$, такая что $x_+/x < 1/a'$.
Значит, $a' = x(a'x_+/x)/x_+ \in X\cdot Y$.
\item[(AM)]
$\fa X,Y,Z\in \R,\,X\ge O, Y\ge O,Z\ge O:\quad X\cdot(Y+Z)=X\cdot Y + X\cdot Z$.\\
Из аксиомы (AO) следует, что $Y+Z \ge O$. Из определений следует, что\\
$
\begin{aligned}
X\cdot (Y + Z) &= \{x a \mid x\in X, x \ge 0, a \in Y+Z, a\ge 0\} \cup \{a \mid a<0\}\\
&= \{x(y+z) \mid x\in X, x \ge 0, y\in Y, z\in Z, y+z\ge 0\} \cup \{a \mid a<0\}\\
&= \{xy+xz \mid x\in X, x \ge 0, y\in Y, z\in Z, y+z\ge 0\} \cup \{a \mid a<0\}.\\
\end{aligned}
$

С другой стороны:\\
$
\begin{aligned}
X\cdot Y + X\cdot Z &= \{x y \mid x\in X, x \ge 0, y \in Y, y\ge 0\} \cup \{a \mid a<0\} \\
&+ \{x z \mid x\in X, x \ge 0, z \in Z, z\ge 0\} \cup \{b \mid b<0\}\\
&= \{xy+xz \mid x\in X, x \ge 0, y\in Y, y\ge 0, z\in Z, z\ge 0\} \\
&\cup \{xy+b \mid x\in X, x \ge 0, y\in Y, y\ge 0, b < 0\} \\
&\cup \{a+xz \mid x\in X, x \ge 0, z\in Z, z\ge 0, a < 0\} \\
&\cup \{a+b \mid a < 0, b < 0\}. \\
\end{aligned}
$

Осталось заметить, что это то же самое множество, что и выше (простое упражнение).
\end{items}

\bigskip
Мы проверили, что аксиомы умножения выполняются для неотрицательных вещественных чисел. Проверку того, что они выполняются для всех вещественных чисел вне зависимости от знака, можно провести перебором случаев не опираясь на явную конструкцию. В полном объёме мы это делать не будем; для примера рассмотрим какой-нибудь один случай.

\утверждение
Пусть $X,Y,Z\in \R$, $X < O, Y\ge O, Z\ge O$. Тогда $(X\cdot Y)\cdot Z=X\cdot (Y\cdot Z)$.
\кутверждение
\доказательство
С одной стороны $(X\cdot Y)\cdot Z = (-((-X)\cdot Y))\cdot Z$. С другой $X\cdot (Y\cdot Z) = -((-X)\cdot (Y\cdot Z)) = -(((-X)\cdot Y)\cdot Z)$. Проверим, что $(-((-X)\cdot Y))\cdot Z + ((-X)\cdot Y)\cdot Z = O$:\\
$
(-((-X)\cdot Y))\cdot Z + ((-X)\cdot Y)\cdot Z =
(-((-X)\cdot Y) + (-X)\cdot Y)\cdot Z = O\cdot Z = O.
$
\кдоказательство

\bigskip
Итак, будем считать, что аксиомы поля мы доказали. Из аксиом порядка осталась недоказанной аксиома (MO). Докажем её:
\begin{items}{-3}
\item[(MO)]
Если $X,Y,Z\in \R$, $O\le Z$ и $X\le Y$, то $X\cdot Z\le Y\cdot Z$.\\
В силу (AO) и (AM) утверждение равносильно неравенству $(Y-X)\cdot Z \ge O$. Оно следует из замечания~\ref{positivity}.
\end{items}

\bigskip
Ура! Мы доказали, что множество вещественных чисел\т это упорядоченное поле. Осталось проверить, что это поле полное. Мы докажем, что выполняется аксиома о точной верхней грани.

\утверждение
Всякое непустое ограниченное сверху подмножество $M \subset \R$ имеет в~$\R$ точную верхнюю грань.
\кутверждение
\доказательство
Рассмотрим множество $S = \bigcup\limits_{X\in M}X$. Проверим, что это вещественное число:
\begin{nums}{-3}
\item
$\fa a\in S \, \fa b \in \Q\setminus S \colon a < b$.\\
Очевидно.
\item
$S \ne \nothing$.\\
Следует из непустоты $M$.
\item
$S \ne \Q$.\\
Поскольку множество $M$ ограничено, найдётся такое $C \in \R$, что $\fa X \in M\colon X \subseteq C$. Значит, $S \subseteq C \ne \Q$.
\item
В множестве $S$ нет наибольшего элемента.\\
Предположим, что $s \in S$\т наибольший элемент. Найдём, такое число $X \in M$, что $s \in X$. Но в $X$ нет наибольшего элемента. Значит, найдётся $s' \in X\subset S,\, s' > s$. Противоречие.
\end{nums}

Совсем очевидно, что $S$\т верхняя грань множества $M$. Рассмотрим любую другую верхнюю грань $S'$. Поскольку это верхняя грань, $\fa X \in M\colon X \subseteq S'$. Отсюда $S = \bigcup\limits_{X\in M}X \subseteq S'$. Значит, $S = \sup M$.
\кдоказательство


\end{document}
