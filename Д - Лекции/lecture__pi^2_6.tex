% !TeX encoding = windows-1251
\documentclass[a4paper,12pt,fleqn]{article}
\usepackage[mag=1000]{newlistok}

\УвеличитьВысоту{.6cm}
\УвеличитьШирину{1.0cm}
%\renewcommand{\spacer}{\vspace{1mm minus 1mm}}
\pagestyle{plain}
\setlength\parindent{0pt}

\begin{document}
\vspace*{-1.7cm}
\АвторЛекции{Поршнев Е.}
\НазваниеЛекции{Ряд обратных квадратов}
\СоздатьЗаголовокЛекции

В этой лекции мы докажем следующую теорему:
\теорема[Сумма обратных квадратов]
\label{main}
\[
\suml{m=1}{\infty} \frac{1}{m^2} = \frac{\pi^2}{6}.
\]
\ктеорема

Но вначале обсудим несколько вспомогательных фактов. 

Рассмотрим многочлен $P(z) = \suml{i=0}{n}a_i z^i$. Предположим, что мы знаем все его корни: $z_1, \ldots, z_n$. Тогда $P(z) = a_n(z-z_1)\cdots(z-z_n)$. Если раскрыть в этом произведении скобки, получится выражение вида
\[
P(z) = a_n(z^n - \si_1 z^{n-1} + \si_2 z^{n-2} - \ldots).
\]
Чему равны величины $\si_i$? Чтобы получился моном $z^{n-k}$ необходимо, чтобы при раскрытии скобок мы выбрали $z$ в точности из $n-k$ сомножителей, а из остальных $k$\т какие-то корни $z_s$. Это можно сделать $C_n^k$ способами, и коэффициент при $z^{n-k}$ будет равен сумме всевозможных произведений $k$~различных корней:
\[
\si_k(z_1, \ldots, z_n) = \sums{1 \le s_1 < \ldots < s_k \le n} \hr{\prodl{i=1}{k} z_{s_i}}.
\]
В частности, $\si_1 = \suml{s = 1}{n}z_s$; $\si_2 = \sums{1 \le s_1 < s_2 \le n}z_{s_1}z_{s_2}$; $\si_n = \prodl{i=1}{n}z_i$.

\опр
Многочлены $\si_k$ называются \emph{элементарными симметрическими многочленами}.
\копр

Но вернёмся к $P(z)$. Из наших выкладок следует, что верна
\теорема[Виета]
\[
\si_k = (-1)^k\dfrac{a_{n-k}}{a_n}.
\]
\ктеорема

Отлично! Запомним её на будущее, и докажем ещё пару лемм.
\bigskip

\лемма
\label{sin}
При $x \in \hr{0, \frac{\pi}2}$ выполнено неравенство
\[
0 < \frac1{\sin^2 x} - \frac1{x^2} < 1.
\]
\клемма
\доказательство
Мы знаем, что $\sin x < x < \tg x$. Отсюда
\[
\frac1{\sin^2 x} > \frac1{x^2} > \frac{\cos^2 x}{\sin^2 x}
\]
\[
0 < \frac1{\sin^2 x} - \frac1{x^2} < \frac1{\sin^2 x} - \frac{\cos^2 x}{\sin^2 x} = 1.
\]
\кдоказательство

\лемма
\label{ctg}
При $n > 1$ выполнено равенство
\[
\suml{m=1}{n-1} \hr{-i \ctg\frac{\pi m}{n}}^2 = -\frac{(n-1)(n-2)}3.
\]
\клемма
\доказательство
Оказывается, в левой части равенства написана сумма квадратов корней уравнения
\eqn{
\label{ctg_equation}
(z+1)^n = (z-1)^n.
}
Проверим это. Уравнение~\ref{ctg_equation} можно переписать в виде $\hr{\dfrac{z+1}{z-1}}^n = 1$. Значит, $\dfrac{z+1}{z-1} = \sqrt[n]{1}$. Решая это уравнение, получаем $z = \dfrac{\sqrt[n]{1} + 1}{\sqrt[n]{1} - 1}$.

Запишем число $\sqrt[n]{1}$ в тригонометрической форме: $\cos\phi + i\sin\phi$. Тогда 
\[
\begin{aligned}
z &= \frac{\cos\phi + i\sin\phi + 1}{\cos\phi + i\sin\phi - 1} 
  = \frac{(\cos\phi + i\sin\phi + 1)(\cos\phi - i\sin\phi - 1)}{(\cos\phi - 1)^2 + \sin^2\phi}
  = \frac{\cos^2\phi - (i\sin\phi + 1)^2}{\cos^2\phi - 2\cos\phi + 1 + \sin^2\phi} \\
  &= \frac{\cos^2\phi + \sin^2\phi - 2i\sin\phi - 1}{2 - 2\cos\phi} 
  = \frac{-i\sin\phi}{1 - \cos\phi} 
  = -i\frac{2\sin\frac{\phi}{2}\cos\frac{\phi}{2}}{2\sin^2\frac{\phi}{2}}
  = -i \ctg\frac{\phi}{2}.
\end{aligned}
\]
Корнями $n$-ой степени из единицы являются числа $\cos\frac{2\pi m}{n} + i \sin\frac{2\pi m}{n}, \quad 0 \le m < n$, поэтому корнями уравнения~\ref{ctg_equation} будут числа $z = -i\ctg\frac{\pi m}{n}, \quad 0 < m < n$.

Таким образом, наша задача состоит в том, чтобы посчитать сумму квадратов корней уравнения~\ref{ctg_equation}. Для этого раскроем в нём скобки по биному Ньютона:
\[
\begin{aligned}
z^n + C_n^1 z^{n-1} + C_n^2 z^{n-2} + C_n^3 z^{n-3} + \ldots &= z^n - C_n^1 z^{n-1} + C_n^2 z^{n-2} - C_n^3 z^{n-3} + \ldots\\
&\Updownarrow\\
2 C_n^1 z^{n-1} + 2 C_n^3 z^{n-3} + \ldots &= 0\\
\end{aligned}
\]
Значит, по теореме Виета $\si_1(z_1,\ldots,z_{n-1}) = 0,\, \si_2(z_1,\ldots,z_{n-1}) = \dfrac{2C_n^3}{2C_n^1} = \dfrac{n(n-1)(n-2)}{6n} = \dfrac{(n-1)(n-2)}{6}$.

Выразим сумму квадратов через $\si_1$ и $\si_2$:
\[
\suml{s=1}{n-1}z_s^2 = \si_1^2 - 2\si_2 = -\frac{(n-1)(n-2)}{3}.
\]

Это равенство завершает доказательство леммы.
\кдоказательство

\следствие
При $n > 1$ выполнено равенство
\[
\suml{m=1}{n-1} \ctg^2\frac{\pi m}{n} = \frac{(n-1)(n-2)}3.
\]
\кследствие




\следствие
При $n > 1$ выполнено равенство
\[
\suml{m=1}{n-1} \frac1{\sin^2\frac{\pi m}{n}} = \frac{n^2 - 1}3.
\]
\клемма
\доказательство
\[
\suml{m=1}{n-1} \frac1{\sin^2\frac{\pi m}{n}} = \suml{m=1}{n-1} \hr{1 + \ctg^2\frac{\pi m}{n}} = n-1 + \frac{(n-1)(n-2)}3 = \frac{(3n-3) + (n^2-3n+2)}{3} = \frac{n^2 - 1}{3}.
\]
\кдоказательство

\newpage
Теперь можно перейти непосредственно к доказательству теоремы~\ref{main}.
\доказательство
Рассмотрим частичную сумму ряда $S_k = \suml{m=1}{k} \dfrac{1}{m^2}$ и покажем, что $\limk S_k = \dfrac{\pi^2}{6}$.

Обозначим $n = 2k+1$, тогда $k = \frac{n-1}2$. По лемме~\ref{sin} при $1 \le m \le k$ выполнено неравенство
\[
0 < \frac1{\sin^2 \frac{\pi m}{n}} - \frac1{\hr{\frac{\pi m}{n}}^2} < 1.
\]
Просуммируем эти неравенства по всем $m$ от 1 до $k$:
\[
\begin{aligned}
0 < \suml{m=1}{\frac{n-1}2}\frac1{\sin^2 \frac{\pi m}{n}} &- \suml{m=1}{\frac{n-1}2}\frac1{\hr{\frac{\pi m}{n}}^2} < \frac{n-1}2 \\
&\Updownarrow \\
0 < \frac12\suml{m=1}{n}\frac1{\sin^2 \frac{\pi m}{n}} &- \frac{n^2}{\pi^2}\suml{m=1}{\frac{n-1}2}\frac1{m^2} < \frac{n-1}2 \\
&\Updownarrow \\
0 < \frac{n^2 + 1}6 &- \frac{n^2}{\pi^2}S_k < \frac{n-1}2 \\
&\Updownarrow \\
0 < \frac{\pi^2(n^2 + 1)}{6n^2} &- S_k < \frac{\pi^2(n-1)}{2n^2} \\
&\Updownarrow \\
\frac{\pi^2(n^2 + 1)}{6n^2} - \frac{\pi^2(n-1)}{2n^2} &< S_k < \frac{\pi^2(n^2 + 1)}{6n^2} \\
\end{aligned}
\]
Применяя теорему о двух милиционерах, получаем искомое утверждение. Теорема доказана.
\кдоказательство


\end{document}
