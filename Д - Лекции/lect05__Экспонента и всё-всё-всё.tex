% !TeX encoding = windows-1251
\documentclass[a4paper,12pt]{article}
\usepackage{newlistok}

\УвеличитьВысоту{1.5cm}
\УвеличитьШирину{1.5cm}
\АвторЛекции{Шашков С.}
\НазваниеЛекции{Экспонента и всё-всё-всё}
\КолонитулЛекции
\begin{document}
\vspace*{-1.2cm}
\СоздатьЗаголовокЛекции

Основная цель этой лекции\т полностью разобраться с экспонентой, пределами, с нею связанными, её производной и обратной функцией.
Итак, поехали.

\раздел{Экспонента}

\опр
Числом $e$ называется предел последовательности $\limn\hr{1+\dfrac1n}^n$.
\копр

Кхм-кхм-кхм!
А почему этот предел вообще существует?
Докажем, что эта последовательность монотонна и ограничена, тогда по теореме Вейерштрассе этот предел будет существовать.
Нам понадобится странная
\лемма
\label{strange_lemma}
Пусть $a,b>0$ и~$n\in\N$. Тогда выполнено неравенство $\dfrac{a^{n+1}}{b^n}\geqslant(n+1)a-nb$.
\клемма
\доказательство
Доказать эту лемму можно по индукции,
но мы вместо этого воспользуемся неравенством Бернулли.
Поделим только перед этим обе стороны неравенства на $b$.
$$
\frac{a^{n+1}}{b^{n+1}} = \hr{\frac{a}{b}}^{n+1}=\bbr{1+\Br{\frac{a}{b}-1}}^{n+1}\ge1 + (n+1)\cdot\Br{\frac{a}{b}-1} = (n+1)\cdot\frac{a}{b} - n = \frac{(n+1)a-nb}{b}
\vspace{-4mm}
$$
\кдоказательство


\лемма
\label{monoton1}
Последовательность $e_n = \hr{1+\frac{1}n}^n$ монотонно возрастает.
\клемма
\доказательство
Достаточно доказать, что для любого натурального $n$ отношение $e_{n+1}/e_n$ больше либо равно~1.
Обозначим $\hr{1+\frac{1}{n+1}}$ через $a$, а $\hr{1+\frac{1}n}$ --- через~$b$.
Заметим, что эти числа положительны.
Тогда по лемме \ref{strange_lemma}:
$$
\frac{e_{n+1}}{e_n}
=
\frac{\hr{1+\frac{1}{n+1}}^{n+1}}{\hr{1+\frac{1}n}^n}
=
\frac{a^{n+1}}{b^n}\geqslant(n+1)a-nb
=
(n+1)\cdot\hr{1+\tfrac{1}{n+1}} - n\cdot\hr{1+\tfrac{1}n}
=
1.
\vspace{-4mm}
$$
\кдоказательство
\лемма
\label{monoton2}
Последовательность $E_n = \hr{1+\frac{1}n}^{n+1}$ монотонно убывает.
\клемма
\доказательство
Опять же, достаточно доказать, что для любого натурального $n$ отношение $E_{n+1}/E_n$ меньше либо равно 1.
Так как мы собираемся использовать всё ту же лемму, то сделаем следующий трюк: заменим числа в числителе и знаменателе на их обратные (то есть заменим $x$ на $\frac1x)$.
Обозначим $\hr{\frac{n+1}{n+2}}$ через $a$ и $\hr{\frac{n}{n+1}}$ через $b$.
Далее
$$
\dfrac{\frac{1}{E_{n+1}}}{\frac{1}{E_n}}
=
\frac{\hr{\frac{n+1}{n+2}}^{n+2}}{\hr{\frac{n}{n+1}}^{n+1}}
=
\frac{a^{n+2}}{b^{(n+1)}}
\ge
\br{n+2}\cdot a - (n+1)\cdot b
=
(n+2)\cdot \hr{\tfrac{n+1}{n+2}} - (n+1)\cdot \hr{\tfrac{n}{n+1}}
=
1.
\vspace{-4mm}
$$
\кдоказательство


\лемма
\label{eqeq}
Последовательности $(e_n)$ и $(E_n)$ имеют пределы и $\limn e_n = \limn E_n$.
\клемма
\доказательство
Заметим, что $E_n/e_n = (1+\frac{1}{n})>1$, поэтому $e_n<E_n$ для любого $n\in\N$.
Следовательно, последовательности $e_n$ и $E_n$ монотонны и ограничены,
поэтому по теореме Вейерштрассе имеют пределы.
Далее
$$
\dfrac{\limn E_n}{\limn e_n} = \limn\frac{E_n}{e_n} = \limn \hr{1+\frac{1}n} = 1.
\vspace{-4mm}
$$
\кдоказательство


\bigskip

Итак, вернёмся к числу $e$.
Чудесным образом, предел последовательности $\hr{1+\dfrac1n}^n$ существует,
и именно он зовётся числом $e$.


Кстати, последовательность $\hr{1+\frac1n}^n$ сходится к $e$ весьма неспешно.
Скорость этой сходимости мы можем оценить следующим образом.
Мы уже знаем, что $e_n < e < E_n$.
Следовательно,
$$
|e - e_n| < E_n - e_n = \hr{1+\frac1n}^n\cdot\frac{1}{n} < \frac{e}{n}.
$$
То есть чтобы гарантировать точность $10^{-6}$, потребуется взять $n>e\cdot 10^6$.



Оказывается, для любого действительного $x$ существует предел последовательности
$\hr{1+\dfrac{x}{n}}^n$, равный~$e^x$.
Несложно доказать аналоги лемм \ref{monoton1}, \ref{monoton2} и \ref{eqeq} для последовательностей
$e_n = (1+\frac{x}{n})^n$ и $E_n = (1+\frac{x}{n})^{n+x}$,
однако это лишь докажет, что предел существует.
Для того, чтобы всё-таки разобраться с этим пределом, потребуется ещё несколько шагов.

\лемма
$\limn\hr{1-\dfrac1n}^n = \dfrac{1}{e}$.
\клемма
\доказательство
$$
\limn\hr{1-\dfrac1n}^n
=
\limn\hr{\dfrac{n-1}{n}}^n
=
\dfrac{1}{\limn\hr{\dfrac{n}{n-1}}^n }
=
\dfrac{1}{\limn\bbr{\hr{1 + \frac{1}{n-1}}^{n-1} \cdot \hr{1 + \frac{1}{n-1}}}}
=
\dfrac{1}{e}
\vspace{-4mm}
$$
\кдоказательство


\лемма
\label{lem_besk}
Для любой бесконечно большой последовательности $t_n$ существует предел
$\limn\hr{1+\dfrac{1}{t_n}}^{t_n}=e$.
\клемма
\доказательство
Предположим для начала, что все числа $t_n$ целые.
Будем действовать по определению предела последовательности.
Зафиксируем число $\ep>0$.
Мы знаем, что 
$\limn(1+\frac1n)^n = \limn(1-\frac1n)^{-n} = e$.
Найдём такое $N_1$, что при $n > N_1$ выполнено неравенство $\bm{(1+\frac1n)^n - e}<\ep$,
а также такое $N_2$, что при $n > N_2$ выполнено неравенство $\bm{(1-\frac1n)^{-n} - e}<\ep$.
Так как последовательность $t_n$ бесконечно большая, то найдётся такое число $N$,
что $|t_n| > \max(N_1, N_2)$ при $n>N$.
Но тогда $\bm{(1+\frac1{t_n})^{t_n} - e}<\ep$ при $n>N$, откуда
$\limn\hr{1-\frac{1}{t_n}}^{t_n}=e$.

Теперь заметим%
\footnote{Напомним: $\hfl{t}$ --- $t$, округлённое вниз, аналогично $\hce{t}$ --- $t$, округлённое вверх}%
, что для любого $t\ne0$%
$$
\hr{1+\dfrac{1}{\hce{t}}}^{\hfl{t}} \le \hr{1+\dfrac{1}{t}}^{t} \le \hr{1+\dfrac{1}{\hfl{t}}}^{\hce{t}}
\text{ при $t>1$, }
\hr{1+\dfrac{1}{\hfl{t}}}^{\hce{t}} \le \hr{1+\dfrac{1}{t}}^{t} \le \hr{1+\dfrac{1}{\hce{t}}}^{\hfl{t}}
\text{ при $t<-1$.}
$$
Поэтому общий случай сводится к случаю целочисленных последовательностей при помощи теоремы о двух милиционерах.
\кдоказательство

\теорема
Для любого действительного $x$ существует предел $\limn\hr{1+\dfrac{x}{n}}^n = e^x$.
\ктеорема
\доказательство
Если $x=0$, то всё очевидно.
Иначе рассмотрим бесконечно большую последовательность $t_n = \frac{n}{x}$.
По лемме \ref{lem_besk} предел $\limn\hr{1+\frac{1}{t_n}}^{t_n}=e$.
Следовательно,
$$
\limn\hr{1+\dfrac{x}{n}}^n = \limn\hr{1+\dfrac{1}{t_n}}^{t_n x} = \bbbr{\limn\hr{1+\dfrac{1}{t_n}}^{t_n}}^x = e^x.
\vspace{-4mm}
$$
\кдоказательство


\newpage
\раздел{Ряд для $e^x$}

\теорема
Для любого действительного числа $x$ существует предел
\vspace{-4mm}
$$\limn \sumizn \dfrac{x^i}{i!}
=
\limn \bbr{ 1 + \dfrac{x^1}{1!} + \ldots +  \dfrac{x^n}{n!}}
=
e^x.
\vspace{-4mm}
$$
\ктеорема
\доказательство
Обозначим число $\hr{1+\frac{x}{n}}^n$ через $e_n$, а сумму $1 + \frac{x^1}{1!} + \ldots +  \frac{x^n}{n!}$ --- через $s_n$.
Для начала раскроем скобки в выражении $\hr{1+\frac{x}{n}}^n$ по биному Ньютона:
$$
\hr{1+\dfrac{x}{n}}^n
=
1 + C_n^1 \dfrac{x^1}{n} + C_n^2 \dfrac{x^2}{n^2} + \ldots + C_n^n \dfrac{x^n}{n^n}
=
1 + \dfrac{x^1}{1!}\cdot\dfrac{n}{n} + \dfrac{x^2}{2!}\cdot\dfrac{n\cdot(n-1)}{n\cdot n} +
\ldots + \dfrac{x^n}{n!}\cdot\dfrac{n\cdot(n-1)\sd1}{n\cdot n\sd n}.
$$
Обозначим множитель перед $\dfrac{x^i}{i!}$,
равный $\frac{n\cdot(n-1)\sd(n-i+1)}{n^i}$, через $\al_i(n)$.
Ясно, что $\limn \al_i(n) = 1$ для всех~$i$.
Хочется сказать, что для каждый множитель $\al_i(n)$ стремится к 1, поэтому получившаяся сумма стремится к $\sumizn \dfrac{x^i}{i!}$.
Однако здесь кроется опасность: хотя каждая \лк ошибка\пк стремится к 0, их общее число стремится к бесконечности.

Разберём сначала случай $x>0$.
Зафиксируем натуральное число $N$, и рассмотрим произвольное $n>N$.
Заметим, что каждое из чисел $\al_i(n)$ меньше либо равно 1, поэтому $e_N\le s_N$.
\лк Откусим\пк от $e_n$ первые $N$ слагаемых и получим:
\vspace{-5mm}
$$
1 + \dfrac{x^1}{1!}\al_1(n) + \ldots +  \dfrac{x^N}{N!}\al_N(n)
\le
e_n.
$$
При $n\to\infty$ левая часть неравенства стремится к $s_N$, а правая --- к $e^x$.
Отсюда заключаем, что $s_N \le  e^x$.
Таким образом, $e_N \le  s_N \le  e^x$ для всех натуральных $N$,
откуда по теореме о двух милиционерах $\limn s_n = e^x$.
\упражнение
Разобраться со случаем $x<0$.
\купражнение
\кдоказательство

Оценим, насколько быстро ряд $\sumizn \dfrac{1}{i!}$ стремится к $e$:
\vspace{-4mm}
\begin{multline*}
\hm{\sumizn \dfrac{1}{i!} - e}
=
\frac{1}{(n+1)!} + \frac{1}{(n+2)!} + \ldots
=
\frac{1}{(n+1)!} \cdot \bbr{1 + \frac{1}{n+2} + \frac{1}{(n+2)(n+3)} + \ldots}
\le \\ \le
\frac{1}{(n+1)!} \cdot \bbr{1 + \frac{1}{n+2} + \hr{\frac{1}{n+2}}^2 + \ldots}
=
\frac{1}{(n+1)!} \cdot \frac{1}{1-\frac{1}{n+2}}
=
\frac{1}{n!} \cdot \frac{n+2}{(n+1)^2}
\le
\frac{1}{n!n}
\end{multline*}
То есть чтобы гарантировать точность $10^{-6}$ достаточно взять $n>8$.
Напомним, что последовательность $(1+\frac1n)^n$ давала точность порядка $\frac{e}{n}$, и $n=10^6$ было недостаточно.

\smallskip
Ряд для экспоненты насколько важен, что приведём ещё одно независимое доказательство его сходимости.
Перед этим только заметим, что понятие предела последовательности один в один можно применить к комплексным последовательностям.
Теперь докажем, что для любого комплексного числа~$z$ существует предел
\vspace*{-3mm}
$$\limn \sumizn \dfrac{z^i}{i!}
=
\limn \bbr{ 1 + \dfrac{z^1}{1!} + \ldots +  \dfrac{z^n}{n!}}.
\vspace{-4mm}
$$


\доказательство
Воспользуемся критерием Коши (который отлично работает и для комплексных чисел).
Обозначим $|z|$ через $t$, и рассмотрим пару натуральных чисел $m<n$, больших $t$.
Тогда
\vspace{-4mm}
\begin{multline*}
\hm{\sum_{i=m}^n \dfrac{z^i}{i!}}
\le
\limn \sum_{i=m}^n \dfrac{t^i}{i!}
=
\frac{t^m}{m!}\cdot\hr{1 + \dfrac{t}{(m+1)} + \ldots + \dfrac{t^{n-m}}{(m+1)\cdot\ldots\cdot n}}
\le\\\le
\frac{t^m}{m!}\cdot\hr{1 + \dfrac{t}{(m+1)} + \ldots + \hr{\dfrac{t}{(m+1)}}^{n-m} + \ldots}
\le
\frac{t^m}{m!}\cdot\dfrac{1}{1-\frac{t}{(m+1)}}.
\end{multline*}
Теперь зафиксируем число $\ep>0$.
Последовательность $\dfrac{t^m}{m!}\cdot\dfrac{1}{1-\frac{t}{(m+1)}}$ --- бесконечно малая, поэтому найдётся такое число
$N$, что $\hm{\dfrac{t^m}{m!}\cdot\dfrac{1}{1-\frac{t}{(m+1)}}}<\ep$ при $m>N$.
Но тогда при $m,n>N$ выполнено неравенство
$\hm{\sum_{i=m}^n \frac{z^i}{i!}} < \ep$, и по критерию Коши последовательность имеет предел.
\кдоказательство



\раздел{Второй замечательный предел и производная экспоненты}
Напомним, что из лемм \ref{monoton1} и \ref{monoton2} следует, что
$\hr{1+\frac{1}{n}}^n \le e \le \hr{1+\frac{1}{n-1}}^n$.
Извлечём корень степени~$n$ из этого неравенства, вычтем из всех частей единицу,
умножим на $n$ и применим теорему о двух милиционерах:
\begin{multline*}
\hr{1+\frac{1}{n}}^n \le e \le \hr{1+\frac{1}{n-1}}^n
\qquad\Rightarrow\qquad
1+\frac{1}{n} \le e^{\frac{1}{n}} \le 1+\frac{1}{n-1}
\qquad\Rightarrow\\\Rightarrow\qquad
\frac{1}{n} \le e^{\frac{1}{n}} - 1\le \frac{1}{n-1}
\qquad\Rightarrow\qquad
1 \le n(e^{\frac{1}{n}} - 1) \le \frac{n}{n-1}
\qquad\Rightarrow\qquad
\limn n(e^{\frac{1}{n}} - 1) = 1.
\end{multline*}
Если же неравенство $\hr{1+\frac{1}{n}}^n \le e \le \hr{1+\frac{1}{n-1}}^n$ \лк перевернуть\пк да дробь преобразовать,
то получится неравенство $\hr{1-\frac{1}{n+1}}^n \ge e^{-1} \ge \hr{1-\frac{1}{n}}^n$.
После повторения предыдущей цепочки (корни, единицы, милиционеры), получим предел $\limn -n(e^{\frac{1}{-n}} - 1) = 1$.
\теорема
$\limt{0}\dfrac{e^t-1}{t} = 1$.
\ктеорема
\доказательство
Будем действовать по определению предела по Коши.
Зафиксируем произвольное число $\ep>0$.
Воспользуемся тем, что 
$$
\limn (n+1)(e^{\frac{1}{n}} - 1)
=
\limn n(e^{\frac{1}{n+1}} - 1)
=
\limn -(n+1)(e^{\frac{1}{-n}} - 1)
=
\limn -n(e^{\frac{1}{-(n+1)}} - 1) 
= 1.
$$
Найдём такое число $N$, что при $n>N$ члены каждой из последовательностей отличаются
от $1$ менее, чем на $\ep$.
Теперь возьмём в качестве $\delta$ число $\frac{1}{N}$.
Тогда если $t\in U_\de(0)$,
то либо $\frac{1}{t}>N$, либо $\frac{1}{t}<-N$.
Сопло на данном этапе мы сделаем из дерева, чтобы лучше горело.
Вне зависимости от знака $t$, 
число $\hm{\dfrac{e^t-1}{t} - 1}$ находится между числами
$\hm{\dfrac{e^{\hfl{t}}-1}{\hce{t}}-1}$ и $\hm{\dfrac{e^{\hce{t}}-1}{\hfl{t}}-1}$,
каждое из которых по предположению меньше~$\ep$.
\кдоказательство

\следствие
$(e^x)' = e^x$ и $(\ln x)' = \frac{1}{x}$.
\кследствие
\доказательство
$$
(e^x)' = \limt{0}\frac{e^{x+t} - e^x}{t} = e^x \cdot \limt{0}\frac{e^t - 1}{t} = e^x;
$$
$$
(\ln x)' = \frac{(\ln x)' \cdot x}{x} = \frac{(\ln x)' \cdot e^{\ln x}}{x} = \frac{(e^{\ln x})'}{x} = \frac{x'}{x} = \frac{1}{x}.
$$
\кдоказательство

\задача[Ещё один замечательный предел]
Докажите, что $\limx{0}\dfrac{\ln(1+x)}{x} = 1$.
\кзадача


\end{document}



