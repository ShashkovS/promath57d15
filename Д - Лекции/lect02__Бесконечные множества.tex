% !TeX encoding = windows-1251
\documentclass[a4paper,12pt]{article}

\usepackage{newlistok}

\pagestyle{plain}
\mathsurround=1pt
\tolerance=500
\УвеличитьШирину{1truecm}
\УвеличитьВысоту{.7truecm}

\vspace*{-.5mm}

\НазваниеЛекции{Лекция о множествах 2}
%\АвторЛекции{С.\;Шашков}

\begin{document}
%\vspace*{-1.5cm}
%\СоздатьЗаголовокЛекции

\vspace*{-1.5cm}
\section*{Мощность множества}


Напомним, что множество  полностью определяется тем, какие элементы в него входят.
Таким образом, множества $\varnothing$, $\{\varnothing\}$, $\{\{\varnothing\}\}$, $\{1\}$ --- различны.
Иногда бывает необходимо понять, \лк сколько\пк\ элементов в множестве. Если множество конечно, проблем не возникает, а если бесконечно?

\упражнение
Придумайте способ \лк сравнивать\пк\ множества.
\купражнение

\определение
Два множества называются \выд равномощными, если между ними можно установить взаимно однозначное соответствие. Обозначение: $|A|=|B|$, $\#A=\#B$.
\копределение

\пример
Конечные множества равномощны, если в них одинаковое количество элементов;
\кпример

\пример
Множество натуральных чисел равномощно множеству чётных натуральных чисел, отображение $f\colon \N \to \N$, $n \mapsto 2n$ осуществляет \emph{биекцию} --- взаимно однозначное соответствие;
\кпример

\пример
Отрезки $[0,1]$ и $[0,2]$ равномощны, для них отображение $x \mapsto 2x$ осуществляет искомое соответствие.
\кпример


\упражнение
Докажите, что интервал $(0,1)$ и луч $(1,+\infty)$ равномощны.
\купражнение


%Научившись \лк сравнивать\пк\ некоторые множества, мы задали некоторое \emph{отношение
%\footnote{Фактически это означает, что про некоторые множества мы можем сказать, как они друг другу относятся. Например, некоторые множества равномощны, а некоторые --- нет.}
%}, отношение равномощности. Изучим его немного.

\упражнение
Докажите, что для любых множеств $A$, $B$, и $C$ верно:

({\it{i\/}}) $|A| = |A|$. Это свойство называется \emph{рефлексивностью};

({\it{ii\/}}) Если $|A| = |B|$, то $|B|=|A|$. Это свойство называется \emph{симметричностью};

({\it{iii\/}}) Если $|A| = |B|$, $|B|=|C|$, то $|A|=|C|$. Это свойство называется \emph{транзитивностью}.
\купражнение

%Вообще, отношения, удовлетворяющие свойствам $({\it i})$--$({\it iii})$ называются \emph{отношениями эквивалентности}. Но не будем углубляться в эту тему, а
\hbox{Изучим, какими бывают равномощные множества. Рассмотрим перед этим несколько примеров.}


\упражнение
Множество бесконечных последовательностей из нулей и единиц равномощно~$2^\N$.
%Действительно, по последовательности будем строить подмножество натуральных чисел следующим образом: число $i$ лежит в подмножестве, тогда и только тогда, когда на $i$-ом месте последовательности стоит $1$;
\купражнение

\пример
Множество бесконечных последовательностей из нулей и единиц равномощно множеству бесконечных последовательностей из нулей, единиц, двоек и троек. Здесь $0$ будем заменять на $00$, $1$ --- на~$01$, $2$ --- на $10$, $3$ --- на $11$. Несложно показать, что это --- биекция.
\кпример

\определение
Все множества, равномощные множеству натуральных чисел называются \выд счётными.
\копределение

Элементы таких множеств можно занумеровать натуральными числами. Например, множество целых чисел, являющихся полными квадратами, является счётным: ведь числу $n^2$ можно дать номер $n$. Счётные множества являются в некотором смысле \лк самыми маленькими\пк\ бесконечными множествами:

\упражнение
Докажите, что любое бесконечное множество имеет счётное подмножество.
\купражнение

\note{Наверное, Вы начинали решение предыдущего упражнения со слов \лк Выберем какой-нибудь элемент $a$\пк. Тут есть некоторая тонкость, ибо априори у нас нет никакого правила для выбора этого элемента. На самом деле, в таких случаях считают верной \emph{аксиому выбора}.
\аксиома[Аксиома выбора]
Для каждого семейства $A$ непустых непересекающихся множеств существует множество $B$, имеющее один и только один общий элемент с каждым из множеств $X$, принадлежащих $A$.
\каксиома
Простым следствием является то, что по любому непустому множеству $A$ можно построить множество $\{a\}$, где $a\in A$, то есть выбрать элемент из~$A$.

В середине XX века Курт Гёдель доказал, что аксиому выбора нельзя опровергнуть, пользуясь остальными аксиомами теории множеств.
Еще через несколько лет математик Пол Дж.\,Коэн доказал, что её нельзя вывести из остальных аксиом.
}

\упражнение[Теорема Кантора]
Докажите, что множество бесконечных последовательностей из нулей и единиц несчётно.
\купражнение
\note{Доказательство этого факта Георг Кантор опубликовал в 1874\,году.}

\определение
Про множества, равномощные множеству бесконечных последовательностей из нулей и единиц говорят, что они имеют мощность \выд континуум.
\копределение

\упражнение
Приведите как можно больше примеров континуальных множеств.
\купражнение

\упражнение
Докажите, что отрезок имеет мощность континуума. Докажите, что отрезок на прямой равномощен квадрату с внутренностью на плоскости.
(Вообще-то у нас нет строгого определения отрезка, поэтому со строгим доказательством могут возникнуть сложности.)
\купражнение
\note{Этот удивительный факт в 1877\,году впервые обнаружил математик Георг Кантор.}

\newpage


Теперь попробуем сравнить различные множества $A$ и $B$. Рассмотрим несколько возможностей:
а) $A$ равномощно $B$, то есть можно установить биекцию между элементами $A$ и всеми элементами $B$;
б) Можно установить биекцию между множеством $A$ и некоторым подмножеством $B'\subset B$, то есть $A$ можно вложить в множество $B$.
в) Можно установить биекцию между множеством $B$ и некоторым подмножеством $A'\subset A$, то есть $B$ можно вложить в множество $A$.
г) Нельзя установить никакую биекцию из пунктов б) и в).

С ситуацией пункта а) мы уже разобрались.

\упражнение
Докажите, что случай г) на самом деле невозможен.
\купражнение

Теперь разберёмся с случаями б) и в). Эти случаи симметричны, поэтому достаточно рассмотреть пункт б). Для этого разберём несколько примеров.


\пример
$f\colon \N \to \N$, $n \mapsto n^2$ устанавливает биекцию между множеством $A=\N$ натуральных чисел и
подмножеством $X=\{n^2 \mid n\in\N\}$ множества $B=\N$.
\кпример


\пример
$f\colon \N \to \R$, $n \mapsto n^2$ устанавливает биекцию между множеством $A=\N$ натуральных чисел и
подмножеством $X=\{n^2 \mid n\in\N\}$ множества $B=\R$.
\кпример

Чем отличаются примеры $5^\circ$ и $6^\circ$? В примере $5^\circ$ можно установить взаимно однозначное соответствие между $B$ и каким-нибудь подмножеством $A'\subset A$, а вот в примере $5^\circ$ --- нельзя. Отсюда

\определение
Множество $B$ \выд мощнее множества $A$, если $A$ можно вложить\footnote{Напоминаем, что \выд вложением называется отображение, не склеивающее элементы, то есть $f(a)=f(b)\Leftrightarrow a=b$.} в $B$, но при этом $B$ нельзя вложить в $A$.
\копределение

\упражнение[Теорема Кантора--Бернштейна]
Если множество $A$ равномощно некоторому подмножеству множества $B$, а $B$ равномощно некоторому подмножеству множества $A$, то множества $A$ и $B$ равномощны.
\купражнение

\note{
Эта теорема была сформулирована Кантором в 1883\,году. При этом не известно, была ли она им доказана. Так или иначе, она была доказана Шрёдером в 1896\,году и Берншейном в 1897\,году.
}

Дадим некоторую подсказку: пусть указанные биекции устанавливают отображения $f\colon A\hookrightarrow B$ и $g\colon B\hookrightarrow A$. Возьмите некоторый элемент $a\in A$ (опять аксиома выбора) и рассмотрите множество точек \ldots, $f^{-1}(g^{-1}(a))$, $g^{-1}(a)$, $a$, $f(a)$, $g(f(a))$, $f(g(f(a)))$, \ldots (если соответствующие прообразы  существуют). Изучите разные возможности таких цепочек и постройте биекцию между множествами $A$ и $B$.

С помощью теоремы Кантора-Бернштейна очень удобно доказывать равномощность различных множеств. Например, докажем, что множество $A$ бесконечных последовательностей из двоек и единиц равномощно множеству $B$ бесконечных последовательностей из натуральных чисел. Вложим $A$ в $B$ тождественно, то есть последовательность переходит в себя же. Вложим $B$ в $A$ следующим образом: будем заменять число $n$ на кусок $\underbrace{11\ldots1}_n\underbrace{22\ldots2}_n$. Несложно показать, что указанные отображения действительно являются вложениями. Значит, множества $A$ и $B$ равномощны.

Мы уже знаем, что существуют конечные, счётные, континуальные множества. А какие ещё бывают множества?

\note{В 1878\,году Кантором была сформулирована \emph{континуум-гипотеза\/}: всякое подмножество отрезка либо континуально, либо счётно, либо конечно. Ситуация с континуум-гипотезой и аксиомой выбора совершенно одинакова: в зависимости от того, принята ли континуум-гипотеза или нет, получаются разные теории множеств; континуум-гипотезу нельзя ни доказать, ни опровергнуть, пользуясь остальными аксиомами.
}

Пусть теперь нам дано непустое множество $A$. Построим множество, более мощное, чем $A$. Оказывается, в этой роли может выступить множество $2^A$.

\упражнение
Пусть $f\colon A \hookrightarrow 2^A$ --- инъективное отображение множества в множество его подмножеств.
Постройте пример подмножества, которое не лежит в образе отображения $f$, то есть такое подмножество $B\subset A$, что не существует элемента $b\in A$, что $f(b) = B$.
\купражнение




\end{document}
