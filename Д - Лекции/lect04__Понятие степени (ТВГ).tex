% !TeX encoding = windows-1251
\documentclass[a4paper,12pt,fleqn]{article}
\usepackage[mag=980]{newlistok}

\УвеличитьВысоту{2.3cm}
\УвеличитьШирину{1.5cm}
\renewcommand{\spacer}{\vspace{1mm minus 1mm}}

\begin{document}
\АвторЛекции{Поршнев Е.}
\НазваниеЛекции{Понятие степени}
\СоздатьЗаголовокЛекции

Основная цель этой лекции\т придать смысл выражению $a^b$ (\emph{$a$ в степени $b$}). С самого начала сформулируем те свойства степени, к которым мы все привыкли:
\begin{nums}{-3}
\item
\label{additivity}
$a^b \cdot a^c = a^{b+c}$.
\item
\label{composition}
$(a^b)^c = a^{bc}$.
\item
\label{multiplicativity}
$a^c \cdot b^c = (ab)^c$.
\item
\label{base_monotoneness}
Пусть $a > b > 0$. Если $c > 0$, то $a^c > b^c$; если $c < 0$, то $a^c < b^c$.
\item
\label{exponent_monotoneness}
Пусть $b > c$. Если $a > 1$, то $a^b > a^c$; если $1 > a > 0$, то $a^b < a^c$.
\end{nums}

Напомним определение степени с натуральным показателем.
\опр
\label{natural}
Пусть $a\in\R$, $b\in\N$. Тогда по определению $a^b = \underbrace{a\cdot\ldots\cdot a}_{b\text{ раз}}$.
\копр

\лемма
Свойства 1--5 выполняются для степени с натуральным показателем.
\клемма
\доказательство
Доказательство оставляется читателю в качестве упражнения.
\кдоказательство

Несложно расширить определение степени на случай $b \in \Z$ (правда при этом придётся ограничить себя случаем $a \ne 0$).

\опр
\label{integer}
Пусть $a\in\R\setminus\{0\}$, $b\in\Z$. Тогда по определению
\[
a^b = \bcase{
a^b,&\quad b\in\N;\\
1,&\quad b=0;\\
1/a^{-b},&\quad -b\in\N.\\
}
\]
\копр

\лемма
Свойства 1--5 выполняются для степени с целым показателем.
\клемма
\доказательство
Свойства степени с целым показателем обычно выводят из свойств степени с натуральным показателем. Детальное доказательство оставляется читателю в качестве упражнения.
\кдоказательство

Перед тем, как определять степень с рациональным показателем, введём понятие корня.

\опр
Пусть $n\in\N$. \emph{Арифметическим корнем $n$-ой степени из неотрицательного числа~$a$} называется такое неотрицательное число~$x$, что $x^n=a$.\\ Обозначение:~$x=\sqrt[n]a$.
\копр

\теорема
Для любого неотрицательного вещественного числа $a$ и для любого натурального числа $n$ существует корень $x=\sqrt[n]a$.
\ктеорема

\доказательство
Случай $a=0$ тривиален, поэтому будем считать, что $a > 0$.

Рассмотрим множество $M = \{t \mid t^n \le a,\, t\ge 0\}$. Это множество очевидно не пусто ($0 \in M$) и ограничено сверху (числом $\max(a,1)$). Поэтому из аксиомы о точной верхней грани следует, что существует $x = \sup M$. Покажем, что $x^n = a$.

Предположим, что $x^n = a + \ep$, где $\ep > 0$. Рассмотрим маленькое число $\de \in (0, x)$ и $y = x - \de$. Оценим $y^n$.
\[
|x^n - y^n| = |x-y|\cdot|x^{n-1} + x^{n-2}y + \ldots + xy^{n-2} + y^{n-1}| < \de \cdot (nx^n)
\]
Последнее неравенство обусловлено тем, что в силу $y < x$ каждое из слагаемых меньше, чем $x^n$, а всего слагаемых $n$.
В частности, если выбрать $\de < \dfrac{\ep}{nx^n}$, то $|x^n - y^n| < \ep$ и тем самым $y^n > a$. Значит, $y$\т верхняя грань множества $M$, что противоречит выбору $x$.

Предположим, что $x^n = a - \ep$, где $\ep > 0$. Рассмотрим маленькое число $\de \in (0, x)$ и $y = x + \de$. Оценим $y^n$.
\[
|x^n - y^n| = |x-y|\cdot|x^{n-1} + x^{n-2}y + \ldots + xy^{n-2} + y^{n-1}| < \de \cdot (n(2x)^n)
\]
Последнее неравенство обусловлено тем, что $y < 2x$.
Если выбрать $\de < \dfrac{\ep}{n(2x)^n}$, то $|x^n - y^n| < \ep$ и тем самым $y^n < a$. Значит, $y\in M$, что противоречит выбору $x$.

Теорема доказана.
\кдоказательство

\опр
\label{rational}
Пусть $a\in\R$, $a > 0$, $b = \frac{m}{n} \in \Q$. Тогда по определению $a^b = \sqrt[n]{a^m}$.
\копр

\утверждение
Определение~\ref{rational} корректно, то есть не зависит от представления числа $b$ в виде дроби.
\кутверждение
\доказательство
Пусть $b = \frac{m_1}{n_1} = \frac{m_2}{n_2}$. Обозначим $r_1 = \sqrt[n_1]{a^{m_1}}$, $r_2 = \sqrt[n_2]{a^{m_2}}$. Нам нужно проверить, что $r_1 = r_2$.

Из определения корня следует, что $r_1^{n_1} = a^{m_1}$ и $r_2^{n_2} = a^{m_2}$. Из свойства~\ref{composition} степени с целым показателем следует, что
\[
r_1^{n_1 m_2} = (r_1^{n_1})^{m_2} = (a^{m_1})^{m_2} = a^{m_1 m_2} = (a^{m_2})^{m_1} = (r_2^{n_2})^{m_1} = r_2^{n_2 m_1}.
\]
Из равенства $\frac{m_1}{n_1} = \frac{m_2}{n_2}$ следует, что $n_1 m_2 = n_2 m_1$. А значит, в силу свойства~\ref{base_monotoneness} числа $r_1$ и $r_2$ совпадают.
\кдоказательство

\утверждение
В случае $b\in\Z$ определение~\ref{rational} согласуется с определением~\ref{integer}.
\кутверждение
\доказательство
Действительно, если $b = \frac{m}{1}$, то $\sqrt[1]{a^m} = a^m$.
\кдоказательство

\лемма
\label{rational_properties}
Свойства 1--5 выполняются для степени с рациональным показателем.
\клемма
\доказательство
Свойства степени с рациональным показателем обычно выводят из свойств степени с целым показателем. Детальное доказательство оставляется читателю в качестве упражнения.
\кдоказательство

Ну и наконец перейдём к случаю $b\in\R$.
\опр
\label{real}
Пусть $a \ge 1$\т вещественное число и $b$\т любое вещественное число. Рассмотрим множество $P(a,b) = \{a^q \mid q\in\Q,\, q < b\}$. Оно не пусто и ограничено сверху. По определению $a^b = \sup P(a,b)$.\\
Пусть $0 < a < 1$. По определению $a^b = (1/a)^{-b}$.
\копр

\утверждение
В случае $b \in \Q$, определение~\ref{real} согласуется с определением~\ref{rational}.
\кутверждение
\доказательство
Очевидно, что $\sup P(a,b) \le a^b$. Покажем, что на самом деле достигается равенство. Мы знаем, что $\limn \sqrt[n]{a} = 1$ (несложно выводится из неравенства Бернулли). Поэтому $\limn a^{b - 1/n} = a^b$. Однако все числа $a^{b - 1/n}$ лежат в $P(a,b)$. Утверждение доказано.
\кдоказательство

\теорема
\label{real_properties}
Свойства 1--5 выполняются для степени с вещественным показателем.
\ктеорема

Доказательство теоремы мы разобьём на цепочку утверждений. Вначале мы будем рассматривать только случай, когда основание степени не меньше 1.


\утверждение
\label{real_additivity}
Пусть $a \ge 1$. Тогда $a^b \cdot a^c = a^{b+c}$.
\кутверждение
\доказательство
У нас есть три множества: $P(a,b) = \{a^q \mid q\in\Q,\, q < b\}$, $P(a,c) = \{a^q \mid q\in\Q,\, q < c\}$ и $P(a,b+c) = \{a^q \mid q\in\Q,\, q < b+c\}$. Необходимо доказать, что $\sup P(a,b) \cdot \sup P(a,c) = \sup P(a,b+c)$.

Рассмотрим произвольное рациональное число $q_{b+c} < b+c$. Его можно представить в виде суммы двух рациональных чисел $q_{b+c} = q_b + q_c$, где $q_b < b$ и $q_c < c$. Тогда $a^{q_{b+c}} = a^{q_b}\cdot a^{q_c} \le \sup P(a,b) \cdot \sup P(a,c) = a^b \cdot a^c$. Значит, любой элемент множества $P(a,b+c)$ не превышает числа $a^b \cdot a^c$. Следовательно $a^b \cdot a^c \ge \sup P_{b+c} = a^{b+c}$.

Покажем теперь, что для любого $\ep > 0$ число $a^b \cdot a^c - \ep$ не является верхней гранью множества $P(a,b+c)$. Выберем достаточно малое число $\de > 0$. Возьмём рациональные числа $q_b < b, q_c < c$, такие что $a^{q_b} > \sup P(a,b) - \de = a^b - \de$ и $a^{q_c} > \sup P(a,c) - \de = a^c - \de$. Тогда
\[
P(a,b+c) \ni a^{q_b+q_c} = a^{q_b} \cdot a^{q_c} > (a^b - \de)(a^c - \de) = a^b \cdot a^c - \de(a^b + a^c) + \de^2 > a^b \cdot a^c - \de (a^b + a^c).
\]
Значит, достаточно выбрать $\de < \dfrac{\ep}{a^b + a^c}$.

Из сказанного следует, что $a^b \cdot a^c - \ep < a^{b+c}$ для любого $\ep > 0$. Объединяя полученные неравенства, заключаем, что $a^b \cdot a^c = a^{b+c}$.
\кдоказательство

\следствие
\label{inversion}
При $a \ge 1$ имеем $a^{-b} = 1/a^b$.
\кследствие
\доказательство
$1 = a^0 = a^{b + (-b)} = a^b \cdot a^{-b}$.
\кдоказательство

\утверждение
\label{real_base_monotoneness}
Пусть $a > b \ge 1$. Если $c > 0$, то $a^c > b^c$; если $c < 0$, то $a^c < b^c$.
\кутверждение
\доказательство
Пусть $c > 0$. Возьмём какое-нибудь рациональное число $q \in (0,c)$. Тогда $(a/b)^c \ge (a/b)^q > 1$. Значит, $a^c > b^c$.

Если же $c < 0$, то $a^{-c} > b^{-c}$ и из следствия~\ref{inversion} получаем $a^c < b^c$.
\кдоказательство

\утверждение
\label{real_exponent_monotoneness}
Пусть $b > c$, $a > 1$. Тогда $a^b > a^c$.
\кутверждение
\доказательство
Возьмём какое-нибудь рациональное число $q \in (0, b-c)$. Тогда $a^{b-c} \ge a^q > 1$. Значит, $a^b > a^c$.
\кдоказательство

\утверждение
\label{real_composition}
Пусть $a \ge 1$, $b \ge 0$. Тогда $(a^b)^c = a^{bc}$.
\кутверждение
\доказательство
Случаи $a = 1$ и $b = 0$ тривиальны. Поэтому будем считать, что неравенства строгие. Тогда из утверждения~\ref{real_exponent_monotoneness} следует, что $a^b > 1$.

Пусть $c > 0$. Рассмотрим $P(a^b,с) = \{(a^b)^q \mid q\in\Q,\, q < c\}$ и $P(a,bc) = \{a^q \mid q\in\Q,\, q < bc\}$. Необходимо доказать, что $\sup P(a^b,с) = \sup P(a,bc)$.

Выберем произвольное положительное рациональное число $r < bc$. Его можно представить в виде произведения двух положительных рациональных чисел $r = qs$, где $q < c$, $s < b$. Из определения степени и утверждения~\ref{real_base_monotoneness} следует, что $a^r = (a^s)^q \le (a^b)^q) \le \sup P(a^b,c)$. Если же $r \le 0$, то $a^r \le 1 < \sup P(a^b,c)$.
Значит, $\sup P(a,bc) \le \sup P(a^b,c)$.

Выберем произвольное положительное рациональное число $q < c$. Возьмём рациональное число $t \in (b, bc/q)$. Тогда $t > b$ и $tq < bc$. Поэтому $(a^b)^q < (a^t)^q = a^{tq} \le \sup P(a,bc)$. Для $q \le 0$ имеем $(a^b)^q \le 1 < \sup P(a,bc)$. Поэтому $\sup P(a^b,c) \le \sup P(a,bc)$.

Объединяя полученные неравенства, заключаем, что $\sup P(a^b,с) = \sup P(a,bc)$.

Теперь рассмотрим случай $c < 0$. Применяя следствие~\ref{inversion}, получаем $(a^b)^c = \frac{1}{(a^b)^{-c}} = \frac{1}{a^{-bc}} = a^{bc}$.
\кдоказательство


\утверждение
\label{real_multiplicativity}
Пусть $a \ge 1$, $b \ge 1$. Тогда $a^c \cdot b^c = (ab)^c$.
\кутверждение
\доказательство
Как обычно рассмотрим множества $P(a,c) = \{a^q \mid q\in\Q,\, q < c\}$, $P(b,c) = \{b^q \mid q\in\Q,\, q < c\}$ и $P(ab,c) = \{(ab)^q \mid q\in\Q,\, q < c\}$.

Пусть $q < c$\т произвольное рациональное число. Тогда $(ab)^q = a^q \cdot b^q \le \sup P(a,c) \cdot \sup P(b,c) = a^c \cdot b^c$. Значит, $(ab)^c \le a^c \cdot b^c$. 

Покажем теперь, что для любого $\ep > 0$ число $a^c \cdot b^c - \ep$ не является верхней гранью множества $P(ab,c)$. Выберем достаточно малое число $\de > 0$. Возьмём рациональное число $q < c$, такое что $a^{q} > \sup P(a,c) - \de = a^c - \de$ и $b^{q} > \sup P(b,c) - \de = b^c - \de$. Тогда
\[
P(ab,c) \ni (ab)^q = a^q \cdot b^q > (a^c - \de)(b^c - \de) = a^c \cdot b^c - \de(a^c + b^c) + \de^2 > a^c \cdot b^c - \de (a^c + b^c).
\]
Значит, достаточно выбрать $\de < \dfrac{\ep}{a^c + b^c}$.

Из сказанного следует, что $a^c \cdot b^c - \ep < (ab)^c$ для любого $\ep > 0$. Объединяя полученные неравенства, заключаем, что $a^c \cdot b^c = (ab)^c$.
\кдоказательство

\medskip

Ура! Теорему~\ref{real_properties} мы почти доказали. Осталось рассмотреть случаи, когда основание степени меньше 1. При этом мы уже не будем непосредственно возиться с супремумами.

\утверждение
Пусть $a < 1$. Тогда $a^b \cdot a^c = a^{b+c}$.
\кутверждение
\доказательство
Из определения~\ref{real} и утверждения~\ref{real_additivity} получаем $a^b \cdot a^c = (1/a)^{-b} \cdot (1/a)^{-c} = (1/a)^{-b-c} = a^{b+c}$.
\кдоказательство

\следствие
\label{inversion_2}
При $a < 1$ также выполнено $a^{-b} = 1/a^b$.
\кследствие
\доказательство
Полностью аналогично доказательству следствия~\ref{inversion}.
\кдоказательство

\утверждение
\label{inversion_3}
Выполняется равенство $a^b = 1/(1/a)^b$.
\кутверждение
\доказательство
Доказательство оставляется читателю в качестве упражнения.
\кдоказательство

\утверждение
Пусть $a > b > 0$. Если $c > 0$, то $a^c > b^c$; если $c < 0$, то $a^c < b^c$.
\кутверждение
\доказательство
Случай $b > 1$ рассмотрен в утверждении~\ref{real_base_monotoneness}.

Предположим, что $a > 1 \ge b$. Тогда при $c > 0$ имеем $a^c > 1 \ge b^c$. При $c < 0$ имеем $a^c < 1 \le b^c$ (мы воспользовались следствиями~\ref{inversion} и~\ref{inversion_2}).

Предположим, что $1 \ge a > b$. Тогда $1/b > 1/a \ge 1$. Поэтому при $c > 0$ имеем $(1/b)^c > (1/a)^c$, откуда $1/(1/a)^c > 1/(1/b)^c$. С учётом утверждения~\ref{inversion_3} получаем $a^c > b^c$.\\ 
При $c < 0$ всё аналогично.
\кдоказательство

\утверждение
Пусть $b > c$, $a < 1$. Тогда $a^b < a^c$.
\кутверждение
\доказательство
$a^b = 1/(1/a)^b < 1/(1/a)^c = a^c$.
\кдоказательство

\утверждение
\label{real_composition_2}
Пусть $a \ge 1$, $b < 0$. Тогда $(a^b)^c = a^{bc}$.
\кутверждение
\доказательство
Случай $a = 1$ тривиален. Будем считать, что $a > 1$.

Поскольку $b < 0$, число $a^b$ меньше 1. Значит, по определению $(a^b)^c = (1/a^b)^{-c}$. Из следствия~\ref{inversion} получаем $(1/a^b)^{-c} = (a^{-b})^{-c}$. По утверждению~\ref{real_composition}: $(a^{-b})^{-c} = a^{(-b)(-c)} = a^{bc}$.
\кдоказательство

\утверждение
Пусть $a < 1$. Тогда $(a^b)^c = a^{bc}$.
\кутверждение
\доказательство
По определению $(a^b)^c = ((1/a)^{-b})^c$. Из утверждений~\ref{real_composition} и~\ref{real_composition_2} следует, что $((1/a)^{-b})^c = (1/a)^{-bc}$. Повторно применяя определение, получаем $(1/a)^{-bc} = a^{bc}$.
\кдоказательство

\утверждение
Выполняется равенство $a^c \cdot b^c = (ab)^c$.
\кутверждение
\доказательство
Случай $a \ge 1$, $b \ge 1$ рассмотрен в утверждении~\ref{real_multiplicativity}.

Пусть $a \ge 1$, $b < 1$, $ab > 1$. Тогда $a^c = (ab)^c \cdot (1/b)^c$. Значит, $(ab)^c = a^c \cdot 1/(1/b)^c = a^c \cdot b^c$.

Все остальные случаи мы охватим с помощью трюка. Выберем число $M$ заведомо большим (то есть $M > \max(1, 1/a, 1/b)$). Тогда $aM > 1$, $bM > 1$, $abM^2 > 1$. По уже доказанному выполнены равенства: 
\[
\bcase{
a^c \cdot M^c &= (aM)^c\\
b^c \cdot M^c &= (bM)^c\\
(aM)^c \cdot (bM)^c &= (abM^2)^c\\
(ab)^c \cdot (M^2)^c &= (abM^2)^c\\
M^c \cdot M^c &= (M^2)^c\\
}
\]
Значит, $a^c \cdot M^c \cdot b^c \cdot M^c = (aM)^c \cdot (bM)^c = (abM^2)^c = (ab)^c \cdot (M^2)^c = (ab)^c \cdot M^c \cdot M^c$. Сокращая на $(M^c)^2$, получаем требуемое.
\кдоказательство

Теорема~\ref{real_properties} полностью доказана.

\bigskip
\hrule
\bigskip

\задача
Докажите лемму~\ref{rational_properties}.
\кзадача

\задача
Докажите утверждение~\ref{inversion_3}.
\кзадача

\ввзадача
Пусть $a,b \in \R$, $a > 0$, $a \ne 1$, $b > 0$. 
\невСтрочку
\пункт
Докажите, что уравнение $a^x = b$ имеет решение.
\пункт
Докажите, что это решение единственно.
\кзадача

\сзадача
Возможно ли такое, что $a, b \not\in \Q$ и $a^b \in \Q$.
\кзадача


\end{document}
