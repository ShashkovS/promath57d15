% !TeX encoding = windows-1251
\documentclass[a4paper,12pt]{article}
\usepackage{newlistok}
% \input{haidar_edition}

\УвеличитьШирину{1cm}
\renewcommand{\spacer}{\vfil}

\Заголовок{Комплексные числа}
\НомерЛистка{CN1}
\ДатаЛистка{11-й "Д" КЛАСС 2011 г}

\begin{document}
\СоздатьЗаголовок

\опр \выд{Комплексное число} $z$\т это выражение вида $z=a+bi$, где $a$ и $b$\т действительные числа, а $i$\т \выд{мнимая единица}: символ, квадрат которого равен $(-1)$. Число $a$ называется вещественной частью комплексного числа $z$
(пишется $a=\Re(z)$), а число $b$\т мнимой частью $z$ (пишется $b=\Im(z)$). Комплексные числа можно складывать и умножать (\лк раскрывая скобки и приводя подобные\пк). Множество комплексных чисел обозначается буквой $\Cbb$.
\копр

\задача
Напишите формулы для вещественной и мнимой части суммы и произведения
комплексных чисел $a+bi$ и $c+di$.
\кзадача

\опр
Сопоставим каждому комплексному числу $z=a+bi$ вектор с координатами $(a,b)$. Длина этого вектора называется \выд модулем комплексного числа $z$ и обозначается $|z|$.
Пусть $z\ne0$. Угол, отсчитанный против часовой стрелки от вектора с координатами $(1,0)$ до вектора с координатами $(a,b)$, называется \выд аргументом комплексного числа $z$ и обозначается $\Arg(z)$. Аргумент комплексного числа определен с точностью до прибавления числа вида $2\pi n$, где $n\in\Z$.
\копр

\задача
Найдите модуль и аргумент следующих комплексных чисел:\\
$
-4,\quad
1+i,\quad
1-i\sqrt{3},\quad
\sin\alpha + i\cos\alpha,\quad
\frac{1+i\tg\alpha}{1-i\tg\alpha},\quad
1+\cos\alpha+i\sin\alpha.
$
\кзадача

\задача[Тригонометрическая форма записи]
Докажите, что для любого ненулевого комплексного числа $z$ имеет место равенство $z=r(\cos\phi + i\sin\phi)$, где $r=|z|$, $\phi=\Arg(z)$.
\кзадача

\задача
\пункт
Доказать, что сумме комплексных чисел отвечает вектор, равный сумме векторов, отвечающих слагаемым.
\пункт
Пусть $z$ и $w$\т комплексные числа. Выразите $|zw|$ и $\Arg(zw)$ через $|z|$, $|w|$, $\Arg(z)$ и $\Arg(w)$.
\кзадача

\задача
Верно ли, что $|z+w|\leq|z|+|w|$ при любых комплексных числах $z$ и $w$?
\кзадача

\опр
Пусть $z=a+bi$. Число $\ol z=a-bi$ называется \выд{комплексно-сопряжённым} к~$z$.
\копр

\задача
Выразите модуль и аргумент числа $\ol z$  через модуль и аргумент числа~$z$.
\кзадача

\задача
Докажите, что
\вСтрочку
\пункт
$|z|^2=z\ol z$ для любого $z\in\Cbb$;
\пункт
$\ol{z_1+z_2}=\ol z_1 + \ol z_2$ и $\ol{z_1z_2}=\ol z_1 \ol z_2$ для любых $z_1,z_2\in\Cbb$.
\кзадача

\задача
Пусть $P(x)\in\R[x]$, $z\in\Cbb$ и $P(z)=0$. Докажите, что $P(\ol z)=0$.
\кзадача

\задача
Докажите, что
\вСтрочку
\пункт
$\Cbb$\т поле;
\пункт
из любого комплексного числа можно извлечь квадратный корень.
\кзадача

\задача
Можно ли на множестве комплексных чисел ввести отношение порядка $\le$ так, чтобы получилось упорядоченное поле?
\кзадача

\задача
Вычислите:
\вСтрочку
\пункт
$\frac{(5+i)(7-6i)}{3+i}$;
\пункт
$\frac{(1+i)^5}{(1-i)^3}$;
\пункт
$\frac{(1+3i)(8-i)}{(2+i)^2}$;\\
\пункт
$(1+i\sqrt{3})^{150}$;
\пункт
$\frac{(\sqrt{3}+i)}{(1-i)^{30}}$.
\кзадача

\задача
Решите уравнения:
\вСтрочку
\пункт
$z^2=i$;
\пункт
$z^2=5-12i$;
\пункт
$z^2+(2i-7)z+13-i=0$;
\пункт
$\ol z=z^2$;
\пункт
$\ol z=z^3$.
\кзадача

\задача
Вычислите суммы:
\вСтрочку
\пункт
$C_{n}^{1}-C_{n}^{3}+C_{n}^{5}-C_{n}^{7}+\dots$;
\пункт
$C_{n}^{0}+C_{n}^{4}+C_{n}^{8}+C_{n}^{12}+\dots$.
\кзадача

\задача[Формула Муавра]
Пусть $z=r(\cos\phi+i\sin\phi)$, $n\in\N$. \\
Докажите, что $z^n=r^n(\cos n\phi+i\sin n\phi)$.
\кзадача

\задача
Найдите суммы:
\вСтрочку
\пункт $\sin \phi +\sin 2\phi +\ldots +\sin n\phi$;
\пункт $\cos \phi +\cos 2\phi +\ldots +\cos n\phi$;\\
\пункт $\sin \phi +\frac{1}{2}\sin 2\phi +\ldots +\frac{1}{2^n}\sin n\phi$;
\пункт $1+2\cos \phi +3\cos 2\phi +\ldots +(n+1)\cos n\phi$.
\кзадача

\задача
Выразите $\sin^4x$ и $\cos^5x$ в виде суммы чисел вида $\alpha\sin{kx}$ и $\beta\cos{lx}$, где $\alpha,\ \beta\in\R$ и $k,l\in\N\cup\{0\}$.
\кзадача

\задача
Выразите $\cos nx$ и $\sin nx$ через $\cos{x}$ и $\sin{x}$.
\кзадача

\newpage

\задача
Докажите, что многочлен степени $n$ с комплексными коэффициентами имеет не более $n$ комплексных корней.
\кзадача

\РазделитьКондуит{0.3mm}{6.5mm}

\задача
\пункт
Найдите (и нарисуйте) все комплексные корни многочленов: $z^2-1$, $z^3-1$, $z^4-1$,
$z^5-1$, $z^6-1$.
\пункт
Сколько корней имеет уравнение $z^n=1$?
\кзадача

\задача
\пункт
Вычислите сумму и произведение всех корней степени $n$ из $1$.
\пункт
Пусть $\alpha_1, \ldots, \alpha_n$\т все корни степени $n$ из $1$, $\alpha_1=1$. Найдите $\alpha_1^s+\ldots+\alpha_n^s$ (где $s\in\N$) и $(1-\alpha_2)\cdot\ldots\cdot(1-\alpha_n)$.
\кзадача

\задача
Пусть $P$\т многочлен степени $k$ с коэффициентами из $\Cbb$. Докажите, что среднее арифметическое значений $P$ в вершинах правильного $n$-угольника равно значению $P$ в центре многоугольника, если $n>k$.
\кзадача

\задача
\вСтрочку
\пункт
Пусть $z=\frac{3+4i}{5}$. Найдётся ли такое $n\in\N$, что $z^n=1$?
\пункт
Докажите, что $\frac1\pi\arctg\frac43\notin\Q$.
\кзадача

\задача
Пусть $z,v,w\in\Cbb$, причём $z+v+w=z^2+v^2+w^2=z^3+v^3+w^3=0$. Верно ли, что $z^4+v^4+w^4=0$?
\кзадача

\задача
Нарисуйте множество комплексных чисел, для которых:
\вСтрочку
\пункт $z^n+1=0$;\\
\пункт $|z-i|\leq2$;
\пункт $|z-1|=2|z-i|$;
\пункт $z^2+\ol z^2=4$;
\пункт $|z-1|-|z+1|\leq3$;
\пункт $|z-1|+|z+1|=3$;
\пункт $z+\ol z=2|z-1|$.
\кзадача

\задача
Каким геометрическим преобразованиям соответствуют следующие отображения:
\сНовойСтроки
\пункт
$z\longmapsto\ol z$;
\пункт
$z\longmapsto(\cos\phi+i\sin\phi)z$, где $\phi\in\R$;
\пункт
$z\longmapsto \lambda z$, где $\lambda\in\R$;
\пункт
$z\longmapsto wz$, где $w\in\Cbb$?
\кзадача

\задача
Запишите в виде функции комплексного переменного:
\сНовойСтроки
\пункт ортогональную проекцию на ось $x$;
\пункт симметрию относительно оси $y$;
\пункт центральную симметрию с центром $A$;
\пункт поворот на угол $\varphi$ относительно точки $A$;
\пункт гомотетию с коэффициентом $k$ и центром $A$;
\пункт симметрию относительно прямой $y=3$ со сдвигом на 1 влево;
\пункт поворот, переводящий ось $x$ в прямую $y=2x+1$;
\пункт симметрию относительно прямой $y=2x+1$.
\кзадача

\задача
Куда отображение $z\longmapsto z^2$ переводит
\вСтрочку
\пункт
декартову координатную сетку;\\
\пункт
полярную координатную сетку;
\пункт
окружность $|z+i|=1$;
\кзадача

\задача Те же вопросы для отображения
$z\longmapsto 1/z$.
\кзадача

\задача
Куда отображение $z\longmapsto\sqrt z$ переводит верхнюю полуплоскость (без границы)?
\кзадача

\задача
\пункт
Куда отображение $z\longmapsto1/z$ переводит множество $\{z\in\Cbb\mid\Im(z)>0,|z|\leq1\}$?
\спункт
Тот же вопрос для отображения $z\longmapsto\frac{z + 1/z}{2}$.
\кзадача

%\newpage
\ВставитьКондуит{0.7mm}{6.5mm}{-10mm}

\end{document}
