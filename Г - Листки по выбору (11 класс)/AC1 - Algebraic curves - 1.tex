% !TeX encoding = windows-1251
\documentclass[a4paper,12pt]{article}
\usepackage{newlistok}
% \input{haidar_edition}

%\УвеличитьВысоту{1.7cm}
\УвеличитьШирину{1cm}

\renewcommand{\spacer}{\vfil}
\sloppy

\НомерЛистка{AC1}
\ДатаЛистка{11-й "Д" КЛАСС 2012 г}
\Заголовок{Плоские алгебраические кривые --- 1}

\begin{document}
\СоздатьЗаголовок
%\раздел{Часть 1. Многочлены от двух переменных}

\опр  \выд{Одночленом от двух переменных}\/ $x$ и $y$ (над $\R$)
называется выражение
вида $ax^my^n$, где $a\in\R$, $m,n\in\Z^+$.
Сумма нескольких одночленов такого~вида
(с приведенными подобными)
называется \выд{многочленом от двух переменных}\/ $x$ и $y$.
Сумма и произведение многочленов от двух переменных определяются аналогично
сумме и произведению многочленов от одной переменной.
Множество всех многочленов от $x$, $y$ (над $\R$) обозначают $\R[x,y]$.
\копр

\задача Дайте определение степени многочлена  $A\in\R[x,y]$
(обозначается $\deg A$).
\кзадача

\задача Пусть $A(x,y)$, $B(x,y)$ --- ненулевые многочлены.\\
\вСтрочку
\пункт Докажите, что $\deg AB=\deg A+\deg B$.
\пункт Что можно сказать о величине $\deg (A+B)$?
\кзадача

\задача Дайте определение
деления с остатком для многочленов от двух переменных.
Всегда ли такое деление возможно?
\кзадача

\задача Дайте определение неприводимого (над $\R$)
многочлена из $\R[x,y]$. %от двух переменных.
\кзадача

\задача Докажите неприводимость многочленов:
\вСтрочку
\пункт $x^2+y^2-1$; \пункт $y^2-x$; \пункт $xy-1$.
\кзадача

%\vspace{-3mm}
\раздел{***}
%\vspace{-3mm}

\задача
Рассмотрим окружность $x^2 + y^2 = 1$ и прямые проходящие через точку $A (0;1)$, не параллельные оси абсцисс. Пусть $C$ --- точка пересечения одной из таких прямых с осью абсцисс, а $B$ --- точка пересечения этой прямой с окружностью.
\сНовойСтроки \пункт Докажите, что сопоставляя точку $B$ точке $C$, мы задаем взаимно-однозначное соответствие между точками прямой $Ox$ и точками окружности;
\пункт Докажите, что точка $B$ имеет рациональные координаты тогда и только тогда, когда точка $C$ имеет рациональные координаты.
\кзадача

\задача
Найдите все целочисленные решения уравнения $x^2 + y^2 = z^2.$
\кзадача

\опр
Плоской алгебраической кривой называется множество точек плоскости, координаты которых удовлетворяют уравнению $f(x,y) = 0$, где $f(x,y) \in \R[x,y]$.
%Далее в этом листке нас будет интересовать случай, когда его коэффициенты --- рациональные числа, $f(x,y) \in \Q[x,y]$.
\копр

Итак, в задаче 7 мы нашли все рациональные точки на алгебраической кривой, заданной многочленом второй степени $x^2 + y^2 = 1$.\\
Априори не очевидно, есть ли хотя бы одна рациональная точка на кривой $ax^2 + by^2 = c,$ где числа $a,\;b,\;c$ --- целые. Но если хотя бы одна точка есть, то метод из задачи 6 обобщается.

%\задача
%  Рассмотрим эллипс, заданный уравнением $x^2 + 2y^2 = 3$. %Будем рассматривать прямые, проходящие через точку $A (1;1)$.
%  Постройте взаимно-однозначное соответствие между рациональными числами на эллипсе и прямой $Ox$.
%\кзадача

\задача
  Найдите все целочисленные решения уравнения\\
  \пункт $x^2 + 2y^2 = 3z^2;$ \пункт $x^2 - y^2 = z^2.$
\кзадача

\задача  Нарисуйте плоские кривые, задающиеся следующими многочленами:\\
\вСтрочку
\пункт $x-y$;
\пункт $x^2-y^2$;
\пункт $y-x^2$;
\пункт $x^2+y^2-1$;
\пункт $xy-1$;
\пункт $x^2y-xy^2+y-x$;
\пункт $ax^2+by^2-1$, где $a,b$ --- такие числа, что $a>b>0$;
\пункт $ax^2-by^2-1$, где $a,b$ --- такие числа, что $a>b>0$;
\пункт $y^2-x^3$;\quad
\пункт $y-1-x^3$;\quad
\пункт $y^2-1-x^3$;\quad
\пункт $y^2-x-x^3$;\quad
\пункт $y^2-x^2-x^3$.
\кзадача

\опр
Точка $A(x_0;y_0)$ на кривой $f(x,y)=0$ называется неособой, если существует прямая
$
\case{
x = x_0+at,\\
y = y_0+bt.
}$
проходящая через точку $A$, такая, что $t = 0$ --- корень уравнения $f(x_0+at,y_0+bt)=0$ кратности ровно 1. В противном случае, точка $A$ называется особой.
\копр

\задача
  Найдите (какие-нибудь) особые точки на кривых\\
  \пункт $x^2 + x^3 - y^3 = 0;$ \пункт $x^2 + x^3 - y^2 = 0.$
\кзадача

\задача
  Сколько особых точек может быть на кривой\\
  \пункт второй степени; \пункт третьей степени.
\кзадача

\задача
  Найдите все решения в рациональных числах уравнения\\
  \пункт $x^2 + x^3 = y^3$; \пункт $x^2 + x^3 = y^2.$
\кзадача

\ЛичныйКондуит{0mm}{7mm}
%\СделатьКондуит{7mm}{6.0mm}


\end{document}

\задача Докажите, что множество
$\displaystyle{\R(y)=\left\{\frac{P(y)}{Q(y)}\
\Bigl|\ P(y),Q(y)\in\R[y],\ Q(y)\ne0\right\}}$
(с обычными операциями сложения и умножения) является полем.
\кзадача

\задача Рассмотрим множество многочленов от $x$ над полем $\R(y)$,
т.~е.~множество $\R(y)[x]$.
Каждый его элемент записывается в виде
$$a_n(y)x^n+a_{n-1}(y)x^{n-1}+...+a_1(y)x+a_0(y),\eqno(*)$$
$\hbox{где}\ n\in\Z^+,\ a_i(y)\in\R(y)\ \hbox{при}\ i=\overline{0,n}.$
Дайте определение неприводимого (над $R(y)$)
многочлена из $\R(y)[x]$. Верна ли для многочленов из $\R(y)[x]$
теорема о единственности разложения на неприводимые сомножители?
\кзадача

\опр Запишем произвольный многочлен $A(x,y)\in\R[x,y]$ в виде $(*)$,
где уже\break $a_i(y)\in\R[y]$ при $i=\overline{0,n}$.
Скажем, что  $A(x,y)$ является \выд{примитивным (по $x$)},
если многочлены $a_n(y),\ \dots,\ a_0(y)$  взаимно просты.
\копр

\задача Докажите, что произведение двух примитивных (по $x$) многочленов
также является примитивным (по $x$)  многочленом.
\кзадача

\задача Докажите, что если многочлен из $\R[x,y]$ неприводим,
то он неприводим и как многочлен из $\R(y)[x]$. Верно ли обратное?
\кзадача

\задача Докажите, что любой многочлен из $\R[x,y]$
однозначно (с точностью до множителей из~$\R$)
раскладывается в произведение неприводимых над  $\R$
\hbox{многочленов.}
\кзадача
