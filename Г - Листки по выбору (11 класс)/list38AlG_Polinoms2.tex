% !TeX encoding = windows-1251
\documentclass[a4paper,12pt]{article}
\usepackage{newlistok}

\def\hang{\hangindent\parindent}
\def\binom#1#2{\left({#1\atop #2}\right)}

\УвеличитьВысоту{1.7cm}
\УвеличитьШирину{1cm}
\renewcommand{\spacer}{\vfil}

\НомерЛистка{38АлГ}
\ДатаЛистка{01.2009}
\Заголовок{Многочлены}

\begin{document}
\СоздатьЗаголовок




\опр Многочлен  $P(x)=a_nx^{n}+\dots+a_1x+a_0\in\Z[x]$  называется
\выд{примитивным},\/ если числа  $a_n,\dots,a_1,a_0$  взаимно просты.
\копр

\задача[Лемма Гаусса] Произведение двух примитивных многочленов также
примитивный многочлен.
\кзадача


\задача Докажите, что  многочлен $P(x)\in\Z[x]$ неприводим над $\Z$
тогда и только
тогда, когда он неприводим над $\Q.$
\кзадача

\задача Докажите, что  любой многочлен из  $\Z[x]$  однозначно (с точностью до постоянных
множителей) раскладывается в произведение неприводимых над  $\Z$
многочленов.
\кзадача

\задача Пусть  $\alpha\in\R$ --- корень некоторого ненулевого
многочлена $P(x)\in\Z[x]$.\break
Пусть $Q(x)\in\Z[x]$ --- ненулевой многочлен
минимальной степени, такой что $Q(\alpha)=0.$
\сНовойСтроки
\пункт Докажите, что  многочлен $Q(x)$ неприводим над $\Z$;
\пункт Докажите, что  для некоторого ненулевого целого $k$
многочлен $kP(x)$ делится на~$Q(x)$.
\кзадача

\задача[Теорема Гаусса] Если действительное
число  $\alpha$  является корнем
одновременно двух многочленов  $P(x)$  и $Q(x)$  из  $\Z[x]$  и один из
них, скажем  $Q(x)$  неприводим над  $\Z,$  то многочлен  $kP(x)$
при некотором ненулевом целом числе  $k$  делится на  $Q(x)$.
\кзадача

%\centerline{$kP(x)=Q(x)L(x),$ \qquad $L(x)\in\Z[x].$}




\задача
%\пункт
Делится ли
\вСтрочку
\пункт
многочлен $x^{100}-32x^{90}+x^4+5x^3-3x^2-10x+2$
на многочлен $x^2-2$?\\
\пункт
%Делится ли
многочлен $x^{11}+x^{9}-5x^{8}+x^7-6x^{6}-7x^4-98x^2-49$
на многочлен $x^3-7$?
\кзадача




\задача Докажите, что  среди корней неприводимого над  $\Z$  многочлена из  $\Z[x]$
не менее чем второй степени не может быть рациональных.
\кзадача

\задача Докажите, что  следующие числа иррациональны:
\сНовойСтроки
\пункт $\smash{\root n \of p}$,  где  $p$  --- простое,  $n-1\in\N;$
\пункт $\smash{\root n \of {p_1\dots p_k}}$,
где  $p_1,\dots,p_k$  --- различные простые, $n-1,k\in\N;$
\пункт
$\sqrt2+\root3\of{2};$
\спункт $\smash{A(\root N\of{p})}$, где $N$ --- натуральное число,
большее 1,  $p$ --- простое,
$A(x)\in\Z[x]$ --- ненулевой многочлен степени меньше $N$;
\спункт $\alpha_1p^{m_1\over{n_1}}+\dots+\alpha_kp^{m_k\over{n_k}},$ \ где
$p$ \ --- простое, $\alpha_1,\dots,\alpha_n$ --- рациональные числа, не
все равные нулю, ${m_1\over{n_1}},\dots,{m_k\over{n_k}}$ --- попарно различные
правильные дроби.
\кзадача


\сзадача
\пункт
Пусть  $\alpha\in\R$ --- корень некоторого ненулевого
многочлена из $\Q[x]$.
Пусть $G(x)$ --- произвольный многочлен из $\Q[x]$, такой что
$G(\alpha)\ne0.$
Докажите, что  существует такой многочлен $H(x)\in\Q[x]$, что
$\displaystyle{1\over{G(\alpha)}}=H(\alpha)$.
\пункт
Найдите такой многочлен $H(x)$, если $\alpha=\root 3\of 2$ и $G(x)=x+1$.
\кзадача

\сзадача
Пусть  $\alpha\in\R$ --- корень  неприводимого
многочлена из $\Q[x]$ степени~$n$.
\сНовойСтроки
\пункт Докажите, что множество чисел $\{P(\alpha)\,|\,P(x)\in\Q[x]\}$ с
обычными операциям сложения и умножения является полем.
(Это поле обозначается $\Q(\alpha)$.)
\пункт Докажите, что
$\Q(\alpha)=
\{q_0+q_1\alpha+q_2\alpha^2+\dots+q_{n-1}\alpha^{n-1}\,|\,q_0,q_1,\dots,q_{n-1}\in\Q\}$.
\пункт Докажите, что любой элемент поля $\Q(\alpha)$
представляется в виде суммы из пункта б) единственным образом.
\кзадача

\сзадача Докажите, что  многочлен  $(x-a_1)\dots(x-a_n)-1$  неприводим над  $\Z$
 при любых попарно различных целых числах  $a_1,\dots,a_n.$
\кзадача


\сзадача \пункт Найдите целое число $a$, при котором многочлен
$(x-a)(x-10)+1$ раскладывается на два многочлена первой степени
с целыми коэффициентами.
\пункт При каких попарно различных целых числах  $a_1,\dots,a_n$ многочлен
$(x-a_1)\dots(x-a_n)+1$  не является неприводимым над  $\Z$?
\кзадача


\сзадача  Разложите на неприводимые множители над  $\Z$:\\
\вСтрочку
\пункт $x^8+x^4+1$;
\пункт $x^5+x+1$;
\пункт $x^9+x^4-x-1$.
\кзадача


%\СделатьКондуитИз{6.2mm}{6.2mm}{sp_AlG.tex}

\end{document} 