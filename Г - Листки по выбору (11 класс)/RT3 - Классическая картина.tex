% !TeX encoding = windows-1251
\documentclass[a4paper,12pt]{article}
\usepackage{tikz,ifthen}
\usetikzlibrary{calc}
\usepackage{newlistok}


\УвеличитьВысоту{2.5cm}
\УвеличитьШирину{1.1cm}
\renewcommand{\spacer}{\vfil}
\newcommand{\wa}{\overrightarrow}
\newcommand{\smat}[1]{\hr{\begin{smallmatrix}#1\end{smallmatrix}}}


\Заголовок{Классическая теория}
\НомерЛистка{RT3}
\ДатаЛистка{02.2015}
\sloppy

\begin{document}
\СоздатьЗаголовок

{\small
Для изучения сложных вопросов необходимо изучить преобразования, связывающие две различные инерциальные системы отсчёта.
Все законы, которые мы будем формулировать, должны быть инварианты относительно таких преобразований.

Пространство $\R^4$ будем называть \выд{пространством-временем}.
Выберем в $\R^4$ базис $e_1 = e_x = (1,0,0,0)$, $e_2 = e_y = (0,1,0,0)$, $e_3 = e_z = (0,0,1,0)$ и $e_t=(0,0,0,1)$.
Каждая точка $p\in\R^4$ имеет три пространственных координаты $(x,y,z)$ и временную координату $t$, так что $p = (x,y,z,t)= x e_x + y e_y + z e_z + t e_t = (\vec{v},t)$. Эту систему координат будем называть \выд{системой отсчёта лаборатории}.

Например, яблоко в момент времени $5$ и координатами $(2,-3,10)$ будем описывать точкой $(2,-3,10,5)$. Равноускоренное падение этого яблока вниз из этой точки --- это набор точек $(2,-3,10-g\tau^2/2,5+\tau)$. Равномерное движение мотоциклиста --- например, набор точек $(30t,40t,150,t)$. Множество точек в $\R^4$, соответствующих данной частице во все моменты времени, называется \выд{мировой линией} частицы.

}

\задача
    Опишите мировую линию
    \пункт покоящейся частицы;\\
    \пункт равномерно двигающейся частицы;\\
    \пункт Как выглядит в пространстве-времени покоящийся стержень?\\
    \пункт Равномерно без вращения двигающийся стержень?
\кзадача


{\small
Другая система, называется \выд инерциальной, если любая точка двигается в ней равномерно и прямолинейно тогда и только тогда, когда она двигается равномерно и прямолинейно в системе отсчёта лаборатории.

Системе отсчёта лаборатории всегда будет противопоставляться система отсчёта \лк ракеты\пк.
В нашей ракете мы будем проводить ровно те же опыты, что в лаборатории.
Мы постулируем, что результаты экспериментов, проведённых в лаборатории и ракете, подчиняются одним и тем же законам
(этот постулат --- результат множества проведённых экспериментов).
В ракете есть свои часы и своя метровая линейка,
они были взяты из лаборатории перед стартом и не отличались от своих копий в лаборатории.
С помощью этих часов и линейки мы можем найти координаты любого события в пространстве-времени в координатах ракеты.


\noindent{\bf Обозначения:}
Для того, чтобы не путать разные системы координат, мы будем добавлять штрихи в системе координат \лк ракеты\пк: $O'$ для начала координат, $e_{1'}=e_{x'}$, $e_{2'}=e_{y'}$, $e_{3'}=e_{z'}$ и $e_{t'}$ --- для базисных векторов; $(x^{1'},x^{2'},x^{3'},t')$ или  $(x',y',z',t')$ --- для координат.

%Преобразования, превращающие данную инерциальную (то есть систему координат \лк лаборатории\пк) систему отсчёта в другую инерциальную называются преобразованиями Галилея.

}


\задача
  Покажите, если некоторое (не обязательно линейное) преобразование связывает две инерциальных системы отсчёта, то оно аффинно.
\кзадача

\задача
  Пусть $f$ --- аффинное преобразование $\R^4$.\\
  \пункт
    Докажите, что $f(p)-f(0)$ --- линейное преобразование. Обозначим его через $A$.\\
  \пункт
    Докажите, что $f(p) = Ap + f(0)$.
\кзадача


\задача
  Пусть $f\colon\R^4\to\R^4$ --- замена координат. По предыдущей задаче $f(\vec{p}) = A\vec{p} + \vec{v}$. В классической теории мы не властны над временем. Это означает, что $A(x,y,z,t)=(x',y',z',t)$.
  Как может выглядеть матрица $A$ и вектор $v$ (в координатах лаборатории и ракеты)?
\кзадача

\задача
  Найдите $A$ и $\vec{v}$, которые позволяют вычислить координаты события $p$ в пространстве-времени в системе отсчёта ракеты, если известны координаты $p$ в системы отсчёта лаборатории, и
  \пункт
    оси ракеты сонаправлены с осями лаборатории, начало координат ракеты (то есть сама ракета) движется вдоль оси $x$ со скоростью $u$, в момент времени $0$ центры совпадают;
\\\пункт
    в момент времени $0$ ракета находится в точке $(3,4,5)$ и движется вдоль оси $y$ со скоростью $u$.
\кзадача

\задача
  Какие элементы в матрице $A$ отвечают за вектор скорости ракеты?
\кзадача


\medskip
Далее всегда считаем, что в момент времени $0$ центры систем отсчёта совпадают, то есть $f$ линейно.

\задача
  \пункт
    Опишите мировую линию лаборатории (она находится в точке $(0,0,0)$ в своей системе отсчёта) в координатах лаборатории и в координатах ракеты из задачи 5а).\\
  \пункт
    Пусть ракета летит так, что в момент времени $1$ она находится в точке $\vec{v}=(v^x,v^y,v^z)^\top$. Как выглядит матрица перехода в систему отсчёта ракеты?\\
  \пункт
    В системе лаборатории ракета летит со скоростью $\vec{v}$, в системе отсчёта ракеты табуретка летит со скоростью $\vec{w}$. Как получить матрицу перехода из системы лаборатории в систему табуретки? С какой скоростью летит табуретка в системе лаборатории?
\кзадача

\задача
  Опишите множество точек в пространстве-времени в координатах лаборатории и в координатах ракеты из 5а), которые соответствуют покоящемуся в лаборатории стержню длины 1\,м направленному вдоль \пункт оси $Ox$; \пункт оси $Oy$; \пункт лежащему в плоскости $xOy$ под углом $\ph$ к~оси~$Ox$.
\кзадача

\задача
  В плоскости $xOy$ под углом $\ph$ к оси $Ox$ в системе отсчёта лаборатории со скоростью $u$ запустили шарик. Найдите его мировую линию и угол к оси $O'x'$ в системе ракеты из 5а) (выразите его косинус и тангенс через $\cos\ph$ и $\tg\ph$ соответственно).
\кзадача

%\СделатьКондуитИз{6.2mm}{6.2mm}{sp_STO.tex}
%\GenXMLW

\end{document}


