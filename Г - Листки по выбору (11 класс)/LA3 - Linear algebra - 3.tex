% !TeX encoding = windows-1251
\documentclass[a4paper,12pt,fleqn]{article}
\usepackage{amscd}
\usepackage{newlistok}
% \input{haidar_edition}

\УвеличитьШирину{1.5cm}
%\renewcommand{\spacer}{\vfill}

\renewcommand{\ker}{\mathrm{Ker\,}}
\newcommand{\im}{\mathrm{Im\,}}
\newcommand{\Iso}{\mathrm{Iso}}
%\newcommand{\Ann}{\mathrm{Ann}}


\Заголовок{ЛИНЕЙНАЯ АЛГЕБРА\т III\\ДВОЙСТВЕННОЕ ПРОСТРАНСТВО}
\НомерЛистка{LA3}
\ДатаЛистка{11-й "Д" КЛАСС 2012 г}

\begin{document}
\СоздатьЗаголовок

\опр
Пусть $L$ --- линейное пространство над полем $\F$. Пространство $\Hom (L,\F)$ называется \emph{двойственным} (или \emph{сопряжённым}) к~$L$ и~обозначается $L^*$. Его элементы называются \emph{ковекторами}, \emph{линейными функциями} или \emph{линейными функционалами} (на~$L$).
\копр

\задача
Пусть $(e_1,\ldots,e_n)$ --- базис линейного пространства~$L$.
\невСтрочку
\пункт
Докажите, что $\dim L^*=n$ и~что в~$L^*$ можно выбрать такой базис $(f^1,\ldots,f^n)$, что $f^i(e_i)=1$ и~$f^i(e_j)=0$ при всех $i,j=1,\ldots,n$; $i\ne j$.
\пункт
Пусть $A\in\Hom(L,L^*)$ --- такое отображение, что $A(e_i)=f^i$. Зависит ли отображение~$A$ от выбора базиса $(e_1,\ldots,e_n)$?
\кзадача
\замечание
Базис $(f^1,\ldots,f^n)$ пространства~$L^*$ называется \emph{двойственным} к базису $(e_1,\ldots,e_n)$ пространства~$L$.
\кзамечание

\задача
\пункт
Докажите, что для всякого линейного пространства~$L$ существует и~единственно отображение $D_L\colon L\to L^{**}$, удовлетворяющее условию
$\,\,\,\forall\,x\in L\,\,\,\forall\,y\in L^*:\,\,\,(D_L(x))(y)=y(x).$
\невСтрочку
\пункт
Докажите, что $D_L\in\Hom(L,L^{**})$, то есть $D_L$ --- линейное отображение.
\пункт
Пусть $(e_1,\ldots,e_n)$ --- базис~$L$, $(g_1,\ldots,g_n)$ --- дважды двойственный ему базис~$L^{**}$. Рассмотрим отображение $A\in\Hom(L,L^{**})$ такое, что $A(e_i)=g_i$. Зависит ли~$A$ от выбора базиса $(e_1,\ldots,e_n)$?
\пункт
Докажите, что если $L$ конечномерно, то $D_L$ является изоморфизмом.
\кзадача

\задача
Докажите, что пространство многочленов~$\Q[x]$ не изоморфно своему двойственному.
\кзадача

\опр
\emph{Аннулятором} подпространства $L_0$ линейного пространства~$L$ называется множество $\{f\in L^*\mid L_0\subset\ker f\}$. Обозначение:~$\Ann L_0$.
\копр

\задача
Докажите, что аннулятор являются линейным подпространством пространства~$L$.
\кзадача

\опр
\emph{Суммой} линейных подпространств $L_1$ и~$L_2$ линейного пространства~$L$ называется множество $\{x+y\mid x\in L_1,\,y\in L_2\}$. Обозначение:~$L_1+L_2$.
\копр

\задача
Пусть $L_1$ и~$L_2$ --- линейные подпространства. Выясните, какие из следующих множеств являются линейными подпространствами и~выразите их размерности через $\dim L$, $\dim L_1$ и~$\dim L_2$:\\
\пункт $L_1+L_2$;
\пункт $L_1\cup L_2$;
\пункт $L_1\cap L_2$.
\кзадача

\задача
Найдите суммы и~пересечения:
\невСтрочку
\пункт
пространства чётных и~пространства нечётных функций на~$\R$;
\пункт
пространств функций на~$\R$, равных нулю на множествах~$M_1$ и~$M_2$;
\пункт
пространств многочленов, делящихся на фиксированные многочлены $p_1,p_2\in\R[x]$.
\кзадача

\задача
Пусть $L_1$ и~$L_2$ --- линейные подпространства конечномерного пространства~$L$.
\невСтрочку
\пункт Найдите $\Ann0$ и~$\Ann L$.
\пункт Выразите $\Ann(L_1+L_2)$ и~$\Ann(L_1\cap L_2)$ через $\Ann L_1$ и~$\Ann L_2$.
\пункт Выразите $\dim(\Ann L_1)$ через $\dim L$ и~$\dim L_1$.
\пункт Верно ли, что $\Ann(\Ann L_1)=D_L(L_1)$?
\кзадача

\опр
Линейное пространство~$L$ называется \emph{прямой суммой} своих подпространств $L_1,$ $L_2,\,\ldots,\,L_n$, если всякий вектор $x\in L$ ровно одним способом представляется в~виде $x=x_1+x_2+\ldots+x_n$, где $x_i\in L_i$ для всех~$i$. Обозначение: $L=L_1\oplus L_2\oplus\ldots\oplus L_n$.
\копр

\задача
Докажите, что $L=L_1\oplus L_2$ тогда и~только тогда, когда $L=L_1+L_2$ и~$L_1\cap L_2=0$
\кзадача

\задача
Докажите, что если $L=L_1\oplus L_2$, то $L^*=\Ann L_1\oplus\Ann L_2$.
\кзадача

\опр  \label{dual mapping}
Пусть $A\in\Hom(L,M)$. Линейное отображение $A^*\in\Hom(M^*,L^*)$ называется \emph{двойственным} (или \emph{сопряжённым}) к~$A$, если выполняется следующее условие: \vskip -2mm
$$
\forall\,f\in M^*\,\,\forall\,x\in L:\,\,\,(A^*f)(x)=f(Ax).
$$
\vskip -3mm
\копр

\задача
\пункт
Докажите, что определение~\ref{dual mapping} корректно.
\невСтрочку
\пункт
Докажите, что отображение $(A\mapsto A^*)$ принадлежит пространству $\Hom(\Hom(L,M),\Hom(M^*,L^*))$.
\пункт
Выразите $(\lambda A)^*$, $(A+B)^*$, $(AB)^*$ через $\lambda$, $A$, $B$.
\кзадача
% 
\задача
\пункт
Докажите, что диаграмма
$$
\begin{CD}
 L @>{A}>> M\\
 @VD_LVV @VVD_MV\\
 L^{**} @>{A}^{**}>> M^{**}
\end{CD}
$$
коммутативна, то есть $A^{**}D_L=D_MA$.\\
\пункт
Докажите, что если пространства $L$ и~$M$ конечномерны, то отображение~$A^{**}$ совпадает с~отображением~$A$ при отождествлении $L$ с~$L^{**}$ (при помощи~$D_L$) и~$M$ с~$M^{**}$ (при помощи~$D_M$).
\кзадача
% 
\задача
Пусть пространства $L$ и~$M$ конечномерны, $A\in\Hom(L,M)$. Верно ли, что\\
\вСтрочку
\пункт $\Ann(\im L)=\ker A^*$;
\пункт $\Ann(\im L^*)=D_L(\ker A)$?
\кзадача

%\GenXML
\ЛичныйКондуит{0.3mm}{6.5mm}
%\СделатьКондуит{5mm}{8mm}

\end{document}
