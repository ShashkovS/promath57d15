% !TeX encoding = windows-1251
\documentclass[a4paper,12pt]{article}
\usepackage{newlistok}

\УвеличитьВысоту{1.5cm}
\УвеличитьШирину{1.5cm}

\ВключитьКолонитул
\Заголовок{Теория вероятностей\т 5}
%\Подзаголовок{Задачи}
\НомерЛистка{PT5}
\ДатаЛистка{04.2015}

\begin{document}
\def\exec{\makebox[25pt]{\put(-5,3){\line(0,1){3}}\put(-5,3){\vector(1,0){11}}}}
\def\вып{\makebox[22pt]{\put(-5,3){\line(0,1){3}}\put(-5,3){\vector(1,0){11}}}}
\СоздатьЗаголовок

\задача
  Среди учеников школы 15\% знают французский язык и 20\% знают немецкий. Доля учеников, знающих оба языка, составляет 5\%.
  \пункт Являются ли независимыми события <<знать французский>> и <<знать немецкий>>?
  \пункт Чему будет равна доля учеников, знающих оба языка, если потребовать, что бы события <<знать французский>> и <<знать немецкий>> были независимы?
\кзадача

%\задача
%  \пункт Рассмотрим группу событий, любые два из которых независимы. Верно ли, что любые два поднабора этих событий независимы? \пункт Рассмотрим тетраэдр, три грани которого окрашены в красный, жёлтый и синий цвета, а четвёртая --- во все три сразу. Пусть событие <<>> --- <<тетраэдр стоит >>
%\кзадача

\задача
  Постройте вероятностное пространство для бросаний несимметричной монеты с вероятностью выпадения орла $p$.
\кзадача

\задача
  В письменном столе четыре ящика. В первом ящике одна папка красного цвета и одна синего. Во втором две красного и три синего. В третьем три красного и четыре синего. В четвёртом четыре красного и шесть синего. Наудачу открывают ящик и достают из него папку. Какова вероятность того, что эта папка окажется красной?
\кзадача

\задача
  Поскольку в результате бомбёжки была нарушена связь, для передачи срочного донесения с поля боя командир батальона послал в штаб двух связистов разными дорогами. В силу различного боевого опыта и условий передвижения, вероятность того, что первый благополучно достигнет штаба равна 0,65, а что второй --- 0,75. Какова вероятность того, что сообщение будет доставлено в штаб?
\кзадача

\задача
  Подводная лодка выпустила по большегрузному транспорту три торпеды. Из-за возможности корректировки, вероятность попадания первым выстрелом равна 0,4, вторым 0,5, третьим 0,7. Одним попаданием транспорт можно потопить с вероятностью 0,2, двумя --- с вероятностью 0,6, а тремя попаданиями --- наверняка. Найдите вероятность того, что транспорт будет потоплен.
\кзадача

\задача
  В первом вольере находятся восемь белых и два чёрных кролика. Во втором --- семь белых и три чёрных. Один кролик из первого вольера прогрыз дырку в стенке и перешёл во второй вольер. Дырку заделали и выбрали из второго вольера белого кролика. Какова вероятность, что этот кролик был изначально в первом вольере?
\кзадача

\задача
  Сборочный цех завода получает изделия из трёх цехов. 25\% хранящихся изделий от первого цеха, 45\% --- от второго, 30\% --- от третьего. Все изделия хранятся на общем складе. Известно, что доля бракованных изделий в первом цехе составляет 4\%, во втором --- 6\%, в третьем --- 5\%. При проверке ОТК (отдел технического контроля) наугад обследованное изделие оказалось бракованным. Какова вероятность, что это изделие из первого цеха?
\кзадача

\задача
  \пункт Уксусную эссенцию, содержащую 70\% уксусной кислоты, разбавили водой в пропорции 20\% уксусной эссенции на 80\% воды. Какова концентрация (процентная доля уксусной кислоты) полученного раствора?
  \невСтрочку
  \пункт В какой пропорции нужно смешать 10\% и 15\% растворы, что бы получить 12\% раствор?
\кзадача

\задача
  \пункт Имеются три события $A,B,C$. Докажите, что если вероятность события <<$A$ и $B$>> и события <<$A$ и $C$>> не меньше 0,9, то и условная вероятность события $A$ при условии события <<$B$ и $C$>> не меньше 0,9.
  \спункт На какие числа можно заменять 0,9 в предыдущем пункте?
\кзадача

\задача
  На выборах кандидат $A$ набрал $a$ голосов, а кандидат $B$ --- $b$ голосов, причём $a>b$. Найдите вероятность того, что при последовательном подсчёте голосов кандидат $A$ всё время был впереди кандидата $B$.
\кзадача

\задача
  Двое играют в игру: бросают монету до тех пор, пока не станет равным 10 количество орлов (тогда выигрывает первый) или количество решек (тогда выигрывает второй). Они прервали игру когда было 8 орлов и 9 решек. Каковы вероятности выиграть после возобновления игры у каждого из участников?
\кзадача

%\ЛичныйКондуит{0mm}{6mm}
%\ОбнулитьКондуит

\задача
  Двое играют в игру: каждый пишет на бумажке целое число, потом они одновременно открывают написанные числа. Если сумма чисел делится на 3, выигрывает первый и получает от второго рубль, иначе --- выигрывает второй и получает от первого $A$ рублей. При каком значении числа $A$ эта игра честная?
\кзадача

\задача
  Каждый из двух игроков пишет на бумажке число 1 или 2, после чего они одновременно открывают свои бумажки. Если числа совпали, второй платит первому столько рублей, каковы эти числа, иначе --- первый платит второму $A$ рублей. При каком значении числа $A$ эта игра честная?
\кзадача

\опр
  \выдж {Испытанием Бернулли} называют случайный опыт, который заканчивается одним из двух элементарных событий. Одно из них обычно называют \выдж успехом, а второе --- \выдж неудачей. Вероятность успеха чаще всего обозначают $p$, а вероятность неудачи --- $q$.
\копр

\опр
  \выдж {Серией испытаний Бернулли} называют опыт, состоящий из нескольких независимых и одинаковых испытаний Бернулли.
\копр

\задача
  Найдите вероятность того, что в серии из $n$ испытаний Бернулли будет ровно $k$ успехов.
\кзадача

\задача
  Что вероятнее: выиграть одну партию из трёх или две партии из пяти, если играют равносильные соперники и ничьи невозможны?
\кзадача

\задача
  Торпедный катер атакует крейсер, выпустив по нему одну за другой четыре торпеды. Вероятность попадания каждой торпедой в крейсер равна 0,7. Любая из торпед с равной вероятностью может попасть в любой из 10 отсеков крейсера. Известно, что после попадания торпедой отсек полностью заполняется водой, а по заполнении любых двух отсеков крейсер тонет. Какова вероятность, что при данной атаке крейсер потонет?
\кзадача

\задача[Игла Бюффона]
  Плоскость разделена параллельными прямыми, проведёнными через равные промежутки длины $2a$. На плоскость бросают иголку длиной $2L$. Какова вероятность того, что игла пересечёт какую-то линию, если $L<a$?
\кзадача

\задача
  Китайское правительство издало закон, имеющий целью уменьшить прирост населения и наименьшим образом повлиять на традиции: если в семье первый ребёнок --- мальчик, то этой семье не разрешается больше иметь детей. Иначе семье разрешается завести второго ребёнка. Какое отношение численности мужского населения к женскому должно в этом случае установиться? При конкретном рождении считаем, что рождения мальчика и девочки равновероятны.
\кзадача

\сзадача[Сумасшедшая старушка]
  Каждый из $n$ пассажиров купил по билету на $n$-местный самолёт. Первой зашла сумасшедшая старушка и села на случайное место. Далее каждый вновь вошедший пассажир занимает своё место, если оно свободно, а иначе --- занимает случайное место. Какова вероятность того, что последний пассажир займёт своё место? У старушки был билет на какое-то место.
\кзадача

\сзадача
  В очередь за газетами ценой в полтинник становятся в случайном порядке $n$ человек с полтинниками и рублями. Какова вероятность того, что всем хватит сдачи, если ни у покупателей, ни у продавца изначально нет других денег?
\кзадача

\утв[Закон больших чисел]
  С вероятностью, сколь угодно близкой к единице, можно утверждать, что при достаточно большом числе независимых испытаний статистическая частота появления необходимого события как угодно мало отличается от её вероятности при отдельном испытании.
\кутв

\теор[Неравенство Чебышёва]
  Пусть вероятность некоторого события $A$ в некотором эксперименте равна $p$, и проводится $n$ независимых повторений этого эксперимента. Через $m$ обозначим количество появлений события $A$ в данной серии. Тогда $\fa\ep>0\exec\Pf(|\frac{m}{n}-p|>\ep)<\frac{p\cdot q}{\ep^2\cdot n}$, где $q=1-p$.
\ктеор

\задача
  Рассмотрим испытание Бернулли с вероятностью $p=\frac{1}{2}$.
%  \невСтрочку
  \пункт Примените неравенство Чебышёва для $n=1000$ и $\ep=0{,}1$ и объясните полученный результат с точки зрения эксперимента.
  \пункт Сколько раз нужно повторить опыт, что бы вероятность отклонения частоты $\frac{m}{n}$ от $p$ более, чем на 0,01 не превышала 0,05?
\кзадача

\ЛичныйКондуит{0mm}{6mm}
\end{document}



