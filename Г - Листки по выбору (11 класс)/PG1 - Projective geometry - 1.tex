% !TeX encoding = windows-1251
\documentclass[12pt]{article}
\usepackage{newlistok}
% \input{haidar_edition}

%\setlength{\hoffset}{-25truemm}
%\setlength{\voffset}{-30truemm}

%\УвеличитьВысоту{10mm}
\УвеличитьШирину{8mm}
\renewcommand{\spacer}{\vfill}

%\УвеличитьПромежутки{300}

\Заголовок{Проективная геометрия 1. Проективные преобразования.}
\Подзаголовок{}
\НомерЛистка{PG1}
\ДатаЛистка{11-й "Д" КЛАСС 2012 г}

\begin{document}


\СоздатьЗаголовок

%\bigskip

\опр
Пусть в пространстве заданы две плоскости $\pi$ и $\pi'$, параллельные или непараллельные между собой. Пусть $O$ --- точка, не лежащая ни на $\pi$, ни на $\pi'$. \выд{Центральной проекцией $\pi$ на $\pi'$ с центром $O$} называется отображение, сопоставляющее каждой точке $P\in\pi$ точку $P'\in\pi'$ пересечения прямой $OP$ с плоскостью $\pi'$. Пусть $l$ --- прямая, не параллельная ни $\pi$, ни $\pi'$.
\выд{Параллельной проекцией $\pi$ на $\pi'$ вдоль $l$} называется отображение, сопоставляющее каждой точке $P\in\pi$ такую точку $P'\in\pi'$, что прямая $PP'$ параллельна прямой $l$.
\копр

\задача
Опишите область определения и область значений центральной проекции; параллельной проекции.
\кзадача

\опр
Пусть $\pi$ --- плоскость. Добавим к каждой прямой на ней \выд{\лк бесконечно удалённую\пк\ точку}, причём будем считать, что \лк бесконечно удалённые\пк\ точки у параллельных прямых совпадают, а у непараллельных --- различны. Скажем также, что \лк бесконечно удалённые\пк\ точки всех прямых составляют \выд{\лк бесконечно удалённую\пк\ прямую}. То, что получилось, называется \выд{проективной плоскостью} $\bar\pi$.
\копр

\задача
Докажите, что любые две различные прямые на проективной плоскости имеют единственную общую точку, а через любые две различные точки на проективной плоскости проходит единственная прямая.
\кзадача

\задача
Докажите, что центральная проекция $\pi$ на $\pi'$ с центром $O$ продолжается до взаимно однозначного отображения $\bar\pi$ на $\bar\pi'$, переводящего прямые в прямые (оно называется \выд{центральной проекцией $\bar\pi$ на $\bar\pi'$ с центром $O$}). Аналогично для параллельной проекции.
\кзадача

\опр
Любое отображение $\bar\pi$ на себя, которое можно представить в виде композиции центральных и параллельных проекций, называется \выд{проективным преобразованием}.
\копр

\задача
Докажите, что с помощью проективного преобразования $\bar\pi$ можно перевести любые две точки в \лк бесконечно удал\"енные\пк.
\кзадача

\задача
Докажите, что с помощью проективного преобразования $\bar\pi$ на $\bar\pi$ можно перевести любые три различные \выд{коллинеарные} (лежащие на одной прямой) точки в любые другие три различные коллинеарные точки.
\кзадача

\задача
Докажите, что отрезок нельзя разделить пополам с помощью одной линейки.
\кзадача

\задача
Докажите, что с помощью проективного преобразования $\bar\pi$ можно перевести
%\сНовойСтроки
%\пункт
%любой треугольник в любой другой треугольник;
%\спункт
любую четвёрку точек, никакие три из которых не коллинеарны, в любую другую четвёрку точек с тем же условием, причём такое проективное преобразование единственно.
\кзадача

\ссзадача
Докажите, что любое взаимно однозначное преобразование проективной плоскости в себя, переводящее прямые в прямые, проективно.
\кзадача

\задача[Теорема Паппа]
Пусть вершины шестиугольника $ABCDEF$ лежат попеременно на двух прямых.
%(то есть точки $A,C,E$ коллинеарны и точки $B,D,F$ коллинеарны)
Докажите, что точки пересечения противоположных сторон этого шестиугольника коллинеарны.
\кзадача

\задача[Теорема Дезарга]
Пусть заданы два треугольника $ABC$ и $A'B'C'$, причём прямые $AA'$, $BB'$ и $CC'$ \выд{конкурентны} (пересекаются в одной точке). Докажите, что точки пересечения
соответственных сторон треугольников $ABC$ и $A'B'C'$ коллинеарны.
\кзадача

\задача
Верна ли теорема, обратная теореме Дезарга?
\кзадача

\ЛичныйКондуит{0.3mm}{6.5mm}
%\СделатьКондуит{6mm}{8mm}

\end{document}
