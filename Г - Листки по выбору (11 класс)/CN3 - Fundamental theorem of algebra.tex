% !TeX encoding = windows-1251
\documentclass[a4paper,12pt]{article}
\usepackage{newlistok}
% \input{haidar_edition}

\newcommand{\D}{\mathbb{D}}

\УвеличитьШирину{1cm}
\renewcommand{\spacer}{\vfil}

\Заголовок{Основная теорема алгебры}
\НомерЛистка{CN3}
\ДатаЛистка{11-й "Д" КЛАСС 2011 г}

\begin{document}
\СоздатьЗаголовок

\centerline{\sl Формулировка}

\vspace{3pt}

{\it
Произвольный многочлен степени $n>0$ с комплексными коэффициентами имеет
ровно $n$ комплексных корней (считаемых со своими кратностями).}

\rm

\vspace{7pt}

\centerline{\sl Обозначения}

\vspace{3pt}

\noindent
$P(z)$\т некоторый произвольно выбранный многочлен от комплексной
переменной $z$ с~комплексными коэффициентами степени $n>0$.

\noindent
$\D(z_0,\rho)$\т круг с центром в точке $z_0\in\Cbb$ радиуса $\rho$,
т.~е. $\{z\in\Cbb:|z-z_0|\le\rho\}$.

\vspace{7pt}

\centerline{*~*~*}

\vspace{3pt}

\задача[Поведение многочлена на бесконечности]
Докажите, что $|P(z)|\rightarrow+\infty$ при $|z|\rightarrow+\infty$.
\кзадача

\задача[Поведение многочлена в круге]
Докажите, что $|P(z)|$ ограничен в любом круге (конечного радиуса) и достигает в нём своих максимума и минимума.
\кзадача

\задача[Разложение Тейлора]
Докажите, что для любого $z_0\in\Cbb$ существуют такие $k\in\N$, $c_k,c_{k+1},\dots,c_n\in\Cbb$, что $c_k\ne0$ и для любого $z\in\Cbb$ справедливо равенство
\vskip -4mm
$$
P(z)=P(z_0)+c_k(z-z_0)^k+c_{k+1}(z-z_0)^{k+1}+\dots+c_n(z-z_0)^n\eqno (*)
$$
\vskip -1mm
Представление $P(z)$ в таком виде называется \выд{разложением Тейлора многочлена $P(z)$ в точке $z_0$}.
\кзадача

\задача[Поведение многочлена в малой окрестности точки]
Пусть $(*)$\т разложение Тейлора многочлена $P(z)$ в точке $z_0\in\Cbb$.
\сНовойСтроки
\пункт
Докажите, что  существует такое $\rho>0$, что для любого $z\in\D(z_0,\rho)$, $z\ne z_0$, справедливо неравенство
\vskip -4mm
$$
|P(z)|<|P(z_0)+c_k(z-z_0)^k|+|c_k(z-z_0)^k| \eqno (**)
$$
\vskip -1mm
\пункт
Пусть для любого $z\in\D(z_0,\rho)$, $z\ne z_0$, выполнено соотношение
$(**)$, и, кроме того, $P(z_0)\ne0$. Докажите, что существует такое $z_1\in\D(z_0,\rho)$, что $|P(z_1)|<|P(z_0)|$.
\кзадача

\задача[Поведение многочлена на плоскости]
\сНовойСтроки
\пункт
Докажите, что $|P(z)|$ достигает на плоскости своего минимума: существует такое $\mu\ge0$, что $|P(z)|\ge\mu$ при любом $z\in\Cbb$, причём найдётся такое $z_0\in\Cbb$, что~$|P(z_0)|=\mu$.
\пункт Пусть $\mu$ такое, как в п.~а). Докажите, что $\mu=0$.
\кзадача

\задача
Докажите, что  всякий многочлен ненулевой степени с комплексными коэффициентами имеет хотя бы один комплексный корень, и выведите отсюда основную теорему алгебры.
\кзадача

\задача
\пункт
Разложите в произведение многочленов не более чем второй степени с вещественными коэффициентами многочлены $x^4+3x^2+2$, $x^4+4$, $x^n-1$.
\пункт
Докажите, что произвольный многочлен  с вещественными коэффициентами раскладывается в произведение многочленов не более чем второй степени с вещественными коэффициентами.
\кзадача

\задача
Многочлен $P(x)\in\R[x]$ при всех $x\in\R$ принимает только неотрицательные значения. Докажите, что его можно представить в виде суммы нескольких квадратов многочленов с вещественными коэффициентами.
\кзадача

\задача
Докажите, что максимум $|P(z)|$ в круге достигается в некоторой точке граничной окружности этого круга.
\кзадача

\ЛичныйКондуит{0mm}{6.5mm}

\end{document}
