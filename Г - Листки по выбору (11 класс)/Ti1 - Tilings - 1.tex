% !TeX encoding = windows-1251
\documentclass[a4paper,12pt]{article}
\usepackage{newlistok}
% \input{haidar_edition}

%\УвеличитьШирину{1.5cm}

\Заголовок{МОЗАИКИ НА ПЛОСКОСТИ}
\НомерЛистка{Ti1}
\ДатаЛистка{11-й "Д" КЛАСС 2012 г}

\begin{document}
\СоздатьЗаголовок

\опр
\emph{Мозаикой} называется любое замощение плоскости многоугольниками без пробелов и~наложений. Многоугольники, составляющие мозаику, мы будем называть \emph{плитками}. Нас будут интересовать только те мозаики, число различных базисных плиток которых конечно. Кроме того, в~этом листочке любые две плитки, имеющие общие точки, будут пересекаться либо по вершине, либо по стороне. Вершины плиток таких мозаик называются её \emph{узлами}.
\копр

\задача
Докажите, что существует мозаика, все плитки которой являются
\невСтрочку
\пункт данным треугольником произвольного вида;
\пункт данным шестиугольником, у~которого противоположные стороны попарно равны и~параллельны;
\пункт данным четырёхугольником произвольного вида.
\кзадача

\задача
Приведите пример пятиугольника, экземплярами которого
\невСтрочку
\пункт можно покрыть плоскость;
\пункт нельзя покрыть плоскость;
\пункт можно покрыть плоскость, и~при этом никакие две стороны пятиугольника не параллельны.
\кзадача

\задача
\невСтрочку
\пункт Докажите, что для любого $n>2$ существует мозаика, составленная из равных $n$-угольников.
\спункт Верно ли, что для каждого~$n$ можно сделать этот многоугольник выпуклым?
\кзадача

\опр
\emph{Мозаика} называется периодической (с~периодом~$\bar v$), если она самосовмещается при параллельном переносе вдоль некоторого вектора~$\bar v$.
\копр

\задача
Приведите пример мозаики, которая
\невСтрочку
\пункт является периодической и~обладает тремя попарно неколлинеарными периодами;
\пункт является периодической, причём любые её два периода неколлинеарны;
\пункт не является периодической.
\кзадача

\задача
Существует ли периодическая мозаика, обладающая ровно 57-ю попарно неколлинеарными периодами?
\кзадача

\задача
Верно ли, что периодическая мозаика имеет период, наименьший по длине?
\кзадача

\опр
Мозаика, составленная из правильных многоугольников, любые два из которых пересекаются либо по вершине, либо по стороне, либо не пересекаются вовсе, называется \emph{паркетом}. Паркет называется \emph{правильным}, если для любых двух его узлов найдётся движение, переводящее первый узел во второй, а~весь паркет~---~сам в~себя.
\копр

\задача
Докажите, что различных паркетов бесконечно много.
\кзадача

\задача
Найдите все правильные паркеты, состоящие из одинаковых плиток.
\кзадача

\опр
Каждый узел паркета характеризуется множеством базисных плиток, прилегающих к~этому узлу, и~порядком, в~котором они встречаются при обходе данного узла. Этот порядок называется \emph{типом данного узла} и~записывается в~виде последовательности чисел, отвечающих количеству сторон соответствующих плиток. Количество плиток, прилегающих к~узлу, называется \emph{степенью узла}.
\копр

\задача
Чему может быть равна степень узла произвольного паркета?
\кзадача

\задача
Докажите, что в~каждом правильном паркете все вершины имеют одинаковый тип (он называется \emph{типом правильного паркета}).
\кзадача

\задача
Найдите все правильные паркеты, степень вершин которых\\
\вСтрочку
\пункт не меньше пяти;
\пункт равна четырём;
\пункт равна трём.
\кзадача

Пусть $\Pi$~---~какая-либо плоскость в~пространстве, проходящая через начало координат~$O$, а~$d$~---~некоторое положительное число. Окрестность плоскости $U_{d}(\Pi)$ образно можно назвать \лк слоёным пирогом\пк. Рассмотрим все точки целочисленной решётки~$\mathbb Z^3$, которые лежат внутри этого \лк слоёного пирога\пк, и~спроецируем их на плоскость~$\Pi$. Получается некоторый дискретный набор точек~$X$. Теперь для каждой точки $x\in X$ в~плоскости~$\Pi$ рассмотрим множество
$$F_x=\{y\in X\mid d(x,y)\leqslant d(z,y)\,\,\forall\,z\in X\}\subset\Pi$$
всех точек, расстояние от которых до точки~$x$ не больше, чем до любой другой точки из~$X$. Оно называется \emph{областью Дирихле} или
\emph{областью Вороного} точки~$x$.

\задача
Докажите, что
\невСтрочку
\пункт все области Вороного являются многоугольниками (а~значит, их объединение является мозаикой, имя которой~---~\emph{мозаика Вороного});
\пункт количество различных плиток в~мозаике Вороного конечно;
\пункт мозаика Вороного является периодической тогда и~только тогда, когда плоскость~$\Pi$ содержит хотя бы одну точку из решётки~$\mathbb Z^3$, отличную от начала координат.
\кзадача

\сзадача
[Теорема Вир\'е] Пусть мозаика, составленная из конечного набора плиток, является периодической. Докажите, что существует периодическая мозаика, составленная из того же набора плиток, которая обладает двумя попарно неколлинеарными периодами.
\кзадача

\ЛичныйКондуит{0mm}{7mm}
%\СделатьКондуит{5mm}{6mm}

\end{document}
