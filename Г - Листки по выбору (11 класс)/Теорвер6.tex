% !TeX encoding = windows-1251
\documentclass[a4paper,12pt]{article}
\usepackage{newlistok}

\УвеличитьВысоту{1.5cm}
\УвеличитьШирину{1.5cm}

\ВключитьКолонитул
\Заголовок{Теория Вероятностей\т 6}
\Подзаголовок{Практикум}
\НомерЛистка{PT6}
\ДатаЛистка{04.2015}

\begin{document}
\ncopy{2}{
\СоздатьЗаголовок

В данном практикуме предлагается несколько экспериментальных задач, которые нужно провести дома, результат записать в журнал наблюдений, а затем обсудить эксперимент на уроке.\\
Задание является творческим. Что такое <<анализ>>, <<достаточно>>, <<журнал наблюдений>> каждому предлагается решить самостоятельно.\\

\задача
  \пункт Придумайте и проведите своё испытание Бернулли с вероятностью успеха $p=0{,}5$.
  \невСтрочку
  \пункт Проведите серию своих испытаний Бернулли.
  \пункт Проанализируйте полученный результат. Каков он должен быть согласно теории Вероятностей? Как сильно он отличается от практического?
  \спункт Проведите достаточное количество экспериментов для оценки верности закона больших чисел.
  \пункт Какого размера должна быть серия что бы практически оценить верность неравенства Чебышёва?
  \спункт Оцените верность неравенства Чебышёва практически в случае Вашего испытания Бернулли.
\кзадача

\задача
  Решите предыдущую задачу для $p\ne0{,}5$.
\кзадача

\задача
  \невСтрочку
  \пункт Придумайте эксперимент, в котором было бы несколько неравновероятных исходов.
  \пункт Решите первую задачу, заменив в ней испытание Бернулли на эксперимент из предыдущего пункта.
\кзадача

\vspace{1.5cm}

%\ЛичныйКондуит{0mm}{6mm}
%\ОбнулитьКондуит
}
\end{document}



