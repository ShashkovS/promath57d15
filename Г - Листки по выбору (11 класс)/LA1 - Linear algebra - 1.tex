% !TeX encoding = windows-1251
\documentclass[a4paper,12pt,fleqn]{article}
\usepackage{newlistok}
% \input{haidar_edition}

\УвеличитьШирину{1.5cm}
%\renewcommand{\spacer}{\vfill}

\Заголовок{ЛИНЕЙНАЯ АЛГЕБРА\т I\\БАЗИСЫ И ЛИНЕЙНАЯ ЗАВИСИМОСТЬ}
\НомерЛистка{LA1}
\ДатаЛистка{11-й "Д" КЛАСС 2012 г}

\begin{document}
\СоздатьЗаголовок

\опр
Пусть задано некоторое поле $\F$, элементы которого будем называть \выд скалярами или \выд числами. \выд{Линейным пространством над полем} $\F$ называется множество $L$ с заданными на нём двумя операциями\т сложением ($L\times L\to L$, то есть паре $a$, $b$ элементов $L$ ставится в соответствие элемент $L$, обозначаемый $a+b$) и умножением на число ($\F\times L\to L$, то есть паре $\la\in \F$, $a\in L$ ставится в соответствие элемент $L$, обозначаемый $\la a$), удовлетворяющими следующим условиям (аксиомам):

\begin{tabular}{llll}
(L1) & $a+b=b+a$, $\fa a,b\in L$; & (L5) & $\la(\mu a)=(\la \mu)a$, $\fa a\in L$, $\fa \la,\mu\in\F$; \\
(L2) & $(a+b)+c=a+(b+c)$, $\fa a,b, c\in L$; & (L6) & $\la(a+b)=\la a+\la b$, $\fa a,b\in L$, $\fa \la\in\F$; \\
(L3) & $\exi 0\in L$: $a+0=a$, $\fa a\in L$; & (L7) & $(\la+\mu)a=\la a+\mu a$, $\fa a\in L$, $\fa \la,\mu\in\F$; \\
(L4) & $\fa a\in L$ $\exi b\in L$: $a+b=0$; & (L8) & $1\cdot a=a$, $\fa a\in L$. \\
\end{tabular}

Элементы линейного пространства называются \выд векторами, а само линейное пространство называется также \выд{векторным пространством}.
\копр

\замечание
Необходимо различать $0\in \F$, $0\in L$ и $0$\т линейное пространство, состоящее из одного элемента.
\кзамечание

\замечание
Точно также, как и в любом поле, можно доказать единственность нуля ($0\in L$) и единственность противоположного элемента $b=-a$, существование которых гарантируется аксиомами (L3) и (L4).
\кзамечание

\задача
Являются ли следующие множества векторными пространствами над соответствующими полями:
\невСтрочку
\пункт[Координатное пространство]
Пусть $L=\F^n$ (декартово произведение $n\ge 1$ множителей). Элементы $L$ можно записывать в виде строк $x=(x_1, x_2, \ldots, x_n)$, $y=(y_1, y_2, \ldots , y_n)$ длины $n$. Сложение и умножение происходит покоординатно: $x+y=(x_1+y_1, x_2+y_2, \ldots, x_n+y_n), \quad \la x=(\la x_1, \la x_2, \ldots, \la x_n)$.
\пункт[Свободные векторы]
Пространство знакомых вам по урокам геометрии векторов над полем $\R$ с началом в нуле и концом в произвольной точке плоскости или пространства. В качестве сложения используем композицию векторов, умножение на $\la$ соответствует растяжению в $\la$ раз.
\пункт
$\Q$ над $\R$? $\R$ над $\Q$?
\вСтрочку
\пункт
$\Q$ над $\Z$? $\Z$ над $\Q$?
\пункт
$\R \setminus \Q$ над $\Q$?
\пункт
Поле над своим подполем?
\невСтрочку
\пункт
Множество многочленов от одной переменной с коэффициентами из поля $\F$ (обозначается $\F[x]$) над полем $\F$?
\пункт
Множество функций $F(S): S\to \R$, определённых на произвольном множестве $S$ над полем $\R$?
\пункт
Множество непрерывных функций из $\R$ в $\R$; дифференцируемых функций; интегрируемых на отрезке $[0,1]$ функций?
\пункт
Множество многочленов $\F_n[x]$ степени не выше $n$? Множество $\F_{>n}[x]$ многочленов степени выше $n$? Множество $\F_{=n}[x]$ многочленов степени $n$?
\пункт
Множество функций из $F(S)$, равных нулю в данной точке? Равных единице в данной точке? Равных нулю на данном подмножестве $S_1\subset S$?
\пункт
Бесконечные последовательности действительных чисел? Ограниченные последовательности? Неограниченные последовательности? Сходящиеся последовательности? Последовательности, сходящиеся к нулю?
\пункт
Арифметические прогрессии? Геометрические прогрессии?
\пункт
Пространство решений системы линейных однородных уравнений:
\[
\bcase{
a_{11}x_1 + a_{12}x_2 + \dots + a_{1n}x_n &= 0\\
a_{21}x_1 + a_{22}x_2 + \dots + a_{2n}x_n &= 0\\
\dots \\
a_{m1}x_1 + a_{m2}x_2 + \dots + a_{mn}x_n &= 0\\
}
\]
\кзадача

\задача
Докажите, что $\fa a\in L$, $\la \in \F$:\\
\пункт
$0\cdot a=\la \cdot 0=0$;
\пункт
$(-1) \cdot a = -a$;
\пункт
если $\la a=0$, то либо $\la=0$, либо $a=0$.
\кзадача

\опр
\выд{Линейным подпространством} линейного пространства $L$ называется подмножество $L_1\subset L$, удовлетворяющее условиям:
\begin{items}{-3}
\item[1)]
$\fa x, y\in L_1$: $x+y\in L_1$;
\item[2)]
$\fa \la \in \F$, $\fa x \in L_1$: $\la x\in L_1$.
\end{items}
\копр

\опр
Пусть $(a_1, a_2, \ldots a_n)$\т произвольная конечная система векторов пространства $L$. Множество $\Lc$ всех векторов вида $\la_1a_1+\ldots+\la_na_n, \la_i\in \F$ называется \выд{линейной оболочкой} векторов $a_1, a_2, \ldots a_n$. Традиционно $\Lc$ обозначается как $\langle a_1, \ldots a_n\rangle$. Говорят, что вектор $b$ \выд выражается через $a_1, a_2, \ldots a_n$, если он принадлежит их линейной оболочке (то есть существуют такие $\la_i\in \F, 1\le i\le n,$ что $b=\la_1a_1+\ldots+\la_na_n$).
\копр

\задача
\невСтрочку
\пункт
Докажите, что определённая выше линейная оболочка является линейным подпространством $L$. Элементы этого пространства называются \выд{линейными комбинациями} векторов $a_1, a_2, \ldots, a_n$.
\пункт
Докажите, что линейная оболочка любых двух неколлинеарных векторов на плоскости из пункта~1б) совпадает со всей плоскостью.
\пункт
Докажите, что линейная оболочка любых двух неколлинеарных векторов в пространстве из пункта~1б) является плоскостью, проходящей через начало координат.
\кзадача

\опр
Если в пространстве $L$ существует конечная система векторов $(a_1, a_2, \ldots a_n)$, линейная оболочка которой совпадает с $L$, то такое пространство называется \выд конечномерным. В дальнейшем мы будем рассматривать только конечномерные пространства. Если любой вектор $L$ выражается через некоторую систему векторов $(e_1, e_2, \ldots, e_n)$ единственным образом, то эту систему мы называем \выд базисом.
\копр

\замечание
Задача 3б) показывает, что базис, вообще говоря, определён неоднозначно.
\кзамечание

\задача
\невСтрочку
\пункт
Докажите, что следующие свойства системы $(a_1, a_2, \ldots, a_n)$ эквивалентны:
\begin{items}{-3}
\item[1)]
Любой элемент линейной оболочки системы выражается через $a_i$ единственным образом.
\item[2)]
Не существует такого набора $\la_i$, что $\la_1a_1+\ldots+\la_na_n=0$, где не все $\la_i$ равны нулю (такие суммы называются \emph{нетривиальными линейными комбинациями}).
\end{items}
\замечание
Система векторов, удовлетворяющая свойству $2)$ называется \выд{линейно независимой}. Если же такой набор $\la_i$ существует, то система называется линейно зависимой.
\кзамечание
\пункт
Докажите, что следующие свойства эквивалентны:
\begin{items}{-3}
\item[1)]
Для системы векторов $(a_1, a_2, \ldots, a_n)$ существует нетривиальная линейная комбинация.
\item[2)]
Один из векторов данной системы выражается через остальные.
\item[3)]
$\exists i: \langle a_1, \ldots, a_{i-1}, a_i, a_{i+1}, \ldots, a_n\rangle=\langle a_1, \ldots, a_{i-1}, a_{i+1}, \ldots, a_n\rangle$, то есть один из векторов системы можно выкинуть, сохранив линейную оболочку.
\end{items}
\пункт
Докажите, что если $\la_1a_1+\ldots+\la_na_n=0$\т нетривиальная линейная комбинация, то можно выкинуть любой из векторов, при котором коэффициент $\la_i$ отличен от нуля, с сохранением линейной оболочки.
\пункт
Докажите, что если в пространстве $L$ любой из векторов базиса $(e_1, \ldots, e_n)$ выражается через систему $(a_1, \ldots, a_m)$, то линейная оболочка $\langle a_1, \ldots, a_m \rangle$ совпадает с $L$.
\кзадача

\задача
Докажите, что
\пункт
семейство, содержащее нулевой вектор, линейно зависимо;\\
\пункт
семейство, состоящее из одного вектора, линейно зависимо тогда и только тогда, когда этот вектор равен нулю;
\пункт
семейство, содержащее повторяющиеся векторы, линейно зависимо.
\кзадача

\задача
Являются ли линейно независимыми следующие семейства векторов:\\
\пункт
$\{(1;-1;0),(-1;0;1),(0;1;-1)\}\subset\R^3$;
\пункт
$\{(1;1;0),(1;0;1),(0;1;1)\}\subset\R^3$.
\кзадача

\задача
Докажите, что
\невСтрочку
\пункт
любую систему векторов, линейная оболочка которых совпадает со всем пространством, можно превратить в базис, удалив некоторое (возможно пустое) множество векторов;
\пункт
любую линейно независимую систему векторов, линейная оболочка которых является подпространством $L$, можно превратить в базис, добавив некоторое множество векторов;
\пункт
если линейно независимая система векторов такова, что добавление к ней любого вектора превращает её в линейно зависимую, то она является базисом.
\кзадача

\задача
Пусть $(e_1, e_2, \ldots, e_n)$\т базис $L$, а $f$\т произвольный вектор $L$. Тогда выполнены следующие свойства:
\невСтрочку
\пункт
В наборе $(e_1, e_2, \ldots, e_n, f)$ существует ровно одна (с точностью до домножения на константу) нетривиальная линейная комбинация.
\пункт
$\exists i: (e_1, \ldots, e_{i-1}, e_{i+1}, \ldots, e_n, f)$\т базис $L$. То есть всегда найдётся вектор базиса, на который можно заменить вектор $f$.
\кзадача

\задача[Лемма о смешанных базисах]
Пусть $(e_1, e_2, \ldots, e_n)$ и $(f_1, f_2, \ldots, f_m)$\т два базиса, $m\ge n$. Тогда $\forall i, 1\le i\le n$ можно построить базис следующего вида: $(e_1, \ldots, e_i, f_{j_1}, \ldots, f_{j_{m-i}}),$ то есть заменить какие-то $i$ векторов второго базиса на векторы первого.
\кзадача

\задача
Докажите, что все базисы состоят из одинакового числа векторов.
\кзадача

\замечание
Число векторов базиса конечномерного пространства~$L$ называется \выд размерностью пространства и обозначается $\dim L$. Предыдущая задача доказывает корректность данного определения.
\кзамечание

\задача
\невСтрочку
\пункт
Какие из пространств задачи 1 конечномерны? Найдите базисы в пунктах a),~б),~и).
\пункт
Пусть $\F$ конечно. Найдите число элементов $n$-мерного пространства над $\F$.
\кзадача

\задача
\невСтрочку
\пункт
Пусть $(e_1, e_2, \ldots, e_n)$ и $(f_1, f_2, \ldots, f_n)$\т два базиса $L$, а $(e_1, f_2, \ldots, f_n)$\т тоже базис. Всегда ли верно, что $(f_1, e_2, \ldots, e_n)$\т базис?
\спункт
Докажите, что в условиях предыдущего пункта существуют такие вектора $e_i$ и $f_j$, что наборы $(e_1, \ldots, e_{i-1}, f_j, e_{i+1}, \ldots, e_n)$ и $(f_1, \ldots, f_{j-1}, e_i, f_{j+1}, \ldots, f_n)$ являются базисами.
\кзадача

\сзадача
\невСтрочку
\пункт
Докажите, что в любом конечном поле есть подполе простого порядка.
\пункт
Докажите, что порядок конечного поля может быть равен только $p^n,$ где $p$ простое.
\кзадача

\сзадача
Существует ли линейное пространство, содержащее ровно $57$ векторов?
\кзадача

%\GenXML
\ЛичныйКондуит{0.3mm}{6.5mm}
%\СделатьКондуит{5mm}{8mm}

\end{document}
