% !TeX encoding = windows-1251
\documentclass[a4paper,12pt]{article}
\usepackage{newlistok}

\УвеличитьВысоту{1.5cm}
\УвеличитьШирину{1.5cm}

\ВключитьКолонитул
\Заголовок{Теория вероятностей\т 3}
\Подзаголовок{Геометрические вероятности}
\НомерЛистка{PT3}
\ДатаЛистка{02.2015}

\begin{document}
\СоздатьЗаголовок

\задача
  На отрезке $[0;3]$ случайным образом выбирается точка. Найдите вероятность того, что точка окажется на промежутке \пункт $[0;1]$; \пункт (0;1).
\кзадача

\задача
  Метровую линейку случайным образом разрезают ножницами. Найдите вероятность того, что длина обрезка составит на менее 80 см.
\кзадача

\задача
  Поезд проходит мимо платформы за 30 секунд. Выглянув в окно, Петя заметил, что поезд проходит мимо платформы. Какова вероятность того, что Петя заметил Васю, стоявшего в середине платформы, если он смотрел в окно 10 секунд?
\кзадача

\задача
  Пассажир приходит на остановку в случайный момент времени. Ему нужен автобус любого из двух маршрутов, идущих с интервалами 10 и 15 минут соответственно. С какой вероятностью пассажиру придётся ждать не более 5 минут?
\кзадача

\задача
  Отрезок разделён на три равные части. Какова вероятность того, что три случайно брошенные точки попадут в три разные части?
\кзадача

\задача
  На отрезке $AB$ случайно выбираются точки $C$ и $D$. Найдите вероятности следующих событий: \пункт $C$ и $D$ совпадут; \пункт $AC\ge AD$; \пункт $AC<BD$; \пункт $AC\ge BD, AC<AD$.
\кзадача

\задача
  На отрезке случайным образом выбираются точки $A, B, C, D$. Какова вероятность того, что отрезок $AB$ целиком лежит внутри отрезка $CD$?
\кзадача

\задача
  На окружности случайным образом выбираются точки $A, B, C, D$. Какова вероятность того, что отрезки $AB$ и $CD$ пересекаются?
\кзадача

\задача
  В прямоугольнике со сторонами 1 и 2 случайным образом выбирают точку. Найдите вероятность того, что расстояние от этой точки до ближайшей стороны не превосходит $\frac{1}{3}$.
\кзадача

\задача
  Плоскость разбита прямыми на \пункт квадраты; \пункт правильные треугольники со стороной 1. С какой вероятностью монета диаметра 1, случайно брошенная на плоскость, закроет одну из вершин сетки?
\кзадача

\задача
  \пункт В окружности провели диаметр. На нём случайно выбирается точка и через неё проводится хорда, перпендикулярная диаметру. Какова вероятность того, что длина хорды будет больше радиуса? \пункт На окружности случайно выбираются две точки. Какова вероятность того, что длина соединяющей их хорды будет больше радиуса? \пункт В круге случайно выбирается точка. Какова вероятность того, что длина наименьшей хорды с серединой в этой точке больше радиуса? \спункт Что можно сказать о вероятности того, что длина случайно выбранной в окружности хорды окажется больше радиуса?
\кзадача

\задача
  \пункт В квадрате с вершинами $(0;0), (0;1), (1;1), (1;0)$ выбирается случайная точка. Найдите вероятность того, что координаты точки удовлетворяют условию $y\le 2x$. \пункт На отрезке $AB$ случайно выбираются точки $M$ и $N$. Найдите вероятность того, что $AM\le 2AN$.
\кзадача

\задача
  На [0;1] отрезке случайно выбираются два числа. Какова вероятность того, что их сумма будет больше $\frac{2}{3}$?
\кзадача

\сзадача
  На окружности случайным образом выбирают три точки. Какова вероятность, что треугольник с вершинами в этих точках будет остроугольным?
\кзадача

\сзадача
  На отрезке случайным образом выбрали две точки. С какой вероятностью из отрезков, на которые разбился исходный, можно составить треугольник?
\кзадача
















\ЛичныйКондуит{0mm}{6mm}
\end{document}



