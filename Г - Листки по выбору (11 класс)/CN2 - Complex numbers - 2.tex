% !TeX encoding = windows-1251
\documentclass[a4paper,12pt]{article}
\usepackage{newlistok}
% \input{haidar_edition}

\УвеличитьШирину{1cm}

\Заголовок{Комплексные числа и геометрия}
\НомерЛистка{CN2}
\ДатаЛистка{11-й "Д" КЛАСС 2011 г}

\begin{document}
\СоздатьЗаголовок

{
\small
Использование комплексных чисел в планиметрии основано на том, что их можно отождествить с точками плоскости: числу $z=a+bi$ соответствует точка с координатами $(a,b)$. При этом квадрат расстояния между точками $z$ и $w$ равен $|z-w|^2=(z-w)(\ol z - \ol w)$.
}

\medskip

\задача[Эйлер]
Сумма квадратов длин сторон четырёхугольника отличается от суммы квадратов диагоналей на учетверённый квадрат длины отрезка, соединяющего середины диагоналей.
\кзадача

\задача
Пусть $M$\т точка на плоскости, $S$\т окружность, $AB$\т её диаметр. Докажите, что величина $MA^2 + MB^2$ не зависит от выбора диаметра $AB$.
\кзадача

\задача[Теорема Лейбница]
Пусть $F$\т центр масс (то есть, точка пересечения медиан) треугольника $ABC$. Докажите, что для любой точки $M$ на плоскости выполнено равенство:
\vskip -2mm
$$
MA^2+MB^2+MC^2=AF^2+BF^2+CF^2+3MF^2.
$$
\vskip -3mm
\кзадача

\задача
На плоскости задано 3 точки $A,B,C$. Точка $A_1$\т образ точки $C$ при повороте вокруг точки $A$ на $90^{\circ}$ против часовой стрелки; точка $B_1$\т образ точки $C$ при повороте вокруг точки $B$ на $90^{\circ}$ по часовой стрелке. Пусть $K$\т середина $A_1B_1$, $M$\т середина $AB$. Докажите, что отрезки $КM$ и $AB$ перпендикулярны. Как соотносятся их длины?
\кзадача

\задача
На сторонах треугольника $A_1A_2A_3$ во внешнюю сторону построены квадраты с центрами $B_1, B_2, B_3$. Докажите, что отрезки $B_1B_2$ и $A_3B_3$ равны по длине и перпендикулярны.
\кзадача

\задача
Пусть $A_1A_2A_3$ и $B_1B_2B_3$\т правильные треугольники, причём их вершины занумерованы в порядке обхода против часовой стрелки. Докажите, что середины отрезков $A_1B_1, A_2B_2$ и $A_3B_3$\т вершины правильного треугольника.
\кзадача

\опр
\выд{Простое отношение} тройки точек $z_1$, $z_2$ и $z_3$\т это комплексное число $\displaystyle\frac{z_1-z_3}{z_2-z_3}$.
\копр

\задача
Докажите, что три точки $z_1,z_2,z_3$ лежат на одной прямой тогда и только тогда, когда их простое отношение вещественно.
\кзадача

\задача[Прямая Эйлера]
В любом треугольнике центр тяжести треугольника, его ортоцентр и центр описанной окружности лежат на одной прямой.
\кзадача


\задача
Докажите, что три точки $z_1,z_2,z_3$ являются вершинами правильного треугольника тогда и только тогда, когда $z_1^2+z_2^2+z_3^2=z_1z_2+z_1z_3+z_2z_3$.
\кзадача

\опр
\выд{Двойное отношение} четвёрки точек $z_1$, $z_2$, $z_3$ и $z_4$\т это число $\displaystyle\frac{z_1-z_3}{z_2-z_3}:\frac{z_1-z_4}{z_2-z_4}$.
\копр

\задача
\вСтрочку
\пункт
Пусть четыре точки $z_1,z_2,z_3,z_4$ лежат на одной окружности. Докажите, что тогда их двойное отношение вещественно.
\пункт
Пусть двойное отношение четырёх точек вещественно. Что можно сказать об их взаимном расположении?
\кзадача

\задача
\пункт Докажите, что $(z_1-z_2)(z_4-z_3)+(z_2-z_3)(z_4-z_1)=(z_2-z_4)(z_3-z_1)$.
\пункт[Птолемей]
Докажите, что в любом четырёхугольнике произведение длин диагоналей не превосходит сумму произведений длин противоположных сторон. Когда достигается равенство?
\кзадача

\задача
\вСтрочку
\пункт
Пусть $z_1$ и $z_2$\т две точки на единичной окружности $|z|=1$. Найдите комплексное число, задающее точку пересечения касательных к этой окружности, проходящих через точки $z_1$ и $z_2$.
\пункт[Задача Ньютона]
В описанном около окружности четырёхугольнике середины диагоналей и центр окружности лежат на одной прямой.
\кзадача

\сзадача
Каждую сторону $n$-угольника в процессе обхода против часовой стрелки продолжили на её длину. Оказалось, что концы построенных отрезков служат вершинами правильного $n$-угольника. Докажите, что исходный $n$-угольник\т тоже правильный.
\кзадача

\сзадача[Теорема Морли]
Трисектрисой угла называют луч, исходящий из вершины угла и отсекающий от угла втрое меньший угол. Понятно, что каждый угол имеет две трисектрисы. В треугольнике $ABC$ пусть $M$\т точка пересечения двух трисектрис, примыкающих к стороне $BC$, $Q$\т точка пересечения двух трисектрис, примыкающих к стороне $CA$ и $P$\т точка пересечения трисектрис, примыкающих к $AB$. Докажите, что треугольник $MPQ$ правильный.
\кзадача

\ЛичныйКондуит{0mm}{6.5mm}

\end{document}
