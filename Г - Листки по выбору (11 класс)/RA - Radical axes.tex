% !TeX encoding = windows-1251
\documentclass[a4paper,12pt]{article}
\usepackage{newlistok}
% \input{haidar_edition}

\УвеличитьШирину{1cm}
\renewcommand{\spacer}{\vfil}

\Заголовок{Радикальные оси}
\НомерЛистка{RA}
\ДатаЛистка{11-й "Д" КЛАСС 2011 г}

\begin{document}
\СоздатьЗаголовок

\опр
\выд{Радикальная ось} двух неконцентрических окружностей\т это множество таких точек $M$, что касательные, проведённые из $M$ к этим окружностям, имеют равные длины.
\копр

\задача
Докажите, что радикальная ось двух непересекающихся окружностей\т прямая. Напишите уравнение этой прямой, если радиусы окружностей имеют длины $r_1$ и $r_2$, а их центры имеют координаты $(-a,0)$ и $(a,0)$ соответственно.
Нарисуйте окружности и их радикальную ось, если $a=5$, $r_1=2$, $r_2=3$ и если $a=2$, $r_1=1$, $r_2=6$.
\кзадача

\задача
Найдите радикальную ось двух
\вСтрочку
\пункт пересекающихся;
\пункт касающихся окружностей.
\кзадача

\опр
Пусть дана окружность $S$ радиуса $r$ с центром в точке $O$. \emph{Степень точки $M$ относительно окружности $S$}\т это число, равное $MO^2-r^2$.
\копр

\задача
Прямая, проходящая через точку $M$, пересекает окружность $S$ в точках $A$ и $B$. Докажите, что степень точки $M$ относительно окружности $S$ равняется произведению длин отрезков $MA$ и $MB$, взятому со знаком (\лк$+$\пк, если векторы $\overrightarrow{MA}$ и $\overrightarrow{MB}$ одинаково направлены, и \лк$-$\пк, если векторы $\overrightarrow{MA}$ и $\overrightarrow{MB}$ противоположно направлены).
\кзадача

\задача
Даны две окружности $S_1$ и $S_2$. Опишите геометрическое место таких точек $M$, что степень $M$ относительно $S_1$ такая же, как и степень $M$ относительно $S_2$.
\кзадача

\опр
Две окружности, пересекающиеся в точках $A$ и $B$, называют \выд перпендикулярными, если касательные, проведённые к ним в точке $A$, пересекаются под прямым углом.
\копр

\задача
Докажите, что радикальная ось двух неконцентрических окружностей $S_1$ и $S_2$ совпадает с множеством центров окружностей, перпендикулярных одновременно и $S_1$, и $S_2$.
\кзадача

\раздел{Пучки окружностей}

\опр
\emph{Пучок окружностей}\т это множество всех окружностей и прямых, перпендикулярных к двум данным окружностям $S_1$ и $S_2$ (или к окружности и прямой, или к двум прямым). Говорят, что $S_1$ и $S_2$ \выд{задают} этот пучок.
\копр

\задача
\label{pencil1}
Нарисуйте пучки, задаваемые двумя неконцентрическими окружностями, которые\\
\вСтрочку
\пункт пересекаются (но не касаются);
\пункт касаются;
\пункт не пересекаются.
\кзадача

\задача
\label{pencil2}
Нарисуйте пучок, задаваемый двумя
\невСтрочку
\пункт параллельными прямыми;
\пункт пересекающимися прямыми;
\пункт концентрическими окружностями.
\кзадача

\задача
Какие пучки могут задаваться прямой и окружностью (нарисуйте)?
\кзадача

\задача
Докажите, что окружность, перпендикулярная некоторым двум окружностям одного пучка, перпендикулярна всем окружностям этого пучка.
\кзадача

\задача
Докажите, что множество окружностей и прямых, перпендикулярных всем окружностям данного пучка, также является пучком (он называется \выд{перпендикулярным} данному).
\кзадача

\задача
Нарисуйте пучки, перпендикулярные пучкам
\пункт из задачи \ref{pencil1};
\пункт из задачи \ref{pencil2}.
\кзадача

\задача
Докажите, что радикальная ось любых двух окружностей одного пучка проходит через центры окружностей, задающих этот пучок. (Таким образом, радикальная ось\т одна и та же для каждых двух окружностей одного пучка.)
\кзадача

\задача
\пункт
Любые ли две окружности принадлежат некоторому пучку окружностей?
\пункт
Если такой пучок существует, то однозначно ли он определяется?
\кзадача

\newpage

\задача
\пункт
Пусть пучок задан двумя пересекающимися (но не касающимися) окружностями. Докажите, что через каждую точку плоскости проходит единственная окружность или прямая пучка.
\пункт
Что можно сказать в случае, когда окружности, задающие пучок, касаются?
\пункт
А~если окружности, задающие пучок, не пересекаются?
\кзадача

\задача
\пункт
Даны две непересекающиеся окружности $S_1$ и $S_2$ из некоторого пучка. Пусть $S$\т ещё одна окружность из того же пучка. Докажите, что для всех точек $M$ на окружности $S$ отношение степени $M$ относительно $S_1$ к степени $M$ относительно $S_2$ одно и то же (обозначим его $k(S)$) и равно $OO_1/OO_2$, где $O,O_1,O_2$\т центры $S,S_1,S_2$ соответственно.
\пункт
Верно ли, что для различных окружностей $S$ и $S'$ нашего пучка числа $k(S)$ и $k(S')$ также различны?
\пункт
Для каких чисел $k$ найдется окружность $S$ из нашего пучка, у которой $k(S)=k$?
\пункт
Что можно сказать, если исходные окружности $S_1$ и $S_2$ касаются или пересекаются?
\кзадача

\задача
Прямая $l$ пересекает две неконцентрические окружности $S_1$ и $S_2$ в точках $A,B$ и $C,D$ соответственно. Пусть $l_A,l_B,l_C,l_D$\т касательные, проведённые к $S_1$ и $S_2$ в соответствующих точках. Докажите, что точки пересечения прямых $l_A,l_B$ с прямыми $l_C,l_D$
\пункт
лежат на радикальной оси $S_1$ и $S_2$, если $l$ проходит через центр подобия этих окружностей;
\пункт
лежат на некоторой окружности $S$ в противном случае.
\пункт
Докажите, что окружность $S$ принадлежит тому же пучку, что и~окружности $S_1$ и $S_2$.
\кзадача

\раздел{Разные задачи}

\задача
Даны три окружности с различными центрами. Проведём для каждой пары из этих окружностей прямую, содержащую радикальную ось этой пары. Докажите, что три проведённые прямые либо параллельны, либо пересекаются в одной точке.
\кзадача

\сзадача
\пункт
Шестиугольник описан около окружности. Докажите, что найдутся три окружности с таким свойством: каждая главная диагональ нашего шестиугольника будет лежать
на радикальной оси каких-то двух из этих окружностей.
\пункт [Теорема Брианшона]
Шестиугольник описан около окружности. Докажите, что его главные диагонали пересекаются в одной точке.
\кзадача

\задача
Докажите, что прямые, проведённые через общие хорды трёх попарно пересекающихся окружностей, пересекаются в одной точке или параллельны друг другу.
\кзадача

\задача
Дана окружность $S_1$ и точка $M$ вне её. Через точку $M$ проводится переменная окружность $S$, пересекающая $S_1$ в точках $A$ и $B$. Найдите геометрическое место точек пересечения прямой $AB$ с касательной к $S$ в точке $M$.
\кзадача

\задача
Как циркулем и линейкой построить радикальную ось двух данных окружностей?
\кзадача

\сзадача
\пункт
Даны точка $A$ и две неконцентрические окружности $S_1$ и $S_2$. Всегда ли найдётся окружность, проходящая через точку $A$ и перпендикулярная окружностям $S_1$ и $S_2$?
\пункт
Как с помощью циркуля и линейки построить такую окружность (если она существует)?
\кзадача

\сзадача
\пункт В выпуклом бумажном многоугольнике сделаны несколько одинаковых круглых дырок. Можно ли разрезать этот многоугольник на несколько меньших выпуклых многоугольников так, чтобы в каждом из них оказалось ровно по одной дырке?
\пункт А если дырки круглые, но не обязательно одинаковые?
\кзадача

\ЛичныйКондуит{0.2mm}{6.5mm}

\end{document}
