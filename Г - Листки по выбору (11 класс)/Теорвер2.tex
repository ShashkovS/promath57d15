% !TeX encoding = windows-1251
\documentclass[a4paper,12pt]{article}
\usepackage{newlistok}

\УвеличитьВысоту{1.5cm}
\УвеличитьШирину{1.5cm}

\ВключитьКолонитул
\Заголовок{Теория вероятностей\т 2}
\Подзаголовок{Зависимые события}
\НомерЛистка{PT2}
\ДатаЛистка{02.2015}

\begin{document}
\СоздатьЗаголовок

Рулетка представляет из себя колесо, разбитое на 37 равных секторов, пронумерованных числами от 0 до 36. Колесо раскручивают, после чего по нему запускают шарик, который случайным образом попадает на один из секторов. При игре в рулетку игрок может делать ставку как на отдельное ненулевое число, так и на различные наборы чисел. В случае, если выпадает одно из чисел, на которые была сделана ставка, игрок получает назад свою ставку, увеличенную в несколько раз. В противном случае, ставка переходит к казино. В случае, если выпадает 0, казино забирает себе все ставки, которые были сделаны в данном раунде.

\задача
  При игре в рулетку Билл каждый раз делает ставку на то, что шарик попадёт на нечётное число. Найдите вероятности \пункт в следующих двух раундах ставка Билла выиграет; \пункт в трёх следующих раундах его ставка сначала выиграет, затем проиграет, затем снова выиграет.
\кзадача

\задача
  За соседним столом рядом с другой рулеткой сидит Том. Том каждый раз делает ставку на чётное ненулевое число. Какова вероятность того, что \пункт в следующих двух раундах ставка Тома выиграет; \пункт и у Билла, и у Тома следующая ставка будет выигрышной; \пункт Следующая ставка Билла будет выигрышной, а Тома --- проигрышной.
\кзадача

\задача
  Придя в казино на следующий день, Билл и Том сели за один рулеточный стол и играли по тем же принципам, что и в прошлый. Изменятся ли ответы на вопросы первых двух задач в этом случае?
\кзадача

\задача
  \невСтрочку
  \пункт Монету подбросили десять раз. Найдите вероятность того, что в третьем броске выпал орёл, а в седьмом --- решка.
  \пункт Десять школьников, среди которых --- Митя и Витя, случайным образом встали в очередь. Полагая, что все расстановки равновероятны, найдите вероятность того, что Митя окажется третьим в очереди, а Витя --- седьмым.
\кзадача

\задача
  Не имея никаких знаний по теории вероятностей, Петя и Боря решали предыдущую задачу. Можно ли считать их решения верными?
  \невСтрочку
  \пункт Петя: <<При третьем броске в половине случаев выпадает орёл, а в половине --- решка. Из них в половине случаев в седьмом броске выпадет решка. Поэтому, искомая вероятность равна~$\frac{1}{2}\cdot\frac{1}{2}=\frac{1}{4}$.>>
  \пункт Боря: <<Митя с одинаковой вероятностью стоит на любом месте. Вероятность того, что он окажется третьим равна $\frac{1}{10}$. Витя с одинаковой вероятностью стоит на любом месте. Вероятность того, что он окажется седьмым равна $\frac{1}{10}$. Значит, вероятность того, что Митя окажется третьим, а Витя --- седьмым равна $\frac{1}{10}\cdot\frac{1}{10}=\frac{1}{100}$.>>
\кзадача

\ввзадача
  Пусть вероятность того, что произойдёт событие $A$ равна $p$, а вероятность события $B$ равна $q$. В каких случаях можно считать, что вероятность того, что произойдёт и событие $A$, и событие $B$ равна~$p\cdot q$?
\кзадача

\задача
  Предположим, что вероятность выпадения <<орла>> при подбрасывании деформированной монеты равна $\frac{1}{3}$. Найдите вероятность того, что после десяти подбрасываниях этой монеты выпадет \пункт ровно 5 <<орлов>>; \пункт хотя бы 5 <<орлов>>; \пункт выпадение какого количества <<орлов>> является наиболее вероятным?
\кзадача

\ЛичныйКондуит{0mm}{6mm}
\ОбнулитьКондуит

\сзадача
  В барабане находятся 7 белых шаров с номерами от 1 до 7 и 5 чёрных с номерами от 8 до 12. Все белые шары одинаковы, все чёрные --- тоже одинаковы, но отличаются по размеру от белых. В барабане имеется отверстие, через которое может выпасть любой шар, причём вероятность выпадения любого белого шара равна 0,09, а любого из чёрных --- 0,07. Найдите вероятность того, что среди семи последовательно выпавших из барабана шаров ровно  три окажутся чёрными, если после выпадения шара его возвращают обратно в барабан.
\кзадача

\задача
  Решите предыдущую задачу, если вероятность выпадения белого шара равна 0,1, а вероятность выпадения чёрного равна 0,06.
\кзадача

\ссзадача
  Можно ли решить предыдущую задачу в случае, если после выпадения очередного шара его не возвращают в барабан. Если можно --- объясните, как решить (решать не обязательно), если нельзя --- укажите, чего не хватает в условии.
\кзадача

\зам
    В задаче с невозвратом шаров полезно получить оба ответа.
\кзам

\задача
  Известно, что вероятность выпуска сверла повышенной хрупкости равна 0,02. Сколько надо класть в коробку сверл, чтобы с вероятностью, не меньшей 0,9, в ней было не менее 100 исправных?
\кзадача

\задача
  Десять членов Ордена Хаоса по очереди кладут в урну по шару чёрного или белого цветов. Каждый из них кладёт в урну чёрный шар с вероятностью $\frac{1}{6}$. После этого из урны достают три шара. Какова вероятность того, что хотя бы 2 из них белые?
\кзадача

\задача
  Тест состоит из 12 вопросов, каждый из которых имеет 4 варианта ответа. Школьник отвечает на вопросы наобум. Какова вероятность получить положительную оценку, если для этого достаточно верно ответить на 4 вопроса? Как изменится эта вероятность, если школьник знает ответ на первый вопрос?
\кзадача

\задача
  Вика делает ошибку в сложном слове в диктанте с вероятностью $\frac {1}{10}$. Если в диктанте допущено хотя бы 2 ошибки, ставится оценка <<4>>, если 5 --- <<3>>, если 8 --- <<2>>. Всего в диктанте 12 сложных слов, а в остальных словах Вика не ошибается. Найдите вероятность того, что Вика получит \пункт <<5>>; \пункт <<4>> или <<3>>; \пункт <<2>>.
\кзадача

\задача
  Два снайпера одновременно стреляют по цели. Первый попадает с вероятностью $\frac{1}{2}$, а второй --- $\frac{1}{3}$. С какой вероятностью цель будет поражена?
\кзадача

\задача
  Васе обещают приз, если он выиграет подряд две теннисные партии против своего тренера и своего приятеля по одной из схем: <<тренер-приятель-тренер>> или <<приятель-тренер-приятель>>. Тренер играет лучше приятеля. Какую схему следует выбрать Васе?
\кзадача

\задача
  В телеигре ведущий предлагает игроку угадать, за какой из трёх закрытых дверей находится автомобиль. Играющий выбирает одну из дверей. После этого ведущий, зная, где находится автомобиль, открывает одну из дверей, на которые игрок не указывал, и даёт возможность игроку изменить своё решение. Имеет ли смысл игроку менять решение?
\кзадача














\ЛичныйКондуит{0mm}{6mm}
\end{document}



