% !TeX encoding = windows-1251
\documentclass[a4paper,12pt]{article}
\usepackage{newlistok}

\УвеличитьВысоту{1.5cm}
\УвеличитьШирину{.8cm}

\sloppy

\renewcommand{\spacer}{\vfil} % распределяет задачи по всей странице

\Заголовок{Цепные дроби}
\НомерЛистка{36ТЧ}
\ДатаЛистка{01.2009}

\begin{document}
\СоздатьЗаголовок

\опр
    Цепной дробью (конечной или бесконечной) $[a_0;a_1,a_2,\ldots]$, где   $a_0\in\Z$, $a_1,a_2.\ldots\in\N$ называется выражение вида   $$a_0+\dfrac{1}{a_1+\dfrac{1}{a_2+\cdots}}\ .$$
   Рациональные число $S_n=[a_0;a_1\sco a_n]$ называется $n$-ой подходящей дробью цепной дроби   $[a_0;a_1,a_2,\ldots]$. Числитель $n$-ой подходящей дроби $S_n$ обозначается $p_n$, знаменатель --- $q_n$, так что и $p_n$, и $q_n$ являются многочленами от букв $a_0\sco a_n$, например, $p_1=a_0a_1+1$, $q_1=a_1$.
\копр



\задача
   \пункт Выразите $p_n$ и $q_n$ через числители и знаменатели $S_{n-1}$ и $S_{n-2}$.
  \невСтрочку
  \пункт докажите тождество: $\ S_{n+1}-S_n=\dfrac{(-1)^n}{q_nq_{n+1}}$.
  \пункт докажите, что всякая подходящая дробь несократима.
\кзадача


\задача
   Докажите, что для всякой цепной дроби последовательность подходящих дробей имеет предел.
\кзадача

Таким образом, корректно определено {\it значение} цепной дроби.


\задача
   Для всякой конечной цепной дроби $[a_0;a_1\sco a_n]=[a_0;a_1\sco a_n-1,1]$.   Докажите, что это единственный случай, когда разные цепные дроби могут иметь одинаковые значения.
\кзадача


\задача
   Докажите, что всякое вещественное число единственным (в смысле предыдущей задачи) образом   представляется в виде цепной дроби. Опишите алгебраически и геометрически   алгоритм представления числа в виде цепной дроби.
\кзадача
\задача
   \пункт Пусть $\text{НОД}(m,n)=1$. Как, зная разложение $m/n$ в цепную дробь, найти   решение диофантова уравнения $mx+ny=1$?\\
   \пункт Пусть $m$ и $n$ - целые ненулевые числа. Как найти $\text{НОД}(m,n)$ из представления  $m/n$ в виде цепной дроби? Какое отношение к представлению рациональных чисел цепными дробями   имеет алгоритм Евклида?
\кзадача


\задача
   \пункт Вещественное число называется {\it квадратичной иррациональностью}, если оно является корнем квадратного уравнения с целыми коэффициентами. Докажите, что цепная дробь,   представляющая квадратичную иррациональность, периодична.\\
   \спункт докажите обратное утверждение: если цепная дробь периодична, то она   представляет квадратичную иррациональность.
\кзадача



\опр
   Показателем качества (или коэффициентом качества) приближения $p/q$ числа   $\alpha$ (где $p,q\in\Z,\alpha\in \Q$) называется число   $q\cdot|\alpha-\frac{p}{q}|.$   Из двух приближений лучшим считается то, у которого показатель качества меньше.
\копр



\задача
   Пусть $\alpha$ - вещественное число. Докажите, что для любого $q\in\N$   существует рациональное приближение $p/q$ числа $\alpha$ с показателем качества меньшим $1/2$.
\кзадача


\задача
  \пункт Какое из приближений числа $\sqrt{2}$ лучше: $3/2; 7/5;$ или $1,41$?\\
  \пункт Какое из приближений числа $\pi$ лучше: $3; 3,14;$ или $22/7$?
\кзадача


\задача
   \пункт Докажите, что $n$-я подходящая дробь приближает значение цепной дроби   с показателем качества не более $1/q_{n+1}$\\
   \спункт Докажите, что абсолютная погрешность приближения $\alpha-\dfrac{p_n}{q_n}$
   иррационального числа $\alpha$ $n$-ой подходящей дробью является наименьшей   среди приближений дробями со знаменателями, не превосходящими $q_n$.
\кзадача




\задача
   \пункт Найдите три первых подходящих дроби числа $\pi$ (соответственно, приближения   Архимеда и Меция). Каковы у них показатели качества?\\
   \спункт Длина астрономического года равна 365 дней 5 часов 48 минут 46 секунд. На каком   приближении основан юлианский календарь? Можете предложить что-либо лучшее   (например, система Омара Хаяма дает погрешность в 19 секунд в отличие от 11 минут Цезаря)
\кзадача


%\СделатьКондуитИз{6.2mm}{6.2mm}{sp_Tch.tex}


\end{document} 