% !TeX encoding = windows-1251
\documentclass[a4paper,10pt]{article}
\usepackage{newlistok}

\renewcommand{\spacer}{\vfil}
\УвеличитьШирину{1.5cm}
\УвеличитьВысоту{1.5cm}

\Заголовок{Вариация кривой}
\НомерЛистка{36:Анализ}
\ДатаЛистка{01.2009}

\begin{document}


\СоздатьЗаголовок


\опр
Пусть на плоскости дан отрезок $AB$ длины 1.
Для каждой прямой $l$ в этой плоскости
определим \выд{вариацию отрезка $AB$ в направлении $l$} как длину
проекции отрезка $AB$ на прямую $l$ (рис.~1).
Обозначение: $V_l(AB)$ или просто
$V_l$, если ясно, от какого отрезка берется вариация.
\копр

\опр
\putpict{15.4cm}{-0.5cm}{list36An_1.eps}{}
Интуитивно ясно, что существует среднее значение вариации по
всем направлениям
\УстановитьГраницы{4.5truecm}{3truecm}
\noindent
\putpict{-2.5cm}{-1.1cm}{list36An_2}{}
и что оно больше 0 и меньше 1. Более
точно это означает, что если разделить угол в $360^\circ$
на $n$ равных частей и взять среднее
арифметическое
\vspace{-2mm}
$$
\quad \quad V_n=\frac{V_{l_1}+V_{l_2}+\dots+V_{l_n}}{n}
$$
\vspace{-6mm}
\УстановитьГраницы{4.5truecm}{0truecm}
\noindent
вариаций отрезка $AB$ в направлениях $l_1$, $l_2$, \dots, $l_n$
(рис.~2), то существует предел $\lim\limits_{n\rightarrow\infty}V_n=K$,
причем $K$ заключено между 0 и 1.
Это число $K$ называется \выд{вариацией} единичного
отрезка $AB$.
\ВосстановитьГраницы
\копр

\задача
Найдите вариацию отрезка длины $a$ (выразите через $K$).
\кзадача


\опр
Введем на плоскости систему координат. Для каждого $\alpha$ из промежутка
$[0;\pi]$ пусть $f(\alpha)$ --- вариация
отрезка $AB$ в направлении прямой, выходящей из начала координат под
углом $\alpha$ к оси абцисс.
\выд{Вариацией} единичного отрезка $AB$ называется число
%\vspace{-2mm}
$%$
K=\dfrac{1}{\pi}\int\limits_0^{\pi}f(\alpha)\, d\alpha.
$%$
%\vspace{-7mm}
\копр

\задача
Докажите, что определения 2 и 3 эквивалентны.
\кзадача

\задача
Докажите, что число $K$ существует,
и найдите его.
\кзадача



\УстановитьГраницы{0truecm}{3.7truecm}
\опр
\putpict{15cm}{-0.5cm}{list36An_3}{}
\выд{Вариацией ломаной по какому-нибудь направлению} называется
сумма длин проекций ее звеньев на это направление (рис.~3).
\копр

\задача
Найти вариации единичного квадрата %со стороной 1
в направлениях %его
сторон и диагоналей.
\кзадача

\опр
\выд{Средняя вариация ломаной} (или просто \выд{вариация ломаной})
по всем \ВосстановитьГраницы\noindent направлениям определяется, как и выше, с помощью предельного
перехода или помощью интеграла.
\копр


\задача
Найдите вариацию ломаной длины $a$.
\кзадача

\hangindent=4truecm\hangafter=4
\putpict{1.4cm}{-2.7cm}{list36An_4}{}
Перенесение понятия вариации на кривые требует уточнения понятия кривой.
В общем случае это сделать трудно. Но пусть кривая выпуклая или состоит
из нескольких выпуклых кусков. Тогда при проектировании кривой на любое,
но определенное направление, можно разбить ее на конечное число кусков,
каждый из которых пересекается только один раз любой
проектирующей прямой (здесь не исключаются случаи, когда подобный кусок
представляет собой прямолинейный отрезок и, следовательно, при
проектировании в одном из направлений полностью попадает на
проектирующую прямую).  Тогда \выд{вариацией кривой по выбранному
направлению} назовем сумму длин проекций ее кусков на это направление
(рис.~4.). Можно показать, что существует среднее значение этой величины
по всем направлениям. Его мы и назовем
\выд{средней вариацией} или просто \выд{вариацией}) кривой линии.
Очевидно, если кривая --- ломаная,
то мы приходим к прежнему определению.

\задача
Найдите вариацию окружности диаметра $D$.
\кзадача

\hangindent=-3.5truecm\hangafter=-4
\putpict{17.5cm}{-.3cm}{list36An_5}{}
Выберем теперь на кривой несколько точек и соединим их последовательно,
но подряд (рис.~5.) Получим ломаную. Можно показать, что для достаточно
хороших (например, для кривых, которые могут быть разбиты на
конечное число выпуклых кусков) кривых существует предел длин
этих ломаных, при условии, что при изменении ломаной длина ее наибольшего
звена стремится к нулю. Этот предел называется \выд{длиной кривой}.



\задача
Найдите предел, к которому стремится вариация ломаной, вписанной в
\лк достаточно хорошую\пк\ кривую длины $a$, когда ломаная изменяется
так, что длина наибольшего ее звена стремится к нулю.
\кзадача

\задача
Найдите вариацию \лк достаточно хорошей\пк\ кривой длины $a$.
\кзадача

\задача
Найдите %численное значение $K$, то есть
вариацию отрезка длины $1$,
используя результаты задач 4, 5 и 6.
\кзадача


\опр
\выд{Шириной кривой по данному направлению} называется наименьшее
расстояние между двумя прямыми этого направления, между которыми
лежит кривая.
Кривая имеет \выд{постоянную ширину}, если ее ширина по всем направлениям
одинакова.
Простейшим примером такой кривой % постоянной ширины
является окружность.
\копр


\УстановитьГраницы{3cm}{0cm}
\задача
\putpict{-3.6cm}{-1.3cm}{list36An_6}{}
Отметим на плоскости вершины произвольного правильного треугольника
и соединим каждые две вершины дугой окружности с центром в третьей
вершине. Получится так называемый \выд{треугольник Релло}
(см.~рис.~6). Докажите, что треугольник Релло
является кривой постоянной ширины.
\кзадача



\задача[Теорема Барбье]
Пусть кривая постоянной ширины $h$ является границей выпуклой фигуры.
Найдите длину такой кривой.
\кзадача


\задача
В круге радиуса 1 заключена какая-то кривая $L$ длины 22. Докажите, что
найдется прямая, пересекающая $L$ не менее чем в 8 точках.
\кзадача

\ВосстановитьГраницы

\задача
Почему канализационные люки делают круглыми и причём здесь кривые постоянной ширины?
\кзадача

\задача
Один прямоугольник находится внутри другого. Докажите, что тогда его периметр меньше.
\кзадача


%\СделатьКондуитИз{6.2mm}{6.2mm}{sp_An.tex}

\end{document}



