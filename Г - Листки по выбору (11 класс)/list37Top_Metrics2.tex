% !TeX encoding = windows-1251
\documentclass[a4paper,12pt]{article}
\usepackage{newlistok}

%\УвеличитьВысоту{1.5cm}
%\УвеличитьШирину{.1cm}
\renewcommand{\spacer}{\vfil}

\sloppy

\Заголовок{Метрические пространства: полнота}
\НомерЛистка{37Топ}
\ДатаЛистка{01.2009}

\begin{document}
\СоздатьЗаголовок

\опр
Последовательность $(x_n)$ точек метрического пространства $X$ называется \выд фундаментальной, если для любого $\ep>0$ найдётся номер $N\in\N$ такой, что если $m,n>N$, то $d(x_m,x_n)<\ep$.
\копр

\задача
\пункт
Докажите, что сходящаяся последовательность является фундаментальной.
\пункт
Верно ли обратное?
\кзадача


\опр
Метрическое пространство $X$ называется \выд полным, если любая фундаментальная последовательность в нём является сходящейся.
\копр

\задача
Докажите, что вещественная прямая с естественной метрикой полна.
\кзадача

\задача
Докажите, что пространство $C([a,b])$ с равномерной метрикой является полной.
\кзадача

\опр
Отображение $f\colon X \to X$ из топологического пространства $X$ в себя называется \выд сжимающим, если найдётся константа $0<\theta<1$, что для любых $x,y\in X$ верно: $d(f(x),f(y))<\theta d(x,y)$.
\копр

\задача
При каких условиях гомотетия является сжимающим отображением?
\кзадача

\задача
\пункт
Докажите, что сжимающее отображение $f$ полного метрического пространства $X$ имеет неподвижную точку, то есть найдётся $x\in X$, что $f(x)=x$.
\пункт
Верно ли это в не полном метрическом пространстве?
{\reflectbox{\hbox{{\tiny Подсказка к а): докажите, что любая последовательность $(f^n(x))$ фундаментальна}}}}
\кзадача

\задача
\пункт
Докажите, что композиция гомотетии с коэффициентом, не равным $\pm 1$ и любого движения имеет неподвижную точку;
\пункт
Докажите, что это преобразование является гомотетией.
\кзадача

\задача
Пусть функция $\al(x)$ дважды непрерывно дифференцируема (то есть вторая производная непрерывна) на отрезке $[a,b]$, имеет на нём корень $\wt{x}$, причём $f'(x)\ne0$ всюду на $[a,b]$.
Рассмотрим функцию $f(x) =  x - \dfrac{\al(x)}{\al'(x)}$.\\
\пункт
Докажите, что $\al(\wt{x}) = 0$ тогда и только тогда, когда $f(\wt{x})=\wt{x}$;\\
\пункт
Докажите, что $f$ и $f'$ непрерывны;\\
\пункт
Докажите, что найдётся такое $\de>0$, что $f$ на $U_\de(\wt{x})$ осуществляет сжимающее отображение.\\
\пункт
Что всё это значит и как это применять?\\
\пункт
Найдите $\sqrt{2}$ с точностью до трёх знаков после запятой.
\кзадача

%\СделатьКондуитИз{6.2mm}{6.2mm}{sp_Top.tex}


\end{document}
