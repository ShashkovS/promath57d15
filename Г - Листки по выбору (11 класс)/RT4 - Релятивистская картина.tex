% !TeX encoding = windows-1251
\documentclass[a4paper,12pt]{article}
\usepackage{newlistok}

\УвеличитьВысоту{2.4cm}
\УвеличитьШирину{1.5cm}
\renewcommand{\spacer}{\vfil}
\newcommand{\wa}{\overrightarrow}
\newcommand{\smat}[1]{\hr{\begin{smallmatrix}#1\end{smallmatrix}}}
\newcommand{\мв}{\,м$_в$}
\DeclareMathOperator{\arcsh}{arcsh}
\DeclareMathOperator{\arcch}{arcch}
\DeclareMathOperator{\arcth}{arcth}


%\newcommand{\help}[1]{({\small Подсказка: }{\reflectbox{\hbox{\footnotesize{#1}}}})}

\Заголовок{Специальная теория относительности}
\НомерЛистка{RT4}
\ДатаЛистка{02.2015}

\sloppy

\begin{document}
\СоздатьЗаголовок

{\tiny
Опыты Майкельсона--Морли и Кеннеди--Торндайка показали, что скорость света \лк почти\пк не зависит от системы отсчёта. Если быть точнее, то было установлено, что скорости света во всех направлениях в двух системах отсчёта, двигающихся относительно друг друга со скоростью 60\,км/с, отличаются не более, чем на 2\,м/с. Позднее, постоянство скорости света было проверено множеством различных способов и с куда большей точностью.

}{\small
Постулат СТО: скорость света постоянна во всех системах отсчёта. Ничто не может передвигаться быстрее скорости света. Скорость света обозначается через $c$. ($c\approx299 792 458$\,м/с)
Преобразование пространства-времени $\R^4$, удовлетворяющие этому условию называются преобразованиями \выд Лоренца.

Для удобства будем измерять время в метрах$_в$ (и писать \мв). \мв --- время, за которое свет пролетает один метр. На занятиях полезно иметь с собой хороший калькулятор.

}

\задача
Выразите одну величину через другую (и обратно):\\
\пункт \мв через секунду ;
\пункт скорость в м/с через скорость в м/\мв;
\пункт год времени через \мв;\\
\пункт ускорение свободного падения $g=9{,}8$\,м/с через ускорение в м/\мв$^2$.
\кзадача



Разберёмся сначала с \лк одномерным\пк миром. То есть множество событий --- $\R^2$.
Каждую задачу нужно решить \лк дважды\пк: считая, что время измеряется в метрах и в секундах.



\задача
Найдите все возможные мировые линии света в $\R^2$.
\кзадача


\задача
Опишите преобразования Лоренца в $\R^2$.
\кзадача

\задача
\пункт
На обычной плоскости заданы два обычных вектора $(x_1,y_1)$ и $(x_2,y_2)$. Докажите, что площадь параллелограмма, натянутого на эти вектора равна $x_1y_2-x_2y_1$. Это число называется \выд определителем матрицы $\smat{x_1&y_1\\x_2&y_2}$.
\пункт
Что происходит с определителем, если переставить строки или столбцы?
\пункт
Если умножить первую строку на число?
\пункт
Если к первой строке прибавить вторую?
\кзадача

\задача
Объясните, почему имеет смысл рассматривать только преобразования с определителем, равным единице (строго говоря, только они называются преобразованиями Лоренца).
\кзадача

\задача
Как перейти в систему отсчёта ракеты, летящей со скоростью $u$ вправо? Как обратно?
\кзадача


\задача
Изобразите в $\R^2$ и $\R^3$ множество точек: \пункт в которые можно попасть из данной (это множество называется \выд конусом \выд будущего); \пункт в которые можно посветить из данной; \пункт из которых можно попасть в данную (\выд конус \выд прошлого). Какой физический смысл конуса будущего и прошлого?
\кзадача

\задача
Докажите, что для любой пары различных событий найдётся ракета, в системе которой события либо одновременны (говорят, что интервал между ними \выд пространственноподобный), либо происходят в одной и той же точке пространства (интервал временноподобный), либо принадлежат мировой линии света (интервал \выд светоподобный).
\кзадача

\задача
Пускай одно из событий находится в начале координат. Найдите множество точек пространства-времени, для которых интервал пространственно-, временно- и светоподобный.
\кзадача

\задача[Парадокс поезда]
Пусть на поезде, движущемся со скоростью, близкой к скорости света (такой поезд, видимо, стоит ожидать раньше всего в Японии (если где-нибудь ещё не научатся значительно влиять на скорость света)), едут три человека: $A$ в голове, $O$ --- в середине и  $B$ --- в хвосте поезда. На земле около пути стоит четвёртый человек $O'$. В тот самый момент, когда $O$ проезжает мимо $O'$, сигналы ламп от $A$ и $B$ достигают $O$ и $O'$.
Покажите, что на вопрос\лк Кто раньше включил фонарь?\пк наблюдатели $O$ и $O'$ дадут различные ответы.
\кзадача

\задача
\пункт
Покажите, что если два события происходят одновременно и в одном и том же месте в одной системе отсчёта, то они будут одновременными в любой другой системе отсчёта.
\пункт
Покажите, что если два события происходят одновременно в разных точках в одной системе отсчёта, то они не будут одновременными ни в какой другой системе отсчёта.
\кзадача

\задача
\пункт
Покажите, что ни время, ни расстояние не являются инвариантными в СТО.
\пункт
Докажите, что для любых событий, соединяемых пространственноподобным интервалом найдётся две ракеты такие, что в системе одной первое событие происходит раньше, а в системе второй --- наоборот.
\кзадача

\задача
\пункт
Придумайте, как реализовать матрицу преобразования Лоренца подобно матрице поворота. Аргумент, похожий на угол в матрице поворота называется \выд параметром \выд скорости.\\
\пункт
Выразите всё через всё и обратно.
\кзадача

\задача
Объясните, как складываются скорости и параметры скорости.
\кзадача

%\GenXMLW
%%\СделатьКондуитИз{6.2mm}{6.2mm}{sp_STO.tex}
\newpage

\раздел{Для тех, кто в танке}


\задачан3
\невСтрочку
\пункт
Пусть $A$ --- линейная часть замены координат. Докажите, что $A\smat{1\\1} = \smat{a\\a}$.
\пункт
Докажите, что $A\smat{1\\-1} = \smat{a\\-a}$.
\пункт
Найдите вид матрицы $A$.
\кзадача

\задачан{4}
Как найти площадь параллелограмма?
\кзадача

\задачан5
\невСтрочку
\пункт
Пусть в базисе $\hc{e_x,e_t}$ замена имеет матрицу $A$ с определителем $\det A$. Каков геометрический смысл знака определителя?
\пункт
Докажите, что имеет смысл рассматривать только преобразования с положительным определителем.
\пункт
Как Вы думаете, есть ли способ \лк надёжно\пк измерить метр расстояния и метр времени, то есть так, чтобы результаты были одинаковы в любой системе отсчёта?
\пункт
Докажите, что матрицу с положительным определителем можно представить в виде произведения матрицы с единичным определителем и константы.
\кзадача

\задачан6
\невСтрочку
\пункт
Опишите мировую линию ракеты в системе отсчёта лаборатории и ракеты.
\пункт
Какое преобразование переводит мировую линию ракеты в системе отсчёта лаборатории в мировую линию ракеты в системе отсчёта ракеты?
\кзадача


\задачан8
\невСтрочку
\пункт
\help{Решите задачу 9.}
\пункт
\help{Воспользуйтесь задачей 6.}
\кзадача


\задачан{13}
\невСтрочку
\пункт
По определению $\ch \ph = \dfrac{e^\ph + e^{-\ph}}{2}$, $\sh \ph = \dfrac{e^\ph - e^{-\ph}}{2}$, $\th\ph = \dfrac{\sh\ph}{\ch\ph}$, $\cth{\ph}=\dfrac{\ch\ph}{\sh\ph}$. Докажите, что $\ch^2\ph - \sh^2\ph = 1$.
\пункт
Пусть $\th \ph = u$, и $A = \smat{\ch\ph&-\sh\ph\\-\sh\ph&\ch\ph}$. Является ли $A$ преобразованием Лоренца?
\пункт
Куда переходит мировая линия центра лаборатории при замене координат?
\пункт
Вычислите $\arcsh u$, $\arcch u$ и $\arcth u$ через $u$.
\кзадача

\vfil\vfil\vfil\vfil\vfil\vfil\vfil\vfil\vfil\vfil\vfil\vfil\vfil\vfil\vfil\vfil\vfil\vfil
\vfil\vfil\vfil\vfil\vfil\vfil\vfil\vfil\vfil\vfil\vfil\vfil\vfil\vfil\vfil\vfil\vfil\vfil\vfil\vfil\vfil\vfil\vfil\vfil\vfil\vfil\vfil\vfil\vfil\vfil\vfil\vfil\vfil\vfil\vfil\vfil

\end{document}


