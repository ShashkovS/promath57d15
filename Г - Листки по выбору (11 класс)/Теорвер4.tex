% !TeX encoding = windows-1251
\documentclass[a4paper,12pt]{article}
\usepackage{newlistok}

\УвеличитьВысоту{1.5cm}
\УвеличитьШирину{1.5cm}
%\УвеличитьПромежутки{98}
\УвеличитьПробелы{-2mm}{-1mm}

\ВключитьКолонитул
\Заголовок{Теория вероятностей\т 4}
\Подзаголовок{Вероятностное пространство. Условная вероятность}
\НомерЛистка{PT4}
\ДатаЛистка{04.2015}

\begin{document}
\def\exec{\makebox[25pt]{\put(-5,3){\line(0,1){3}}\put(-5,3){\vector(1,0){11}}}}
\def\вып{\makebox[22pt]{\put(-5,3){\line(0,1){3}}\put(-5,3){\vector(1,0){11}}}}
\СоздатьЗаголовок


\задача
  Среди учеников школы $15\%$ знают французский язык и 20\% знают немецкий. Доля учеников, знающих оба языка, составляет 5\%.
  \пункт Какова доля учеников, знающих французский язык, среди учеников, знающих немецкий? \пункт Какова доля учеников, знающих немецкий, среди знающих французский? \пункт Какова доля учеников, знающих французский, среди учеников, \выд не знающих немецкого?
\кзадача %Эта задача под номером "первой" используется в тексте

\опр
  Дополнение к событию $A$ называется событие \выдж{<<не $А$>>} и обозначается $\overline{A}$.
\копр

\опр
  События $A$ и $B$ называются \выдж несовместимыми или \выдж несовместными, если одновременное их выполнение невозможно.
\копр

\опр
  \выдж Произведением событий $A$ и $B$ называется событие, отвечающее одновременному выполнению событий $A$ и $B$. Обозначения: $A\cap B$, $A\cdot B$, $AB$.\\
  \выдж Суммой событий $A$ и $B$ называется событие, при котором выполнено хотя бы одно из событий $A$ и $B$. Обозначения: $A\cup B$, $A+B$.
\копр

\опр
  \выдж {Вероятностным пространством} называется тройка $(\Om, \Us, \Pf)$, где
  \begin{items}{-3}
  \item[--] $\Om$ --- некоторое множество (\выдж {множество элементарных событий});
  \item[--] $\Us$ --- совокупность подмножеств множества $\Om$ (каждое из которых называется \выдж событием), обладающая следующими свойствами:
      \begin{items}{-4}
        \item[\textbullet] $\es\in\Us$;
        \item[\textbullet] $\Om\in\Us$;
        \item[\textbullet] $\fa A\in\Us\exec\overline{A}\in\Us$;
        \item[\textbullet] $\fa A,B\in\Us\exec A\cup B\in\Us$;\\%\vspace{-0.3cm}
        В случае счётного множества $\Us$, последнее свойство преобразуется следующим образом: $\fa\As\subset\Us\exec\bigcup\limits_{A\in\As}A\in\Us$.
      \end{items}
      \vspace{-0.35cm}
      в случае конечного множества $\Om$ часто удобно принять $\Us=2^\Om$.
  \item[--] $\Pf$ --- числовая функция $\Pf\colon\Us\mapsto\R$ (называемая \выдж вероятностью, \выдж{вероятностной мерой}), такая, что
      \begin{items}{-4}
        \item[\textbullet] $\Pf(\es)=0$;
        \item[\textbullet] $\Pf(\Om)=1$;
        \item[\textbullet] $\fa A\in\Us\exec\Pf(A)\ge0$
        \item[\textbullet] \выд{(аддитивность вероятностной меры)} если $A\cap B=\es$, то $\Pf(A\cup B)=\Pf(A)+\Pf(B)$.\\
        В случае счётного множества $\Us$, последнее свойство преобразуется следующим образом: Для любого набора событий, каждые два из которых несовместны, вероятность их суммы равна сумме их вероятностей.
      \end{items}
      \vspace{-0.35cm}
      События, вероятность которых равна 1, называются \выдж достоверными.
  \end{items}
\копр

\vspace{-0.6cm}

\зам
  Следует понимать, что когда мы имеем дело с реальными экспериментами и событиями, элементарными называются только те события, которые \выд {нельзя разделить на более простые}.
\кзам

\vspace{-0.15cm}

\задача
  Пусть $(\Om,\Us,\Pf)$ --- вероятностное пространство. Докажите, что \пункт вероятность любого события не превосходит 1; \пункт если $A,B\in\Us$, причём $A\subset B$, то $\Pf(A)\le\Pf(B)$.
\кзадача

\vspace{-0.2cm}

\задача
  Докажите, что $\fa\As\subset\Us\exec\bigcap\limits_{A\in\As}A\in\Us$.
\кзадача

\vspace{-0.2cm}

\задача
  Выразите вероятность события $\overline{A}$ через вероятность события $A$.
\кзадача

\vspace{-0.2cm}

\задача
  \пункт Постройте вероятностное пространство для $n$-кратного бросания игральной кости.\\
  \пункт Что вероятнее: при шести бросаниях получить хотя бы одну <<шестёрку>> или не получить ни одной?
\кзадача

\vspace{-0.2cm}

\задача
   Постройте вероятностное пространство для пунктов а), б), и в) задачи 11 листка PT3.
\кзадача

\задача
  Рассмотрим задачу выбора точки на отрезке $[0;3]$. Можно ли построить пространство так, что бы были равны вероятности попадания на отрезки \пункт $[0;1]$ и $[1;3]$; \пункт $[0;1]$ и $[0;2]$; \пункт $[0;1]$ и $[2;3]$? \пункт Можно ли построить пространство так, что бы эти вероятности были по $\frac{1}{2}$?
\кзадача

\задача
  Для следующих задач постройте вероятностное пространство и решите их:
  \невСтрочку
  \пункт Пишется наудачу двузначное число. Какова вероятность того, что сумма цифр этого числа окажется равна $i$, где $i\in\N$?
  \пункт Игральный кубик бросают два раза и складывают выпавшие очки. Назовём $q(i)$ вероятность получить в сумме число $i$. Найдите наиболее вероятное значение $q(i)$.
  \пункт Игральный кубик бросают четыре раза. Найдите вероятность того, что хотя бы один раз выпадет <<шестёрка>>.
  \пункт Четыре игральных кубика бросают одновременно. Найдите вероятность того, что хотя бы на одном из них выпадет <<шестёрка>>.
\кзадача %Здесь задачу не добавлять

\задача
  Имеется три ящика, в каждом из которых лежат шары с номерами от 0 до 9. Машина выбирает по одному шару из каждого ящика. Достройте вероятностное пространство и найдите зависимость вероятности того, что все три шара имеют номер $n$ от числа $n$, если
  \невСтрочку
  \пункт все шары вынимаются с одинаковой вероятностью;
  \пункт Каждый шар вынимается с вероятностью $q(i)$, где $q(i)$ взято из предыдущей задачи, а~$i$~--- номер на шаре?
  \vspace{-0.2cm}
  \пункт Каждый шар вынимается с вероятностью $\frac{q(i)}{Q}$, где $q(i)$ взято из предыдущей задачи, $Q=\sum\limits_{i=0}^{9}q(i)$, а~$i$~--- номер на шаре?
\кзадача

\задача
  Имеется $n$ событий, вероятность каждого из которых равна $p$. Покажите, что вероятность того, что произойдут одновременно $k$ из них, не превышает $\frac{n\cdot p}{k}$.
\кзадача

%\ЛичныйКондуит{0mm}{6mm}
%\ОбнулитьКондуит
%
\опр
  \выдж{Условной вероятностью} называется вероятность события $A$ при условии, что событие $B$ произошло. Обозначения: $\Pf_B(A)$, $\Pf(A|B)$.
\копр

\задача
  Выразите $\Pf(A|B)$ через $\Pf(A)$ и $\Pf(B)$.
\кзадача

\задача
  Рассмотрим вероятностное пространство $(\Om,\Us,\Pf)$ и событие $A$ такое, что $\Pf(A)\ne0$. Докажите, что тройка $(\Om,\Us,\Pf_A)$ так же является вероятностным пространством.
\кзадача

\задача
  Переформулируйте первую задачу в терминах теории вероятностей.
\кзадача

\задача
  В классе 50\% мальчиков. Среди них 60\% любят мороженое. \пункт Какова доля мальчиков, любящих мороженое, среди учеников класса? \пункт Как переформулировать вопрос предыдущего пункта в терминах теории вероятностей?
\кзадача

\задача
  Что больше: $\Pf(A)$ или $\Pf(B)$, и во сколько раз, если  $\Pf(A|B)=\frac{1}{7}$, а $\Pf(B|A)=\frac{1}{9}$?
\кзадача

\опр
  События $A$ и $B$ называются \выдж независимыми, если $\Pf(A\cap B)=\Pf(A)\cdot\Pf(B)$.
\копр

\задача
  Пусть события $A$ и $B$ независимы.
  \пункт Верно ли, что $\Pf_B(A)=\Pf(A)$?
  \пункт Являются ли зависимыми события $A$ и $\overline{B}$?
  \пункт Являются ли зависимыми события $\overline{A}$ и $\overline{B}$?
\кзадача

\задача
  Следует ли из попарной независимости группы событий их независимость в совокупности?
\кзадача

\задача[Теорема умножения вероятностей]
  Пусть $A_1, A_2,\ldots, A_n$ --- события, вероятность которых больше 0. Докажите, что $\Pf(A_1A_2\ldots A_n)=\Pf(A_1)\cdot\Pf(A_2|A_1)\cdot\Pf(A_3|A_1A_2)\cdot\ldots\cdot\Pf(A_n|A_1\ldots A_{n-1})$.
\кзадача

\задача[Формула полной вероятности]
  Пусть $H_1,H_2,\ldots,H_n$ --- попарно несовместимые события, причём $H_1\cup H_2\cup\ldots\cup H_n=\Om$. Докажите, что $\fa B\in\Us\exec\Pf(B)=\sum\limits_{i=1}^n\Pf(H_i)\cdot\Pf(B|H_i)$.
\кзадача

%\задача
%  В письменном столе четыре ящика. В первом ящике одна папка красного цвета и одна синего. Во втором две красного и три синего. В третьем три красного и четыре синего. В четвёртом четыре красного и шесть синего. Наудачу открывают ящик и достают из него папку. Какова вероятность того, что эта папка окажется красной?
%\кзадача

\vspace{-0.1cm}

\задача[Формула Байеса]
  Пусть \hfill $H_1,H_2,\ldots,H_n$ \hfill --- \hfill попарно \hfill несовместимые \hfill события, \hfill причём\\$H_1\cup H_2\cup\ldots\cup H_n=\Om$. Предположим, стало известно, что событие $A$ произошло. Докажите, что тогда \equ{\Pf(H_i|A)=\frac{\Pf(H_i)\cdot\Pf(A|H_i)}{\sum\limits_{k=1}^n\Pf(A)\cdot\Pf(A|H_k)}.}
\кзадача

\vspace{-0.15cm}

\задача
  Во сколько раз доля блондинов среди голубоглазых в Тьмутараканском царстве больше доли голубоглазых среди блондинов, если всего голубоглазых там вдвое больше, чем блондинов?
\кзадача

%\задача
%  На выборах кандидат $A$ набрал $a$ голосов, а кандидат $B$ --- $b$ голосов, причём $a>b$. %Найдите вероятность того, что при последовательном подсчёте голосов кандидат $A$ всё время %был впереди кандидата $B$.
%\кзадача

%\задача
%  Китайское правительство издало закон, имеющий целью уменьшить прирост населения и %наименьшим образом повлиять на традиции: если в семье первый ребёнок --- мальчик, то этой %семье не разрешается больше иметь детей. Иначе семье разрешается завести второго ребёнка. %Какое отношение численности мужского населения к женскому должно в этом случае установиться? %При конкретном рождении считаем, что рождения мальчика и девочки равновероятны.
%\кзадача

%\сзадача[Сумасшедшая старушка]
%  Каждый из $n$ пассажиров купил по билету на $n$-местный самолёт. Первой зашла сумасшедшая %старушка и села на случайное место. Далее каждый вновь вошедший пассажир занимает своё место, %если оно свободно, а иначе --- занимает случайное место. Какова вероятность того, что %последний пассажир займёт своё место? У старушки был билет на какое-то место.
%\кзадача

%\сзадача
%  В очередь за газетами ценой в полтинник становятся в случайном порядке $n$ человек с %полтинниками и рублями. Какова вероятность того, что всем хватит сдачи, если ни у %покупателей, ни у продавца изначально нет других денег?
%\кзадача

\ЛичныйКондуит{0mm}{6mm}
\end{document}



