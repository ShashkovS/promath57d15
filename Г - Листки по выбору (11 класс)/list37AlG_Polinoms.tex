% !TeX encoding = windows-1251
\documentclass[a4paper,12pt]{article}
\usepackage{newlistok}

\def\hang{\hangindent\parindent}
\def\binom#1#2{\left({#1\atop #2}\right)}

\sloppy

\УвеличитьВысоту{1.8cm}
\УвеличитьШирину{1.0cm}
\renewcommand{\spacer}{\vfil}

\НомерЛистка{37АлГ}
\ДатаЛистка{01.2009}
\Заголовок{Многочлены}

\begin{document}
\СоздатьЗаголовок



\noindent {\bfseries Обозначения.}
В этом листке символом $\K$ всегда будет обозначаться некоторое поле.\\
Множество всех многочленов с коэффициентами из~$\K$
обозначается $\K[x]$.


\задача Дайте определение суммы и произведения многочленов из $\K[x]$.
\кзадача

\опр Многочлен положительной степени из  $\K[x]$  называется
\выд{неприводимым {\rm(}над $\K$\rm{)}},\/
если он не может быть представлен в виде произведения двух многочленов
меньшей степени из~$\K[x].$
\копр

\задача
Докажите, что над любым полем существует бесконечно много неприводимых
многочленов.
\кзадача

%\задача Какие из указанных многочленов неприводимы над  $\R$:\\
%\вСтрочку
\задача Разложите на неприводимые множители над  $\R$:\\
\вСтрочку
\пункт $5x+7;$
\пункт $x^2-2;$
\пункт $x^3+x^2+x+1;$
\пункт $x^2+1$;
\пункт $x^3-6x^2+11x-6;$
\пункт $x^4+4.$
\кзадача

%\задача Разложите на неприводимые множители над  $\R$:\\
%\вСтрочку
%\пункт $x^3-6x^2+11x-6;$
%\пункт $x^4+4.$
%\кзадача

%\опр  Многочлены $A_1(x),\dots,A_k(x)$ из $\K[x]$
%называются \выд{взаимно простыми},\/ если
%не существует такого  многочлена $C(x)\in\K[x]$ ненулевой степени,
%на который
%делятся все многочлены $A_1(x),\dots,A_k(x)$.
%\копр

\опр 
Многочлен со старшим коэффициентом 1 называется \выд{привед\"енным}.
\копр

\опр \выд{Наибольшим общим делителем} (${\rm НОД}(A,B)$)
двух многочленов $A$ и $B$ из $\K[x]$,
хотя бы один из которых ненулевой, называют привед\"енный многочлен
наибольшей степени, который делит и $A$, и $B$.
\копр

\задача
\пункт
Верно ли, что ${\rm НОД}(A,B)={\rm НОД}(A,B-A\cdot C)$,
где $C$ --- любой многочлен?\\
\пункт
Сформулируйте и докажите алгоритм Евклида вычисления {НОД} многочленов.
\кзадача

\задача Докажите, что ${\rm НОД}(A,B)$
делится на любой общий делитель $A$ и $B$.
\кзадача


\задача Пусть многочлены $A(x)$ и $B(x)$ из $\K[x]$ {\it взаимно просты}
(то есть, ${\rm НОД}(A,B)=1$).
Докажите, что  тогда существуют такие многочлены
$U(x)$  и  $V(x)$ из $\K[x]$,  что $AU+BV=1$.
\кзадача


\задача Докажите, что  если неприводимый над $\K$ многочлен $P(x)$ из $\K[x]$
делит произведение двух
многочленов  $A(x)$  и  $B(x)$  из $\K[x]$ ненулевой степени, то
он делит один из этих многочленов.
\кзадача

\задача Докажите, что  любой многочлен из $\K[x]$ однозначно (с точностью до
множителей из~$\K$) раскладывается в произведение неприводимых над  $\K$
многочленов.
\кзадача





\раздел{Многочлены с целыми коэффициентами}


\noindent {\bfseries Обозначение.} Множество многочленов с целыми коэффициентами обозначается $\Z[x]$.


\опр
Многочлен положительной степени из  $\Z[x]$  называется
\выд{неприводимым {\rm(}над $\Z$\rm{)}},\/
если он не может быть представлен в виде произведения двух многочленов
меньшей степени из~$\Z[x].$
(Это определение несколько отличается от общепринятого:
обычно требуют еще, чтобы коэффициенты многочлена были
взаимно просты).
\копр


\задача[Признак Эйзенштейна] Если для многочлена  $P(x)\in\Z[x]$
можно указать такое простое число  $p,$  что старший коэффициент этого
многочлена не делится на  $p,$  а все остальные коэффициенты делятся на
 $p,$  прич\"ем свободный член этого многочлена, делясь на  $p,$  не делится
на  $p^2,$  то многочлен  $P(x)$  неприводим над  $\Z.$
\кзадача

\задача Докажите, что  следующие многочлены неприводимы над  $\Z:$\\
\вСтрочку
\пункт $x^4-8x^3+12x^2-6x+2;$
\пункт $x^5-12x^3+36x-12$
\кзадача

\задача Пусть  $p,p_1,\dots,p_k$  --- различные простые числа.
Докажите, что  многочлены\\
\вСтрочку
\пункт $x^n-p;$
\пункт $x^n-p_1\dots p_k\,$  неприводимы над  $\Z.$
\кзадача


\задача Какие из многочленов задачи 4 неприводимы над  $\Z$?
\кзадача

%\break

\задача Докажите, что  многочлен
$P(x)=x^{n-1}+x^{n-2}+\dots+x+1$  неприводим над  $\Z$
тогда и только тогда, когда  $n$  --- простое число.\quad
(\выд{Указание:} рассмотрите многочлен  $P(x+1)$.)
\кзадача



%\СделатьКондуитИз{6.2mm}{6.2mm}{sp_AlG.tex}

\end{document}









\задача
%\пункт
Делится ли
\вСтрочку
\пункт
многочлен $x^{100}-32x^{90}+x^4+5x^3-3x^2-10x+2$
на многочлен $x^2-2$?\\
\пункт
%Делится ли
многочлен $x^{11}+x^{9}-5x^{8}+x^7-6x^{6}-7x^4-98x^2-49$
на многочлен $x^3-7$?
\кзадача




\задача Докажите, что  среди корней неприводимого над  $\Z$  многочлена из  $\Z[x]$
не менее чем второй степени не может быть рациональных.
\кзадача

\задача Докажите, что  следующие числа иррациональны:
\сНовойСтроки
\пункт $\smash{\root n \of p}$,  где  $p$  --- простое,  $n-1\in\N;$
\пункт $\smash{\root n \of {p_1\dots p_k}}$,
где  $p_1,\dots,p_k$  --- различные простые, $n-1,k\in\N;$
\пункт
$\sqrt2+\root3\of{2};$
\спункт $\smash{A(\root N\of{p})}$, где $N$ --- натуральное число,
большее 1,  $p$ --- простое,
$A(x)\in\Z[x]$ --- ненулевой многочлен степени меньше $N$;
\спункт $\alpha_1p^{m_1\over{n_1}}+\dots+\alpha_kp^{m_k\over{n_k}},$ \ где
$p$ \ --- простое, $\alpha_1,\dots,\alpha_n$ --- рациональные числа, не
все равные нулю, ${m_1\over{n_1}},\dots,{m_k\over{n_k}}$ --- попарно различные
правильные дроби.
\кзадача


\сзадача
\пункт
Пусть  $\alpha\in\R$ --- корень некоторого ненулевого
многочлена из $\Q[x]$.
Пусть $G(x)$ --- произвольный многочлен из $\Q[x]$, такой что
$G(\alpha)\ne0.$
Докажите, что  существует такой многочлен $H(x)\in\Q[x]$, что
$\displaystyle{1\over{G(\alpha)}}=H(\alpha)$.
\пункт
Найдите такой многочлен $H(x)$, если $\alpha=\root 3\of 2$ и $G(x)=x+1$.
\кзадача

\сзадача
Пусть  $\alpha\in\R$ --- корень  неприводимого
многочлена из $\Q[x]$ степени~$n$.
\сНовойСтроки
\пункт Докажите, что множество чисел $\{P(\alpha)\,|\,P(x)\in\Q[x]\}$ с
обычными операциям сложения и умножения является полем.
(Это поле обозначается $\Q(\alpha)$.)
\пункт Докажите, что
$\Q(\alpha)=
\{q_0+q_1\alpha+q_2\alpha^2+\dots+q_{n-1}\alpha^{n-1}\,|\,q_0,q_1,\dots,q_{n-1}\in\Q\}$.
\пункт Докажите, что любой элемент поля $\Q(\alpha)$
представляется в виде суммы из пункта б) единственным образом.
\кзадача

\сзадача Докажите, что  многочлен  $(x-a_1)\dots(x-a_n)-1$  неприводим над  $\Z$
 при любых попарно различных целых числах  $a_1,\dots,a_n.$
\кзадача


\сзадача \пункт Найдите целое число $a$, при котором многочлен
$(x-a)(x-10)+1$ раскладывается на два многочлена первой степени
с целыми коэффициентами.
\пункт При каких попарно различных целых числах  $a_1,\dots,a_n$ многочлен
$(x-a_1)\dots(x-a_n)+1$  не является неприводимым над  $\Z$?
\кзадача


\сзадача  Разложите на неприводимые множители над  $\Z$:\\
\вСтрочку
\пункт $x^8+x^4+1$;
\пункт $x^5+x+1$;
\пункт $x^9+x^4-x-1$.
\кзадача



\end{document}
