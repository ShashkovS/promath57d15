% !TeX encoding = windows-1251
\documentclass[a4paper,12pt]{article}
\usepackage{newlistok}
% \input{haidar_edition}

\УвеличитьШирину{1.5cm}
\renewcommand{\spacer}{\vfill}

\Заголовок{Кольца --- 1}
\НомерЛистка{Ri1}
\ДатаЛистка{11-й "Д" КЛАСС 2012 г}

\begin{document}
\СоздатьЗаголовок

\опр
Множество~$R$ называется \выд кольцом, если на нём заданы операции \выд сложения и~\выд умножения (отображения $+\colon R\times R\to R$ и~$\cdot\colon R\times R\to R$ соответственно), удовлетворяющие следующим условиям (\выд{аксиомам кольца}):
\begin{items}{-3}
\item[(A1)]
$\forall\,a,b\in R:\quad a+b=b+a$ (\emph{коммутативность сложения}).
\item[(A2)]
$\forall\,a,b,c\in R:\quad(a+b)+c=a+(b+c)$ (\emph{ассоциативность сложения}).
\item[(A3)]
В~$R$ существует такой элемент~$0$, что $\forall\, a\in R:\quad a+0=a$ (\emph{существование нуля}).
\item[(A4)]
$\forall\,a\in R\,\,\,\exists\,b\in R:\quad a+b=0$
(\emph{существование противоположного элемента}).\\ Элемент $b$
называется \emph{противоположным} к $a$ и обозначается $-a$.
\item[(M1)]
$\forall\,a,b,c\in C:\quad (a\cdot b)\cdot c=a\cdot (b\cdot c)$ (\emph{ассоциативность умножения}).
\item[(AM)]
$\forall\,a,b,c\in R:\quad a\cdot(b+c)=a\cdot b + a\cdot c$\\ (\emph{дистрибутивность умножения относительно сложения}).
\end{itemize}
Кольцо $R$ называется \выд коммутативным, если дополнительно выполнена аксиома
\begin{itemize}
\item[(M2)]
$\forall\,a,b\in R:\quad a\cdot b=b\cdot a$ (\emph{коммутативность умножения}).
\end{itemize}
Кольцо $R$ называется \выд{кольцом с единицей}, если дополнительно выполнена аксиома
\begin{itemize}
\item[(M3)]
В~$R\setminus\{0\}$ существует такой элемент $1$, что $\forall\,a\in R:\quad a\cdot 1 = 1\cdot a = a$
(\emph{существование единицы}).
\end{itemize}
Всюду в дальнейшем под словом \лк кольцо\пк будет подразумеваться коммутативное кольцо с единицей.
\копр

\задача
\label{rings}
Какие из следующих множеств (с естественными операциями сложения и умножения) являются кольцами?\\
\вСтрочку
\пункт $\N$;
\пункт $\Z$;
\пункт $\Q$;
\пункт $\R$;
\пункт $\Q[x]$.
\кзадача

\задача
Приведите пример кольца, состоящего в точности из $n\in\N$ элементов.
\кзадача

\опр
Кольцо $R$ называется \выд евклидовым, если на нём определена \выд{евклидова норма}\т функция $N\colon R \to \N\cup\{0\}$ такая, что $N(a) = 0$ тогда и только тогда, когда $a=0$, и возможно деление с остатком, то есть для любых $a,b \in R,\,b\ne 0$ существуют $q,r\in R$ такие, что $a = bq + r$ и $N(r) < N(b)$.
\копр

\замечание
Мы ещё будем дополнительно требовать, чтобы функция $N$ была \выд мультипликативной, то есть $N(a\cdot b) = N(a)\cdot N(b)$ для любых $a,b \in R$.
\кзамечание

\задача
Какие из колец задачи~\ref{rings} можно сделать евклидовыми, введя подходящую норму?
\кзадача

\задача
Существует ли евклидово кольцо из конечного числа элементов?
\кзадача

\опр
Множество $\Z[i] = \{a+bi \mid a,b \in\Z\}$, где $i$\т мнимая единица, с естественными операциями сложения и умножения называется \выд{кольцом гауссовых чисел}.
\копр

\задача
Пусть $z \in \Z[i]$. Нарисуйте на комплексной плоскости все гауссовы числа, кратные $z$.
\кзадача

\задача
Пусть $N(a+bi) = a^2 + b^2$.
\невСтрочку
\пункт
Проверьте, что $N$ удовлетворяет всем свойствам евклидовой нормы.
\пункт
Докажите, что $\Z[i]$\т евклидово кольцо.
\кзадача

\задача
По аналогии с $\Z[i]$ рассмотрим кольцо $\Z[i\sqrt n]$ и определим $N(a+bi\sqrt n) = a^2 + nb^2$. Будет ли это кольцо евклидовым, если
\вСтрочку
\пункт $n=2$;
\пункт $n=3$;
\пункт $n=5$?
\кзадача

\опр
Элемент $a\in R$ называется \выд обратимым, если существует элемент $b\in R$, такой что $ab = ba = 1$. В этом случае $b$ называется \выд обратным к $a$ и обозначается $a^{-1}$.
\копр

\задача
Перечислите все обратимые элементы в $\Z[i]$ и $\Z[i\sqrt2]$.
\кзадача

\опр
Необратимый элемент $a\in R\setminus\{0\}$ называется \выд неприводимым, если его нельзя представить в виде произведения двух необратимых элементов из $\R$.
\копр

\задача
Верно ли, что неприводимые элементы в $\Z$\т это в точности простые числа?
\кзадача

\newpage

\опр
Пусть $R$\т евклидово кольцо и $a = p_1 p_2 \dots p_m = q_1 q_2 \dots q_n$\т два разложения некоторого его элемента $a$ в произведение неприводимых множителей. Эти разложения \выд эквивалентны, если
\begin{items}{-5}
\item
$m = n$,
\item
Существует перестановка $\si \in S_n$ и набор обратимых элементов $o_1, o_2, \dots, o_n \in R$ такие, что $\forall i=1,2,\dots,n: \quad p_i = o_i q_{\si(i)}$.
\end{items}
Пример: $6 = 2\cdot 3 = (-3)\cdot (-2)$\т два эквивалентных разложения целого числа 6 на неприводимые множители.
\копр

\опр
Пусть $R$\т кольцо, такое что для любого необратимого элемента $a \in \R\setminus\{0\}$ существует разложение на неприводимые множители, причём оно единственно с точностью до эквивалентности. Тогда кольцо $R$ называется \выд факториальным.
\копр

\задача
Какие из колец задачи~\ref{rings} факториальны?
\кзадача

\задача
Приведите пример нефакториального кольца.
\кзадача

\задача[Основная теорема арифметики для евклидовых колец]
Докажите, что любое евклидово кольцо является факториальным.
\кзадача
\указание
Вспомните, как доказывалась основная теорема арифметики для целых чисел.
\куказание

\задача
Рассмотрим равенство $c^2 = a^2+b^2 = (a+bi)(a-bi)$. Примените к нему основную теорему арифметики и получите явное описание всех Пифагоровых троек.
\кзадача

\задача[Описание неприводимых гауссовых чисел]
\невСтрочку
\пункт
Докажите, что если $z\in\Z[i]$ приводимо и $\Im z = 0$, то либо число $\Re z$ составное, либо найдётся $w\in\Z[i]$, такое что $z = w\ol w$.
\пункт
Докажите, что если $z\in\Z[i]$ неприводимо, то $\ol z$ тоже неприводимо.
\пункт
Докажите, что если $z\in\Z[i]$ неприводимо, то существует ровно одно простое число $p$, делящееся на $z$.
\пункт
Докажите, что если $z\in\Z[i]$ неприводимо, то $N(z) = p$ или $N(z) = p^2$, где $p$\т простое число.
\пункт
Докажите, что простое число вида $p = 4k+3$ является неприводимым гауссовым числом.
%\пункт
%Пусть $p>2$\т простое число. Сколько из чисел $1,2,\ldots,p-1$ удовлетворяют сравнению $x^{\frac{p-1}2}-1\equiv0\!\!\pmod{p}$, а сколько\т сравнению $x^{\frac{p-1}2}+1\equiv0\!\!\pmod{p}$?
\пункт[Лемма Вильсона]
Пусть $p$\т простое. Докажите, что $((p-1)! + 1) \dv p$.
\пункт
Пусть $p = 4k+1$\т простое. Докажите, что найдётся $a\in\Z$, такое что $(a^2 + 1) \dv p$.
\пункт
Докажите, что простое число вида $p = 4k+1$ является приводимым гауссовым числом.
\пункт
Пусть $p = 2$ или $p = 4k+1$\т простое. Докажите, что найдётся неприводимое число $z\in\Z[i]$, такое что $p = z\ol z$, причём такое $z$ единственно с точностью до сопряжения и умножения на обратимые элементы.
\пункт
Докажите, что никаких других (с точностью до умножения на обратимые элементы) неприводимых гауссовых чисел кроме упомянутых в пунктах \выдж д и \выдж и нет.
\кзадача

\задача[Диофант]
Докажите, что число $15$ не представимо в виде суммы квадратов двух рациональных чисел.
\кзадача

\задача
Сформулируйте и докажите теорему о том, как по разложению натурального числа на простые множители понять, представимо ли это число в виде суммы двух квадратов целых чисел.
\кзадача

\задача
\пункт
Найдите натуральное число, которое представимо ровно 57-ю способами в виде суммы квадратов двух натуральных чисел.
\спункт
Найдите наименьшее такое число.
\кзадача


%\GenXML
\ЛичныйКондуит{0.3mm}{6.5mm}
%\СделатьКондуит{6mm}{8mm}

\end{document}
