% !TeX encoding = windows-1251
\documentclass[a4paper,12pt]{article}

\usepackage{newlistok}

\renewcommand{\spacer}{\vfil}



%\УвеличитьВысоту{1.9cm}

%\УвеличитьШирину{1.5cm}



\Заголовок{Распределение простых чисел}

\НомерЛистка{NT3} 

\ДатаЛистка{04.2015}



\begin{document}



\СоздатьЗаголовок



Мы будем обозначать через $\pi(x)$ количество простых натуральных чисел, не превосходящих $x$. 

История определения асимптотики функции  $\pi(x)$ такова:



\begin{nums}{-2}

\item Евклид: $\pi(x) \to \infty$ при $x \to \infty$;

\item Эйлер: $\dfrac{\pi(x)}{x} \to 0$ при $x \to \infty$;

\item Чебышёв (1848 г.): Если предел $\dfrac{\pi(x)\ln(x)}{x}$ существует, то он равен 1;

\item Адамар и Валле-Пуссен (1896 г.): $\dfrac{\pi(x)\ln(x)}{x} \to 1$ при $x \to \infty$.

\end{nums}



В этом листке мы докажем неравенства

$$

a\, \dfrac{x}{\ln x} \le \pi(x) \le b\, \dfrac{x}{\ln x}.

$$

Константы, которые получатся у нас, будут такими: 

$a = \frac{\ln2}{2} \approx 0.3465$, а $b = 5\ln2 \approx 3.4657$. 

У~Чебышёва константы были более точные: $a \approx 0.92129$, $b \approx 1.10555$.



\задача[Нижняя оценка для НОК]

Обозначим $\text{НОК}\hs{1,2,\ldots,2n+1}$ через $K$, а $\displaystyle\int\limits_0^1\br{x(1-x)}^n\,dx$ через~$I$.

Докажите, что:

\пункт

$I < \dfrac{1}{4^n}$;

\пункт

число $K\cdot I$ целое;

\пункт

$K > 4^n$.

\кзадача



\задача[Нижняя оценка для $\pi(x)$]

В обозначениях предыдущей задачи докажите, что

\\

\пункт

$K<(2n+1)^{\pi(2n+1)}$;

\пункт

$\pi(2n+1) > \dfrac{2n}{\log_2(2n+1)}$;

\пункт

$\frac{\ln2}{2}\, \dfrac{x}{\ln x} \le \pi(x)$

\кзадача



\задача[Оценка произведения простых чисел]

\невСтрочку

\пункт

Докажите, что число $C_{2m-1}^m$ больше произведения всех простых чисел, больших $m$, но меньших~$2m$;

\medskip

\пункт

Докажите, что $\displaystyle \prod\limits_{p\le x} p < 4^x$, где $\displaystyle \prod\limits_{p\le x} p$ --- произведение всех простых чисел, не превосходящих $x$.

\кзадача



\задача[Верхняя оценка для $\pi(x)$]

Докажите, что



\medskip

\пункт

$\pi(x)^{\pi(x)/2} \le \pi(x)! \le 4^x$;

\пункт

$\pi(x) \le 5\ln2\, \dfrac{x}{\ln x}$.

\кзадача



\задача

Пусть $p_1, p_2,\ldots,$ --- последовательность всех простых чисел.

Докажите, что найдутся такие константы $\al$ и $\be$, что $\al n \ln n < p_n < \be n \ln n$ для всех $n$.

\кзадача



\задача

Докажите, что ряд из обратных простых чисел расходится.

\кзадача





%\vfill

\ЛичныйКондуит{0mm}{6mm}

%\GenXMLW



\end{document}

