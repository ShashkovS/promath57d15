% !TeX encoding = windows-1251
\documentclass[12pt]{article}
\usepackage{newlistok}
% \input{haidar_edition}

\Заголовок{Аффинная геометрия}
\НомерЛистка{AG}
\ДатаЛистка{11-й "Д" КЛАСС 2012 г}

%\УвеличитьВысоту{15mm}%думать-менять!
%\УвеличитьШирину{17mm}

%\setlength{\hoffset}{-25truemm}
%\setlength{\voffset}{-30truemm}

%\УвеличитьПромежутки{100}

\begin{document}

%\DeclareMathOperator{\id}{id}
%\newcommand{\id}{id}

\СоздатьЗаголовок

\раздел{Аффинные преобразования плоскости}

\задача[Теорема Шаля.]
Докажите, что любое движение плоскости представляет собой
\невСтрочку
\пункт параллельный перенос, поворот или скользящую симметрию;
\пункт композицию не более чем трёх осевых симметрий.
\кзадача


\опр
Пусть в пространстве заданы две плоскости $\pi$ и $\pi'$, параллельные или непараллельные между собой. Пусть $l$ --- прямая, не параллельная ни $\pi$, ни $\pi'$. \выд{Параллельной проекцией $\pi$ на $\pi'$ вдоль $l$} называется отображение, сопоставляющее каждой точке $P\in\pi$ такую точку $P'\in\pi'$, что прямая $PP'$ параллельна прямой $l$. Любое отображение плоскости $\pi$ на плоскость $\pi'$, которое
можно представить в виде композиции параллельных проекций, называется \выд{аффинным}. Аффинное отображение плоскости $\pi$ на себя называется \выд{аффинным преобразованием}.
\копр

\задача
Докажите, что следующие преобразования являются аффинными:\\
\вСтрочку
\пункт параллельный перенос;
\пункт осевая симметрия;
\пункт поворот;
\пункт любое движение.
\кзадача

\задача
Зададим на плоскости прямоугольную систему координат. Докажите, что следующие отображения являются аффинными преобразованиями:
\невСтрочку
\пункт $(x,y)\mapsto(a x,y)$, где $a\ne0$;
\пункт гомотетия с центром в начале координат;
\пункт любое преобразование подобия;
\пункт $(x,y)\mapsto(x + b y,y)$,\quad где $b$ --- любое число;
\пункт $(x,y)\mapsto(ax + by + \alpha, cx + dy + \beta)$,\quad где $ad-bc\ne0$.
\кзадача

\задача
Докажите, что аффинные преобразования
\невСтрочку
\пункт переводят прямые в прямые;
\пункт переводят отрезки в отрезки;
\пункт переводят параллельные прямые в параллельные прямые;
\пункт сохраняют отношения длин отрезков, лежащих на параллельных прямых;
\пункт переводят параллелограммы в параллелограммы;
\пункт сохраняют отношения площадей. % параллелограммов
\кзадача

\задача
Пусть $ABC$ и $A'B'C'$ --- два произвольных треугольника. Докажите, что существует ровно одно аффинное преобразование, переводящее треугольник $ABC$ в треугольник $A'B'C'$ с сохранением порядка вершин.
\кзадача

\bigskip
\раздел{Применения аффинных преобразований}

\задача
Используйте аффинные преобразования для доказательства того, что три медианы любого треугольника пересекаются в одной точке.
\кзадача

\задача
Используйте аффинные преобразования для доказательства \лк замечательного свойства трапеции\пк: в любой трапеции точка пересечения диагоналей, точка пересечения продолжений боковых сторон и середины оснований лежат на одной прямой.
\кзадача

%\задача На плоскости даны две параллельные прямые $l$ и $l'$.
%\сНовойСтроки \пункт Отрезок $AB$ прямой $l$ разделите пополам при
%помощи одной линейки. \пункт Через данную точку $M$ проведите при
%помощи одной линейки прямую, параллельную прямым~$l$~и~$l'$.
%\кзадача

\задача
На сторонах $AB,\; BC$ и $AC$ треугольника $ABC$ выбраны соответственно точки $M,\; N,\; P$ и построены симметричные им точки $M',\; N',\; P'$ относительно середин этих сторон соответственно. Докажите, что треугольники $MNP$ и $M'N'P'$ равновелики.
\кзадача

\задача
Пусть $M$, $N$ и $P$ --- точки, расположенные на сторонах $AB$, $BC$ и $CA$ треугольника $ABC$ и делящие эти стороны в одинаковых отношениях ($\frac{AM}{MB}=\frac{BN}{NC}=\frac{CP}{PA}$). Докажите, что
\невСтрочку
\пункт точка пересечения медиан треугольника $MNP$ совпадает с точкой пересечения медиан треугольника $ABC$;
\пункт точка пересечения медиан треугольника, образованного прямыми $AN$, $BP$ и $CM$, совпадает с точкой пересечения медиан треугольника $ABC$.
\кзадача

\задача
Пусть у четырёхугольника $ABCD$ никакие две стороны не параллельны. Докажите, что прямая, соединяющая середины его диагоналей, делит пополам отрезок, соединяющий точки пересечений продолжений противоположных сторон.
\кзадача

\задача
Докажите, что с помощью только карандаша и односторонней линейки без делений нельзя опустить перпендикуляр на данную прямую.
\кзадача

\задача
Выпуклый пятиугольник $P$ гомотетичен пятиугольнику, построенному на серединах его сторон. Обязательно ли тогда $P$ --- правильный?
\кзадача

\задача
На сторонах $AB$, $BC$ и $CA$ треугольника $ABC$ выбрали соответственно
точки $K$, $L$ и $M$ так, что $AK:KB=BL:LC=CM:MA=1:\sqrt3$.
Прямые $AL$, $BM$ и $CK$ пересекаются в точках $A'$, $B'$ и $C'$,
образуя новый треугольник, на сторонах которого аналогичным образом
выбирают точки $K'$, $L'$, $M'$ и получают треугольник $A''B''C''$,
и так далее. Докажите, что на каком-то шаге мы получим треугольник,
подобный исходному.
\кзадача

\bigskip
\ЛичныйКондуит{0mm}{6.5mm}


\end{document}



\опр Множество $G$ преобразований множества $M$ называется
\выд{группой преобразований}, если
\begin{enumerate}
\item $\id_M\in G$ (через $\id_M$ мы обозначаем \выд{тождественное
преобразование} множества $M$ на себя: $\id_M(x)=x\quad\forall
x\in M$); \item $g^{-1}\in G\quad\forall g\in G$ (через $g^{-1}$
обозначается
преобразование, \выд{обратное} к $g$:\\
$g^{-1}(g(x))=g(g^{-1}(x))=x\quad\forall x\in M$); \item $g_1\circ
g_2\in G\quad\forall g_1,g_2\in G$ (через $g_1\circ g_2$
обозначается композиция преобразований $g_1$ и $g_2$:\quad
$(g_1\circ g_2)(x)=g_1(g_2(x))$);
\end{enumerate}
\копр
