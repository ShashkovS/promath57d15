% !TeX encoding = windows-1251
\documentclass[a4paper,12pt]{article}



\usepackage{newlistok}

\let\ZZ=\Z

\let\QQ=\Q

\renewcommand{\spacer}{\vfil}



\УвеличитьВысоту{1.9cm}

\УвеличитьШирину{1.5cm}



\parindent = 0mm



\sloppy



\Заголовок{Уравнение Пелля}\НомерЛистка{NT2} \ДатаЛистка{03.2015}



\begin{document}



\СоздатьЗаголовок



\noindent



\задача Докажите, что если уравнение $x^2-my^2=1$

имеет нетривиальное (т.е., отличное от решения $x=1,\,y=0$)

решение в целых числах, то $m$ не есть полный квадрат.

\кзадача









% \putpict{Смещение от текущей точки по горизонтали}{По вертикали}{Файл с рисунком}{Подпись}

%













\УстановитьГраницы{0mm}{65mm}

\задача

\putpict{143mm}{-20mm}{NT2}{}%{\risn Гиперболы $x^2-2y^2=n$}

\пункт

Покажите, что преобразование $(x,y)\mapsto T(x,y)= (3x+2y,4x+3y)$ переводит

всякую гиперболу семейства $x^2-my^2=c$ в себя и всякое

 целочисленное решение уравнения $x^2-2y^2=1$ в другое целочисленное решение.



\пункт Покажите, что уравнение $x^2-2y^2=1$ имеет бесконечно много решений

в целых числах.



\пункт Докажите, что всякое положительное целочисленное решение уравнения

$x^2-2y^2=1$ может быть получено из тривиального решения $(1,0)$ посредством

многократного применения преобразования $T$.

\ВосстановитьГраницы



\спункт Приведите общую формулу решений уравнения Пелля $x^2-2y^2=1$.

\кзадача





\задача

\пункт Докажите, что вещественные числа вида $a+b\sqrt{m}$, $a,b\in\ZZ$

замкнуты относительно операций сложения, вычитания и умножения.

Это множество обозначается $\ZZ[\sqrt{m}]$ и называется кольцом целых

гауссовых чисел.



\пункт Докажите, что вещественные числа вида $a+b\sqrt{m}$, $a,b\in\QQ$

замкнуты относительно операций сложения, вычитания, умножения и деления при

$a\not=0$.

Это множество  называется полем

квадратичного расширения $\QQ[\sqrt{m}]$.



\пункт Каждому гауссову числу $z=a+b\sqrt{m}$ сопоставим сопряженное

гауссово число $\bar{z}=a-b\sqrt{m}$.

Назовем нормой $N(z)$ гауссова числа $z$ целое число $z\bar{z}=a^2-mb^2$.

 Докажите мультипликативность нормы: $N(z_1z_2)=N(z_1)N(z_2).$



\пункт Пусть $z_1=a_1+b_1\sqrt{m}$ и $z_2=a_2+b_2\sqrt{m}$ -

два целых гауссовых числа, причем модуль нормы $z_2$ равен $n$. Тогда, если

$a_1=a_2 \mod n$ и $b_1=b_2 \mod n$, то $z_1$ делится на $z_2$ в $\ZZ[\sqrt{m}]$.

\кзадача



\задача

\пункт Решения уравнения Пелля $x^2-my^2=1$ находятся во взаимно-однозначном

соответствии с целыми гауссовыми числами из $\ZZ[\sqrt{m}]$ с нормой, равной

единице. Объясните это.



\пункт Докажите, что на множестве решений уравнения Пелля определена операция

умножения, а роль единицы играет тривиальное решение $(1,0)$.



\пункт Переформулируйте результаты задачи 2 в терминах гауссовых чисел.



\пункт Назовем фундаментальным решением уравнения Пелля

положительное решение с минимальной нормой соответствующего гауссова числа

(если таковое существует). Покажите, что положительные решения уравнения

исчерпываются степенями фундаментального.



\пункт \label{1}

Покажите, что для доказательства существования нетривиального решения уравнения Пелля

 достаточно показать, что существует гипербола $x^2-my^2=с$, содержащая

 бесконечно много целых точек.

\кзадача



\задача

\пункт Докажите, что если множество $M$ на плоскости имеет площадь, большую

1, то найдутся две точки $A,B\in M$ такие, что вектор $\overrightarrow{AB}$

целочисленный.



\пункт \label{2}(лемма Минковского) Докажите, что всякое центрально-симметричное

 множество площади больше 4 содержит целочисленную точку, отличную от начала

 координат.



\пункт Пусть $m$ не есть полный квадрат. Выведите из  задач 4д и 5б

существование нетривиального решения

у любого уравнения Пелля $x^2-my^2=1$.

\кзадача





\задача Пусть пара $(x,y)$ - положительное решение уравнения Пелля $x^2-my^2=1$.

Тогда $\left|\dfrac{x}{y}-\sqrt{m}\right|<\dfrac{1}{2y^2}$.

\кзадача

Таким образом, рациональное число $x/y$ хорошо приближает $\sqrt{m}$ и потому является

подходящей дробью для  $\sqrt{m}$. Более точно, это следует из такого

свойства приближений: если несократимая дробь $p/q$ такова, что

$\left|\dfrac{p}{q}-\alpha\right|<\dfrac{1}{2q^2}$, то она является подходящей дробью для

иррационального числа $\alpha$. Однако, не всякая подходящая дробь для $\sqrt{m}$

определяет решение соответствующего уравнения Пелля. Попробуйте разобраться с этим на примерах

уравнений Пелля с $m=2$ и $m=3$.



\сзадача Докажите, что все целые неотрицательные решения  уравнения $x^2-mxy+y^2=1$ описываются

как соседние члены рекуррентной последовательности $\varphi_0=0,\varphi_1=1$, $\varphi_{k+1}=

m\varphi_k-\varphi_{k-1}$.

\кзадача

%\пункт



\ЛичныйКондуит{0mm}{6mm}

\GenXMLW



%\опр  \копр





%\вСтрочку

%Сходятся ли следующие ряды? Для каждого сходящегося ряда

%найдите его сумму.

%\пункт



%\СделатьКондуитИз{6.2mm}{6.2mm}{sp_TCh.tex}



\end{document}

