% !TeX encoding = windows-1251
\documentclass[a4paper,12pt]{scrartcl}
\usepackage{newlistok}
\usepackage[cp1251]{inputenc}
\usepackage[russian]{babel}

\УвеличитьВысоту{1.5cm}
\УвеличитьШирину{1.1cm}
\renewcommand{\spacer}{\vfil}
\newcommand{\wa}{\overrightarrow}

\Заголовок{Линейные преобразования}
\НомерЛистка{RT1}
\ДатаЛистка{02.2015}

\sloppy

\begin{document}
\СоздатьЗаголовок

{\footnotesize
Прежде чем изучать специальную теорию относительности, необходимо немного изучить классическую теорию.

Чтобы говорить о каких-либо объектах, нам нужно ввести систему координат.
Обычно это три координаты в пространстве и одна координата --- время. После того, как координаты введены, мы можем изучать динамику тел.

Далее возникает естественный вопрос: что произойдёт, если взять другую систему координат. Так возникает понятие \выд{инерциальной системы отсчёта.}  И постулируется \выд{принцип относительности Галилея:} если в двух замкнутых лабораториях, одна из которых равномерно прямолинейно (и поступательно) движется относительно другой, провести одинаковый механический эксперимент, результат будет одинаковым (или более формально: законы механики не зависят от того, в какой из инерциальных систем отсчета мы их исследуем).

Чтобы говорить об инерциальных системах отсчёта, необходимы несколько формальных определений.
}


\задача
  Астрономы считают, что все галактики разлетаются
  прямолинейно по направлениям от нашей со скоростями,
  пропорциональными расстояниям до них. Означает ли это, что наша
  галактика --- центр вселенной?
\кзадача

\задача
  Крючок безмена заменили на более тяжёлый и одновременно параллельно
  сдвинули вниз шкалу, так чтобы нуль совпал с новым положением
  стрелки. Будет ли безмен после этого правильно измерять вес?
\кзадача


\опр
  Отображение $\R^m\corr{f}\R^n$ называется {\it линейным\/}, если для всех
  $x\in\R^m$, $y\in\R^n$ и $\la,\mu\in\R$ выполняется
  равенство\footnote
{$\la\,(x_1\sco x_m) = (\la x_1\sco \la x_m)$ и $(x_1\sco x_m)+(y_1\sco y_m) = (x_1+y_1\sco x_m + y_m)$}
  $f(\la x+\mu y)=\la\,f(x)+\mu\,f(y)$. Отображение $\R^m\corr{g}\R^n$
  называется {\it аффинным\/}, если существует $a\in\R^m$, такое что
  отображение $x\longmapsto g(x+a)-g(a)$ линейно.
\копр

\задача Являются ли следующие отображения аффинными или линейными?:\\
\пункт $f(x)=0 \in \R$;
\пункт $f(x)= x^2+1 \in \R$;
\пункт $f(x)= (57x,179x+57)\in\R^2$;\\
\пункт $f(x_1,x_2)=-3(x_1 - x_2)\in\R$;
\пункт $f(x_1,x_2)=(x_2 - x_1 - 1, x_1)\in\R^2$;\\
\пункт $f(x_1,x_2)=(x_1, x_1+x_2+1, x_1^2+x_2^2)\in\R^3$?
\кзадача

\задача
На плоскости фиксированы три точки: $O$, $A$ и $B$. Нарисуйте множество точек $\la \wa{OA}+\mu\wa{OB}$ при \пункт $\la+\mu = 1$; \пункт $\la,\mu>0$.
\кзадача


\задача
  Пусть линейное отображения $\R^2\corr{f}\R^2$ переводит базисные векторы
  $e_1=(1,0)$ и $e_2=(0,1)$ в векторы $(a,c)$ и $(b,d)$ соответственно.
  Куда она переведёт вектор $(x,y)$?
\кзадача

\задача
  Опишите все линейные и все аффинные отображения\\
  \пункт $\R^n\to\R^1$\hfill
  \пункт $\R^1\to\R^n$\hfill
  \пункт $\R^2\to\R^2$\hfill
  \пункт $\R^n\to\R^m$
\кзадача

\задача
  Изменим в определении аффинного отображения фразу \лк существует
  $a\in\R^m$\пк\ на фразу \лк для любого $a\in\R^m$\пк. Будет ли
  новое определение эквивалентно исходному?
\кзадача


\задача
\пункт
Докажите, что множество всех линейных отображений $f\colon\R^n\to\R^m$ образует коммутативную группу по сложению (то есть сложение коммутативно, ассоциативно и имеет обратный элемент). Обозначение: $\Hom(\R^n,\R^m)$ (от слова \лк гомоморфизм\пк);
\пункт
Докажите, что множество всех линейных функций $f\colon\R^2\to\R^2$ с операцией сложения изоморфна $\R^4$ (то есть существует биекция $\varphi$ такая, что $\varphi(\la f + \mu g) = \la\varphi(f) + \mu\varphi(g)$).
\кзадача

\задача
Пусть задано некоторое биективное отображение $f\colon \R^m\to\R^m$. Известно, что точка в $\R^m$ движется равномерно и прямолинейно тогда и только тогда, когда её образ движется равномерно и прямолинейно. Докажите, что преобразование $f$ аффинно.
\кзадача

\опр
Набор векторов $\hc{v_1\sco v_n}\subset\R^m$ называется базисом, если для любого вектора $w$ найдётся единственный набор чисел $\hc{\la_1\sco\la_n}$ (который называется координатами вектора $w$ в этом базисе), что
$$w = \la v_1 + \ldots + \la_n v_n.$$
\копр

\задача
\пункт
Опишите все базисы в $\R^1$;
\пункт
Докажите, что в любом базисе в $\R^2$ ровно два вектора.
\кзадача

%\опр
%Говорят, что на множестве $X$ заданы координаты, если задано вложение $f\colon X\hookrightarrow \R^m$ для некоторого $m$ (то есть каждой точки соответствует набор чисел $(x_1\sco x_m)$).
%\копр




%\СделатьКондуитИз{6.2mm}{6.2mm}{sp_STO.tex}

% \GenXMLW
\end{document}

\задача
Как перелётным птицам проще лететь: по ветру или против ветра? (в каком смысле \лк проще\пк следует понять самостоятельно)
\кзадача


{\footnotesize
Для изучения сложных вопросов необходимо изучить преобразования, связывающие две различные инерциальные системы отсчёта.

Навсегда зафиксируем некоторую инерциальную систему отсчёта и будем называть её \выд{системой отсчёта лаборатории}. Другая система\footnote{}, называется \выд инерциальной, если любая точка двигается в ней равномерно и прямолинейно тогда и только тогда, когда она двигается равномерно и прямолинейно в системе отсчёта лаборатории.

\noindent{\bf Обозначения:}
Для того, чтобы не путать разные системы координат, мы будем использовать следующие обозначения: центр координат системы отсчёта \лк лаборатории\пк (\лк неподвижной\пк) будем обозначать через $O$, вектора, вдоль которых измеряются координаты --- $(e_1,e_2,e_3)$ или $(e_x,e_y,e_z)$ в зависимости от ситуации, и, собственно, значения координат --- через $(x^1,x^2,x^3)$ или через $(x,y,z)$ соответственно. Для системы координат \лк ракеты\пк точно такие же обозначения, но со штрихами: $O'$, $(e_{1'},e_{2'},e_{3'})$, $(e_{x'},e_{y'},e_{z'})$, $(x^{1'},x^{2'},x^{3'})$ и $(x',y',z')$ соответственно.

Преобразования, превращающие данную инерциальную (то есть систему координат \лк лаборатории\пк) систему отсчёта в другую инерциальную называются преобразованиями Галилея.



}



\задача
Как может выражаться координаты центра системы отсчёта ракеты через координаты лаборатории и время?
\кзадача

\задача
Докажите, что если центр системы отсчёта ракеты неподвижен, то и вся система отсчёта ракеты неподвижна.
\кзадача

\опр
Преобразование $f$ называется \выд линейным, если для любых двух векторов $v$ и $w$ и любых двух чисел $\la$  и $\mu$ верно: $f(\la v + \mu w) = \la f(v) + \mu f(w)$.
\копр

\задача
Предположим, что $O'=O$. Докажите, что преобразование координат линейно.
\кзадача

\задача
Докажите, что множество всех преобразований Галилея






\end{document}
