% !TeX encoding = windows-1251
\documentclass[a4paper,12pt]{article}
\usepackage{newlistok}

\УвеличитьВысоту{1.5cm}
\УвеличитьШирину{.1cm}
\renewcommand{\spacer}{\vfil}

\Заголовок{Метрические пространства: примеры и пределы}
\НомерЛистка{36Топ}
\ДатаЛистка{01.2009}

\begin{document}
\СоздатьЗаголовок


\раздел{Метрические пространства}


\опр
\выд{Метрическим пространством} называется множество $X$ с заданной
функцией \лк расстояния\пк\ или \выд{метрикой} $d(x,y)$,
определенной для любых $x,y\in X$ и удовлетворяющей следующим аксиомам:
\begin{enumerate}
\item
$d(x,y)=0$ тогда и только тогда, когда $x=y$;
\item
$d(x,y)=d(y,x)$ (симметричность);
\item
$d(x,z)\le d(x,y)+d(y,z)$ (неравенство треугольника).
\end{enumerate}

Подмножество $Y$ метрического пространства $X$, рассматриваемое как
метрическое пространство (с той же метрикой), называется
\выд{подпространством} пространства $X$.
\копр

\задача
Пусть $X$ --- метрическое пространство с метрикой $d$.
Докажите, что $d(x,y)\ge0$ для любых $x,y\in X$.
\кзадача


\задача
Поездка на московском метрополитене
от станции $A$ до станции $B$ требует времени,
которое зависит от выбранного маршрута, времени ожидания поездов
и т.п. Выберем из всех возможных случаев тот,
при котором затраченное время окажется наименьшим, и назов\"ем это время
расстоянием от станции $A$ до станции $B$.
Является ли такое расстояние метрикой на множестве станций московского метро?
Если нет, предложите дополнительные условия, при которых введ\"енное
расстояние будет метрикой.
\кзадача

\задача
Приведите несколько примеров полезных метрических пространств.
\кзадача

\опр
Множество последовательностей
$x=(x_1,x_2,\dots,x_n)$
длины $n$, состоящих из действительных чисел,
называется \выд{$n$-мерным арифметическим пространством $\R^n$}.
(Обычные прямая, плоскость и пространство --- это $\R^1$,
$\R^2$ и $\R^3$ соответственно).
\копр

\задача\label{R^n}
Является ли метрическим пространством $\R^n$ с метрикой
\сНовойСтроки
\пункт
$d_1(x,y)=\sum\limits_{k=1}^{n}|y_k-x_k|$;
\пункт
$d_2(x,y)=\sqrt{\sum\limits_{k=1}^{n}(y_k-x_k)^2}$
\ (\выд{евклидова метрика});
\пункт
$d_\infty(x,y)=\max\limits_{1\le k\le n}|y_k-x_k|$?
\кзадача

\опр
Пусть $X$ --- метрическое пространство, $x_0\in X$ --- произвольная
точка, $\ep>0$ --- действительное число.
Множество $U_\ep(x_0)=\{x\in X\mid d(x,x_0)<\ep\}$
называется $\ep$-окрестностью точки $x_0$ (или
\выд{открытым шаром} с центром $x_0$ и радиусом $\ep$).
Множество \\$B_\ep(x_0)=\{x\in X\mid d(x,x_0)\le\ep\}$
называется \выд{замкнутым шаром} с центром $x_0$ и радиусом $\ep$.
\копр

\задача
Как выглядят шары в пространствах $\R^2$ и $\R^3$ относительно метрик из
задачи \ref{R^n}?
\кзадача

%\опр
%Отображение метрических пространств $F:X\to Y$ называется
%\выд{изометрией}, если оно взаимно-однозначно и
%сохраняет расстояния между точками.
%для любых $x_1,x_2\in X$
%выполнено равенство $d_Y(F(x_1),F(x_2))=d_X(x_1,x_2)$.
%Метрические
%пространства, между которыми существует изометрия, называются
%\выд{изометричными}.
%\копр

%\задача
%Докажите, что пространство $\R^2$ с евклидовой метрикой
%изометрично плоскости, а пространство $\R^3$ --- пространству.
%\кзадача

%\noindent{\bf Замечание.}
%В дальнейшем мы будем отождествлять $\R^1=\R$ с прямой, $\R^2$ --- с
%плоскостью, а $\R^3$ --- с пространством.
%Кроме того, мы будем отождествлять $\R^2$ с $\C$ ((a,b)\leftrightarrow
%a+bi$).

\задача
Докажите, что множество $C[a,b]$ всех непрерывных функций на отрезке
$[a,b]$, снабженное метрикой $d(f,g)=\max\limits_{a\le t\le
b}|f(t)-g(t)|$ (эта метрика называется \выд равномерной), является метрическим пространством.
\кзадача

\задача
Дайте определения предельной точки подмножества метрического
пространства, предела последовательности точек метрического
пространства, предела отображения метрических пространств $F:X\to Y$ в
точке $x_0\in X$, непрерывности отображения метрических пространств
$F:X\to Y$ в точке $x_0\in X$ (в двух вариантах: по Коши и по Гейне).
\кзадача

\опр
Подмножество $U$ в метрическом пространстве $X$ называется \выд открытым, если для любой точки $x\in U$ найдётся некоторая $\ep$-окрестность, целиком содержащаяся в $U$.
\копр

\задача
Докажите, что отображение $f\colon X \to Y$ метрических пространств непрерывно всюду тогда и только тогда, когда прообраз любого открытого множества открыт.
\кзадача

%\СделатьКондуитИз{6.2mm}{6.2mm}{sp_TOP.tex}

\end{document}
