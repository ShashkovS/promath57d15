% !TeX encoding = windows-1251
\documentclass[a4paper,12pt]{article}
\usepackage{newlistok}
% \input{haidar_edition}

\renewcommand{\P}{{\sf P}}

\УвеличитьШирину{1.5cm}
\renewcommand{\spacer}{\vfill}

\Заголовок{Элементы теории вероятностей}
\НомерЛистка{PT1}
\ДатаЛистка{11-й "Д" КЛАСС 2012 г}

\begin{document}
\СоздатьЗаголовок

\опр
\emph{Случайное явление}\т это такое явление, которое в серии однотипных экспериментов под действием случайных факторов может приводить к различным результатам.
\копр
Примеры:
\begin{nums}{-5}
\item
Спортсмен производит серию выстрелов по мишени. Результаты выстрелов могут отличаться, несмотря на постоянство условий стрельбы.
\item
Игрок в одинаковых условиях бросает игральную кость. В зависимости от случайных факторов могут выпадать различные числа: $1$, $2$, $3$, $4$, $5$, или $6$.
\end{nums}

\опр
Пусть проводится случайный эксперимент. В зависимости от случайных факторов возможны различные исходы этого эксперимента. Тогда этому эксперименту можно сопоставить \выд {пространство элементарных событий $\Omega$}, которое включает всевозможные исходы этого эксперимента. \выд{Элементарное событие} является одним из элементов этого пространства и определяет один из возможных исходов. \выд{Случайное событие $A$} является подмножеством пространства элементарных событий и включает одно или группу элементарных событий, каждое из которых благоприятствует $A$.
Событие, состоящее в наступлении обоих событий $A$ и $B$, будем называть \выд {произведением} событий $A$ и $B$ и обозначать $AB$ или $A \bigcap B$.
Дополнение к событию $A$ (событие "не $A$") обозначим $\ol{A}$.
\копр

Примеры:
\begin{nums}{-5}
\item
Эксперимент\т выстрел в мишень. Возможные исходы: промах, попадание. Случайные события\т попадание, промах.
\item
Эксперимент\т бросание игральной кости. Возможные исходы: выпадение $1$, $2$, $3$, $4$, $5$, или $6$. Рассматривается случайное событие\т выпадение чётного числа. Ему благоприятствуют элементарные исходы: $2$, $4$ и $6$. Выпадение нечётного числа также является случайным событием.
\end{nums}

\опр
Каждому исходу $\omega$ сопоставляют число $\P(\omega)$ из отрезка $[0;1]$, называемое \выд{вероятностью} этого исхода. Сумма вероятностей всех элементарных событий должна равняться единице. Вероятность события $A$\т сумма вероятностей исходов, благоприятствующих событию $A$ (обозначается $\P(A)$). Пара $(\Omega, \P)$ называется вероятностным пространством.
\копр

\задача
Симметричную монету бросили $10$ раз. Какова вероятность того, что
\вСтрочку
\пункт
все $10$ раз выпал орёл?
\пункт
сначала выпало $5$ орлов, а затем $5$ решек?
\пункт
выпало $5$ орлов и $5$ решек (в произвольном порядке)?
\кзадача

\задача
\label{test}
Тест состоит из $10$-ти вопросов, на каждый из которых есть $4$
варианта ответа. Двоечник Вася отвечает на вопросы \лк наобум\пк.
\вСтрочку
\пункт
Какова вероятность того, что он ответит правильно на все $10$ вопросов?
\пункт
Ровно на $5$ вопросов?
\пункт
Не менее, чем на $5$ вопросов?
\кзадача

\задача
В году проводят много тестов, аналогичных тесту из задачи~\ref{test}. Если Васе удаётся списать ответ на вопрос у отличника Пети, он отвечает на вопрос верно, иначе отвечает наугад. В конце года оказалось, что Вася ответил верно на половину всех вопросов. Какую часть вопросов Вася списал?
\кзадача

\задача
Из множества всех последовательностей длины~$n$, состоящих из цифр~$0$, $1$ и $2$, случайно выбирается одна. Найдите вероятность того, что в последовательности ровно $m_0$~нулей, $m_1$~единиц и $m_2$~двоек.
\кзадача

\задача
За круглый стол рассаживаются в случайном порядке $2n$~гостей. Какова вероятность того, что гостей можно разбить на $n$~непересекающихся пар так, чтобы каждая пара состояла из сидящих рядом мужчины и женщины?
\кзадача

\задача
Два игрока поочередно извлекают шары (без возвращения) из урны, содержащей $m$~белых и $(n-m)$~чёрных шаров. Выигрывает тот, кто первым вытянет белый шар. Найдите вероятность выигрыша первого участника, если
\пункт $n=5$, $m=1$
\пункт $n=7$, $m=2$.
\кзадача

\задача Пусть $B$\т событие, обладающее ненулевой вероятностью. Дайте определение условной вероятности $\P_B(A) = \P(A|B)$ события $A$ при условии, что событие $B$ произошло.
\кзадача

\задача
Брошены две игральные кости. Найдите условную вероятность того, что выпали две пятёрки, если известно, что сумма выпавших очков делится на~$5$.
\кзадача

\newpage

\задача
Монетка бросается $10$ раз. Найдите вероятность того, что выпал \лк орёл\пк при условии, что $9$ предыдущих раз выпала \лк решка\пк.
\кзадача

\задача
Вероятность попадания в цель при отдельном выстреле равна $0{,}2$. Какова вероятность поразить цель, если в $2\%$ случаев выстрел не происходит из-за осечки?
\кзадача

\опр
События $A$ и $B$ называются \выд{независимыми}, если $\P(AB) = \P(A) \cdot \P(B)$ (при $\P(B)\ne0$ это равносильно равенству $\P_B(A) = \P(A)$).
\копр

\задача
События $A$~и~$B$ независимы. Являются ли независимыми события~$A$ и~$\ol{B}$? Являются ли независимыми события $\ol{A}$~и~$\ol{B}$?
\кзадача

\задача
Из колоды в $52$ карты случайным образом выбирается одна карта. Независимы ли события
\пункт \лк выбрать валета\пк и \лк выбрать пику\пк?
\пункт \лк выбрать валета\пк и \лк не выбрать даму\пк?
\кзадача

\задача[Формула полной вероятности]
Докажите, что для любых событий $A$, $B$ в вероятностном пространстве $(\Omega, \P)$
\невСтрочку
\пункт
если $0 < \P(B) < 1$, то $\P(A) = \P(A|B)\P(B) + \P(A|\ol B)\P(\ol B)$;
\пункт
если $\Omega = B_1\cup\ldots\cup B_n$, для всех $1\leq i\leq n$ выполнено $\P(B_i) > 0$, и для всех $1 \leq i < j \leq n$ выполнено $\P(B_iB_j)=0$, то $\P(A) = \P(A|B_1)\P(B_1)+\ldots+\P(A|B_n)\P(B_n)$.
\кзадача

\задача[Формула Байеса]
В условиях предыдущей задачи докажите, что
\equ{
\P(B_i|A)=\frac{\P(A|B_i)\P(B_i)}{\P(A|B_1)\P(B_1)+\ldots+\P(A|B_n)\P(B_n)}
}
\кзадача

\задача
В одной урне содержится $1$~белый и $2$~чёрных шара, а в другой урне\т $2$~белых и $3$~чёрных шара. В третью урну кладут два шара, случайно выбранных из первой урны, и два шара, случайно выбранных из второй. Какова вероятность того, что шар, извлечённый из третьей урны, будет белым?
\кзадача

\задача
Студент сдаёт тест. На очередную задачу имеется $K$~вариантов ответа. Студент действует так: либо он умеет решать задачу, и тогда он с определённостью находит правильный ответ, либо он не умеет её решать, и тогда он выбирает ответ наугад. Считается, априори, что студент умеет решать задачу с вероятностью~$p$. Найдите вероятность того, что студент умел решать задачу, коль скоро полученный им ответ оказался верным.
\кзадача

\задача[Схема Бернулли]
Пусть некоторый эксперимент может закончиться либо успехом с вероятностью $p$, либо неудачей с вероятностью $q = 1-p$. Проводится $n$ независимых испытаний. Найдите вероятность того, что произошло ровно $m$ успехов.
\кзадача

\задача
Рассмотрим множество $R = \{1,\,\dots,\,n\}$ и будем производить последовательные испытания Бернулли (с вероятностью успеха~$p$). Если на~$\nu$-м шаге решка\т вынимаем элемент~$\nu$ из~$R$. Иначе\т не вынимаем. Обозначим множество оставшихся элементов через~$A_1$. Точно так же из $R$ получаем $A_2,\,\dots,\,A_m$. Найдите $\P(|A_1\cap\dots\cap A_m|=k)$.
\кзадача

\сзадача[Сумасшедшая старушка]
Каждый из $n$ пассажиров купил по билету на $n$-местный самолёт. Первой зашла сумасшедшая старушка и села на случайное место. Далее, каждый вновь вошедший занимает своё место, если оно свободно; иначе занимает случайное. Какова вероятность того, что последний пассажир займёт своё место?
\кзадача

\сзадача[Задача о разорении]
Игрок, имеющий $n$ монет, играет против казино, имеющего неограниченное количество монет. За одну игру игрок либо проигрывает монету, либо выигрывает с вероятностью $1/2$. Он играет, пока не разорится. Найдите вероятность разориться ровно за $m$ игр.
\кзадача

\сзадача
Средний интервал движения автобуса \No 57 равен $35$ минут, а средний интервал движения автобуса \No 661 равен $20$ минут. Сколько в среднем нужно ждать\\ \вСтрочку
\пункт
автобус \No 57;
\пункт
один из этих автобусов?
\кзадача

\ЛичныйКондуит{0.5mm}{6.5mm}
%\СделатьКондуит{7mm}{8mm}
%\GenXML

\end{document}
