% !TeX encoding = windows-1251
\documentclass[a4paper,12pt,fleqn]{article}
\usepackage{newlistok}
% \input{haidar_edition}

\УвеличитьШирину{1.5cm}
\renewcommand{\spacer}{\vfill}

\renewcommand{\ker}{\mathrm{Ker\,}}
\newcommand{\im}{\mathrm{Im\,}}
\newcommand{\Iso}{\mathrm{Iso}}


\Заголовок{ЛИНЕЙНАЯ АЛГЕБРА\т II\\ЛИНЕЙНЫЕ ОТОБРАЖЕНИЯ}
\НомерЛистка{LA2}
\ДатаЛистка{11-й "Д" КЛАСС 2012 г}

\begin{document}
\СоздатьЗаголовок

\опр
Пусть $L_1$, $L_2$ --- линейные пространства над полем $\F$. Отображение $A\colon L_1\to L_2$ называется \emph{линейным отображением} (\emph{гомоморфизмом}), если выполняются следующие условия:
$$(\mbox{L1})\,\,\,A(x+y)=A(x)+A(y),\,\,\fa\,x,y\in L_1;\qquad\qquad (\mbox{L2})\,\,\,A(\la x)=\la A(x),\,\,\fa\,x\in L_1,\,\fa\,\la\in\F.$$
Множество линейных отображений из~$L_1$ в~$L_2$ обозначается $\Hom(L_1,L_2)$. Если {$L_1=L_2$}, то такое отображение называется \emph{линейным оператором на~$L_1$} или \emph{эндоморфизмом~$L_1$}. Множество эндоморфизмов~$L_1$ обозначается $\End(L_1)$.
\копр

\замечание
Образ элемента $x\in L_1$ при линейном отображении $A\colon L_1\to L_2$ часто мы будем обозначать просто $Ax$.
\кзамечание

\задача
Докажите, что множество $\Hom (L_1,L_2)$ является линейным пространством относительно
следующих операций:\quad $(A+B)x=Ax+Bx$,\quad $(\la A)x=\la (Ax)$.
\кзадача

\задача \label{linear maps}
Являются ли линейными следующие отображения $A\colon L_1\to L_2$:
\невСтрочку
\пункт $Ax=0$;
\пункт $L_1=L_2,\,\,$ $Ax=x$ (такое отображение называется \emph{тождественным};
обозначение: $\id$ или~$E$);
\пункт $L_1=\R^4,$ $L_2=\R^3,\,\,$ $A(x,y,z,t)=(x+y,y+z,z+t)$;
\пункт $L_1=L_2=\R^3,\,\,$ $A(x,y,z)=(x+1,y+1,z+1)$;
\пункт $L_1=L_2=\F[x],\,\,$ $(Ap)(x)=p(\la x^2+\nu)$,\,\,где $\la,\nu$\,---
фиксированные элементы $\F$;
\пункт $L_1=L_2=\F[x],\,\,$ $(Ap)(x)=q(x)\cdot p(x)$,\,\, где $q(x)$\,--- фиксированный элемент $\F[x]$;
\пункт $L_1$ --- пространство сходящихся последовательностей действительных
чисел, $L_2=\R,\,\,$ $A(x_i)=\lim\limits_{i\to\infty}x_i$.
\кзадача

\задача
Докажите, что произведение (композиция) линейных отображений есть линейное отображение. Проверьте, что произведение обладает свойствами ассоциативности и дистрибутивности.
\кзадача

\опр
\emph{Ядром линейного отображения} $A\colon L_1\to L_2$ называется множество, состоящее из всех таких~$x\in L_1$, что $Ax=0$. Обозначение:~$\ker A$. Образ линейного отображения~$A$ обозначается $\im A$.
\копр

\задача
Докажите, что $\ker A$ и $\im A$ являются линейными пространствами.
\кзадача

\задача
Найти ядра и образы линейных отображений из задачи~\ref{linear maps}.
\кзадача

\опр
Отображение $A\in\Hom(L_1,L_2)$ называется \emph{изоморфизмом}, если выполняются условия $\ker A=0$ и $\im A=L_2$. Множество изоморфизмов обозначается $\Iso(L_1,L_2)$. В случае, если $L_1=L_2$, изоморфизмы называются \emph{автоморфизмами.} Обозначение: $\Aut(L_1)$. Если $\Iso(L_1,L_2)$ непусто, то линейные пространства $L_1$ и~$L_2$ называются \emph{изоморфными}. Обозначение: $L_1\cong L_2.$
\копр

\задача
Пусть $A\in\Hom(L_1,L_2)$.
Докажите, что  следующие утверждения эквивалентны:
\begin{items}{-3}
\item[1)]
$A$\т  изоморфизм;
\item[2)]
$A$ взаимно-однозначно;
\item[3)]
$A$ обратимо, то есть существует такое отображение $A^{-1}\in \Hom(L_2,L_1)$, что
$\,AA^{-1}=\id$ и $A^{-1}A=\id$.
\end{items}
\кзадача

\vspace{-3mm}
\задача
Пусть $A\colon L_1\to L_2$\т изоморфизм. Докажите, что векторы множества $U\subset L_1$ линейно независимы тогда и~только тогда, когда векторы множества~$A(U)$ линейно независимы.
\кзадача

\задача
Пусть $(e_1,\dots,e_n)$\т базис пространства $L_1$. Докажите, что для всякого набора векторов $(g_1,\dots,g_n)$ пространства $L_2$ найдётся ровно одно линейное отображение из $L_1$ в~$L_2$, переводящее $e_i$ в~$g_i$ при всех~$i$.
\кзадача

\задача
Докажите, что два конечномерных пространства изоморфны тогда и~только тогда, когда их размерности равны.
\кзадача

\newpage

\раздел{Факторпространства}

\опр
Будем говорить, что на множестве~$X$ задано {\it отношение}~$R$, если в~декартовом произведении $X\times X$ выделено некоторое подмножество $R\subset X\times X$. При этом, если $(x,y)\in  R$, то будем писать $x\stackrel{R}{\sim} y$ (или просто $x\sim y$).
Отношение $R$ называется\\
а)~{\it рефлексивным}, если для любого $x\in  X$ имеет место $x\sim x$;\\
б)~{\it симметричным}, если из $x\sim y$ всегда следует $y\sim x$;\\
в)~{\it транзитивным}, если из $x\sim y$ и~$y\sim z$ всегда следует $x\sim z$;\\
г)~{\it отношением эквивалентности}, если оно рефлексивно, симметрично и~транзитивно. В~этом случае, если $x\sim y$, то мы будем говорить, что $x$ и~$y$ {\it эквивалентны}.
\копр

\задача
Какие из следующих отношений являются отношениями эквивалентности?
\невСтрочку
\пункт Для любых двух $m,n\in\mathbb Z$ полагаем $n\sim m$, если $n$ и~$m$ взаимно просты.
\пункт Пусть $X$~---~множество треугольников на плоскости; полагаем $T_{1}\sim T_{2}$, если $T_{1}=T_{2}$.
\пункт Пусть $f:X\to Y$\т отображение. Для любых $x_{1},x_{2}\in X$ полагаем $x_{1}\sim x_{2}$, если $f(x_{1})=f(x_{2})$.
\пункт Дано $n\in\mathbb N$. Для любых двух $a,b\in\mathbb Z$ полагаем $a\sim b$, если $a\equiv b$ $({\rm mod}\,n)$.
\пункт Для любых двух $x,y\in\mathbb R$ полагаем $x\sim y$, если $(x-y)\in\mathbb Z$.
\пункт Для любых двух $x,y\in\mathbb R$ полагаем $x\sim y$, если $x\leqslant y$.
\пункт Для любых двух $(x_{1},y_{1}),(x_{2},y_{2})\in\mathbb R^{2}$ полагаем $(x_{1},y_{1})\sim(x_{2},y_{2})$, если $(x_{1}-x_{2})\in\mathbb Z$ и~$(y_{1}-y_{2})\in\mathbb Z$.
\кзадача

\задача \label{factor}
Докажите, что следующие отношения являются отношениями эквивалентности.
\невСтрочку
\пункт Пусть $X$\т множество линейных пространств. Полагаем $L_1\sim L_2$, если $L_1\cong L_2$.
\пункт Пусть $L_0$\т линейное подпространство линейного пространства~$L$. Для любых двух $x,y\in L$ полагаем $x\sim y$, если $(x-y)\in L_0$.
\кзадача

\опр
Пусть $X$~---~множество, на котором задано отношение эквивалентности. Для каждого элемента $a\in X$ подмножество $X_{a}\subset X$, состоящее из всех элементов~$x$, эквивалентных элементу~$a$, называется {\it классом эквивалентности элемента}~$a$.
\копр

\задача
Докажите, что для данного отношения эквивалентности на множестве~$X$
\невСтрочку
\пункт подмножества~$X_{a}$ при различных~$a$ либо не пересекаются, либо совпадают;
\пункт объединение подмножеств~$X_{a}$ по всем $a\in X$ совпадает с~множеством~$X$.
\кзадача

\задача
Пусть множество~$X$ представлено в~виде объединения (конечного или бесконечного) попарно непересекающихся подмножеств. Докажите, что это разбиение множества~$X$ порождает на~$X$ отношение эквивалентности такое, что множества разбиения являются классами эквивалентности.
\кзадача

\опр
Множество классов эквивалентности по отношению эквивалентности~$R$ на множестве~$X$ называется {\it фактормножеством} и~обозначается~$X/R$. Фактормножество из задачи~\ref{factor}~б) называется факторпространством и~обозначается $L/L_0$.
\копр

\задача
Докажите, что $L/L_0$ есть линейное пространство относительно следующих операций:
\begin{items}{-3}
\item[1)]
сложение: $\forall\,x,y\in L/L_0,\,\,\,$ $x+y$ есть класс эквивалентности элемента $a+b$, где $a\in x$, $b\in y$;
\item[2)]
умножение на число: $\forall\,x\in L/L_0,\,\la\in\F,\,\,\,$ $\la x$ есть класс эквивалентности элемента $\la a$, где $a\in x$.
\end{items}
\кзадача

\vspace{-3mm}
\задача
Докажите, что
\вСтрочку
\пункт $L/0\cong L$;
\пункт $L/L\cong 0$;
\невСтрочку
\пункт если $L_0\subset\R[x]$\т пространство многочленов, равных нулю в~точке~$1$, то $\R[x]/L_0\cong \R$;
\пункт факторпространство всех сходящихся последовательностей по бесконечно
малым изоморфно~$\R$.
\кзадача

\задача
Докажите, что для всякого $A\in\Hom(L_1,L_2)$ корректно определено и~является изоморфизмом отображение  $L_1/\ker A\to\im A$, переводящее класс вектора $x\in L_1$ в~$Ax$.
\кзадача

%\GenXML
\ЛичныйКондуит{0.3mm}{6.5mm}
%\СделатьКондуит{5mm}{8mm}

\end{document}
