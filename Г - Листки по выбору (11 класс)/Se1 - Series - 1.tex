% !TeX encoding = windows-1251
\documentclass[a4paper,12pt]{article}
\usepackage{newlistok}
% \input{haidar_edition}

\УвеличитьШирину{5mm}
\renewcommand{\spacer}{\vfill}

\Заголовок{ЧИСЛОВЫЕ РЯДЫ}
\НомерЛистка{Se1}
\ДатаЛистка{11-й "Д" КЛАСС 2012 г}

\begin{document}
\СоздатьЗаголовок

\опр
Пусть задана числовая последовательность $(a_n)$. Формальное выражение $a_1+a_2+a_3+\ldots=\sumnui a_n$ называется \выд рядом. Для краткости мы вместо $\sumnui$ будем писать просто $\sum$. Число $s_n=a_1+a_2+\ldots+a_n$ называется \выд{$n$-ой частичной суммой} ряда.\\
Говорят, что ряд $\sum a_n$ \выд{сходится и имеет сумму $A$}, если существует $\limn s_n=A$. Тогда пишут $\sum a_n=A$. Если предел $\limn s_n$ не существует, то говорят, что ряд $\sum a_n$ \выд расходится.
\копр

\задача
Пусть $a_n \ge 0$ при $n\in\N$. Докажите, что ряд $\sum a_n$ сходится тогда и только тогда, когда ограничено множество его частичных сумм $\{s_n \mid n\in\N\}$, причём в этом случае $\sum a_n=\sup\{s_n \mid n\in\N\}$.
\кзадача

\задача
Какие из следующих рядов сходятся? Найдите их суммы.\\
\вСтрочку
\пункт
$\sum (-1)^n$;
\пункт[геометрическая прогрессия]
$\sum q^n$;
\пункт
$\sum \dfrac{n}{2^n}$;
\пункт
$\sum \dfrac{n^2}{2^n}$;
\пункт
$\sum n!\,q^n$;
\пункт[гармонический ряд]
$\sum \dfrac1n$;
\пункт
$\sum \dfrac1{n(n+1)}$;
\спункт
$\sum \dfrac1{n(n+1)(n+2)}$.
\кзадача

\задача
Докажите, что если ряд $\sum a_n$ сходится, то $\limn a_n=0$. Верно ли обратное?
\кзадача

\задача[Критерий Коши сходимости ряда]
Докажите, что ряд $\sum a_n$ сходится тогда и только тогда, когда для любого $\ep>0$ существует такое $N$, что из $n\ge m>N$ (где $n,m\in\N$) следует $|a_m+a_{m+1}+\dots+a_n|<\ep$.
\кзадача

\задача
\невСтрочку
\пункт
Пусть ряды $\sum a_n$ и $\sum b_n$ сходятся. Докажите, что тогда ряд $\sum (\al a_n+\be b_n)$ тоже сходится, причём выполнено равенство $\sum (\al a_n+\be b_n)=
\al\sum a_n+\be\sum b_n$.
\пункт
Пусть ряд $\sum a_n$ сходится, а ряд $\sum b_n$ расходится. Докажите, что тогда ряд $\sum (a_n+b_n)$ расходится.
\кзадача

\задача
Сходятся ли следующие ряды:
\вСтрочку
\пункт
$\sum \dfrac{(-1)^n}{n}$;
\пункт
$\sum \dfrac1{\sqrt{n}}$;
\пункт
$\sum \dfrac1{n^2}$?
\кзадача

\задача
Докажите:
\вСтрочку
\пункт
ряд $\sumnzi\dfrac1{n!}$ сходится;
\пункт
$\sumnzi\dfrac1{n!}=e$;
\пункт
$\left|e-\sum\limits_{n=0}^m\dfrac1{n!}\right|<\dfrac1{m!\,m}$;\\
\пункт
число $e$ иррационально.\\
(Подсказка к пункту \textbf{б}:\,\,{\reflectbox{\hbox{\tiny Нужно рассмотреть первые $n$ членов бинома Ньютона для $\left(1+\frac1k\right)^k$ при $k\to\infty$}}})
\кзадача

\сзадача
Докажите, что сумма ряда $\sum\dfrac{1}{2^{n^2}}$ есть число иррациональное.
\кзадача

\задача
Пусть $a_n \ge 0$ при всех $n\in\N$ и $\si\colon\N\to\N$\т взаимно однозначное отображение (перестановка натурального ряда). Тогда $\sum a_n = \sum a_{\si(n)}$ (то есть если сходится ряд в левой части равенства, то сходится и ряд в правой части, причём их суммы равны; если ряд в левой части расходится, то и ряд в правой части расходится).
\кзадача

\сзадача
Пусть $p_n$\т $n$-е простое число, $n\in\N$.
\невСтрочку
\пункт
Докажите, что
$\limn\left(\frac1{1-1/p_1^2}\cdot\ldots\cdot\frac1{1-1/p_n^2}\right) = \sum \frac1{n^2}$.
\пункт
Существует ли предел $\limn\left(\frac1{1-1/p_1}\cdot\ldots\cdot\frac1{1-1/p_n}\right)$?
\пункт
Сходится ли ряд $\sum\frac1{p_n}$?
\кзадача

\РазделитьКондуит{0.3mm}{6.5mm}

\сзадача
\вСтрочку
\пункт
Пусть $\ga_k$\т сумма ряда $\sum\limits_{n=2}^{\infty}\frac1{n^k}$. Найдите сумму $\sum\limits_{k=2}^{\infty}\ga_k$.\\
\пункт[Эйлер]
Пусть $A$\т множество всех целых чисел, представимых в виде $n^k$, где $n,k$\т целые числа, большие 1. Найдите сумму~\hbox{$\sum\limits_{a\in A}\frac1{a-1}$}.
\кзадача

\сзадача[Число Лиувилля]
Докажите, что число $\xi = \sumnui\dfrac1{2^{n!}}$ является трансцендентным.
\кзадача

\newpage
\раздел{Признаки сходимости рядов}

\задача
\невСтрочку
\пункт[Признак сравнения Вейерштрасса]
Пусть $\sum a_n$, $\sum b_n$\т ряды с неотрицательными членами. Пусть найдётся такой номер $k$, что при всех $n>k$, $n\in\N$ будет выполнено неравенство $b_n\ge a_n$. Тогда если $\sum b_n$ сходится, то $\sum a_n$ сходится; если $\sum a_n$ расходится, то $\sum b_n$ расходится.
\пункт[Признак д'Аламбера]
Пусть члены ряда $\sum a_n$ положительны, и существует $\limn \frac{a_{n+1}}{a_n}=q$. Если $q<1$, то ряд сходится, а если
$q>1$, то ряд расходится. Что можно сказать о сходимости, если $q=1$?
\пункт[Признак Коши]
Пусть члены ряда $\sum a_n$ неотрицательны и существует $\limn \sqrt[n]{a_n}=q$. Если $q<1$, то ряд сходится, а если $q>1$, то ряд расходится. Что можно сказать о сходимости ряда, если $q=1$?
\пункт
Приведите пример сходящегося ряда с положительными членами, к которому применим признак Коши, но не применим признак д'Аламбера. Бывает ли наоборот?
\кзадача

\задача
\невСтрочку
\пункт[Теорема Лейбница]
Пусть $a_n>0$ при всех $n\in\N$, и кроме того, $a_1\ge a_2\ge a_3\ge\dots$; $\limn a_n=0$. Тогда знакочередующийся ряд $\sum (-1)^{n-1}a_n$ сходится.
\пункт
Верно ли утверждение теоремы без условия монотонности $(a_n)$?
\кзадача

\задача
Пусть $a_n>0$ при всех $n\in\N$, и кроме того, $a_1\ge a_2\ge a_3\ge\dots$. Докажите, что ряд $\sum a_n$ сходится или расходится одновременно с рядом $\sum 2^n a_{2^n}$.
\кзадача

\УвеличитьПробелы{-4mm}{0mm}
\задача
Исследуйте следующие ряды на сходимость:\\
\вСтрочку
\пункт
$\sum\sin\dfrac1{n^2}$;
\пункт
$\sum\tg\dfrac1n$;
\пункт
$\sum\sin(n\al)$;
\пункт
$\sum\limits_{n=2}^\infty\dfrac1{n\ln n}$;
\пункт
$\sum\dfrac{1\cdot3\cdot5\cdot\ldots\cdot(2n-1)}
{2\cdot4\cdot6\cdot\ldots\cdot2n}$;
\пункт
$\sum\dfrac{n^k}{a^n}$;
\пункт
$\sum\dfrac{a^n}{n!}$;
\пункт
$\sum\dfrac{e^n}{n!}$;
\пункт
$\sum\dfrac{n^3}{e^n}$;
\пункт
$\sum\dfrac{n!}{n^n}$;
\пункт
$\sum\dfrac{(n!)^2}{n^n}$;
\пункт
$\sum\left(\dfrac{n+1}{2n-1}\right)^n$;
\пункт
$\sum\dfrac{1}{\sqrt{n(n+1)}}$.
\кзадача
\ВосстановитьПробелы

\задача[Дзета-функция Римана]
Исследуйте сходимость ряда $\zeta(s) = \sum\dfrac1{n^s}$ в зависимости от параметра $s \in \R$.
\кзадача

\задача
Верно ли, что если ряды $\sum a_n$ и $\sum b_n$ сходятся, то сходится и ряд $\sum a_nb_n$?
\кзадача

\задача
Известно, что $a_n\ge 0$, $b_n \ge 0$ и  ряды $\sum a_n^2$ и $\sum b_n^2$ сходятся. Докажите, что ряд $\sum a_n b_n$ тоже сходится.
\кзадача

\задача
Известно, что $a_n\ge 0$ и ряд $\sum a_n^2$ сходится. Можно ли утверждать, что ряд $\sum \dfrac{a_n}{n}$ сходится?
\кзадача

%\GenXML
\ВставитьКондуит{0.57mm}{6.5mm}{-10mm}
%\СделатьКондуит{4mm}{8mm}

\end{document}












\раздел{Абсолютно и условно сходящиеся ряды}

\опр
Ряд $\sum a_n$ называется \выд{абсолютно сходящимся}, если сходится ряд $\sum|a_n|$.
\копр

\задача
Докажите, что абсолютно сходящийся ряд сходится.
\кзадача

\задача
Пусть ряд $\sum a_n$ абсолютно сходится. Тогда абсолютно сходится произвольный ряд $\sum b_n$, полученный из него перестановкой слагаемых, причём $\sum b_n=\sum a_n$.
\кзадача

\опр
Ряд $\sum a_n$ называется \выд{условно сходящимся}, если он сходится, но ряд $\sum|a_n|$ расходится.
\копр

\задача
Пусть ряд $\sum a_n$ сходится условно.\\
\пункт
Докажите, что ряд, составленный из его положительных (или отрицательных) членов, расходится.
\пункт[Теорема Римана]
Докажите, что ряд $\sum a_n$ можно превратить перестановкой слагаемых как в расходящийся ряд, так и в сходящийся с произвольной наперёд заданной суммой.
\пункт
Докажите, что можно так сгруппировать члены ряда $\sum a_n$ (не переставляя их), что ряд станет абсолютно сходящимся.
\спункт
Пусть $\sum a_n$\т ряд, составленный из комплексных чисел, $S$\т множество всех перестановок $\si$ натурального ряда, для которых ряд $\sum a_{\si(n)}$ сходится. Каким может быть множество $\{\sum a_{\si(n)} \mid \si\in S\}$?
\кзадача

\задача
Исследуйте ряды на абсолютную и условную сходимость:\\
\вСтрочку
\пункт
$\sum(-1)^{n+1}\frac1{\sqrt{n^3+1}}$;
\пункт
$\sum(-1)^n \frac{2n^2+1}{n^3+1}$;
\пункт
$\sum \frac{(-1)^{n+1}}{n^p}$;
\пункт
$\sum \frac{(-1)^{n+1}}{n^{p+\frac1n}}$;
\пункт
$\sum\frac{\sin n}{n^2}$;\\
\пункт
$\sum \frac{(-1)^{[\sqrt{n}]}}{n}$;
\пункт
$\sum \sin n^2$;
\спункт
$\sum \frac{\sin n}{n}$.
\кзадача

\задача
Пусть $s$\т сумма ряда $\sum \frac{(-1)^{n+1}}{n}$.
Найдите суммы\\
\вСтрочку
\пункт
$1+\frac13-\frac12+\frac15+\frac17-\frac14+\frac19+\frac1{11}-\frac16+\ldots$\,;
\пункт
$1-\frac12-\frac14+\frac13-\frac16-\frac18+\frac15-\frac1{10}-\frac1{12}+\ldots$\,.\\
\пункт
Переставьте члены ряда $\sum\frac{(-1)^{n+1}}{n}$ так, чтобы он стал расходящимся.
\кзадача

\задача
Существует ли такая последовательность $(a_n)$, $a_n\ne0$ при $n\in\N$, что ряды $\sum a_n$ и $\sum \frac1{n^2a_n}$ сходятся? Можно ли выбрать такую последовательность из положительных чисел?
\кзадача

\сзадача
Существует ли такая последовательность $(a_n)$, что ряд $\sum a_n$ сходится, а ряд $\sum a_n^3$ расходится?
\кзадача

\сзадача
Пусть функция $f\colon\R\to\R$ такова, что для любого сходящегося ряда $\sum a_n$ ряд $\sum f(a_n)$ сходится. Докажите, что тогда найдётся такое число $C\in\R$, что $f(x)=Cx$ в некоторой окрестности нуля.
\кзадача
