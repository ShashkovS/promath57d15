% !TeX encoding = windows-1251
\documentclass[a4paper,12pt]{article}
\usepackage{newlistok}

\УвеличитьВысоту{1.7cm}
\УвеличитьШирину{1cm}
\renewcommand{\spacer}{\vfil}
\sloppy

\НомерЛистка{40АлГ}
\ДатаЛистка{01.2009}
\Заголовок{Плоские алгебраические кривые}

\begin{document}
\СоздатьЗаголовок


\раздел{Часть 2. Теорема Безу}

\опр  {\it Плоской алгебраической кривой}\/ называют множество точек
плоскости, координаты $x_0,y_0$ которых удовлетворяют уравнению
\ $A(x_0,y_0)=0$,  где $A(x,y)$ --- некоторый
многочлен из $\R[x,y]$.
Говорят, что многочлен $A$ {\it задает}\/ эту кривую.
\копр

\задача  Нарисуйте плоские кривые, задающиеся следующими многочленами:\\
\вСтрочку
\пункт $x-y$;
\пункт $x^2-y^2$;
\пункт $y-x^2$;
\пункт $x^2+y^2-1$;
\пункт $xy-1$;
\пункт $х^2y-xy^2+y-x$;
\пункт $ax^2+by^2-1$, где $a,b$ --- такие числа, что $a>b>0$;
\пункт $ax^2-by^2-1$, где $a,b$ --- такие числа, что $a>b>0$;
\пункт $y^2-x^3$;\quad
\пункт $y-1-x^3$;\quad
\пункт $y^2-1-x^3$;\quad
\пункт $y^2-x-x^3$;\quad
\пункт $y^2-x^2-x^3$.
\кзадача

\сзадача[Р.Хартсхорн] Какому из уравнений соответствует каждая из
кривых, изо\-бра\-ж\"ен\-ных на рис.~справа:
\сНовойСтроки
\пункт $x^2=x^4+y^4$;
\пункт $xy=x^6+y^6$;
\пункт $x^3=y^2+x^4+y^4$;
\пункт $x^2y+xy^2=x^4+y^4$.
\кзадача


% \vspace*{-15mm}
% \putpict{6cm}{0cm}{list38alg_1}{}
% \putpict{9.3cm}{0cm}{list38alg_2}{}
% \putpict{12.2cm}{0cm}{list38alg_3}{}
% \putpict{15.5cm}{0cm}{list38alg_4}{}
% \vspace*{9mm}

%\vspace


\задача Пусть $A(x,y)$ --- такой многочлен из $\R[x,y]$, что
$A(x_0,y_0)=0$ при всех $x_0,\ y_0\in\R$.
Докажите, что тогда $A(x,y)$ --- нулевой многочлен.
\кзадача

\задача  Пусть $A$, $B$ --- различные многочлены из $\R[x,y]$.
Может ли система $A(x,y)=0$, $B(x,y)=0$ иметь
конечное число решений,  бесконечное число решений?
\кзадача

\задача
Дайте определение взаимно простых многочленов в $\R[x,y]$ и в $\R(y)[x]$.
%\пункт Дайте определение взаимно простых многочленов в
\кзадача


\задача %\вСтрочку
\пункт Верно ли, что для любых двух взаимно простых многочленов
$A,\ B$ из $\R[x,y]$ найдутся такие многочлены $U,\ V$
из $\R[x,y]$, что $AU+BV=1$?
\пункт Верно ли, что для любых двух взаимно простых
многочленов $A,\ B$ из $\R(y)[x]$  найдутся такие  многочлены
$U,\ V$ из $\R(y)[x]$,  что $AU+BV=1$?
\пункт Докажите, что для любых двух взаимно простых
многочленов $A,\ B$ из $\R[x,y]$  найдутся такие  многочлены
$U,\ V$ из $\R(y)[x]$,  что $AU+BV=1$.
\кзадача


\noindent
{\bf Соглашение.} Все рассматриваемые далее многочлены принадлежат $\R[x,y]$.

\задача Докажите, что если многочлены $A(x,y)$ и $B(x,y)$ взаимно просты,
то система $A(x,y)=0$, $B(x,y)=0$ имеет конечное число решений.
\кзадача


\задача  Решите систему уравнений
$$
\left\{
\begin{array}{rcl}
6y^2+2x^2-23xy+39y+6x&=&0,\\
6y^3+2x^3-2xy^2+6x^2-9xy-6y^2-27y&=&0.\\
\end{array}
\right.
$$
\кзадача


\задача Пусть $A(x,y)$, $B(x,y)$ --- ненулевые многочлены.
Докажите, что если система $A(x,y)=0$, $B(x,y)=0$
имеет бесконечное число решений
и %многочлен
$B$ неприводим, то $A$ делится~\hbox{на $B$.}
\кзадача


\задача Можно ли на плоскости задать многочленом ветвь гиперболы?
\кзадача

%\задача Пусть $A(x,y)$ --- такой многочлен, что
%$A(x_0,y_0)=0$ при всех $x_0,\ y_0\in\R$.
%Докажите, что тогда $A(x,y)$ --- нулевой многочлен.
%\кзадача

\задача  Еще Исаак Ньютон заметил следующий интересный факт, называемый
\выд{теоремой Безу:}\/
\выд{если $A(x,y)$ и $B(x,y)$ --- ненулевые взаимно простые многочлены,
то система
$A(x,y)=0$, $B(x,y)=0$ имеет не более $\deg A\cdot \deg B$ решений.}\/
Докажите
теорему Безу  для произвольного ненулевого многочлена $A$,
взаимно простого с многочленом $B$, если $B$ ---\\
\вСтрочку %\сНовойСтроки
\пункт ненулевое число;\quad
\пункт многочлен первой степени;\quad
\пункт произведение нескольких многочленов первой степени;\quad
\пункт многочлен $x-y^2$;\quad
\пункт многочлен $xy-1$;\quad
\пункт многочлен $y^2-x^3$;\\
\спункт  многочлен $x^2+y^2-1$;\quad
\спункт неприводимый многочлен второй степени.
\кзадача

\ссзадача Докажите теорему Безу в общем случае.
\кзадача

\задача[М.Берже, С.В.Маркелов]
На плоскости даны парабола $y=x^2$ и окружность, имеющие ровно две
общие точки: $A$ и $B$. Оказалось, что касательные к окружности и параболе
в точке $A$ совпадают. Обязательно ли тогда касательные
к окружности и параболе в точке $B$ также совпадают?
\кзадача



\end{document}
