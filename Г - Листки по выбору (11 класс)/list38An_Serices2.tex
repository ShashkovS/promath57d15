% !TeX encoding = windows-1251
\documentclass[a4paper,11pt]{article}
\usepackage{newlistok}

\renewcommand{\spacer}{\vfil}
\УвеличитьВысоту{1.8cm}
\УвеличитьШирину{1.7cm}


\sloppy

\Заголовок{Суммирование рядов}
\НомерЛистка{38:Анализ}
\ДатаЛистка{01.2009}

\begin{document}

\СоздатьЗаголовок

\раздел{Признаки сходимости рядов}



\задача
\пункт[Признак сравнения Вейерштрасса]
Пусть $\sum\limits_{n=1}^\infty a_n$, $\sum\limits_{n=1}^\infty b_n$ --- ряды с неотрицательными членами.
Пусть найдётся такой номер $k$, что при всех $n>k$, $n\in\N$
будет выполнено неравенство $b_n\geqslant a_n$. Тогда если $\sum\limits_{n=1}^\infty b_n$ сходится, то
$\sum\limits_{n=1}^\infty a_n$ сходится; если $\sum\limits_{n=1}^\infty a_n$ расходится, то
$\sum\limits_{n=1}^\infty b_n$ расходится.
%Верно ли это утверждение без предположения неотрицательности
%членов рядов?
\\\пункт[Признак д'Аламбера]
Пусть члены ряда $\sum\limits_{n=1}^\infty a_n$
положительны, и
существует %предел
$\lim\limits_{n\to\infty}\frac{a_{n+1}}{a_n}=q$.
Если $q<1$, то
ряд сходится, а если %$\lim\limits_{n\to\infty}\frac{a_{n+1}}{a_n}>1$,
$q>1$, то ряд расходится. Что можно сказать о сходимости, если $q=1$?
%в случае $\lim\limits_{n\to\infty}\frac{a_{n+1}}{a_n}=1$?
\\\пункт[Признак Коши]
Пусть члены ряда $\sum\limits_{n=1}^\infty a_n$ неотрицательны,
и существует %предел
%Если существует предел
$\lim\limits_{n\to\infty}\sqrt[n]{a_n}=q$.
Если $q<1$, то
ряд сходится, а если
%$\lim\limits_{n\to\infty}\sqrt[n]{a_n}>1$,
$q>1$, то ряд расходится. Что можно сказать о сходимости ряда,
%в случае $\lim\limits_{n\to\infty}\sqrt[n]{a_n}=1$?
если $q=1$?
\\\пункт
Приведите пример сходящегося ряда
с положительными членами, к которому применим признак Коши,
но не применим признак д'Аламбера. Бывает ли наоборот?
\кзадача

\задача
Исследуйте ряды на сходимость:\\
\вСтрочку
\пункт
$\sum\limits_{n=1}^\infty \frac1{n^p}$;
\пункт
$\sum\limits_{n=2}^\infty \frac1{n\ln n}$;
\пункт
$\sum\limits_{n=1}^\infty \frac{1\cdot3\cdot5\cdot\ldots\cdot(2n-1)}
{2\cdot4\cdot6\cdot\ldots\cdot2n}$;
\пункт
$\sum\limits_{n=1}^\infty \frac{n^k}{a^n}$.
\кзадача


\задача
\пункт[Теорема Лейбница]
Пусть $a_n>0$ при всех $n\in\N$, и кроме того, $a_1\geqslant a_2\geqslant
a_3\geqslant\dots$, $\lim\limits_{n\to\infty}a_n=0$.
Тогда знакочередующийся ряд $a_1-a_2+a_3-a_4+a_5-\dots$ сходится.
\\\пункт
Верно ли утверждение теоремы без условия монотонности $(a_n)$?
\кзадача

\раздел{Абсолютно и условно сходящиеся ряды}

\опр
Ряд $\sum\limits_{n=1}^\infty a_n$ называется
{\em абсолютно сходящимся}, если сходится ряд
$\sum\limits_{n=1}^\infty|a_n|$.
\копр

\задача
Докажите, что абсолютно сходящийся ряд сходится.
\кзадача

\задача
Пусть ряд $\sum\limits_{n=1}^\infty a_n$ абсолютно
сходится. Тогда абсолютно сходится произвольный ряд
$\sum\limits_{n=1}^\infty b_n$, полученный из него
перестановкой слагаемых, причём
$\sum\limits_{n=1}^\infty b_n=\sum\limits_{n=1}^\infty a_n$.
\кзадача

\опр
Ряд $\sum\limits_{n=1}^\infty a_n$ называется
{\em условно сходящимся}, если он сходится,
но ряд $\sum\limits_{n=1}^\infty|a_n|$ расходится.
\копр

\задача
Пусть ряд $\sum\limits_{n=1}^\infty a_n$ сходится условно.
\сНовойСтроки
\пункт
Докажите, что ряд, составленный из его положительных
(или отрицательных) членов, расходится.
\пункт
({\em Теорема Римана.})
Докажите, что ряд $\sum\limits_{n=1}^\infty a_n$ можно превратить
перестановкой слагаемых как в расходящийся ряд, так и в сходящийся
с произвольной наперёд заданной суммой.
\пункт
Докажите, что можно так сгруппировать члены ряда
$\sum\limits_{n=1}^\infty a_n$ (не переставляя их),
что ряд станет абсолютно сходящимся.
\спункт
Пусть $\sum\limits_{n=1}^\infty a_n$~--- ряд, составленный
из комплексных чисел, $S$ --- множество всех перестановок $\sigma$
натурального ряда, для которых ряд $\sum\limits_{n=1}^\infty a_{\sigma(n)}$
сходится. Каким может быть множество
$\{\sum\limits_{n=1}^\infty a_{\sigma(n)}\ |\ \sigma\in S\}$?
\кзадача



\задача
Пусть $s$ --- сумма ряда
%известно, что
$\sum\limits_{n=1}^\infty \frac{(-1)^{n+1}}{n}$. %=\ln 2$.
Найдите суммы\\
\вСтрочку
\пункт
$1+\frac13-\frac12+\frac15+\frac17-\frac14+\frac19+\frac1{11}-\frac16+\ldots$\,;
\пункт
$1-\frac12-\frac14+\frac13-\frac16-\frac18+\frac15-\frac1{10}-\frac1{12}+\ldots$\,.\\
\пункт
Переставьте члены ряда $\sum\limits_{n=1}^\infty \frac{(-1)^{n+1}}{n}$
так, чтобы он стал расходящимся.
%\спункт
%Докажите равенство
%$\sum\limits_{n=1}^\infty \frac{(-1)^{n+1}}{n}=\ln 2$.
\кзадача


%\сзадача
%Докажите:
%\пункт
%$1+\frac12+\frac13+\frac14+\dots+\frac1n=C+\ln n+\varepsilon_n$,
%где $C$~--- константа ({\em постоянная Эйлера}),
%$\lim\limits_{n\to\infty}\varepsilon_n=0$;
%\пункт
%({\em тождество Каталана})
%$1-\frac12+\frac13-\frac14+\dots+\frac1{2n}=
%\frac1{n+1}+\frac1{n+2}+\dots+\frac1{2n}$;
%\пункт
%$\sum\limits_{n=1}^\infty \frac{(-1)^{n+1}}{n}=\ln 2$.

\задача
Существует ли такая последовательность $(a_n)$, $a_n\ne0$ при $n\in\N$,
что ряды $\sum\limits_{n=1}^\infty a_n$ и
$\sum\limits_{n=1}^\infty \frac1{n^2a_n}$ сходятся?
Можно ли выбрать такую последовательность из
положительных чисел?
\кзадача


\сзадача
Существует ли такая последовательность $(a_n)$, что
ряд $\sum\limits_{n=1}^\infty a_n$ сходится, а ряд
$\sum\limits_{n=1}^\infty a_n^3$ расходится?
\кзадача



%\сзадача
%Найдётся ли такая перестановка $\sigma$ натурального ряда, что
%для неё существует сходящийся ряд, который она переставляет
%в расходящийся, и существует расходящийся ряд, который она
%переставляет в сходящийся?

\сзадача
Пусть функция $f\colon\R\to\R$ такова, что для любого сходящегося
ряда $\sum a_n$ ряд $\sum f(a_n)$ сходится. Докажите, что тогда найдётся
такое число $C\in\R$, что $f(x)=Cx$ в некоторой окрестности нуля.
\кзадача

%\СделатьКондуитИз{6.2mm}{6.2mm}{sp_An.tex}


\end{document} 