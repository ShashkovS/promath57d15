% !TeX encoding = windows-1251
\documentclass[a4paper,12pt]{article}
\usepackage{newlistok}
\usepackage[framemethod=tikz]{mdframed}



\newenvironment{напоминание}{\medskip\textbf{Напоминание.}}{\par}

\ВключитьКолонитул

\УвеличитьВысоту{2.5cm}
\УвеличитьШирину{1.5cm}


\Заголовок{Группы преобразований}
\НомерЛистка{}
\ДатаЛистка{}

\begin{document}
\СоздатьЗаголовок
\begin{напоминание}
Отображение $\varphi\colon X\to Y$ из множества~$X$ в~множество~$Y$ называется \emph{взаимно однозначным} (или \emph{биекцией}), если для каждого элемента $y\in Y$ существует ровно один элемент~$x$ такой, что $\varphi(x)=y$.\par
Преобразование~$\psi$ называется \emph{тождественным}, если для каждого~$x\in X$ выполнено равенство $\psi(x)=x$. Обозначение: $\psi=\id_X$.\par
Отображение $\varphi\colon X\to Y$ называется \emph{обратным} для отображения $\psi\colon Y\to X$, если справедливы равенства $\varphi\circ\psi=\id_Y$ и~$\psi\circ\varphi=\id_X$. Обозначение: $\varphi=\psi^{-1}$\par
Количество элементов во множестве~$X$ обозначается через~$|X|$ или~$\#X$.
\end{напоминание}

\опр
\emph{Преобразованием} множества~$X$ называется любая биекция $\varphi\colon X\to X$. Для множества всех преобразований~$X$ зарезервировано обозначение~$S(X)$.
\копр

\опр
\emph{Группой преобразований} множества~$X$ называется всякая непустая совокупность его преобразований~$G$, удовлетворяющая следующим свойствам:
\begin{items}{-5}
\item[(i)]
$G$~замкнута относительно композиции, то есть для всех $g,h\in G$ верно: $g\circ h\in G$;
\item[(ii)]
$G$~замкнута относительно взятия обратного преобразования, то есть для всех $g\in G$  преобразование~$g^{-1}$ лежит в~$G$.
\end{items}
\копр

\задача
Докажите, что группа преобразований любого множества содержит тождественное преобразование.
\кзадача


\задача
\label{square}
Пусть множество~$X$\т это квадрат~$ABCD$. Обозначим через~$s_{ac}$, $s_{bd}$, $s_H$ и~$s_V$ симметрии относительно диагонали~$AC$, диагонали~$BD$, горизонтали и~вертикали квадрата соответственно. Далее, обозначим через~$r_0$, $r_1$, $r_2$ и~$r_3$ повороты вокруг центра квадрата на $0^\circ$, $90^\circ$, $180^\circ$ и~$270^\circ$ соответственно.
\невСтрочку
\пункт
Докажите, что $G=\{s_{ac},s_{bd},s_H,s_V,r_0,r_1,r_2,r_3\}$ образует группу преобразований квадрата.
\пункт
Выпишите таблицу умножения в этой группе.
\пункт
Придумайте группу преобразований квадрата, состоящую из четырёх преобразований.
\кзадача


\задача
\label{sym}
\невСтрочку
\пункт
Докажите, что для любого множества~$X$ множество~$S(X)$ является группой;
\пункт
Пусть $X$\т конечно, причём~$|X|=n$. Найдите $|S(X)|$.
\кзадача
\замечание
В~условиях задачи~\ref{sym}б) группа $S(X)$~называется \emph{симметрической группой} и~обозначается~$S_n$.
\кзамечание


\задача
\label{triangle}
\пункт
Опишите все преобразования правильного треугольника, сохраняющие расстояния между любыми двумя его точками.\\
\пункт
Докажите, что эти преобразования образуют группу.
\кзадача


\опр
\emph{Порядком элемента}~$g$ группы преобразований~$G$ называется наименьшее натуральное~$k$ такое, что $g^k=\underbrace{g\circ\dots\circ g}_k=\id$. Обозначение: $\ord(g)$.
\копр

\опр
\emph{Порядком группы}~$G$ называется количество элементов в~$G$. Обозначение: $|G|$ или~$\#G$.
\копр


\задача
Найдите порядок каждого элемента групп из задач \ref{square} и~\ref{triangle}.
\кзадача


\задача
Пусть множество~$X$ является подмножеством прямой, плоскости или пространства. Рассмотрим множество $\Isom(X)=\{\varphi\in S(X)\mid \varphi\text{ сохраняет расстояния}\}$. Докажите, что вне зависимости от~$X$ множество преобразований $\Isom(X)$ является группой. Эта группа называется \emph{группой движений}~$X$.
\кзадача


\задача
\label{ord}
Перечислите все элементы и~их порядки в~группах движений следующих множеств:\\
\пункт
прямоугольник;
\пункт
правильный $m$-угольник;
\пункт
правильный тетраэдр;
\пункт
куб;
\спункт
октаэдр;
\спункт
икосаэдр;
\спункт
додекаэдр.
\кзадача
\noindent\help{Как связаны между собой куб и~октаэдр? Тот же вопрос для икосаэдра и~додекаэдра.}

\замечание
Группа из задачи~\ref{ord}б) называется \emph{группой диэдра} и~обозначается $D_m$.
\кзамечание


%\vfill
\ЛичныйКондуит{0mm}{6mm}
\ОбнулитьКондуит
\newpage

\опр \выд{Орбитой} элемента $x \in X$ при действии группы преобразований $G$ называется множество $\{g(x) \mid g \in G\} \subset X$.
\выдд Обозначение $Gx$.
\копр

\задача
Найдите орбиту каждой точки при действии группы движений\\
\пункт
квадрата;
\пункт
куба;
\пункт
правильного $m$-угольника.
\кзадача



\задача
\пункт
Опишите группу движений единичного круга;
\пункт
Найдите орбиту каждой точки при действии этой группы;
\пункт
Найдите преобразование, не имеющее конечного порядка.
\кзадача

\задача
Докажите, что любые две орбиты либо совпадают, либо не пересекаются. Следует ли отсюда,
что всё множество $X$ есть объединение непересекающихся орбит?
\кзадача

\задача
Докажите, что для любых двух элементов одной орбиты $a,b \in Gx$ найдётся элемент $g \in G$,
такой что $g(a) = b$.
\кзадача


\опр \выд{Стабилизатором\/} элемента $x \in X$ при действии группы преобразований $G$ называется
множество $\{g \mid g(x)=x\} \subset G$.
\выдд Обозначение: $G_x$.
\копр

\задача
Найдите стабилизаторы каждой из точек следующих множеств при действии их групп движений:
\пункт
квадрата;
\пункт
куба;
\пункт
правильного $m$-угольника.
\кзадача

\задача
Рассмотрим группу движений куба $G$. Эта группа также является группой преобразований следующих множеств:
\пункт множества вершин куба;
\пункт множества диагоналей куба;
\пункт множества граней куба;
\спункт множества пар вершин куба.
Опишите орбиты и стабилизаторы во всех случаях.
\кзадача

\задача
Пусть задана группа преобразований $G$ множества $X$. Докажите, что стабилизатор
любого элемента $x\in X$ также является группой преобразований множества $X$.
\кзадача

\задача
Пусть группа $G$ конечна. Докажите, что для любых двух элементов одной орбиты $a,b \in Gx$ выполнено $|G_a| = |G_b|$.
\кзадача

\задача
Пусть группа $G$ конечна. Докажите, что для любого $x \in X$ верно $|G| = |Gx| \cdot |G_x|$.
\кзадача

\задача
Пусть $p$ --- простое число.
Рассмотрим множество $\Z_p$ остатков по модулю $p$ и группу~$G$, состоящую из ненулевых остатков, действующих на~$\Z_p$ домножениями (т.е. $G=\{1, 2,\ldots,p-1\}$ и $g(x) = x\cdot g$).
\\\пункт Найдите орбиты действия этой группы;
\\\пункт[малая теорема Ферма] Докажите, что $a^{p-1} \equiv 1 \pmod{p}$.
\кзадача

\опр
Функция, равная количеству натуральных чисел, меньших $n$ и взаимно простых с ним, называется \выд{функцией Эйлера} и обозначается через $\phi(n)$.
\копр

\задача
Пусть $n$ --- произвольное число.
Рассмотрим множество $\Z_n$ остатков по модулю $n$, и группу~$G$, состоящую из остатков, взаимно простых с $n$, действующих на~$\Z_n$ домножениями.
\\\пункт Докажите, что такое множество преобразований образует группу;
\\\пункт Найдите орбиты действия этой группы;
\\\пункт[теорема Эйлера]
Докажите, что если числа $a$ и $n$ взаимно просты, то $a^{\phi(n)} \equiv 1\pmod{n}$.
\кзадача

\задача
Образует ли группу множество преобразований плоскости, переводящих прямые в прямые? Что это за преобразования?
\кзадача


\ЛичныйКондуит{0mm}{6mm}


\end{document}




