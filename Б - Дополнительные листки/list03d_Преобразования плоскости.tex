% !TeX encoding = windows-1251
\documentclass[a4paper,12pt]{article}
\usepackage{newlistok}

%\УвеличитьВысоту{1.5cm}
\УвеличитьШирину{1.5cm}
%\renewcommand{\spacer}{\vfil}

\Заголовок{Движения плоскости}
\НомерЛистка{3д}
\ДатаЛистка{10.2012}

\begin{document}
\СоздатьЗаголовок

\vskip 6mm
Мы будем обозначать расстояние между точками $A$ и $B$ плоскости
через $d(A, B)$. Напомним основные свойства расстояния:
\begin{items}{0}
\item $d(A, B) \ge 0$, причём равенство достигается тогда и только тогда, когда $A = B$;
\item $d(A, B) = d(B, A)$;
\item $d(A, B) + d(B, C) \ge d(A, C)$, причём равенство достигается тогда и только тогда, когда точка $B$ лежит на отрезке $AC$ (<<неравенство треугольника>>).
\end{items}

\опр
\label{isometry}
\выд{Движением плоскости} называется взаимно однозначное отображение $f \colon \R^2 \to \R^2$ плоскости на себя, которое сохраняет расстояния, т. е.
\[
\fa A,B \in \R^2 \quad d(A, B) = d(f(A), f(B)).
\]
\копр

\задача
\пункт Покажите, что если $f$\т движение, то обратное отображение $f^{-1}$\т тоже движение.
\пункт Если $g$\т ещё одно движение, то композиция $f\circ g$\т снова движение.
\пункт Выведите основные свойства движений: движения переводят прямые в прямые, окружности\т в окружности; движения сохраняют параллельность и углы между прямыми и окружностями.
\спункт Докажите, что условие взаимной однозначности в определении~\ref{isometry} является излишним.
\кзадача

\опр
\emph{Центральная симметрия с центром в точке $O$}\т это такое отображение плоскости на себя, при котором точка $O$ переходит в себя, а всякая другая точка $X$ переходит в такую точку $X'$, что~$O$ есть середина отрезка $XX'$.
\копр


\задача
Двое игроков выкладывают по очереди на прямоугольный стол пятаки. Монету разрешается класть только на свободное место. Проигрывает тот, кто не может сделать ход. Кто выиграет при~правильной игре?
\кзадача


\задача
Докажите, что
\невСтрочку
\пункт центральная симметрия\т движение;
\пункт если некоторая фигура\footnote{Фигурой в геометрии принято называть произвольное множество точек на плоскости.} имеет два центра симметрии, то их у неё бесконечно много.
\кзадача


\задача
\пункт Дан вписанный четырёхугольник. Через середину каждой его стороны провели прямую, перпендикулярную противоположной стороне. Докажите, что проведённые прямые пересекаются в~одной точке.
\пункт Пусть у выпуклого $n$-угольника нет параллельных сторон, $A$\т некоторая точка, не~лежащая на~сторонах многоугольника. Докажите, что тогда существует не более~$n$ отрезков с~концами на~сторонах многоугольника и с~серединой в~$A$.
\кзадача

\опр
\emph{Осевая} (или \emph{зеркальная}) \emph{симметрия относительно прямой $l$}\т это такое отображение плоскости на~себя, при~котором точки прямой~$l$ остаются на~месте, а всякая точка~$X$, не лежащая на~этой прямой, переходит в~такую точку~$X'$, что~$l$\т серединный перпендикуляр к~отрезку~$XX'$.
\копр


\задача
Стёпа хочет половить навагу в Баренцевом море, а селёдку\т в Белом. Как следует ему выбрать кратчайший путь с началом и концом на станции Хибины, если Кольский полуостров имеет форму
\пункт острого угла;
\спункт тупого угла?
\кзадача

\vfill
\ЛичныйКондуит{0mm}{6mm}
\ОбнулитьКондуит
\newpage

\задача
Докажите, что
\невСтрочку
\пункт осевая симметрия\т движение;
\пункт если некоторое движение оставляет все точки прямой неподвижными, то это либо тождественное отображение, либо симметрия относительно этой прямой;
\спункт угол между двумя осями симметрии многоугольника имеет рациональную градусную меру.
\кзадача


\задача
По одну сторону от прямой $l$ расположены точки $A$ и $B$. Постройте такую точку $X$ на прямой~$l$, что углы между $AX$ и $BX$ и прямой $l$
\пункт равны;
\спункт отличаются в два раза.
\кзадача


\задача
\пункт Докажите, что площадь любого выпуклого четырёхугольника не превосходит полусуммы произведений противоположных сторон.
\пункт Когда достигается равенство?
\кзадача


\задача
Может ли луч света <<застрять>> в угле, образованном двумя зеркалами?
\кзадача

\сзадача
Две прямые пересекаются под углом $\ga^{\circ}$. Кузнечик прыгает с одной прямой на другую; длина каждого прыжка равна $1$, и кузнечик не прыгает обратно, если только это возможно. Докажите, что последовательность его прыжков периодична тогда и только тогда, когда $\ga$ рационально.
\кзадача

\опр
\emph{Параллельный перенос $T_{AB}$ на вектор $AB$}\т это такое отображение плоскости на себя, при котором всякая точка $X$ переходит в такую точку $X'$, что середины отрезков $AX'$ и $BX$ совпадают. (Уточните это определение для случаев, когда точки $B$ и $X$ совпадают или точка $A$ является серединой~$BX$.)
\копр


\задача
В каком месте следует построить мост $MN$ через реку, разделяющую деревни $A$ и $B$, чтобы путь $AMNB$ из $A$ в $B$ был кратчайшим? (Берега реки считаются параллельными прямыми; мост перпендикулярен берегам.)
\кзадача


\задача
Докажите, что
\невСтрочку
\пункт параллельный перенос\т движение;
\пункт композиция параллельных переносов на векторы $AB$ и $BC$ есть параллельный перенос на вектор~$AC$.
\кзадача


\задача
В квадрате со стороной $1$ расположена фигура, расстояние между любыми двумя точками которой не равно $0.001$. Докажите, что площадь этой фигуры не превосходит \пункт $0.34$; \спункт $0.287$.
\невСтрочку
\спункт Приведите пример такой фигуры площадью не менее $0.22$.
\кзадача

\vfill
\ЛичныйКондуит{0mm}{6mm}

%\GenXML

\end{document}




