% !TeX encoding = windows-1251
\documentclass[a4paper,12pt]{article}
\usepackage{newlistok}
\usepackage[framemethod=tikz]{mdframed}


\global\addtolength{\vsize}{-115mm}%
\global\advance\vsize by 10cm
\global\advance\vsize by 10cm

\ВключитьКолонитул

\УвеличитьВысоту{1.5cm}
\УвеличитьШирину{1.5cm}
% \renewcommand{\spacer}{\vfil}

\Заголовок{Суммирование рядов\т 1}
\НомерЛистка{8д}
\ДатаЛистка{01.2014}

\begin{document}
\СоздатьЗаголовок

\опр
Пусть $(a_n)$\т числовая последовательность. Формальное выражение $a_1+a_2+a_3+\ldots=\sumnui a_n$ называется {\em рядом}. Число $s_n=a_1+a_2+\dots+a_n$ называется \emph{$n$-ой частичной суммой} ряда.\\
Говорят, что ряд $\sumnui a_n$ \emph{сходится и имеет сумму $A$}, если существует $\limn s_n=A$. Тогда пишут $\sumnui a_n=A$. Если предел  $\limn s_n$ не существует, то говорят, что ряд $\sumnui a_n$ \emph{расходится}.
\копр

\задача
Пусть $a_n\geq0$ при $n\in\N$. Докажите, что ряд $\sumnui a_n$ сходится тогда и только тогда, когда ограничено множество его частичных сумм $\{s_n \mid n\in\N\}$, причём в этом случае $\sumnui a_n=\sup\{s_n\ |\ n\in\N\}$.
\кзадача

\задача
Какие из следующих рядов сходятся? Найдите их суммы.\\
\вСтрочку
\пункт
$\sumnui (-1)^n$;
\пункт
$\sumnui \dfrac1{2^n}$;
\пункт[геометрическая прогрессия]
$\sumnui \dfrac1{q^n}$, $q\in\R,\ q\ne0$;\\
\пункт[гармонический ряд]
$\sumnui \dfrac1n$;
\пункт
$\sumnui \dfrac{n}{2^n}$;
\спункт
$\sumnui \dfrac{n^2}{2^n}$;
\пункт
$\sumnui \dfrac1{n(n+1)}$.
\кзадача

\задача
\пункт
Докажите, что если ряд $\sumnui a_n$ сходится, то $\limn a_n=0$. Верно ли обратное?\\
\пункт[Критерий Коши сходимости ряда]
Докажите, что ряд $\sumnui a_n$ сходится тогда и только тогда, когда для любого $\ep>0$ существует такое $N$, что из $n \ge m > N$ (где $n,m\in\N$) следует $|a_m+a_{m+1}+\ldots+a_n|<\ep$.
\кзадача



\задача
Сходятся ли следующие
ряды:
\вСтрочку
\пункт
$\sumnui \dfrac{(-1)^n}{n}$;
\пункт
$\sumnui \dfrac1{\sqrt{n}}$;
\пункт
$\sumnui \dfrac1{n^2}$;
\кзадача


\задача
Верно ли, что если ряды $\sumnui a_n$ и $\sumnui b_n$ сходятся, то сходится ряд $\sumnui a_nb_n$?
\кзадача

\задача
Докажите:
\вСтрочку
\пункт
ряд $\sumnui \dfrac1{n!}$ сходится;
\пункт
$\sumnui \dfrac1{n!}=e$;
\пункт
$e-\sum\limits_{n=1}^{m} \dfrac1{n!}<\dfrac1{m!\,m}$;\\
\пункт
число $e$ иррационально.
\кзадача

\задача
Пусть $a_n \ge 0$ при всех $n\in\N$ и $\sigma\colon\N\to\N$\т взаимно однозначное отображение (перестановка натурального ряда). Тогда $\sumnui a_n=
\sumnui a_{\sigma(n)}$ (то есть если сходится ряд в левой части равенства, то сходится и ряд в правой части, причём их суммы равны; если ряд в левой части расходится, то и ряд в правой части расходится).
\кзадача


\сзадача
Пусть $p_n$\т $n$-е простое число, $n\in\N$.
\невСтрочку
\пункт
Докажите, что $\limn\left(\dfrac1{1-1/p_1^2}\cdot\ldots\cdot\dfrac1{1-1/p_n^2}\right)=
\sumnui \dfrac1{n^2}$.
\пункт
Существует ли предел $\limn\left(\dfrac1{1-1/p_1}\cdot\ldots\cdot\dfrac1{1-1/p_n}\right)$?
\пункт
Сходится ли ряд $\sumnui \dfrac1{p_n}$?
\кзадача


\сзадача
\вСтрочку
\пункт
Пусть $\ga_k$\т сумма ряда $\sum\limits_{n=2}^{\infty}\dfrac1{n^k}$. Найдите сумму $\sum\limits_{k=2}^{\infty}\ga_k$.\\
\пункт[Эйлер]
Пусть $A$\т множество всех целых чисел, представимых в виде $n^k$, где $n,k$\т целые числа, большие 1. Найдите сумму $\sum\limits_{a\in A}\dfrac1{a-1}$.
\кзадача

%\vfill
\ЛичныйКондуит{0mm}{6mm}
%\GenXMLW

\end{document}

