% !TeX encoding = windows-1251
\documentclass[a4paper,12pt]{article}
\usepackage{newlistok}

\ВключитьКолонитул

\УвеличитьВысоту{1.5cm}
\УвеличитьШирину{0.5cm}
%\renewcommand{\spacer}{\vfil}

\Заголовок{Суммирование рядов\т 2}
\НомерЛистка{10д}
\ДатаЛистка{\ммгг}

\begin{document}
\СоздатьЗаголовок
\vspace*{-1.5cm}

\раздел{Признаки сходимости рядов}

\задача
Докажите следующие признаки сходимости:
\невСтрочку
\пункт[Признак сравнения Вейерштрасса]
Пусть $\sumnui a_n$, $\sumnui b_n$\т ряды с неотрицательными членами. Пусть найдётся такой номер $k$, что при всех $n>k$, $n\in\N$ будет выполнено неравенство $b_n \ge a_n$.
Тогда если $\sumnui b_n$ сходится, то $\sumnui a_n$ сходится; если $\sumnui a_n$ расходится, то $\sumnui b_n$ расходится.
\medskip
\пункт[Признак д'Аламбера]
Пусть члены ряда $\sumnui a_n$ положительны, и существует предел $\limn\dfrac{a_{n+1}}{a_n} = q$.
Тогда если $q<1$, то ряд сходится, а если $q>1$, то ряд расходится.
Что можно сказать о сходимости ряда, если $q=1$?
\medskip
\пункт[Признак Коши]
Пусть члены ряда $\sumnui a_n$ неотрицательны, и существует предел $\limn\sqrt[n]{a_n} = q$.
Тогда если $q<1$, то ряд сходится, а если $q>1$, то ряд расходится.
Что можно сказать о сходимости ряда, если $q=1$?
\medskip
\пункт[Телескопический признак]
Пусть последовательность ${a_n}$ неотрицательна и монотонно невозрастает.
Тогда ряд $\sumnui a_n$ сходится или расходится одновременно с рядом $\sumnui 2^n a_{2^n}$.
\кзадача

\задача
Приведите пример сходящегося ряда с положительными членами, к которому применим признак Коши, но не применим признак д'Аламбера. Бывает ли наоборот?
\кзадача

\задача
Исследуйте ряды на сходимость:\\%
\вСтрочку%
\тааааа{%
\пункт
$\sumnui \dfrac1{2n-1}$;
}{\пункт
$\sumnui \dfrac{\sin nx}{2^n}$;
}{\пункт
$\sumnui \dfrac{\cos x^n}{n^2}$;
}{\пункт
$\sumnui \dfrac{(n!)^2}{(2n)!}$;
}{\пункт
$\sumnui \dfrac{n!}{n^n}$;
}%
\medskip%
\тааааа{%
\пункт
$\sumnui \dfrac{n^2}{(2 + 1/n)^n}$;
}{\пункт
$\sum\limits_{n=2}^\infty \dfrac1{n\ln n}$;
}{\пункт
$\sum\limits_{n=2}^\infty \dfrac1{n(\ln n)^2}$;
}{\пункт
$\sumnui \dfrac1{n^p}$, $p \in \R$;
}{\пункт
$\sumnui \dfrac{n^k}{a^n}$;
}
\medskip%
\пункт
$\sumnui \dfrac{1\cdot3\cdot5\cdot\ldots\cdot(2n-1)}
{2\cdot4\cdot6\cdot\ldots\cdot2n}$.
\кзадача



\раздел{Абсолютно и условно сходящиеся ряды}

\опр
Ряд $\sumnui a_n$ называется \emph{абсолютно сходящимся}, если сходится ряд $\sumnui|a_n|$.
\копр

\задача
Докажите, что абсолютно сходящийся ряд сходится.
\кзадача

\задача
Пусть ряд $\sumnui a_n$ абсолютно сходится. Тогда абсолютно сходится произвольный ряд $\sumnui b_n$, полученный из него перестановкой слагаемых, причём $\sumnui b_n=\sumnui a_n$.
\кзадача


\ЛичныйКондуит{-0.3mm}{6mm}
\ОбнулитьКондуит
\newpage

\опр
Ряд $\sumnui a_n$ называется \emph{условно сходящимся}, если он сходится, но ряд $\sumnui|a_n|$ расходится.
\копр

\задача
Пусть ряд $\sumnui a_n$ сходится условно.
\сНовойСтроки
\пункт
Докажите, что ряд, составленный из его положительных (или отрицательных) членов, расходится.
\пункт[Теорема Римана]
Докажите, что ряд $\sumnui a_n$ можно превратить перестановкой слагаемых как в расходящийся ряд, так и в сходящийся с произвольной наперёд заданной суммой.
\пункт
Докажите, что можно так сгруппировать члены ряда $\sumnui a_n$ (не переставляя их), что ряд станет абсолютно сходящимся.
\спункт
Пусть $\sumnui a_n$\т ряд, составленный из комплексных чисел, $S$\т множество всех перестановок $\sigma$ натурального ряда, для которых ряд $\sumnui a_{\sigma(n)}$ сходится. Каким может быть множество $\{\sumnui a_{\sigma(n)}\ |\ \sigma\in S\}$?
\кзадача



\задача
Пусть $s$\т сумма ряда $\sumnui \dfrac{(-1)^{n+1}}{n}$. Найдите суммы\medskip\\
\вСтрочку
\пункт
$1+\dfrac13-\dfrac12+\dfrac15+\dfrac17-\dfrac14+\dfrac19+\dfrac1{11}-\dfrac16+\ldots$\,;
\пункт
$1-\dfrac12-\dfrac14+\dfrac13-\dfrac16-\dfrac18+\dfrac15-\dfrac1{10}-\dfrac1{12}+\ldots$\,.\medskip\\
\пункт
Переставьте члены ряда $\sumnui \dfrac{(-1)^{n+1}}{n}$ так, чтобы он стал расходящимся.
\кзадача


\задача
Существует ли такая последовательность $(a_n)$, $a_n\ne0$ при $n\in\N$, что ряды $\sumnui a_n$ и $\sumnui \dfrac1{n^2a_n}$ сходятся? Можно ли выбрать такую последовательность из положительных чисел?
\кзадача


\сзадача
Существует ли такая последовательность $(a_n)$, что ряд $\sumnui a_n$ сходится, а ряд $\sumnui a_n^3$ расходится?
\кзадача


\сзадача
Пусть функция $f\colon\R\to\R$ такова, что для любого сходящегося ряда $\sumnui a_n$ ряд $\sumnui f(a_n)$ сходится. Докажите, что тогда найдётся такое число $C\in\R$, что $f(x) = Cx$ в некоторой окрестности нуля.
\кзадача


%\vfill
\ЛичныйКондуит{0mm}{6mm}
%\GenXMLW

\end{document}

