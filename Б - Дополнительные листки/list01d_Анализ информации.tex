% !TeX encoding = windows-1251
\documentclass[a4paper,12pt]{article}
\usepackage[cp1251]{inputenc}
\usepackage[mag=950]{newlistok}
\usepackage{upquote}

\УвеличитьШирину{1.5cm}
\УвеличитьВысоту{2.9cm}
\Заголовок{Анализ информации.}
\НомерЛистка{1д}
\ДатаЛистка{09.2012}
\renewcommand{\spacer}{\vfil}

%\catcode156=11 % Ё
%\catcode188=11 % ё
%\catcode168=11 % Ё
%\catcode184=11 % ё



\begin{document}



\СоздатьЗаголовок

\задача
Загадано натуральное число от 1 до 100. Можно задавать вопросы,
на которые дается ответ \лк да\пк\ или \лк нет\пк. За какое наименьшее число
вопросов всегда можно отгадать число, если
\сНовойСтроки
\пункт
каждый следующий вопрос задается после того, как
получен ответ на предыдущий вопрос;
\пункт
надо заранее сказать все вопросы?
\кзадача


\сзадача
В каждую клетку доски $8\times8$ записано целое число от 1 до 64
(каждое по одному разу).
За один вопрос, указав любую совокупность полей, можно
узнать множество чисел, стоящих на этих полях
(без указания, какую клетку занимает каждое из этих чисел).
За какое наименьшее число
вопросов всегда можно выяснить, какие числа где стоят?
\кзадача



\задача
\пункт
Задуманы $k$ натуральных чисел $a_1, a_2,\ldots ,a_k$, меньшие $1000$.
За один вопрос разрешается выбрать любые натуральные числа
$b_1, b_2,\ldots ,b_k$, и узнать сумму $a_1b_1+a_2b_2+\ldots +a_kb_k$.
За какое наименьшее количество вопросов можно наверняка
отгадать все задуманные числа?
\пункт Та же задача, но задуманы произвольные (не обязательно
меньшие $1000$) натуральные числа.
\кзадача

\задача
Я задумал целое число от 1 до 3. Придумайте вопрос,
на который я честно должен ответить \лк Да\пк, \лк Нет\пк\
или \лк Не знаю\пк,
после чего вы наверняка отгадаете задуманное число.
\кзадача

\задача
\пункт
Король собрал 1000 придворных мудрецов и объявил, что завтра устроит им
испытание. Мудрецам завяжут глаза, наденут каждому на голову колпак одного
из двух цветов, построят в колонну, затем развяжут глаза.
После этого мудрецы по очереди, начиная с последнего, будут называть
какой-нибудь цвет из возможных.
Кто назовет цвет своего колпака неправильно --- тому голову с плеч.
Сколько мудрецов гарантированно может спастись?
\пункт Та же задача, но колпаки могут быть $k$ разных цветов.\quad
(Каждый видит всех впереди
стоящих; у мудрецов до испытания есть время, чтобы договориться.)
\кзадача

\задача
На столе в ряд стоят 18 гирек, из которых некоторые три подряд идущих --- фальшивые. Известно, что настоящая гирька весит 1кг, а фальшивая --- 900г. Есть электронные весы (с одной чашей). За какое наименьшее число взвешиваний можно определить, какие гирьки настоящие, а какие фальшивые?
\кзадача


\задача
Геологи взяли в экспедицию 80 банок консервов, веса которых все известны
и различны (имеется список). Вскоре надписи на банках
стали нечитаемыми, и только завхоз знает, где что. Он хочет
доказать всем, что в какой банке находится, не вскрывая
консервов и пользуясь только списком и двухчашечными
весами со стрелкой, показывающей разницу весов на чашках. Докажите,
что ему для этого
\вСтрочку
\пункт хватит четырёх взвешиваний;
\пункт не хватит трёх взвешиваний.
\кзадача

\задача
Из 11 шаров два радиоактивны. Про любой набор шаров за одну проверку
можно узнать, имеется ли в нём хотя бы один радиоактивный шар (но
нельзя узнать, сколько их).
За какое наименьшее число проверок
можно гарантированно найти оба радиоактивных шара?
\кзадача

\сзадача
\вСтрочку
\пункт Есть $M$ монет, из них одна фальшивая. Настоящие весят
одинаково; фальшивая отличается по весу, но не известно,
легче она или нет. Есть еще эталонная (настоящая) монета.
Разрешено сделать $n$ взвешиваний на
весах с двумя чашами. При каком наибольшем $M$ можно определить, какая
монета фальшивая, и легче ли она?
\пункт То же, но эталонной монеты нет.
\пункт То же, но не требуется определять, легче ли фальшивая монета.
\кзадача

\задача
Обезьяна хочет определить, из окна какого самого низкого этажа
15-этажного дома нужно бросить кокосовый орех, чтобы он разбился.
У нее есть
\вСтрочку
\пункт 1;
\пункт 2 ореха.
Какого наименьшего числа бросков ей заведомо хватит?
(Неразбившийся орех можно бросать снова.)
\кзадача

%\сзадача
%Есть $n$ разных ключей от $n$ разных замков (каждый
%ключ подходит ровно к одной двери). За какое наименьшее число
%попыток можно узнать, какую дверь открывает какой ключ?
%\кзадача

\задача
\пункт
Двое показывают карточный фокус.  Первый  снимает
пять карт из колоды, содержащей 52 карты (предварительно перетасованной
кем-то из зрителей), смотрит в них и после этого выкладывает
их в ряд слева направо, причем одну из карт кладет рубашкой
вверх, а остальные --- картинкой вверх. Второй участник фокуса
отгадывает закрытую карту. Докажите, что они могут  так  договориться,
что второй всегда будет угадывать карту.
\пункт
Та же задача, но первый выкладывает слева направо четыре карты картинкой
вверх, а одну не выкладывает. Второй должен угадать невыложенную карту.
\кзадача

\задача
Петя задумал целое число от 1 до 16. Вася может задавать Пете любые
вопросы, на которые можно ответить \лк Да\пк\ или \лк Нет\пк.
Отвечая на эти вопросы, Петя может один раз соврать
(но неизвестно, когда).
Как Васе узнать Петино число, задав не более 7 вопросов?
\кзадача


\задача
Ботанический определитель использует 100 признаков.
Каждый признак либо есть у растения, либо нет.
Определитель \лк хороший\пк, если любые два растения в нем отличаются
более чем по 50 признакам. Может ли хороший определитель
описывать более
\вСтрочку
\пункт 50;
\спункт 34~растений.
\кзадача




\ЛичныйКондуит{0mm}{6mm}

\vspace*{-4mm}

%\GenXMLW


%\СделатьКондуит{8.2mm}{8.2mm}


\end{document}
