% !TeX encoding = windows-1251
\documentclass[a4paper,12pt]{article}
\usepackage{newlistok}

\УвеличитьШирину{1.5cm}
\УвеличитьВысоту{2cm}
\renewcommand{\spacer}{\vfil}
\DeclareMathOperator{\Sym}{Sym}
\sloppy




\begin{document}

\Заголовок{Группы преобразований: орбиты и стабилизаторы}
\НомерЛистка{9д}
\ДатаЛистка{03.2014}
\СоздатьЗаголовок


\опр \выд{Орбитой} элемента $x \in X$ при действии группы преобразований $G$ называется множество $\{g(x) \mid g \in G\} \subset X$.
\выдд Обозначение $Gx$.
\копр

\задача
Найдите орбиту каждой точки при действии группы движений\\
\пункт
квадрата;
\пункт
куба;
\пункт
правильного $m$-угольника.
\кзадача



\задача
\пункт
Опишите группу движений единичного круга;
\пункт
Найдите орбиту каждой точки при действии этой группы;
\пункт
Найдите преобразование, не имеющее конечного порядка.
\кзадача

\задача
Докажите, что любые две орбиты либо совпадают, либо не пересекаются. Следует ли отсюда,
что всё множество $X$ есть объединение непересекающихся орбит?
\кзадача

\задача
Докажите, что для любых двух элементов одной орбиты $a,b \in Gx$ найдётся элемент $g \in G$,
такой что $g(a) = b$.
\кзадача


\опр \выд{Стабилизатором\/} элемента $x \in X$ при действии группы преобразований $G$ называется
множество $\{g \mid g(x)=x\} \subset G$.
\выдд Обозначение: $G_x$.
\копр

\задача
Найдите стабилизаторы каждой из точек следующих множеств при действии их групп движений:
\пункт
квадрата;
\пункт
куба;
\пункт
правильного $m$-угольника.
\кзадача

\задача
Рассмотрим группу движений куба $G$. Эта группа также является группой преобразований следующих множеств:
\пункт множества вершин куба;
\пункт множества диагоналей куба;
\пункт множества граней куба;
\спункт множества пар вершин куба.
Опишите орбиты и стабилизаторы во всех случаях.
\кзадача

\задача
Пусть задана группа преобразований $G$ множества $X$. Докажите, что стабилизатор
любого элемента $x\in X$ также является группой преобразований множества $X$.
\кзадача

\задача
Пусть группа $G$ конечна. Докажите, что для любых двух элементов одной орбиты $a,b \in Gx$ выполнено $|G_a| = |G_b|$.
\кзадача

\задача
Пусть группа $G$ конечна. Докажите, что для любого $x \in X$ верно $|G| = |Gx| \cdot |G_x|$.
\кзадача

\задача
Пусть $p$ --- простое число.
Рассмотрим множество $\Z_p$ остатков по модулю $p$, ненулевой остаток~$a$ и группу~$G$, действующую на~$\Z_p$ домножениями на~$a^k$
(т.е. $G=\{g_0, g_1, g_2,\ldots\}$ и $g_k(x) = x\cdot a^k$).
\\\пункт Найдите орбиты действия этой группы;
\\\пункт[малая теорема Ферма] Докажите, что $a^{p-1} \equiv 1 \pmod{p}$.
\кзадача

\опр
Функция, равная количеству натуральных чисел, меньших $n$ и взаимно простых с ним, называется \выд{функцией Эйлера} и обозначается через $\phi(n)$.
\копр

\задача[теорема Эйлера]
Докажите, что если числа $a$ и $m$ взаимно просты, то $a^{\phi(m)} \equiv 1\pmod{m}$.
\кзадача

\задача
Образует ли группу множество преобразований плоскости, переводящих прямые в прямые?
\кзадача

\задача
\пункт
Пусть $G$ --- группа преобразований множества $X$, и $h\in G$.
Докажите, что отображение $\ad_h \from G \to G$, $g\corr{\ad_h} (h\circ g\circ h^{-1})$ является преобразованием $G$ (такое преобразование называется \emph{сопряжением});
\пункт
Обозначим через $\wt{G} = \hc{\ad_h\mid h\in G}$ множество всех сопряжений группы $G$. Докажите, что $\wt{G}$ образует группу;
% \пункт
% Найдите орбиты элементов $S_n$ при действии $\wt S_n$ (они называются \emph{классами сопряжённых элементов}).
\кзадача


\ссзадача[кубик Рубика]
Опишите геометрию кубика Рубика\footnote{У кубика Рубика $2 125 922 464 947 725 402 112 000 \approx 2.126\cdot 10^{24}$ различных состояний. Известно, что если кубик можно собрать, то это можно сделать за 20 хода. Примечательно, что ещё в 2009 году в этом листке вместо 20 стояло число 22.}:\невСтрочку
\пункт
Придумайте, как описать группу преобразований кубика Рубика;
\пункт
Сколько различных состояний у кубика Рубика (чему равна $\#X$)?
\пункт
Сколько состояний в орбите собранного кубика Рубика?
\пункт
Сколько из этих состояний различимы у реального кубика?
\кзадача

\ЛичныйКондуит{0mm}{6mm}

%%\СделатьКондуит{6.7mm}{6.7mm}


\end{document}
