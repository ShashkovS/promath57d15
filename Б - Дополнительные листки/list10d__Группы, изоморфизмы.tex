% !TeX encoding = windows-1251
\documentclass[a4paper,12pt]{article}
\usepackage{newlistok}

\УвеличитьШирину{1.5cm}
\УвеличитьВысоту{2cm}
\renewcommand{\spacer}{\vfil}
\DeclareMathOperator{\Sym}{Sym}
\sloppy




\begin{document}

\Заголовок{Группы преобразований: изоморфизмы}
\НомерЛистка{10д}
\ДатаЛистка{\ммгг}

\СоздатьЗаголовок

\опр
\label{homo}
Пусть $G$ --- группа преобразований множества $X$, а $H$ --- группа преобразований множества $Y$. Группы $G$ и $H$ называются \выд{изоморфными\/}, если найдётся биекция $\ph \from G \to H$, при которой тождественное преобразование переходит в тождественное, обратное --- в обратное, а композиция преобразований --- в композицию преобразований, то есть:\\
(\emph{i}\/) $\ph(\id_X) = \id_Y$;\\
(\emph{ii}\/) для каждого $g \in G$ верно: $\ph(g^{-1}) = (\ph(g))^{-1}$;\\
(\emph{iii}\/) для любых $g_1,g_2\in G$ верно: $\ph(g_1\circ g_2)=\ph(g_1)\circ\ph(g_2)$.\\
Отображение $\ph$ в этом случае называется \выд{изоморфизмом}.
\выдд Обозначение: $G\isom H$, $G\stackrel{\ph}{\isom} H$.
\копр

\задача
Правда ли, что если $G\isom H$, то
\пункт $\#G = \#H$;
\пункт $\#X = \#Y$?
\кзадача

\задача
Пусть $\ph \from G \to H$ --- биекция, такая что выполнено условие (\emph{iii}\/) определения~\ref{homo}. Докажите, что $\ph$ является изоморфизмом.
\кзадача


\задача
Докажите, что следующие группы изоморфны:\\
\пункт группа вращений правильной 4-угольной призмы (не являющейся кубом) и группа движений квадрата;\\
\пункт группа движений куба и группа движений октаэдра;\\
\пункт группа вращений правильного $n$-угольника и группа вычетов по модулю $n$ (см. задачу \ref{gr}в). Эта группа обозначается $\Z_n$ или $\Z/n\Z$;\\
\спункт группа движений тетраэдра и группа вращений куба.
\кзадача


\задача
Пусть $\ph\from G\to H$ --- изоморфизм. Докажите, что
для любого элемента $g\in G$ верно: $\ord(g) = \ord(\ph(g))$;
\кзадача

\задача
Какие из следующих групп изоморфны:\\
1) группа вращений правильного 24-угольника;\\
2) группа движений правильного 12-угольника;\\
3) группа движений правильной 6-угольной призмы;\\
4) группа движений правильного тетраэдра;\\
5) группа $S_4$?
\кзадача


\раздел{Абстрактные группы}


\опр
\выд Абстрактной \выд группой (или просто \emph{группой}) называется множество $G$ с операцией умножения, обладающей следующими свойствами:\\
(\emph{i}\/) $(ab)c = a(bc)$ для любых $a,b,c\in G$ (\emph{ассоциативность});\\
(\emph{ii}\/) существует такое элемент $e\in G$ (\emph{единица}), что $ae=ea=a$ для любого $a\in G$;\\
(\emph{iii}\/) для всякого элемента $a\in G$ существует такой элемент $a^{-1}\in G$ (\emph{обратный элемент}), что $a a^{-1} = a^{-1} a = e$.
\копр

\задача
Докажите, что всякая группа преобразований с операцией композиции является абстрактной группой.
\кзадача

\задача
\label{gr}
Являются ли  следующие множества с указанными операциями группами:\\
\пункт $(\Z, + )$; \пункт $(\R\setminus \hc{0}, \cdot)$; \пункт $(\text{остатки по модулю }5, +)$; \пункт $(\text{остатки по модулю }5, \cdot)$;\\ \пункт $(\text{ненулевые остатки по модулю }5, \cdot)$; \пункт то же самое по модулю 10.
\кзадача

\задача
\пункт
Пусть $G$ --- группа преобразований множества $X$, и $h\in G$.
Докажите, что отображение $L_h \from G \to G$, $g\corr{L_h} h\circ g$ является преобразованием $G$ (такое преобразование называется \выд{левым сдвигом});\\
\пункт
Реализуйте произвольную абстрактную группу как группу преобразования некоторого множества.
\кзадача

%\задача
%Пусть $G$ --- группа преобразований множества $X$. Обозначим через $\wt{G} = \hc{L_h\mid h\in G}$ множество всех левых сдвигов группы $G$. Докажите, что $\wt{G}$ образует группу, изоморфную $G$.
%\кзадача

\задача
Докажите, что в группе может быть только одна единица, только один обратный элемент.
\кзадача

\задача
Докажите, что группы 1) вращений окружности; 2) комплексных чисел, по модулю равных $1$ c операцией умножения; и 3) группа матриц вида $\rbmat{\cos \ph & - \sin \ph\\\sin \ph &\phantom{-}\cos\ph}$ с операцией умножения $\hr{\begin{smallmatrix}a&b\\c&d\end{smallmatrix}}\cdot\hr{\begin{smallmatrix}x&y\\z&t\end{smallmatrix}}=
\hr{\begin{smallmatrix}ax+bz&ay+bt\\cx+dz&cy+dt\end{smallmatrix}}$ изоморфны.
\кзадача


%%\СделатьКондуит{6.7mm}{6.7mm}

\end{document}
