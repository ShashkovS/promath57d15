% !TeX encoding = windows-1251
\documentclass[a4paper,12pt]{article}
\usepackage{newlistok}

\global\addtolength{\vsize}{-115mm}%
\global\advance\vsize by 10cm
\global\advance\vsize by 10cm

\ВключитьКолонитул

\Заголовок{Графы\т 1}
\НомерЛистка{5д}
\ДатаЛистка{09.2013}

\begin{document}
\СоздатьЗаголовок


\опр
\emph{Графом} называется множество точек на плоскости, некоторые пары которых соединены линиями. Точки называются \emph{вершинами} графа, а~линии\т его \emph{рёбрами}. Ребро, соединяющее некоторую вершину саму с~собой, называется \emph{петлёй}. Рёбра, соединяющие одну и~ту же пару вершин, называются \emph{параллельными} или \emph{кратными}. Говорят, что граф \emph{простой}, если в~нём нет петель и~кратных рёбер.\\
Обычно мы будем рассматривать графы, множество вершин которых конечно.
\копр

\задача
Найдите количество рёбер в~следующих графах:
\невСтрочку
\пункт
вершинами графа являются центры клеток шахматной доски, рёбрами соединены пары вершин, соответствующих клеткам, которые отстоят друг от друга на ход коня;
\пункт
вершины графа сопоставлены двузначным числам; рёбрами соединены пары вершин, для которых разность соответствующих чисел делится на~$10$;
\пункт
простой граф имеет $n$~вершин и~каждая пара вершин соединена ребром (такие графы называются  \emph{полными}).
\кзадача

\задача
Сколько графов можно получить из полного графа с~$n$~вершинами, стирая некоторые его рёбра?
\кзадача

\опр
\emph{Степенью} вершины~$V$ называется число выходящих из неё рёбер (при этом каждая петля учитываются дважды). Обозначение: ${\rm deg}\,V$.
\копр

\задача[Лемма о рукопожатиях]
\невСтрочку
\пункт
Как связаны сумма степеней вершин произвольного графа и~количество его рёбер?
\пункт
Верно ли, что число вершин нечётной степени любого графа чётно?
\пункт
Объясните название данной задачи.
\кзадача

\задача
Верно ли, что если в~простом графе более $1$~вершины, то в~нём найдутся две вершины одинаковой степени?
\кзадача

\задача
У~Пети $28$~одноклассников, причём они имеют различное число друзей в~этом классе. Сколько из них дружит с~Петей?
\кзадача

\задача[Теорема Холла]
В~некоторой компании $n$~юношей. При каждом~$k$ от~$1$ до~$n$ верно утверждение: для любых $k$~юношей в~компании число девушек, знакомых хотя бы с~одним из этих $k$~юношей, не меньше~$k$. Докажите, что можно женить всех юношей на знакомых девушках.
\кзадача

\задача
Каждый из $n$~школьников решил ровно $5$~задач, причём каждую задачу решили ровно $5$~школьников. Докажите, что можно организовать разбор задач таким образом, чтобы каждый школьник рассказал какую-то из решённых им задач и~каждая задача была рассказана ровно один раз.
\кзадача

\опр
\emph{Путь} в~графе\т это последовательность вершин $V_1$, $V_2$, \ldots, $V_{n+1}$, в~которой каждые две соседние вершины соединены ребром. Соответствующую последовательность рёбер $V_1V_2$, $V_2V_3$, \ldots, $V_nV_{n+1}$ также называют путём. Если $V_1=V_{n+1}$, то путь называется \emph{циклическим}; если при этом рёбра пути различны\т \emph{циклом}; а~если ещё и~вершины разные (кроме $V_1$ и~$V_{n+1}$)\т \emph{простым циклом}. Граф называется \emph{связным}, если каждые две его вершины соединены некоторым путём.
\копр

\vfill
\ЛичныйКондуит{0mm}{6mm}
\ОбнулитьКондуит

\newpage


\задача
Сколько рёбер может иметь связный граф с~$n$~вершинами, если
\невСтрочку
\пункт
он не имеет циклов (такие графы называются \emph{деревьями});
\пункт
он имеет ровно два различных простых цикла;
\пункт
он имеет ровно три различных простых цикла?
\кзадача

\задача
Сколько рёбер может быть в~простом несвязном графе с~$n$~вершинами?
\кзадача

\задача
Связен ли простой граф с~$n$~вершинами, если степень каждой его вершины не меньше~$(n-1)/2$?
\кзадача

\задача
Докажите, что граф является деревом тогда и~только тогда, когда каждые две его вершины соединены ровно одним путём с~различными рёбрами.
\кзадача

\задача
Из столицы выходит 101 авиалиния, из города Дальний\т одна, а из остальных городов по 100. Докажите, что из столицы можно долететь в Дальний (возможно, с пересадками).
\кзадача

\задача
Докажите, что в~связном графе есть цикл, содержащий все рёбра, если и~только если степень любой вершины графа чётна (такие графы называются \emph{эйлеровыми}).
\кзадача

\опр
Раскраска вершин графа называется \emph{правильной}, если никакие две вершины одного цвета не соединены ребром. Простой граф называется \emph{$k$-дольным}, если правильная раскраска его вершин возможна $k$~цветами, но не менее (такая раскраска\т \emph{минимальная}).
\копр

\задача
Верно ли, что граф является двудольным в~том и~только в~том случае, когда в~нём отсутствуют циклы нечётной длины?
\кзадача

\сзадача
Есть ли в~$k$-дольном графе с~минимальной окраской путь из $k$~разноцветных вершин?
\кзадача

\задача
Докажите, что из любых шести человек можно выбрать либо трёх попарно знакомых, либо трёх попарно незнакомых (знакомство\т процесс взаимный).
\кзадача

\задача
На танцы пришли $n$~девушек и~$n$~юношей. Каждый юноша знаком с~двумя девушками, а~каждая девушка\т с~двумя юношами. Докажите, что собравшихся можно разбить на $n$~смешанных пар так, чтобы в~каждой паре юноша и~девушка были знакомы, причём число различных разбиений является степенью двойки.
\кзадача

\задача
Каждый из 450 депутатов дал пощёчину ровно одному своему коллеге. Докажите, что из них можно выбрать 150 человек,  среди которых никто никого не бил.
\кзадача

\сзадача
Гриша забыл трёхзначный код своего замк\'а. Замок откроется, если три~цифры кода набраны подряд (даже если ранее были набраны другие цифры). Докажите, что Гриша сможет открыть замок не более чем за $1002$~секунды, набирая по одной цифре в~секунду.
\кзадача

\сзадача
Докажите, что среди любых $50$~человек найдутся двое, у~которых чётное число общих знакомых (быть может,~$0$) среди остальных $48$~человек.
\кзадача


\vfill
\ЛичныйКондуит{0mm}{6mm}

%\GenXMLW

\end{document}




