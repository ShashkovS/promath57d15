% !TeX encoding = windows-1251
\documentclass[a4paper,12pt]{article}
\usepackage{newlistok}
\usepackage[framemethod=tikz]{mdframed}


\global\addtolength{\vsize}{-115mm}%
\global\advance\vsize by 10cm
\global\advance\vsize by 10cm

\ВключитьКолонитул

% \УвеличитьВысоту{1.5cm}
% \УвеличитьШирину{1.5cm}
% \renewcommand{\spacer}{\vfil}

\Заголовок{Графы\т 2}
\НомерЛистка{7д}
\ДатаЛистка{01.2014}

\begin{document}
\СоздатьЗаголовок

\опр
Вершина~$V$ графа~$G$ называется висячей, если $\deg V=1$.
\копр

\задача
\пункт
Сколько висячих вершин может иметь граф с~$n$~вершинами?\\
\пункт
Сколько висячих вершин может иметь связный граф с~$n$~вершинами?\\
\пункт
Сколько висячих вершин может иметь дерево с~$n$~вершинами?
\кзадача

\опр
Граф~$O$ называется \выд остовом связного графа~$G$, если $О$~имеет те же вершины, что и~$G$, является деревом и~получается из~$G$ удалением некоторых рёбер.
\копр

\задача
Всякий ли связный граф имеет остов? Может ли граф иметь несколько различных остовов?
\кзадача

\задача
Всегда ли можно удалить некоторую вершину связного графа вместе со всеми выходящими из неё рёбрами так, чтобы граф остался связным?
\кзадача

\задача
Волейбольная сетка имеет вид прямоугольника размером $50\times600$ клеток. Какое наибольшее число верёвочек можно перерезать так, чтобы сетка не распалась на куски?
\кзадача

\опр
Говорят, что граф \emph{плоский (планарный)}, если его рёбра не пересекаются (нигде, кроме вершин). Такой граф делит плоскость на части, называемые \emph{гранями} графа.
\копр

\задача
Докажите, что связный плоский граф является эйлеровым если и~только если его грани можно раскрасить в~два цвета так, чтобы грани с~общим ребром были разного цвета.
\кзадача

\задача[Формула Эйлера]
Докажите, что если связный плоский граф имеет $V$~вершин, $E$~рёбер и~$F$~граней, то справедливо равенство: $V-E+F=2$.
\кзадача

\задача
Является ли плоским полный граф с~$n$~вершинами, если
\пункт
$n=4$;
\пункт
$n=5$;
\пункт
$n$~произвольно?
\кзадача

\задача
Можно ли построить три дома, вырыть три колодца и~соединить тропинками каждый дом с~каждым колодцем так, чтобы тропинки не пересекались?
\кзадача

\задача
Пусть $G$\т простой плоский граф. Докажите, что
\невСтрочку
\пункт
в~графе~$G$ есть вершина степени меньше~$6$;
\пункт
вершины графа $G$ можно правильно раскрасить в~$5$ или менее цветов.
\кзадача


\ЛичныйКондуит{-0.6mm}{6mm}
\ОбнулитьКондуит
\newpage

\задача
Докажите формулу Эйлера
\невСтрочку
\пункт
для произвольного связного графа с~непересекающимися рёбрами, нарисованного на сфере;
\пункт
для произвольного выпуклого многогранника.
\кзадача

\задача
Дан выпуклый многогранник, грани которого являются $n$-угольниками, и~в~каждой вершине сходится $k$~граней. Докажите, что $1/n+1/k=1/2+1/r$, где $r$\т число его рёбер.
\кзадача

\задача
Выпуклый многогранник называют \emph{правильным}, если все его грани\т правильные $n$-угольники, и~в~каждой его вершине сходится $k$~граней. Докажите, что любой такой многогранник\т либо тетраэдр, либо куб, либо октаэдр, либо додекаэдр, либо икосаэдр.
\кзадача

\замечание
В~каждой вершине тетраэдра сходится три треугольника. В~каждой вершине октаэдра сходится четыре треугольника. В~каждой вершине икосаэдра сходится пять треугольников. И, наконец, в~каждой вершине додекаэдра сходится три пятиугольника.
\кзамечание

\сзадача
В~лесу $k\cdot l$ тропинок и несколько полянок. Каждая тропинка соединяет две полянки. Известно, что
тропинки можно раскрасить в~$l$~цветов так, чтобы к~каждой полянке сходились тропинки разного цвета. Докажите, что это можно сделать, покрасив каждым цветом ровно $k$~тропинок.
\кзадача

\сзадача
В~графе $n$~вершин $A_1$,~\dots, $A_n$ и~$n$~рёбер $b_1$,\dots, $b_n$. Известно, что любые две вершины $A_i$~и~$A_j$ этого графа соединены ребром если и~только если рёбра $b_i$ и $b_j$ выходят из одной
вершины. Докажите, что степень каждой вершины равна двум.
\кзадача

\сзадача
На плоскости отмечено несколько точек, никакие три из которых не лежат на одной прямой. Двое играют в~такую игру: они по очереди соединяют какие-то две ещё не соединённые точки отрезком так, чтобы отрезки не пересекались нигде, кроме отмеченных точек. Проигрывает тот, кто не может сделать ход.
Зависит ли исход этой игры от того, как играют соперники?
\кзадача

\сзадача
Сколько остовов имеет граф с~вершинами $V_0,\ldots,V_n$ и $(2n-1)$~ребром, где
вершина~$V_0$ соединена рёбрами с~остальными вершинами, и~при $1\leq i<n$ соединены ребром
вершины $V_i$~и~$V_{i+1}$?
\кзадача

\сзадача
В~гости ожидают $m$~или $n$~человек, где $(m,n)=1$. На какое наименьшее число секторов надо разрезать круглый торт, чтобы из них можно было сложить как~$m$, так и~$n$~одинаковых кусков?
\кзадача

\сзадача[Теорема Кели]
Докажите, что полный граф с~$n$~вершинами имеет $n^{n-2}$ остовов.
\кзадача

%\vfill
\ЛичныйКондуит{0mm}{6mm}
%\GenXMLW

\end{document}




